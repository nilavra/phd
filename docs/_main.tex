%%%%%%%%%%%%%%%%%%%%%%%%%%%%%%%%%%%%%%%%%%%%%%%%%%%%%%%%%%%%%%%
%% U TEXAS THESIS TEMPLATE
%
% Originally by Keith A. Gillow (gillow@maths.ox.ac.uk), 1997
% Modified by Sam Evans (sam@samuelevansresearch.org), 2007
% Modified by John McManigle (john@oxfordechoes.com), 2015
% Modified by Ulrik Lyngs (ulrik.lyngs@cs.ox.ac.uk), 2018-, for use with R Markdown
% Modified by Nilavra Bhattacharya (nilavra@utexas.edu), 2023-, for use with R Markdown
%
% Nilavra Bhattacharya, 24 Jan 2023: Following previous authors: Ulrik Lyngs and John McManigle, broad permissions are granted to use, modify, and distribute this software
% as specified in the MIT License included in this distribution's LICENSE file.
%
% Ulrik Lyngs, 25 Nov 2018: Following John McManigle, broad permissions are granted to use, modify, and distribute this software
% as specified in the MIT License included in this distribution's LICENSE file.
%
% John commented this file extensively, so read through to see how to use the various options.  Remember that in LaTeX,
% any line starting with a % is NOT executed.

%%%%% PAGE LAYOUT
% The most common choices should be below.  You can also do other things, like replace "a4paper" with "letterpaper", etc.

% 'twoside' formats for two-sided binding (ie left and right pages have mirror margins; blank pages inserted where needed):
%\documentclass[a4paper,twoside]{templates/ociamthesis}
% Specifying nothing formats for one-sided binding (ie left margin > right margin; no extra blank pages):
%\documentclass[a4paper]{ociamthesis}
% 'nobind' formats for PDF output (ie equal margins, no extra blank pages):
%\documentclass[a4paper,nobind]{templates/ociamthesis}

% As you can see from the line below, oxforddown uses the a4paper size, 
% and passes in the binding option from the YAML header in index.Rmd:
\documentclass[letterpaper, nobind]{templates/ociamthesis}


%%%%% ADDING LATEX PACKAGES
% add hyperref package with options from YAML %
\usepackage[pdfpagelabels]{hyperref}
% handle long urls
\usepackage{xurl}
% change the default coloring of links to something sensible
\usepackage{xcolor}
% single double and other spacing
\usepackage{setspace}

% for bengali fonts
% \babelprovide[import]{bangla}
% \babelfont[bangla]{rm}{kalpurush.ttf}

% \usepackage{fontspec} 
% \usepackage{polyglossia} 
% \setdefaultlanguage{english} 
% \setotherlanguages{bengali}
% \newfontfamily{\bengalifont}[Script=Bengali]{kalpurush.ttf}

% \setotherlanguages{hindi,sanskrit,bengali}
% \setmainfont{Times New Roman} 
% \newfontfamily{\devanagarifont}[Script=Devanagari]{Lohit Devanagari.ttf} 


\definecolor{mylinkcolor}{RGB}{0,0,139}
\definecolor{myurlcolor}{RGB}{0,0,139}
\definecolor{mycitecolor}{RGB}{0,33,71}

\hypersetup{
  hidelinks,
  colorlinks,
  linktocpage=true,
  linkcolor=mylinkcolor,
  urlcolor=myurlcolor,
  citecolor=mycitecolor
}


% add float package to allow manual control of figure positioning %
\usepackage{float}

% enable strikethrough
\usepackage[normalem]{ulem}

% use soul package for correction highlighting
\usepackage{color, soulutf8}
\definecolor{correctioncolor}{HTML}{CCCCFF}
\sethlcolor{correctioncolor}
\newcommand{\ctext}[3][RGB]{%
  \begingroup
  \definecolor{hlcolor}{#1}{#2}\sethlcolor{hlcolor}%
  \hl{#3}%
  \endgroup
}
% stop soul from freaking out when it sees citation commands
\soulregister\ref7
\soulregister\cite7
\soulregister\citet7
\soulregister\autocite7
\soulregister\textcite7
\soulregister\pageref7

%%%%% FIXING / ADDING THINGS THAT'S SPECIAL TO R MARKDOWN'S USE OF LATEX TEMPLATES
% pandoc puts lists in 'tightlist' command when no space between bullet points in Rmd file,
% so we add this command to the template
\providecommand{\tightlist}{%
  \setlength{\itemsep}{0pt}\setlength{\parskip}{0pt}}
 
% allow us to include code blocks in shaded environments

% User-included things with header_includes or in_header will appear here
% kableExtra packages will appear here if you use library(kableExtra)
\usepackage{booktabs}
\usepackage{longtable}
\usepackage{array}
\usepackage{multirow}
\usepackage{wrapfig}
\usepackage{float}
\usepackage{colortbl}
\usepackage{pdflscape}
\usepackage{tabu}
\usepackage{threeparttable}
\usepackage{threeparttablex}
\usepackage[normalem]{ulem}
\usepackage{makecell}
\usepackage{xcolor}


%UL set section header spacing
\usepackage{titlesec}
% 
\titlespacing\subsubsection{0pt}{24pt plus 4pt minus 2pt}{0pt plus 2pt minus 2pt}


%UL set whitespace around verbatim environments
\usepackage{etoolbox}
\makeatletter
\preto{\@verbatim}{\topsep=0pt \partopsep=0pt }
\makeatother


%%%%%%% PAGE HEADERS AND FOOTERS %%%%%%%%%
\usepackage{fancyhdr}
\setlength{\headheight}{15pt}
\fancyhf{} % clear the header and footers
\pagestyle{fancy}
\renewcommand{\chaptermark}[1]{\markboth{\thechapter. #1}{\thechapter. #1}}
\renewcommand{\sectionmark}[1]{\markright{\thesection. #1}} 
\renewcommand{\headrulewidth}{0pt}

\fancyhead[LO]{\emph{\leftmark}} 
\fancyhead[RE]{\emph{\rightmark}} 




% UL page number position 
\fancyfoot[C]{\emph{\thepage}} %regular pages
\fancypagestyle{plain}{\fancyhf{}\fancyfoot[C]{\emph{\thepage}}} %chapter pages




%%%%% SELECT YOUR DRAFT OPTIONS
% This adds a "DRAFT" footer to every normal page.  (The first page of each chapter is not a "normal" page.)
\fancyhead[R]{\emph{Draft \today}}

% IP feb 2021: option to include line numbers in PDF

% for line wrapping in code blocks
\usepackage{fancyvrb}
\usepackage{fvextra}
\DefineVerbatimEnvironment{Highlighting}{Verbatim}{breaklines=true, breakanywhere=true, commandchars=\\\{\}}

% This highlights (in blue) corrections marked with (for words) \mccorrect{blah} or (for whole
% paragraphs) \begin{mccorrection} . . . \end{mccorrection}.  This can be useful for sending a PDF of
% your corrected thesis to your examiners for review.  Turn it off, and the blue disappears.
\correctionstrue


%%%%% BIBLIOGRAPHY SETUP
% Note that your bibliography will require some tweaking depending on your department, preferred format, etc.
% If you've not used LaTeX before, I recommend just using pandoc for citations -- this is what's used unless you specific e.g. "citation_package: natbib" in index.Rmd
% If you're already a LaTeX pro and are used to natbib or something, modify as necessary.

% this allows the latex template to handle pandoc citations
\newlength{\cslhangindent}
\setlength{\cslhangindent}{1.5em}
\newlength{\csllabelwidth}
\setlength{\csllabelwidth}{3em}
\newlength{\cslentryspacingunit} % times entry-spacing
\setlength{\cslentryspacingunit}{\parskip}
\newenvironment{CSLReferences}[2] % #1 hanging-ident, #2 entry spacing
 {% don't indent paragraphs
  \setlength{\parindent}{0pt}
  % turn on hanging indent if param 1 is 1
  \ifodd #1
  \let\oldpar\par
  \def\par{\hangindent=\cslhangindent\oldpar}
  \fi
  % set entry spacing
  \setlength{\parskip}{1mm}
  \setlength{\baselineskip}{6mm}
 }%
 {}
\usepackage{calc}
\newcommand{\CSLBlock}[1]{#1\hfill\break}
\newcommand{\CSLLeftMargin}[1]{\parbox[t]{\csllabelwidth}{#1}}
\newcommand{\CSLRightInline}[1]{\parbox[t]{\linewidth - \csllabelwidth}{#1}\break}
\newcommand{\CSLIndent}[1]{\hspace{\cslhangindent}#1}




% Uncomment this if you want equation numbers per section (2.3.12), instead of per chapter (2.18):
%\numberwithin{equation}{subsection}


%%%%% THESIS / TITLE PAGE INFORMATION
% Everybody needs to complete the following:
\title{LongSAL: A Longitudinal Search as Learning Study With University Students {[}work in progress{]}}
\author{Nilavra Bhattacharya}


% Master's candidates who require the alternate title page (with candidate number and word count)
% must also un-comment and complete the following three lines:

% Uncomment the following line if your degree also includes exams (eg most masters):
%\renewcommand{\submittedtext}{Submitted in partial completion of the}
% Your full degree name.  (But remember that DPhils aren't "in" anything.  They're just DPhils.)
\degree{Doctor of Philosophy}

% Term and year of submission, or date if your board requires (eg most masters)



%%%%% YOUR OWN PERSONAL MACROS
% This is a good place to dump your own LaTeX macros as they come up.

% To make text superscripts shortcuts
\renewcommand{\th}{\textsuperscript{th}} % ex: I won 4\th place
\newcommand{\nd}{\textsuperscript{nd}}
\renewcommand{\st}{\textsuperscript{st}}
\newcommand{\rd}{\textsuperscript{rd}}

%%%%% THE ACTUAL DOCUMENT STARTS HERE
\begin{document}

%%%%% CHOOSE YOUR LINE SPACING HERE
% This is the official option.  Use it for your submission copy and library copy:
\setlength{\textbaselineskip}{22pt plus2pt}
% This is closer spacing (about 1.5-spaced) that you might prefer for your personal copies:
%\setlength{\textbaselineskip}{18pt plus2pt minus1pt}

% You can set the spacing here for the roman-numbered pages (acknowledgements, table of contents, etc.)
\setlength{\frontmatterbaselineskip}{17pt plus1pt minus1pt}

% UL: You can set the line and paragraph spacing here for the separate abstract page to be handed in to Examination schools
\setlength{\abstractseparatelineskip}{13pt plus1pt minus1pt}
\setlength{\abstractseparateparskip}{0pt plus 1pt}

% UL: You can set the general paragraph spacing here - I've set it to 2pt (was 0) so
% it's less claustrophobic
\setlength{\parskip}{2pt plus 1pt}

%
% Customise title page
%
\def\crest{}
\renewcommand{\university}{The University of Texas at Austin}
\renewcommand{\submittedtext}{}
\renewcommand{\thesistitlesize}{\fontsize{22pt}{28pt}\selectfont}
\renewcommand{\gapbeforecrest}{25mm}
\renewcommand{\gapaftercrest}{25mm
}


% Leave this line alone; it gets things started for the real document.
\setlength{\baselineskip}{\textbaselineskip}


%%%%% CHOOSE YOUR SECTION NUMBERING DEPTH HERE
% You have two choices.  First, how far down are sections numbered?  (Below that, they're named but
% don't get numbers.)  Second, what level of section appears in the table of contents?  These don't have
% to match: you can have numbered sections that don't show up in the ToC, or unnumbered sections that
% do.  Throughout, 0 = chapter; 1 = section; 2 = subsection; 3 = subsubsection, 4 = paragraph...

% The level that gets a number:
\setcounter{secnumdepth}{4}
% The level that shows up in the ToC:
\setcounter{tocdepth}{4}


%%%%% ABSTRACT SEPARATE
% This is used to create the separate, one-page abstract that you are required to hand into the Exam
% Schools.  You can comment it out to generate a PDF for printing or whatnot.

% JEM: Pages are roman numbered from here, though page numbers are invisible until ToC.  This is in
% keeping with most typesetting conventions.
\begin{romanpages}

% Title page is created here
% original code:
% \maketitle

% --------- start: UTexas Frontmatter (committee membership, title page) -------

% ~~~~ committee membership page ~~~~
% \thispagestyle{empty}
\begin{center}
  The Dissertation Committee for Nilavra Bhattacharya\\
  certifies that this is the approved version of the following Dissertation:\\
  \vspace*{30pt}
  \begin{spacing}{2}
    {\Large{\textbf{LongSAL: A Longitudinal Search as Learning Study With University Students {[}work in progress{]}}}}
  \end{spacing}
\end{center}

\vspace*{55pt}

\phantom{x}\hspace{45ex} {\large{\textbf{Committee:}}}\\
% \vspace*{12pt}

\begin{flushright}
  Jacek Gwizdka, Supervisor\\
  \vspace*{24pt}
  Soo-Young Rieh\\
  \vspace*{24pt}
  Matthew Lease\\
  \vspace*{24pt}
  Robert Capra
\end{flushright}


% ~~~~ title page ~~~~
\newpage
% \thispagestyle{empty}
\begin{center}
  
  \begin{spacing}{2}
    {\Large{\textbf{LongSAL: A Longitudinal Search as Learning Study With University Students {[}work in progress{]}}}}
  \end{spacing}
  
  \vspace*{24pt}

  \begin{spacing}{1.4}
    by\\
    
    \vspace*{24pt}
    
    {\Large{\textbf{
      Nilavra Bhattacharya\\
      নীলাভ্র ভট্টাচার্য্য
    }}}\\
    
    \vspace*{72pt}
    
    {\Large{\textbf{Dissertation}}}\\
    
    \vspace*{24pt}
    
    Presented to the Faculty of the Graduate School of\\
    The University of Texas at Austin\\
    in Partial Fulfillment\\
    of the Requirements\\
    for the Degree of\\
    
    \vspace*{30pt}
    
    {\Large{\textbf{Doctor of Philosophy}}}\\
    
    \vfill

    {\large{The University of Texas at Austin\\
    May 2023}}

  \end{spacing}
\end{center}


% --------- end: UTexas Frontmatter (committee membership, title page) -------


%%%%% DEDICATION

%%%%% ACKNOWLEDGEMENTS


\begin{acknowledgements}
 	This section will be fleshed out in more detail after the initial committee-submission on Feb 27, 2023.
 For now, I wish to thank the following people :

 \begin{itemize}
 \tightlist
 \item
   Committee Members: Jacek Gwizdka, Soo Young Rieh, Matt Lease, Rob Capra
 \end{itemize}
\end{acknowledgements}



%%%%% ABSTRACT


% ----------- original abstract ----------
% \renewcommand{\abstracttitle}{Abstract}
% \begin{abstract}
% 	Learning today is about navigation, discernment, induction, and synthesis of the wide body of information on the Internet present ubiquitously at every student's fingertips. Learning, or addressing a gap in one's knowledge, has been well established as an important motivator behind information-seeking activities. The Search as Learning research community advocates that online information search systems should be reconfigured to become educational platforms to foster learning and sensemaking. Modern search systems have yet to adapt to support this function. An important step to foster learning during online search is to identify behavioural patterns that distinguish searchers gaining more vs.~less knowledge during search. Previous efforts have primarily studied searchers in the short term, typically during a single lab session. Many researchers have expressed their concern over this ephemeral approach, as learning takes place over time, and is not fleeting. We propose an exploratory longitudinal study to analyze the long-term searching behaviour of students enrolled in a university course, over the span of a university semester. Our research aims are to identify if and how students' searching behaviour changes over time, as they gain new knowledge on a subject; and how do processes like motivation, metacognition, self-regulation, and other individual differences moderate their `searching as learning' behaviour. Findings from this exploratory longitudinal study will help to build improved search systems that foster human learning and sensemaking, and are more equitable in the face of learner diversity.
% \end{abstract}

% --------- start: UTexas abstract -------
\begin{center}
  \textbf{Abstract}\\
  
  \vspace{18pt}
  
  \begin{spacing}{2}
    {\Large{\textbf{LongSAL: A Longitudinal Search as Learning Study With University Students {[}work in progress{]}}}}
  \end{spacing}

  \vspace{18pt}

  \begin{spacing}{1.4}
    Nilavra Bhattacharya নীলাভ্র ভট্টাচার্য্য, PhD TBD\\
    The University of Texas at Austin, 2023\\
    \vspace{24pt}
    Supervisor: Jacek Gwizdka
  \end{spacing}

\end{center}

\begin{spacing}{1.5}
  \indent
  % \indent
  Learning today is about navigation, discernment, induction, and synthesis of the wide body of information on the Internet present ubiquitously at every student's fingertips. Learning, or addressing a gap in one's knowledge, has been well established as an important motivator behind information-seeking activities. The Search as Learning research community advocates that online information search systems should be reconfigured to become educational platforms to foster learning and sensemaking. Modern search systems have yet to adapt to support this function. An important step to foster learning during online search is to identify behavioural patterns that distinguish searchers gaining more vs.~less knowledge during search. Previous efforts have primarily studied searchers in the short term, typically during a single lab session. Many researchers have expressed their concern over this ephemeral approach, as learning takes place over time, and is not fleeting. We propose an exploratory longitudinal study to analyze the long-term searching behaviour of students enrolled in a university course, over the span of a university semester. Our research aims are to identify if and how students' searching behaviour changes over time, as they gain new knowledge on a subject; and how do processes like motivation, metacognition, self-regulation, and other individual differences moderate their `searching as learning' behaviour. Findings from this exploratory longitudinal study will help to build improved search systems that foster human learning and sensemaking, and are more equitable in the face of learner diversity.
\end{spacing}

% --------- end: UTexas abstract -------



%%%%% MINI TABLES
% This lays the groundwork for per-chapter, mini tables of contents.  Comment the following line
% (and remove \minitoc from the chapter files) if you don't want this.  Un-comment either of the
% next two lines if you want a per-chapter list of figures or tables.
\dominitoc % include a mini table of contents

% This aligns the bottom of the text of each page.  It generally makes things look better.
\flushbottom

% This is where the whole-document ToC appears:
\tableofcontents

\listoffigures
	\mtcaddchapter
  	% \mtcaddchapter is needed when adding a non-chapter (but chapter-like) entity to avoid confusing minitoc

% Uncomment to generate a list of tables:
\listoftables
  \mtcaddchapter
%%%%% LIST OF ABBREVIATIONS
% This example includes a list of abbreviations.  Look at text/abbreviations.tex to see how that file is
% formatted.  The template can handle any kind of list though, so this might be a good place for a
% glossary, etc.

% The Roman pages, like the Roman Empire, must come to its inevitable close.
\end{romanpages}

%%%%% CHAPTERS
% Add or remove any chapters you'd like here, by file name (excluding '.tex'):
\flushbottom

% all your chapters and appendices will appear here
\hypertarget{introduction}{%
\chapter{Introduction}\label{introduction}}

\hypertarget{sec-intro-overview}{%
\section{Searching as Learning: Overview}\label{sec-intro-overview}}

Searching for information is a fundamental human activity. In the modern world, it is frequently conducted by users interacting with online search systems (e.g., web search engines), or more formally, \textbf{Information Retrieval} (IR) systems.
As early as in 1980, Bertam Brookes, in his `fundamental equation' of information and knowledge, had stated that an information searcher's current state of knowledge is changed to a new knowledge structure by exposure to information (\protect\hyperlink{ref-brookes1980foundations}{Brookes, 1980, p. 131}).
This indicates that searchers acquire new knowledge in the search process, and the same information will have different effects on different searchers' knowledge states.
Fifteen years later, Marchionini (\protect\hyperlink{ref-marchionini1995information}{1995}) described information seeking as ``a process, in which humans purposefully engage in order to change
their state of knowledge''.
Thus, we have known for quite a while that search is driven by higher-level human needs, and IR systems are a means to an end, and not the end in itself.
\textbf{Interactive information retrieval} (IIR), a.k.a. human-computer information retrieval (HCIR) (\protect\hyperlink{ref-marchionini2006toward}{Marchionini, 2006}) refers to the study and evaluation of users' interaction with IR systems and users' satisfaction with the retrieved information (\protect\hyperlink{ref-borlund2013interactive}{Borlund, 2013}).

Despite their technological marvels, modern IR systems falls short in several aspects of fully satisfying the higher level human need for information.
In essence, IR systems are software that take, as input, some query, and return as output some ranked list of resources.

\begin{quote}
\emph{Within the context of information seeking, (search engines and IR systems) \textbf{feel} like they play a prominent role in our lives, when in actuality, they only play a small role: the \textbf{retrieval} part of information \ldots{}}

\begin{itemize}
\item
  \emph{Search engines \textbf{don't help us identify what we need} -- that's up to us; search engines don't question what we ask for, though they do recommend queries that use similar words.}
\item
  \emph{Search engines \textbf{don't help us choose a source} -- though they are themselves a source, and a heavily marketed one, so we are certainly compelled to choose search engines over other sources, even when other sources might have better information.}
\item
  \emph{Search engines \textbf{don't help us express our query} accurately or precisely -- though they will help with minor spelling corrections.}
\item
  \emph{Search engines do help retrieve information---this is the primary part that they automate.}
\item
  \emph{Search engines \textbf{don't help us evaluate the answers we retrieve} -- it's up to us to decide whether the results are relevant, credible, true; Google doesn't view those as their responsibility.}
\item
  \emph{Search engines \textbf{don't help us sensemake} -- we have to use our minds to integrate what we've found into our knowledge.}
\end{itemize}

\hfill --- Ko (\protect\hyperlink{ref-ko2021seeking}{2021})
\end{quote}

In recent years, the IIR research community has been actively promoting the \textbf{Search as Learning} (SAL) research direction.
This fast-growing community of researchers propose that search environments should be augmented and reconfigured to foster learning, sensemaking, and long-term knowledge-gain.
Various workshops and seminars have been organized to develop research agendas at the interaction of IIR and the Learning Sciences (\protect\hyperlink{ref-agosti2014evaluation}{Agosti et al., 2014}; \protect\hyperlink{ref-allan2012frontiers}{Allan et al., 2012}; \protect\hyperlink{ref-collins2017search}{Collins-Thompson et al., 2017}; \protect\hyperlink{ref-freund2013searching}{Freund et al., 2013}, \protect\hyperlink{ref-freund2014searching}{2014}; \protect\hyperlink{ref-gwizdka2016search}{Gwizdka et al., 2016}).
Additionally, special issues on Search as Learning have also been published in the \emph{Journal of Information Science} (\protect\hyperlink{ref-hansen2016editorial}{Hansen \& Rieh, 2016}) and in the \emph{Information Retrieval Journal} (\protect\hyperlink{ref-eickhoff2017introduction}{Eickhoff et al., 2017}).
Articles in these special issued presented landmark literature reviews (\protect\hyperlink{ref-rieh2016searching}{Rieh et al., 2016}; \protect\hyperlink{ref-vakkari2016searching}{Vakkari, 2016}), research agendas, and ideas
in this direction.
Overall, these works generally advocate that future research in this domain should aim to:

\begin{itemize}
\tightlist
\item
  understand the contexts in which people search to learn
\item
  understand factors that can influence learning outcomes
\item
  understand how search behaviours can predict learning outcomes
\item
  develop search systems to better support learning and sensemaking
\item
  help searchers be more critical consumers of information
\item
  understand the cognitive biases fostered by existing search systems
\item
  develop search engine ranking algorithms and interface tools that foster long term knowledge gain
\end{itemize}

Parallelly, the Educational Science and the Learning Science research communities have also been organizing workshops and formulating research
agendas to conceptualize forms of `new learning' (\protect\hyperlink{ref-cope2013new}{Cope \& Kalantzis, 2013}; \protect\hyperlink{ref-kalantzis2012newa}{Kalantzis \& Cope, 2012}; \protect\hyperlink{ref-newlondon1996pedagogy}{New London Group, 1996}) that are afforded by innovations in digital technologies and e-learning ecologies (\protect\hyperlink{ref-cope2017elearningc}{Cope \& Kalantzis, 2017}).
Higher education researchers have been increasingly studying how students' information search and information use behaviour affect and support their learning (\protect\hyperlink{ref-weber2019informationseeking}{Weber et al., 2019}, \protect\hyperlink{ref-weber2018can}{2018}; \protect\hyperlink{ref-zlatkin2021students}{Zlatkin-Troitschanskaia et al., 2021}).
Efforts are underway to conceptualize a theoretical framework around new forms of e-Learning that is aided and afforded by digital technologies (\protect\hyperlink{ref-amina2017active}{Amina, 2017}; \protect\hyperlink{ref-cope2017elearningc}{Cope \& Kalantzis, 2017}).
In the community's own words: ``learning today is more about navigation, discernment, induction, and synthesis'' of the wide body of information present ubiquitously at every student's fingertips (\protect\hyperlink{ref-amina2017active}{Amina, 2017}).
Therefore, ``knowing the source, finding the source, and using the information aptly is important to learn and know now more than ever before'' (\protect\hyperlink{ref-cope2013new}{Cope \& Kalantzis, 2013}).
All of these interests in the intersection of searching and learning goes to emphasize that understanding learning during search is critical to
improve human-information interaction.

\hypertarget{sec-intro-problem-statement}{%
\section{Problem Statement}\label{sec-intro-problem-statement}}

A major limitation in the area of Search as Learning, Interactive IR (IIR), and more broadly, in Human-Computer Interaction (HCI) research is
that, the user is examined in the short-term, typically over the course of a single experimental session in a lab
(\protect\hyperlink{ref-karapanos2021advances}{Karapanos et al., 2021}; \protect\hyperlink{ref-kelly2009evaluation}{Kelly et al., 2009}; \protect\hyperlink{ref-HCIUXres81_online}{Koeman, 2020}; \protect\hyperlink{ref-zlatkin2021students}{Zlatkin-Troitschanskaia et al., 2021}).
Very few studies exist in the search-as-learning domain that have observed \emph{the same participant} over a longer period of time than a single search session (\protect\hyperlink{ref-kelly2006measuring_a}{Kelly, 2006a}, \protect\hyperlink{ref-kelly2006measuring_b}{2006b}; \protect\hyperlink{ref-kuhlthau2004seeking}{Kuhlthau, 2004}; \protect\hyperlink{ref-vakkari2001changes}{Vakkari, 2001}; \protect\hyperlink{ref-white2009characterizing}{White et al., 2009}; \protect\hyperlink{ref-wildemuth2004effects}{Wildemuth, 2004}).
This ephemeral approach has acute implications in any domain where learning is involved because ``learning is a \emph{process} that leads to \emph{change} in knowledge \ldots{} (which) unfolds over time'' (\protect\hyperlink{ref-ambrose2010howa}{Ambrose et al., 2010}), and ``\ldots does not happen all at once''(\protect\hyperlink{ref-white_2016_iwss_learning}{White, 2016b}).

\textbf{To the best of the author's knowledge, almost no new longitudinal studies were reported in major search-as-learning literature in the last five years, that systematically studied students' information search behaviour and information-use over the long term, in their \emph{in-situ} naturalistic environment and contexts, and linked those behaviours quantitatively to the students' learning outcomes and individual differences.}

Higher education students are increasingly using the Internet as their main learning environment and source of information when studying. Yet, the short term nature of research in this domain creates significant gaps in our knowledge regarding how students' information search behaviour and information use develop over time, and how it affects their learning (\protect\hyperlink{ref-zlatkin2021students}{Zlatkin-Troitschanskaia et al., 2021}).

\begin{quote}
\emph{When research in this area relies so heavily on (short-term) lab studies, can we realistically say we are comprehensively studying human-tech interactions -- when many of those interactions take place over long periods of time in real-world contexts? \ldots{} An over-reliance on short studies risks inaccurate findings, potentially resulting in prematurely embracing or disregarding new concepts.}

\hfill --- Koeman (\protect\hyperlink{ref-HCIUXres81_online}{2020})
\end{quote}

Current search engines and information retrieval systems ``do not help us know what we want to know, \ldots do not help us know if what we've found is relevant or true; and they do not help us make sense of the retrieved information.
All they do is quickly retrieve what other people on the internet have shared'' (\protect\hyperlink{ref-ko2021seeking}{Ko, 2021}).
Unless we have more long-term understanding of the nature of knowledge gain during search, the limitations of current search systems will continue to persist.
Increased knowledge and understanding of students', and more broadly searchers', information searching and learning behaviour over time will help us to overcome the limitations of current IR systems, and transform them into rich learning spaces where ``search experiences and learning experiences are intertwined and even synergized'' (\protect\hyperlink{ref-url_rieh_homepage}{Rieh, 2020}).
The internet and digital educational technologies offer great opportunities to transform learning and the education experience.
Enabled by our increased comprehension of the longitudinal searching-as-learning process, improved and validated by empirical data, we can create a new wave of fundamentally transformative educational technologies and ``e-learning ecologies, that will be more engaging for learners, more effective (than traditional classroom practices), more resource efficient, and more equitable in the face of learner diversity'' (\protect\hyperlink{ref-cope2017elearningc}{Cope \& Kalantzis, 2017}).

\hypertarget{sec-intro-purpose}{%
\section{Purpose of this Dissertation Proposal}\label{sec-intro-purpose}}

To address the gaps in our knowledge of how information searching influences students' learning process over time, this dissertation proposal proposes to conduct a semester-long longitudinal study (approx. 16 weeks) with university student participants.
The overarching research aim is to identify how students' online searching behaviour correlate with their learning outcomes for a particular university course.
Building upon principles from the Learning Sciences (\protect\hyperlink{ref-ambrose2010howa}{Ambrose et al., 2010}; \protect\hyperlink{ref-council2000how}{National Research Council, 2000}; \protect\hyperlink{ref-novak2010learninga}{Novak, 2010}; \protect\hyperlink{ref-sawyer2005cambridge}{Sawyer, 2005}),
and empirical evidences from the Information Sciences (\protect\hyperlink{ref-rieh2016searching}{Rieh et al., 2016}; \protect\hyperlink{ref-vakkari2016searching}{Vakkari, 2016}; \protect\hyperlink{ref-white2016interactions}{White, 2016a}),
this dissertation proposal aims to:

\begin{itemize}
\tightlist
\item
  situate students as learners in their naturalistic contexts, and characterized by their individual differences
\item
  measure students' information search and information use behaviour over time
\item
  correlate the information search behaviour with the learning outcomes for the university course
\end{itemize}

Learning, or addressing a gap in one's knowledge, has been well established as an important motivator behind information-seeking activities
Section \ref{sec-intro-overview}.
Therefore, search systems that support rapid learning across a number of searchers, and a range of tasks, can be considered as more effective search systems (\protect\hyperlink{ref-white2016interactions}{White, 2016a, p. 310}).
This dissertation proposal takes a step in this direction.
``It opens great expectations for many-sided, great contribution to our knowledge on the relations between search process and learning outcomes'' (anonymous reviewer for \protect\hyperlink{ref-bhattacharya2021longitudinal}{Bhattacharya, 2021}).

\hypertarget{sec-intro-outline}{%
\section{Outline}\label{sec-intro-outline}}

This dissertation proposal document is structured as follows.
First, principles of learning and relevant background from the domain of Educational Sciences are presented in Chapter 2.
Next, relevant empirical evidences from the Information Searching Literature are discussed in Chapter 3.
Chapter 4 presents the research questions, the overarching hypotheses, and discusses their rationale in the context of
the existing research gaps.
Chapter 5 describes the research methods, including the longitudinal study design, experimental procedures, data collection and analyses plans, anticipated limitations, and expected schedule to complete the dissertation.

\hypertarget{ch-bg-learn}{%
\chapter{Background: Knowledge and Learning}\label{ch-bg-learn}}

This first chapter on background literature discusses relevant concepts
from the disciplines of Education and Learning Sciences. First, we
introduce some relevant terminology, and the concepts of deep or
meaningful learning. Then we discuss several research backed principles
that have been shown to lead to meaningful learning. Next, we discuss
how learning, sensemaking, and searching for information are related,
and how modern technologies provide affordances for new forms of
learning and knowledge work in the 21st century. We also discuss some
concepts about individual differences of learners as well as techniques
that can promote better learning. In the last section, we state what
implications these findings have for shaping the proposed study in this
dissertation proposal.

\hypertarget{sec-bg-learn-terminology}{%
\section{Terminology}\label{sec-bg-learn-terminology}}

The Webster dictionary\footnote{\url{https://www.merriam-webster.com/dictionary/knowledge}} defines \textbf{knowledge} in two ways. The first
definition is ``the range of one's information or understanding''.
Vakkari (\protect\hyperlink{ref-vakkari2016searching}{2016}) says it is ``the totality what a person knows,
that is, a \textbf{personal knowledge} or \textbf{belief system}. It may include
both justified, true beliefs and less justified, not so true beliefs,
which the person more or less thinks hold true.'' Webster's second
definition of knowledge is ``the sum of what is known: the body of truth,
information, and principles acquired by humankind''. We can regard this
as \textbf{universal knowledge}.

\textbf{Learning} is a \emph{process}, that leads to a \emph{change} in (personal)
knowledge, beliefs, behaviours, and attitudes (\protect\hyperlink{ref-ambrose2010howa}{Ambrose et al., 2010}). Thus,
learning always aims to increase one's personal knowledge, and can often
draw from the body of universal knowledge. In some cases, the change in
personal knowledge can also lead to change in universal knowledge, such
as when new discoveries are made, or new philosophies are proposed.
Human learning is an innate capacity. It is longitudinal and unfolds
over time. Learning is lifelong and life-wide, and has a lasting impact
on how humans think and act (\protect\hyperlink{ref-ambrose2010howa}{Ambrose et al., 2010}; \protect\hyperlink{ref-kalantzis2012newa}{Kalantzis \& Cope, 2012}).
Learning can be informal or formal. \textbf{Informal learning} is the casual
learning taking place in everyday life, and is incidental to the
everyday life experience. \textbf{Formal learning} is the deliberate,
conscious, systematic, and explicit acquiring of knowledge
(\protect\hyperlink{ref-kalantzis2012newa}{Kalantzis \& Cope, 2012}).

\textbf{Education} is a form of formal learning. It is the systematic
acquiring of knowledge. In today's world, the institutions of education
are formally constructed places (classrooms), times (of the day and of
life) and social relations (teachers and students); for instance,
schools, colleges, and universities. The scientific discipline of
Education concerns itself with the systematic investigation of the ways
in which humans know and learn. It is the science of ``coming to know''
(\protect\hyperlink{ref-kalantzis2012newa}{Kalantzis \& Cope, 2012}).

\textbf{Pedagogy} describes small sequences of learner activities that
promote learning in educational settings (\protect\hyperlink{ref-kalantzis2012newa}{Kalantzis \& Cope, 2012}).
Traditional approaches to (classroom) pedagogy, especially the \emph{didactic
pedagogy}, primarily involves a teacher telling, and a learner
listening. The teacher is in command of the knowledge, and their mission
is to transmit this knowledge to the learners, in a one-way flow. It is
hoped that the learners will dutifully absorb the knowledge laid before
them by the teacher. The balance of agency weighs heavily towards the
teacher. ``There is a special focus on long-term memory, or retention,
measurable by the ritual of closed-book, summative examination''
(\protect\hyperlink{ref-cope2017elearningc}{Cope \& Kalantzis, 2017}).

Cognitive scientists had discovered that learners retain material
better, and are able to generalize and apply it to a broader range of
contexts, when they learn \textbf{deep knowledge} rather than \textbf{surface
knowledge}, and when they learn how to use that knowledge in real-world
social and practical settings (\protect\hyperlink{ref-sawyer2005cambridge}{Sawyer, 2005}). Deep learning \footnote{
  of the human kind}
takes place when ``the learner chooses conscientiously to integrate new
knowledge to knowledge that the learner already possesses'' and involves
``substantive, non-arbitrary incorporations of concepts into cognitive
structure'' (\protect\hyperlink{ref-novak2002meaningful}{Novak, 2002, p. 549}) and may eventually lead to the
development of transferable knowledge and skills. A parallel terminology
for deep learning (\protect\hyperlink{ref-marton1976qualitative_b}{Marton \& Säaljö, 1976}; \protect\hyperlink{ref-marton1976qualitative_a}{Marton \& Säljö, 1976})
is \textbf{meaningful learning}
(\protect\hyperlink{ref-ausubel1968educational}{Ausubel et al., 1968}; \protect\hyperlink{ref-novak2002meaningful}{Novak, 2002}), and they are often
contrasted with \emph{surface learning} or \emph{rote learning}.
Table
discusses some more
details on deep or meaningful learning, and the limitations of
traditional classroom practices to promote deep learning.
Figure
describes (using a concept map) how
meaningful learning can be achieved and sustained, and our annotations
highlight how Search-as-learning systems can foster the same.

\hypertarget{sec-bg-learn-principles}{%
\section{Principles of Meaningful Learning}\label{sec-bg-learn-principles}}

Ambrose et al. (\protect\hyperlink{ref-ambrose2010howa}{2010}) have proposed several principles of (student)
learning that lead to creation of deeper knowledge in learners, and help
educators understand why certain teaching approaches may help or hinder
learning. These principles are based on research and literature from a
range of disciplines in psychology, education, and anthropology, and the
authors claim they are domain independent, experience independent, and
cross-culturally relevant.

\begin{enumerate}
\def\labelenumi{\arabic{enumi}.}
\tightlist
\item
  Students' \textbf{prior knowledge} can help or hinder learning.
\item
  How students \textbf{organize knowledge} influences how they learn and apply what they know.
\item
  Students' \textbf{motivation} determines, directs, and sustains what they do to learn.
\item
  Goal-directed practice coupled with \textbf{targeted feedback} enhances the quality of students' learning.
\item
  Students' current level of development interacts with the social, emotional, and intellectual \textbf{context} around the student to impact learning.
\item
  To become \textbf{self-directed} learners, students must learn to \textbf{monitor and adjust} their approaches to learning.
\end{enumerate}

In line with the above, the US National Research Council identified
several key principles about \textbf{experts' knowledge} (\protect\hyperlink{ref-council2000how}{National Research Council, 2000}),
that illustrate the outcome of successful learning:

\begin{enumerate}
\def\labelenumi{\arabic{enumi}.}
\item
  Experts notice features and \textbf{meaningful patterns} of information
  that are not noticed by novices.
\item
  Experts have acquired a great deal of content knowledge that is
  \textbf{organized} in ways that reflect a deep understanding of their
  subject matter.
\item
  Experts' knowledge cannot be reduced to sets of isolated facts or
  propositions but, instead, reflects contexts of \textbf{applicability}:
  that is, the knowledge is `conditionalized' on a set of
  circumstances.
\item
  Experts are able to \textbf{flexibly retrieve} important aspects of their
  knowledge with little attentional effort.
\item
  Though experts know their disciplines thoroughly, this does not
  guarantee that they are able to teach others.
\item
  Experts have varying levels of flexibility in their approach to new
  situations.
\end{enumerate}

The principles of learning illustrate that both the \emph{context} of
learning, and the \emph{individual differences} of learners moderate the
learning process. The findings about expert knowledge suggests that
\emph{incorporating new information into existing knowledge structures} in a
meaningful manner is a key aspect of learning. We discuss these concepts
in more detail in the following sections.

\hypertarget{sec-bg-learn-sensemaking}{%
\section{Meaningful Learning as Sensemaking}\label{sec-bg-learn-sensemaking}}

In this section, we discuss how meaningful learning can be further
qualified using the concepts of sensemaking.
\textbf{Sensemaking}\footnote{
  ``Brenda Dervin, one of the originators of the sense-making methodology, prefers the spelling with a hyphen, while the community in computer science and more technical people in information science (e.g., SIGCHI) use sensemaking without a hyphen'' (\protect\hyperlink{ref-zhang2014towards}{Zhang \& Soergel, 2014}).} is a
process that occurs when learners \emph{connect} their \emph{previously developed}
knowledge, ideas, abilities, and experiences together to address the
uncertainty presented by a newly introduced phenomenon, problem, or
piece of information (\protect\hyperlink{ref-ngss_sensemaking}{Next Generation Science Standards, 2021}). A significant portion of
learning is sensemaking, especially those which use recorded information
or systematic discovery to learn concepts, ideas, theories, and facts in
a domain (such as science or history) (\protect\hyperlink{ref-zhang2014towards}{Zhang \& Soergel, 2014}). The phrase
``figure something out'' is often synonymous with sensemaking. Sensemaking
is generally about actively trying to figure out the way the world
works, and/or exploring how to create or alter things to achieve desired
goals (\protect\hyperlink{ref-ngss_sensemaking}{Next Generation Science Standards, 2021}). (\protect\hyperlink{ref-dervin2010sensemaking}{Dervin \& Naumer, 2010}) distinguish work on
sensemaking in four fields: ``Human Computer Interaction (HCI) (Russell's
sensemaking); Cognitive Systems Engineering (Klein's sensemaking);
Organizational Communication (Weick's sensemaking; Kurtz and Snowden's
sense-making); and Library and Information Science (Dervin's
sense-making)''.

Many theories of learning and sensemaking revolve around the concept of
fitting new information into an existing or adapted knowledge structure
(\protect\hyperlink{ref-zhang2014towards}{Zhang \& Soergel, 2014}). The central idea is that knowledge is stored in
human memory as \emph{structures} or \emph{schemas}, which comprise interconnected
concepts and relationships. When new information is encountered or
acquired, the learner or sensemaker needs to actively construct a
revised or entirely new knowledge structure. Examples of some such
theories include: the \emph{assimilation theory (theory of meaningful
learning)}
(\protect\hyperlink{ref-ausubel1968educational}{Ausubel et al., 1968}; \protect\hyperlink{ref-ausubel2012acquisition}{Ausubel, 2012}; \protect\hyperlink{ref-novak2002meaningful}{Novak, 2002}; \protect\hyperlink{ref-novak2010learninga}{Novak, 2010});
the \emph{schema theory}
(\protect\hyperlink{ref-rumelhart1981accretion}{Rumelhart \& Norman, 1981}; \protect\hyperlink{ref-rumelhart1977representation}{Rumelhart \& Ortony, 1977}); and the
\emph{generative learning theory}
(\protect\hyperlink{ref-grabowski1996generative}{Grabowski, 1996}; \protect\hyperlink{ref-wittrock1989generative}{Wittrock, 1989}); all of which have
their foundations in the Piagetian concepts of \emph{assimilation} and
\emph{accommodation} (\protect\hyperlink{ref-piaget1936origins}{Piaget, 1936}).

\textbf{Assimilation} means addition of new information into an existing
knowledge structure. A ``synonym'' (\protect\hyperlink{ref-vakkari2016searching}{Vakkari, 2016}) for
assimilation is \textbf{accretion}, which is the gradual addition of factual
information to an existing knowledge structure, without structural
changes. Accretion does not change concepts and their relations in the
structure, but may populate a concept with new instances or facts.
\textbf{Accommodation} means modifying or changing existing knowledge
structures, by adding or removing concepts and their connections in the
knowledge structure. Accommodation is subdivided into \emph{tuning} /
\emph{weak-revision}, and \emph{restructuring}, based on the degree of structural
changes (\protect\hyperlink{ref-zhang2014towards}{Zhang \& Soergel, 2014}). \textbf{Tuning} or \textbf{weak revision} does not
include replacing concepts or connections between concepts in the
structure, but tuning of the scope and meaning of concepts and their
connections. This may include, for example, generalizing or specifying a
concept. \textbf{Restructuring} means radically changing and replacing
concepts and their connections in the existing knowledge structure, or
creating of new structures. Such radical changes often take place when
prior knowledge conflicts with new information. New structures are
constructed either to reinterpret old information or to account for new
information (\protect\hyperlink{ref-vakkari2016searching}{Vakkari, 2016}; \protect\hyperlink{ref-zhang2014towards}{Zhang \& Soergel, 2014}). A comparison of
these types of conceptual changes can be found in (\protect\hyperlink{ref-zhang2014towards}{Zhang \& Soergel, 2014} Table 3).

\hypertarget{sec-bg-concept-maps}{%
\subsection{Concept Maps to enhance Sensemaking}\label{sec-bg-concept-maps}}

As we saw in the previous section, deep learning / meaningful learning /
sensemaking is a process in which new information is connected to a
relevant area of a learner's existing knowledge structure. However, the
\emph{learner must choose} to do this, and must actively seek a way to
integrate the new information with existing relevant information in
their cognitive structure
(\protect\hyperlink{ref-ausubel1968educational}{Ausubel et al., 1968}; \protect\hyperlink{ref-novak2010learninga}{Novak, 2010}). Learning facilitators
(e.g., teachers) can encourage this choice by using the concept mapping
technique.

A \textbf{concept-map} is a two-dimensional, hierarchical node-link diagram
(a \emph{graph} in Computer Science parlance) that depicts the structure of
knowledge within a discipline, as viewed by a student, an instructor, or
an expert in a field or sub-field. The map is composed of concept
labels, each enclosed in a box (graph \emph{nodes}); a series of labelled
linking lines (\emph{labelled edges}); and an inclusive, general-to-specific
organization (\protect\hyperlink{ref-halttunen2005assessing}{Halttunen \& Jarvelin, 2005}). Concept-maps assess how well
students see the `'big picture'', and where there are knowledge-gaps and
misconceptions. A \emph{mind map} is a diagram similar to a concept map,
comprising nodes and links between nodes. However, mind maps emerge from
a single centre, and have a more hierarchical, tree like structure.
Concept maps are more free-form, allowing multiple hubs and clusters.
Also, mind-maps have unlabelled links, and are subjective to the
creator. There are no ``correct'' relationships between nodes in a mind
map. Figure
shows the key features of a concept
map, with the help of a concept map.

\textbf{Concept maps are therefore, arguably the most suited mechanism to
represent the cognitive knowledge structures, connections, and patterns
in a learner's mind}. Conventional tests, such as multiple choice
questions, are best at assessing students' recall of facts and guessing
skills. Their format treats information as distinct and separate items,
rather than interconnected pieces of a bigger picture. Concept maps on
the other hand, encourage learners to identify and make connections
between concepts that they know, and concepts that are new to them.
Concept maps have been used for over 50 years to provide a useful and
visually appealing way of illustrating and assessing learners'
conceptual knowledge
(\protect\hyperlink{ref-egusa2010usingb}{Egusa et al., 2010}, \protect\hyperlink{ref-egusa2014howd}{2014a}, \protect\hyperlink{ref-egusa2014howe}{2014b}, \protect\hyperlink{ref-egusa2017evaluating}{2017}; \protect\hyperlink{ref-halttunen2005assessing}{Halttunen \& Jarvelin, 2005}; \protect\hyperlink{ref-novak2010learninga}{Novak, 2010}; \protect\hyperlink{ref-novak1984learning}{Novak \& Gowin, 1984}).

Analysis of concept maps can reveal interesting patterns of learning and
thinking. Some of these measures that have been used by
(\protect\hyperlink{ref-halttunen2005assessing}{Halttunen \& Jarvelin, 2005}) are: addition, deletion, and differences in
top-level concept-nodes; depths of hierarchy; and number of concepts
that were ignored or changed fundamentally. In this regard,
(\protect\hyperlink{ref-novak1984learning}{Novak \& Gowin, 1984}) have presented well-established scoring schemes to
evaluate concept-maps: 1 point is awarded for each correct relationship
(i.e.~concept--concept linkage); 5 points for each valid level of
hierarchy; 10 points for each valid and significant cross-link; and 1
point for each example.

Having discussed how deep learning / meaningful learning / sensemaking
involves creation of knowledge structures in the learner's mind, and
suitably adding new pieces of information in the knowledge structure, we
now discuss how these processes are influenced in the 21st century with
the presence of new media, digital technologies, and information
retrieval systems.

\hypertarget{sec-bg-learn-active-knowledge-multiliteracy}{%
\section{`New' Learning as Online Information Searching}\label{sec-bg-learn-active-knowledge-multiliteracy}}

Digital media technologies and e-learning `ecologies' can enable new
forms and models of learning, that are fundamentally different from the
traditional classroom practices of didactic pedagogy
(\protect\hyperlink{ref-cope2017elearningc}{Cope \& Kalantzis, 2017}). Some key concepts associated with these forms of
`new learning' are described below. These concepts from the Educational
Sciences domain tie back strongly to the issues, challenges, and
research agenda being investigated by researchers in the Search as
Learning and Information Retrieval domain (Section \ref{sec-intro-overview}.

\hypertarget{sec-bg-learn-active-knowledge-making}{%
\subsection{Active Knowledge Making}\label{sec-bg-learn-active-knowledge-making}}

The Internet and new forms of media provide us the opportunity to create
learning environments where learners are no longer mainly \emph{consumers} of
knowledge, but also \emph{modifiers}, \emph{producers}, and \emph{exchangers} of
knowledge. In \textbf{active knowledge making}, learners can, and often need
to, find information on their own using online resources. They are not
restricted to the textbook alone. The Internet is often a definitive
resource for information on any given topic. A learner can search the
web (to learn) at any time, from anywhere, on any web-enabled device.

As knowledge producers, learners search and analyze multiple sources
with differing and contradictory perspectives, and develop their own
observations and conclusions. In this process, they become researchers
themselves and learn to collaborate with peers in knowledge production.
Collaboration gives learners the opportunity to work with others as
coauthors of knowledge, peer reviewers, and discussants to completed
works. Because learners bring their own views, outlooks, and
experiences, the knowledge artefact they create is often uniquely voiced
instead of a templated ``correct'' response (\protect\hyperlink{ref-amina2017active}{Amina, 2017}).

\begin{quote}
Learners become \textbf{active knowledge producers} (for instance, project-based learning, using multiple knowledge sources, and research based knowledge making), and not merely knowledge consumers (as exemplified in the `transmission' pedagogies of traditional textbook learning or e-learning focused on video or e-textbook delivery). Active knowledge making practices underpin contemporary emphases on innovation, creativity and problem solving, which are quintessential `knowledge economy' and `knowledge society' attributes.
\hfill --- Cope \& Kalantzis (\protect\hyperlink{ref-cope2017elearningc}{2017})
\end{quote}

\hypertarget{sec-bg-learn-artefact}{%
\subsection{Artefacts for Learning Assessment}\label{sec-bg-learn-artefact}}

Traditionally, the focus of learning outcomes has been long term memory.
Students and learners were expected to remember a collection of facts,
definitions, proofs, equations, and other associated details. For a
significant amount of modern knowledge-work today, \textbf{memory is actually
less important}. Information is so readily accessible now that it is no
longer necessary to remember the information. Because of the
technological phenomenon, the mass of information is available
ubiquitously \footnote{
  as long as there is internet connection} to a learner (or a knowledge worker), in every moment
of learning. Empirical details such as facts, definitions, proofs, or
equations do not need to be remembered today, because they can always be
looked up again (\protect\hyperlink{ref-amina2017active}{Amina, 2017}; \protect\hyperlink{ref-cope2017elearningc}{Cope \& Kalantzis, 2017}).

This creates an interesting shift in the focus of learning and knowledge
work today: \emph{``if we are not going to measure and value long-term memory in education, what are we going to assess?''}
Cope \& Kalantzis (\protect\hyperlink{ref-cope2017elearningc}{2017}) suggest that \textbf{we assess the knowledge artefacts} that learners
produce. In active knowledge making, the final work \footnote{
  be it a project report, poster, presentation, video, software,research paper, website, etc.} can be proof of
the learning outcome and represent a learner's ability to use the
resources that are available (\protect\hyperlink{ref-amina2017active}{Amina, 2017}). \textbf{Measure of learning can be measure of information quality and information use in artefacts.} This shows a shift in pedagogy and assessment and an
increase in personalization and individualization of learning
(\protect\hyperlink{ref-pea2014learning}{Pea \& Jacks, 2014}). Memorizing the information on a topic is less
important, compared to the writing, synthesizing, analyzing, and
\textbf{sensemaking} of the available information that has been referenced in
the work. This shifts the focus of assessment to the quality of the
artefacts and the processes of their construction. Moreover, as
technology increases the ability to capture detailed data from formal
and informal learning activities, it can give us a new view of how
learners progress in acquiring knowledge, skills, and attributes
(\protect\hyperlink{ref-dicerbo2014impacts}{DiCerbo \& Behrens, 2014}). Because learning is a continuous, longitudinal
process, these advanced, technologically enhanced assessments are more
useful in understanding the learning process and knowledge development
(\protect\hyperlink{ref-amina2017active}{Amina, 2017}).

Assessing open-ended artefacts does come with its challenges and
limitations. First, assessing and grading artefacts requires the
development of detailed qualitative coding guides
(\protect\hyperlink{ref-wilson2013comparison}{Wilson \& Wilson, 2013}). This process involves defining grading criteria
and measuring inter-coder agreement to ensure that the coding guide is
reliable. Prior studies have scored summaries along dimensions such as
the inclusion of facts, relationships between facts, and evaluative
statements (\protect\hyperlink{ref-lei2015effect}{Lei et al., 2015}; \protect\hyperlink{ref-roy2021note}{Roy et al., 2021}; \protect\hyperlink{ref-wilson2013comparison}{Wilson \& Wilson, 2013}).
Second, the quality of responses may be difficult to compare across
learners. Since this type of assessment imposes very few constraints on
the learners' responses, it may cause some learners to \emph{satisfice}, and
not convey everything that was learned. Additionally, writing skills are
likely to vary across learners, and some may not be able to effectively
articulate everything that was learnt.

\hypertarget{sec-bg-learn-info-eval}{%
\subsection{`Information Search and Evaluation' as and for Learning}\label{sec-bg-learn-info-eval}}

Learning today is more about \textbf{navigation, discernment, induction, and
synthesis}, and less about memory and deduction (\protect\hyperlink{ref-cope2013new}{Cope \& Kalantzis, 2013}).
However, knowing the source, finding the source, and using the
information critically is important to learn and know now more than ever
before (\protect\hyperlink{ref-amina2017active}{Amina, 2017}). Learners must know the social sources of
knowledge and understand and correctly use quotations, paraphrases,
remixes, links, citations, and the like in the works that they develop.
Searching and sourcing from the web entails a process of developing and
completing a work that inevitably makes learners \textbf{knowledge
producers}, as long as they can navigate and critically discern the
value of multiple sources. This is a skill that must be learned, as many
sources of information are not valid, reliable, or authentic
(\protect\hyperlink{ref-mcgrew2018can}{McGrew et al., 2018}; \protect\hyperlink{ref-wineburg2016students}{Wineburg \& McGrew, 2016}). Understanding the different
sources and identifying the more reliable ones are essential for
effective teaching and learning
(\protect\hyperlink{ref-mcgrew2017challenge}{McGrew et al., 2017}; \protect\hyperlink{ref-mcgrew2021skipping}{McGrew, 2021}). This is a critical aspect
because the inability to cite properly or to use reliable resources
provides learners with misconstrued information and ideas
(\protect\hyperlink{ref-amina2017active}{Amina, 2017}; \protect\hyperlink{ref-breakstone2021students}{Breakstone et al., 2021}; \protect\hyperlink{ref-mcgrew2017challenge}{McGrew et al., 2017}).

The Stanford History Education Group (SHEG) conceptualised the \textbf{Civic
Online Reasoning} (COR) curriculum \footnote{\url{https://cor.stanford.edu}} to enable students to
effectively search for and evaluate online information
(\protect\hyperlink{ref-breakstone2021students}{Breakstone et al., 2021}; \protect\hyperlink{ref-breakstone2018we}{Breakstone et al., 2018}; \protect\hyperlink{ref-mcgrew2020learning}{McGrew, 2020}). The
curriculum centres on asking three questions of any digital content:
\emph{(i)} who is behind a piece of information? \emph{(ii)} what is the evidence
for a claim? \emph{(iii)} what do other sources say? The curriculum has
lessons and assessments for information evaluation skills such as
lateral reading (\protect\hyperlink{ref-wineburg2017lateral}{Wineburg \& McGrew, 2017}), identifying news versus
opinions, checking domain names, identifying sponsored content,
evaluating evidence, and practising click restraint (\protect\hyperlink{ref-mcgrew2021click}{McGrew \& Glass, 2021}).
The lessons were developed and piloted by the Stanford History Education
Group (\protect\hyperlink{ref-mcgrew2018can}{McGrew et al., 2018}; \protect\hyperlink{ref-mcgrew2020learning}{McGrew, 2020}; \protect\hyperlink{ref-mcgrew2021click}{McGrew \& Glass, 2021}). Taken
together, these strategies will allow academics and students to better
evaluate digital content, from the perspectives of professional fact
checkers.

The purview of the \emph{Civic Online Reasoning} curriculum is more targeted
than the expansive fields of media and digital literacy{[}\^{}7{]}, (which can
embrace topics ranging from cyberbullying to identity theft). Civic
Online Reasoning focuses squarely on how to sort fact from fiction
online, a prerequisite for responsible civic engagement in the
twenty-first century
(\protect\hyperlink{ref-breakstone2021students}{Breakstone et al., 2021}; \protect\hyperlink{ref-kahne2012digital}{Kahne et al., 2012}; \protect\hyperlink{ref-mihailidis2013media}{Mihailidis \& Thevenin, 2013}).

\hypertarget{sec-bg-learn-promoting-learning}{%
\section{Promoting Better Learning}\label{sec-bg-learn-promoting-learning}}

\begin{quote}
\emph{It is not the technology that makes a difference; it is the pedagogy.}

\hfill --- Cope \& Kalantzis (\protect\hyperlink{ref-cope2017elearningc}{2017})
\end{quote}

Having discussed how meaningful learning takes place, and how it is
influenced by the presence of digital media and the mass of information
on the Internet, let us now look deeper into the learners as persons
themselves. In this section, we discuss how different cognitive and
metacognitive practices and aspects of learners can promote better
learning. These phenomena have important implications for any digital
systems that aim to foster learning.

\hypertarget{sec-bg-learn-articulation}{%
\subsection{Externalization and Articulation}\label{sec-bg-learn-articulation}}

The learning sciences have discovered that when learners externalize and
articulate their developing knowledge, they learn more effectively
(\protect\hyperlink{ref-council2000how}{National Research Council, 2000}). Best learning takes place when learners articulate
their unformed and still developing understanding, and continue to
articulate it throughout the process of learning. This phenomenon was
first studied in the 1920s by Russian psychologist Lev Vygotsky.
Articulating and learning go hand in hand, in a mutually reinforcing
feedback loop. Often learners do not actually learn something until they
start to articulate it. While thinking out loud, they learn more rapidly
and deeply than while studying quietly (\protect\hyperlink{ref-sawyer2005cambridge}{Sawyer, 2005}). The
learning sciences community is actively researching how to support
students in their ongoing process of articulation, and which forms of
articulation are the most beneficial to learning. Articulation is more
effective if it is scaffolded -- channelled so that certain kinds of
knowledge are articulated, and in a certain form that is most likely to
result in useful reflection (\protect\hyperlink{ref-sawyer2005cambridge}{Sawyer, 2005}). Students need help
in articulating their developing understandings, as they do not yet know
how to think about thinking, or talk about thinking; their knowledge
state is \emph{anomalous} (\protect\hyperlink{ref-belkin1982ask}{Belkin et al., 1982}).

\hypertarget{sec-bg-learn-metacognition}{%
\subsection{Metacognition and Reflection}\label{sec-bg-learn-metacognition}}

One of the reasons that articulation is so helpful to learning is that
it promotes \emph{reflection} or \emph{metacognition}. \textbf{Metacognition}, commonly
referred to as thinking about thinking, involves thinking at a higher
level of abstraction, which in turn improves thinking and learning
(\protect\hyperlink{ref-blanken2017metacognition}{Blanken-Webb, 2017}). It is ``the process of reflecting on and
directing one's own thinking'' (\protect\hyperlink{ref-council2000how}{National Research Council, 2000, p. 78}), and involves
thinking about the process of learning, and thinking about knowledge.
This ties forward to the self-regulation that effective learners exhibit
(Section \ref{sec-bg-learn-self-regulation}). Effective learners are aware
of their learning process, and can measure how efficiently they are
learning as they study.

The literature on metacognition broadly identifies two fundamental
components of metacognition: knowledge about cognition, and regulation
of cognition. \textbf{Knowledge about cognition} includes three subprocesses
that facilitate the \emph{reflective} aspect of metacognition: declarative
knowledge (knowledge about self and about strategies), procedural
knowledge (knowledge about how to use strategies), and conditional
knowledge (knowledge about when and why to use strategies). \textbf{Regulation
of cognition} include a number of subprocesses that facilitate the
\emph{control} aspect of learning. Five component skills of regulation have
been discussed extensively in the literature, including planning,
information management strategies, comprehension monitoring, debugging
strategies, and evaluation. The operational definitions of these
components are described in Table
.
Schraw \& Dennison (\protect\hyperlink{ref-schraw1994assessing}{1994})
developed the \textbf{Metacognitive Awareness Inventory} (MAI) survey and a
scoring guide to measure these self-reported components and subprocesses
of metacognition. The original survey consists of 52 true/false
questions (Appendix \ref{app-mai}), such as ``I consider several alternatives to a
problem before I answer'', ``I understand my intellectual strengths and
weaknesses'', ``I have control over how well I learn'', and ``I change
strategies when I fail to understand''. The instrument has been widely
used in research, and has its reliability and validity measures
available. Later, Terlecki \& McMahon (\protect\hyperlink{ref-terlecki2018call}{2018}) proposed a revised version of the
MAI, using five-point Likert-scales, ranging from ``I never do this'' to
``I do this always''. They argue that when measuring change in
metacognition over time, the Likert-scale based `how often' questions
are more effective than dichotomous `Yes/No' questions
(\protect\hyperlink{ref-terlecki2020revising}{Terlecki, 2020}; \protect\hyperlink{ref-terlecki2018call}{Terlecki \& McMahon, 2018}).

\hypertarget{motivation}{%
\subsection{Motivation}\label{motivation}}

\hypertarget{ch_bg_search}{%
\chapter{Background: Information Searching}\label{ch_bg_search}}

\hypertarget{research-questions-and-hypotheses}{%
\chapter{Research Questions and Hypotheses}\label{research-questions-and-hypotheses}}

\hypertarget{methods-longitudinal-study}{%
\chapter{Methods: Longitudinal Study}\label{methods-longitudinal-study}}

\hypertarget{sec-method-exp-design}{%
\section{Study Design}\label{sec-method-exp-design}}

\hypertarget{apparatus}{%
\section{Apparatus}\label{apparatus}}

\hypertarget{yasbil-browsing-logger}{%
\subsection{YASBIL Browsing Logger}\label{yasbil-browsing-logger}}

\hypertarget{qualtrics-survey-software}{%
\subsection{Qualtrics Survey Software}\label{qualtrics-survey-software}}

\hypertarget{zoom-video-conferencing-software}{%
\subsection{Zoom Video-conferencing Software}\label{zoom-video-conferencing-software}}

\hypertarget{sec_method_search_task_template}{%
\section{Search Task Template}\label{sec_method_search_task_template}}

\hypertarget{sec_method_procedure}{%
\section{Procedure}\label{sec_method_procedure}}

Insert diagram and check how it looks



\begin{figure}

{\centering \includegraphics[width=1\linewidth]{figs/fig-study-proc-diss} 

}

\caption[Short Caption for LoF]{Very very very very very very very very long caption.}\label{fig:fig-study-proc-diss}
\end{figure}

Reference it

\hypertarget{sec-method-sur1}{%
\subsection{SUR1: Entry Survey}\label{sec-method-sur1}}

\hypertarget{ses1-initial-session}{%
\subsection{SES1: Initial Session}\label{ses1-initial-session}}

\hypertarget{ses2a---ses2d-longitudinal-tracking-sessions}{%
\subsection{SES2a - SES2d: Longitudinal Tracking Sessions}\label{ses2a---ses2d-longitudinal-tracking-sessions}}

\hypertarget{sec-method-sur2}{%
\subsection{SUR2: Mid-Term Survey}\label{sec-method-sur2}}

\hypertarget{ses3-final-session}{%
\subsection{SES3: Final Session}\label{ses3-final-session}}

\hypertarget{sur3-exit-survey}{%
\subsection{SUR3: Exit Survey}\label{sur3-exit-survey}}

\hypertarget{data-analysis}{%
\chapter{Data Analysis}\label{data-analysis}}

Note about pronouns:
all participants are referred to using gender-neutral they/them pronouns.

Final feedback:
P022Pisa said
\textgreater{} \emph{It is great to be able to participate in the research this semester. Using the extension somehow brings me postive feedback and that helps me in study I303. So I wanna say thank you}\\
\textgreater{} - P022Pisa

\hypertarget{data-cleaning-and-processing}{%
\section{Data Cleaning and Processing}\label{data-cleaning-and-processing}}

\hypertarget{data-analysis-approach}{%
\section{Data Analysis Approach}\label{data-analysis-approach}}

see crescenzi thesis

\hypertarget{url-categorization}{%
\section{URL Categorization}\label{url-categorization}}

\hypertarget{latent-profile-analysis}{%
\section{Latent Profile Analysis}\label{latent-profile-analysis}}

\begin{itemize}
\tightlist
\item
  add WMC / memory span to features
\item
  Use LIME / SHAP and counterfactual explanations to understand which components contribute to change in Profile Membership
\end{itemize}

\hypertarget{dwell-time-analysis}{%
\section{Dwell Time Analysis}\label{dwell-time-analysis}}

\hypertarget{results}{%
\chapter{Results}\label{results}}

\begin{itemize}
\tightlist
\item
  Also see Yung Sheng's Dissertation
\item
  think hard about which data component has not been touched / analysed
\end{itemize}

\hypertarget{rq1---search-behaviours}{%
\section{RQ1: - search behaviours?}\label{rq1---search-behaviours}}

\hypertarget{q---query-reformulation}{%
\subsection{Q - query reformulation}\label{q---query-reformulation}}

\begin{itemize}
\tightlist
\item
  see Yung Sheng's Dissertation
\end{itemize}

\hypertarget{l---source-selection}{%
\subsection{L - source selection}\label{l---source-selection}}

\hypertarget{i---interacting-with-sources}{%
\subsection{I - interacting with sources}\label{i---interacting-with-sources}}

\hypertarget{sheg-tasks---information-evaluation}{%
\subsection{SHEG tasks - information evaluation}\label{sheg-tasks---information-evaluation}}

\begin{quote}
We've confused young people's ability to operate digital devices with the sophistication they need to discern whether the information those devices yield is something that can be relied upon

\url{https://twitter.com/suzettelohmeyer/status/1617909351766757376}
\url{https://www.grid.news/story/misinformation/2023/01/23/will-information-literacy-in-schools-fix-our-misinformation-problem/}
\end{quote}

\hypertarget{rq2-mention-here}{%
\section{RQ2: mention here}\label{rq2-mention-here}}

\hypertarget{rq3-mention-here}{%
\section{RQ3: mention here}\label{rq3-mention-here}}

\hypertarget{rq4-mention-here}{%
\section{RQ4: mention here}\label{rq4-mention-here}}

\hypertarget{conclusions-contributions-and-future-work}{%
\chapter{Conclusions, Contributions, and Future Work}\label{conclusions-contributions-and-future-work}}

see Jacek's thesis

\hypertarget{research-summary}{%
\section{Research Summary}\label{research-summary}}

\hypertarget{summary-of-results}{%
\section{Summary of Results}\label{summary-of-results}}

\hypertarget{methodology}{%
\section{Methodology}\label{methodology}}

\hypertarget{contributions}{%
\section{Contributions}\label{contributions}}

\hypertarget{limitations}{%
\section{Limitations}\label{limitations}}

\begin{itemize}
\tightlist
\item
  No PDF
\item
  N=16 to N=10
\item
  Also check anticipated limitations section from proposal
\end{itemize}

\hypertarget{future-work}{%
\section{Future Work}\label{future-work}}

\startappendices

\hypertarget{ch_pilot_study}{%
\chapter{Prior Work: Pilot Study}\label{ch_pilot_study}}

\hypertarget{ses1-initial-session-1}{%
\section{SES1: Initial Session}\label{ses1-initial-session-1}}

\hypertarget{app-signup-survey}{%
\chapter{SUR1: Entry Survey}\label{app-signup-survey}}

\hypertarget{app-demographics}{%
\section{Demographics}\label{app-demographics}}

\begin{enumerate}
\def\labelenumi{\arabic{enumi}.}
\tightlist
\item
  Please select the degree level/name of the program you are in.
\item
  Please state which year of the program you are in.
\item
  Please state your major(s)
\item
  Do you have native-level familiarity with English language? Yes / No
  / Other:
\item
  Please state your age (in years)
\item
  Please state your gender
\item
  With which ethnicities do you identify? Please check all that apply:

  \begin{itemize}
  \tightlist
  \item
    African
  \item
    African American / Black
  \item
    Asian - East
  \item
    Asian - South East
  \item
    Asian - South
  \item
    Asian - Middle East
  \item
    Caucasian / White
  \item
    Hispanic / Latinx
  \item
    Native American
  \item
    Pacific Islander
  \item
    Mixed
  \item
    Other: \_\_\_\_
  \end{itemize}
\item
  Are you an international student? Yes / No; If Yes, where are you originally from?
\item
  Please enter an email address that you check regularly. We will send communications and compensation information to this email address.
\item
  Your name as you would like us to address you.
\end{enumerate}

\hypertarget{app-search-it-proficiency}{%
\section{Search and IT Proficiency}\label{app-search-it-proficiency}}

\begin{enumerate}
\def\labelenumi{\arabic{enumi}.}
\tightlist
\item
  Which device(s) and browser(s) do you normally use to surf the internet?
\end{enumerate}

\begin{longtable}[]{@{}
  >{\raggedright\arraybackslash}p{(\columnwidth - 14\tabcolsep) * \real{0.1250}}
  >{\raggedright\arraybackslash}p{(\columnwidth - 14\tabcolsep) * \real{0.1250}}
  >{\raggedright\arraybackslash}p{(\columnwidth - 14\tabcolsep) * \real{0.1250}}
  >{\raggedright\arraybackslash}p{(\columnwidth - 14\tabcolsep) * \real{0.1250}}
  >{\raggedright\arraybackslash}p{(\columnwidth - 14\tabcolsep) * \real{0.1250}}
  >{\raggedright\arraybackslash}p{(\columnwidth - 14\tabcolsep) * \real{0.1250}}
  >{\raggedright\arraybackslash}p{(\columnwidth - 14\tabcolsep) * \real{0.1250}}
  >{\raggedright\arraybackslash}p{(\columnwidth - 14\tabcolsep) * \real{0.1250}}@{}}
\toprule()
\begin{minipage}[b]{\linewidth}\raggedright
Choice
\end{minipage} & \begin{minipage}[b]{\linewidth}\raggedright
Chrome (1)
\end{minipage} & \begin{minipage}[b]{\linewidth}\raggedright
Safari (2)
\end{minipage} & \begin{minipage}[b]{\linewidth}\raggedright
Firefox (3)
\end{minipage} & \begin{minipage}[b]{\linewidth}\raggedright
Edge (4)
\end{minipage} & \begin{minipage}[b]{\linewidth}\raggedright
Opera (5)
\end{minipage} & \begin{minipage}[b]{\linewidth}\raggedright
Other (6)
\end{minipage} & \begin{minipage}[b]{\linewidth}\raggedright
None (7)
\end{minipage} \\
\midrule()
\endhead
Desktop & & & & & & & \\
Laptop & & & & & & & \\
Tablet & & & & & & & \\
Smartphone & & & & & & & \\
& & & & & & & \\
\bottomrule()
\end{longtable}

\begin{enumerate}
\def\labelenumi{\arabic{enumi}.}
\setcounter{enumi}{1}
\item
  How comfortable are you with using Mozilla Firefox to search information on the internet?

  \begin{enumerate}
  \def\labelenumii{\arabic{enumii}.}
  \tightlist
  \item
    I do not know how to use Mozilla Firefox.
  \item
    I have never used Mozilla Firefox.
  \item
    I feel very uncomfortable to use Mozilla Firefox.
  \item
    I feel uncomfortable to use Mozilla Firefox.
  \item
    I feel neither comfortable nor uncomfortable to use Mozilla Firefox.
  \item
    I feel comfortable to use Mozilla Firefox.
  \item
    I feel very comfortable to use Mozilla Firefox.
  \item
    Other: \_\_\_\_
  \end{enumerate}
\item
  Which search engines do you normally use?

  \begin{enumerate}
  \def\labelenumii{\arabic{enumii}.}
  \tightlist
  \item
    Google
  \item
    Bing
  \item
    Baidu
  \item
    Yahoo!
  \item
    Yandex
  \item
    DuckDuckGo
  \item
    Other:
  \end{enumerate}
\end{enumerate}

The following items are adapted from the \textbf{Digital Health Literacy Instrument (DHLI)} by Van Der Vaart \& Drossaert (\protect\hyperlink{ref-van2017development}{2017}).

\emph{On a scale of 1 to 5 \ldots{}\\
(1) Very difficult / Very seldom -- Difficult / Seldom -- Neutral -- Easy / Often -- Very easy / Very often (5)}

\emph{How easy or difficult is it for you to\ldots{}}

\begin{enumerate}
\def\labelenumi{\arabic{enumi}.}
\setcounter{enumi}{3}
\tightlist
\item
  Use the keyboard of a computer (e.g., to type words)?
\item
  Use the mouse (e.g., to put the cursor in the right field or to click)?
\item
  Use the buttons or links and hyperlinks on websites?
\end{enumerate}

\emph{When you search the Internet for information, how easy or difficult is it for you to \ldots{}}

\begin{enumerate}
\def\labelenumi{\arabic{enumi}.}
\setcounter{enumi}{6}
\tightlist
\item
  Make a choice from all the information you find?
\item
  Use the proper words or search query to find the information you are looking for
\item
  Find the exact information you are looking for?
\item
  Decide whether the information is reliable or not?
\item
  Decide whether the information is written with commercial interests (e.g., by people trying to sell a product)?
\item
  Check different websites to see whether they provide the same information?
\item
  Decide if the information you found is applicable to your situation?
\item
  Apply the information you found in your daily life?
\item
  Use the information you found to make decisions about your life
\end{enumerate}

\emph{When you search the Internet for information, how often does it happen that\ldots{}}

\begin{enumerate}
\def\labelenumi{\arabic{enumi}.}
\setcounter{enumi}{15}
\tightlist
\item
  You lose track of where you are on a website or the Internet?
\item
  You do not know how to return to a previous page?
\item
  You click on something and get to see something different than you expected?
\end{enumerate}

The following items are adapted from the \textbf{Search Self-Efficacy Scale (SSE)} by Brennan et al. (\protect\hyperlink{ref-brennan2016factor}{2016}).

\emph{On a scale of 1 to 5, how confident are you that you can \ldots{}\\
(1) Not at all confident -- Neither confident nor unconfident -- Totally confident (5)}

\begin{enumerate}
\def\labelenumi{\arabic{enumi}.}
\setcounter{enumi}{18}
\tightlist
\item
  Identify the major requirements of the search from the initial statement of the topic.
\item
  Correctly develop search queries to reflect my requirements.
\item
  Use special syntax in advanced searching (e.g., AND, OR, NOT).
\item
  Evaluate the resulting list to monitor the success of my approach.
\item
  Develop a search query which will retrieve a large number of appropriate articles.
\item
  Find an adequate number of articles.
\item
  Find articles similar in quality to those obtained by a professional searcher.
\item
  Devise a query which will result in a very small percentage of irrelevant items on my list.
\item
  Efficiently structure my time to complete the task.
\item
  Develop a focused search query that will retrieve a small number of appropriate articles.
\item
  Distinguish between relevant and irrelevant articles.
\item
  Complete the search competently and effectively.
\item
  Complete the individual steps of the search with little difficulty.
\item
  Structure my time effectively so that I will finish the search in the allocated time.
\end{enumerate}

\hypertarget{app-course-load}{%
\section{Course Load and Other Engagements}\label{app-course-load}}

\begin{enumerate}
\def\labelenumi{\arabic{enumi}.}
\tightlist
\item
  How many total weekly hours of coursework are you registered for this semester?
\item
  How many weekly hours do you anticipate putting in for studying this course?
\item
  What are your other time commitments, as hours per week? (enter 0 if not applicable)

  \begin{itemize}
  \tightlist
  \item
    jobs
  \item
    extra-curriculars
  \item
    other
  \end{itemize}
\item
  Do you hold a position of responsibility (officer / committee member) in any (student) organisation? Yes / No
\end{enumerate}

\hypertarget{app-note-taking-strategies}{%
\section{Note-taking Strategies}\label{app-note-taking-strategies}}

Adapted from \emph{Listening and Note Taking Survey} by
(\protect\hyperlink{ref-note-taking-survey-penn-state}{\textbf{note-taking-survey-penn-state?}}), and \emph{Note Taking Strategies Inventory}
by (\protect\hyperlink{ref-note-taking-strategies-umass}{\textbf{note-taking-strategies-umass?}}).

\emph{For each question, choose the response that best describes your actions
(not the one that describes what you think you should be doing). There
are no right or wrong answers. In general (not specifically for this
course)}

\begin{enumerate}
\def\labelenumi{\arabic{enumi}.}
\tightlist
\item
  I take notes using (check all that apply)

  \begin{itemize}
  \tightlist
  \item
    Paper and Pen / Pencil
  \item
    Laptop / Desktop
  \item
    Tablet with Keyboard
  \item
    Tablet with Stylus / Digital Pen
  \end{itemize}
\item
  When taking notes on the laptop, I minimize distractions by:
\end{enumerate}

\begin{center}\rule{0.5\linewidth}{0.5pt}\end{center}

\emph{On a scale of 1 to 5 \ldots{}\\
(1) Never -- Rarely -- Sometimes -- Often -- Always (5)}

\begin{enumerate}
\def\labelenumi{\arabic{enumi}.}
\setcounter{enumi}{2}
\tightlist
\item
  I read my assignments before I go to lecture.
\item
  I find lectures interesting and/or challenging.
\item
  My lecture notes are well organized.
\item
  I recognize main ideas in lectures.
\item
  I recognize supporting details of main ideas.
\item
  I recognize patterns in lectures, e.g., cause-effect, concept-example.
\item
  My lecture notes are complete.
\item
  I recognize relationships between lecture and readings.
\item
  I integrate my lecture notes with my reading notes.
\item
  I summarize my notes, both lecture and reading, in my own words.
\item
  I review my notes immediately after class.
\item
  I conduct weekly reviews of my notes.
\item
  I edit my notes within 24 hours after class.
\item
  I take notes
\item
  I put dates on my notes
\item
  I makes notes in the margins of the text when I read (on paper / digital medium, e.g.~iPad and Apple Pencil)
\item
  I pause periodically while reviewing notes to summarize or paraphrase the information.
\item
  I use diagrams in my notes
\item
  I use different colours when writing notes
\item
  I create outlines, concept maps or organizational charts of how ideas fit together.
\item
  I write down questions I want to ask the instructor
\item
  I reorganize and fill in notes I took in class
\item
  I put things in my own words
\item
  I rewrite my notes
\item
  I use abbreviations in my notes
\item
  I write out my own descriptions of the main concepts
\item
  I keep track of things I do not understand and note when they finally become clear and what made that happen
\item
  I understand my notes
\item
  I refer back to my notes
\end{enumerate}

\begin{center}\rule{0.5\linewidth}{0.5pt}\end{center}

\begin{enumerate}
\def\labelenumi{\arabic{enumi}.}
\setcounter{enumi}{31}
\tightlist
\item
  How do you organise your notes? \_\_\_\_\_
\item
  Have you ever wished that you had written better notes? Why?

  \begin{itemize}
  \tightlist
  \item
    Yes: \_\_\_\_
  \item
    No: \_\_\_\_
  \end{itemize}
\item
  How long do you store your notes for?

  \begin{enumerate}
  \def\labelenumii{\arabic{enumii}.}
  \tightlist
  \item
    Till the end of the semester
  \item
    End of academic year
  \item
    End of college
  \item
    Lifelong
  \item
    Other: \_\_\_\_
  \end{enumerate}
\item
  How do you search for a bit of information in your notes?
\end{enumerate}

\hypertarget{app-imi}{%
\section{Motivation}\label{app-imi}}

Adapted from Intrinsic Motivation Inventory (IMI) (\protect\hyperlink{ref-ryan1982control}{Ryan, 1982}).
Items will be randomly ordered.

\textbf{Scoring directions:}
Score each response from 1 (not at all true) to 5 (very true).
Then reverse score the items marked with \textbf{(R)}.
To do that, subtract the item response from 6, and use the resulting number as the item score.
Then, calculate subscale scores by averaging across all the items on that subscale.
The subscale scores are then used in the analyses of relevant research questions.

\emph{For each of the following statements, please indicate how true it is
for you, using the following scale:}\\
\emph{(1) not at all true --- somewhat true --- very true (5)}

\hypertarget{interestenjoyment}{%
\subsection{Interest/Enjoyment}\label{interestenjoyment}}

\begin{enumerate}
\def\labelenumi{\arabic{enumi}.}
\tightlist
\item
  I will enjoy taking this course very much.
\item
  This course will be fun to do.
\item
  I think this will be a boring course. \textbf{(R)}
\item
  This course will not hold my attention at all. \textbf{(R)}
\item
  I would describe this course as very interesting.
\item
  I think this course will be quite enjoyable.
\end{enumerate}

\hypertarget{perceived-competence}{%
\subsection{Perceived Competence}\label{perceived-competence}}

\begin{enumerate}
\def\labelenumi{\arabic{enumi}.}
\tightlist
\item
  I think I will be pretty good at this course.
\item
  I think I will be doing pretty well at this course, compared to other students.
\item
  After working at this course for awhile, I will feel pretty competent.
\item
  I think I will be satisfied with my performance in this course.
\item
  I think I am pretty skilled at this course.
\item
  This is a course that I think would not be able to do very well. \textbf{(R)}
\end{enumerate}

\hypertarget{effortimportance}{%
\subsection{Effort/Importance}\label{effortimportance}}

\begin{enumerate}
\def\labelenumi{\arabic{enumi}.}
\tightlist
\item
  I plan to put a lot of effort into this course.
\item
  I don't think I will try very hard to do well at this course. \textbf{(R)}
\item
  I will try very hard on this course.
\item
  It is important to me to do well in this course.
\item
  I do not plan to put much energy into this course. \textbf{(R)}
\end{enumerate}

\hypertarget{valueusefulness}{%
\subsection{Value/Usefulness}\label{valueusefulness}}

\begin{enumerate}
\def\labelenumi{\arabic{enumi}.}
\tightlist
\item
  I believe the course and the final project activities could be of some value to me.
\item
  I think that doing the final project activities is useful for me.
\item
  I think the final project is important activity to do because it can equip me with skills that are necessary for making ethical decisions in my adult and professional life.
\item
  I would be willing to do research on the final project topic again because it has some value to me.
\item
  I think doing the final project activities will help me in my adult and professional life
\item
  I believe doing the final project activities will be beneficial to me.
\item
  I think this is an important course.
\end{enumerate}

\hypertarget{app-srq}{%
\section{Self-regulation}\label{app-srq}}

Self-Regulation Questionnaire (SRQ) by (\protect\hyperlink{ref-brown1999self}{Brown et al., 1999}).

\emph{Please answer the following questions by selecting the option that best
describes how you are. There are no right or wrong answers. Work quickly
and don't think too long about your answers.\\
\strut \\
(1) Strongly Disagree -- Disagree -- Neutral -- Agree -- Strongly Agree (5)}

\begin{enumerate}
\def\labelenumi{\arabic{enumi}.}
\tightlist
\item
  I usually keep track of my progress toward my goals.
\item
  My behavior is not that different from other people's. \textbf{(R)}
\item
  Others tell me that I keep on with things too long. \textbf{(R)}
\item
  I doubt I could change even if I wanted to. \textbf{(R)}
\item
  I have trouble making up my mind about things. \textbf{(R)}
\item
  I get easily distracted from my plans. \textbf{(R)}
\item
  I reward myself for progress toward my goals.
\item
  I don't notice the effects of my actions until it's too late. \textbf{(R)}
\item
  My behavior is similar to that of my friends. Evaluating
\item
  It's hard for me to see anything helpful about changing my ways. \textbf{(R)}
\item
  I am able to accomplish goals I set for myself.
\item
  I put off making decisions. \textbf{(R)}
\item
  I have so many plans that it's hard for me to focus on any one of them. \textbf{(R)}
\item
  I change the way I do things when I see a problem with how things are going.
\item
  It's hard for me to notice when I've ``had enough'' (alcohol, food, sweets, internet, social media) \textbf{(R)}
\item
  I think a lot about what other people think of me.
\item
  I am willing to consider other ways of doing things.
\item
  If I wanted to change, I am confident that I could do it.
\item
  When it comes to deciding about a change, I feel overwhelmed by the choices. \textbf{(R)}
\item
  I have trouble following through with things once I've made up my mind to do something. \textbf{(R)}
\item
  I don't seem to learn from my mistakes. \textbf{(R)}
\item
  I'm usually careful not to overdo it when working, eating, drinking, or being on social media.
\item
  I tend to compare myself with other people.
\item
  I enjoy a routine, and like things to stay the same. \textbf{(R)}
\item
  I have sought out advice or information about changing.
\item
  I can come up with lots of ways to change, but it's hard for me to decide which one to use. \textbf{(R)}
\item
  I can stick to a plan that's working well.
\item
  I usually only have to make a mistake one time in order to learn from it.
\item
  I don't learn well from punishment. \textbf{(R)}
\item
  I have personal standards, and try to live up to them.
\item
  I am set in my ways. \textbf{(R)}
\item
  As soon as I see a problem or challenge, I start looking for possible solutions.
\item
  I have a hard time setting goals for myself. \textbf{(R)}
\item
  I have a lot of willpower.
\item
  When I'm trying to change something, I pay a lot of attention to how I'm doing.
\item
  I usually judge what I'm doing by the consequences of my actions.
\item
  I don't care if I'm different from most people. \textbf{(R)}
\item
  As soon as I see things aren't going right I want to do something about it.
\item
  There is usually more than one way to accomplish something.
\item
  I have trouble making plans to help me reach my goals. \textbf{(R)}
\item
  I am able to resist temptation.
\item
  I set goals for myself and keep track of my progress.
\item
  Most of the time I don't pay attention to what I'm doing. \textbf{(R)}
\item
  I try to be like people around me.
\item
  I tend to keep doing the same thing, even when it doesn't work. \textbf{(R)}
\item
  I can usually find several different possibilities when I want to change something.
\item
  Once I have a goal, I can usually plan how to reach it.
\item
  I have rules that I stick by no matter what.
\item
  If I make a resolution to change something, I pay a lot of attention to how I'm doing.
\item
  Often I don't notice what I'm doing until someone calls it to my attention. \textbf{(R)}
\item
  I think a lot about how I'm doing.
\item
  Usually I see the need to change before others do.
\item
  I'm good at finding different ways to get what I want.
\item
  I usually think before I act.
\item
  Little problems or distractions throw me off course. \textbf{(R)}
\item
  I feel bad when I don't meet my goals.
\item
  I learn from my mistakes.
\item
  I know how I want to be.
\item
  It bothers me when things aren't the way I want them.
\item
  I call in others for help when I need it.
\item
  Before making a decision, I consider what is likely to happen if I do one thing or another.
\item
  I give up quickly. \textbf{(R)}
\item
  I usually decide to change and hope for the best. \textbf{(R)}
\end{enumerate}

\textbf{Scoring Directions:}
Score each response from 1 (strongly disagree) to 5 (strongly agree), and calculate the following seven subscale scores by
summing the items on that subscale.
Items marked \textbf{(R)} are reverse-coded (i.e.~1 = strongly agree and 5 = strongly disagree).
To do that, subtract the item response from 6, and use the resulting number as the item score.

\begin{enumerate}
\def\labelenumi{\arabic{enumi}.}
\tightlist
\item
  \emph{Receiving relevant information:} 1, 8, 15, 22, 29, 36, 43, 50, 57
\item
  \emph{Evaluating the information and comparing it to norms:} 2, 9, 16, 23, 30, 37, 44, 51, 58
\item
  \emph{Triggering change:} 3, 10, 17, 24, 31, 38, 45, 52, 59
\item
  \emph{Searching for options:} 4, 11, 18, 25, 32, 39, 46, 53, 60
\item
  \emph{Formulating a plan:} 5, 12, 19, 26, 33, 40, 47, 54, 61
\item
  \emph{Implementing the plan:} 6, 13, 20, 27, 34, 41, 48, 55, 62
\item
  \emph{Assessing the plan's effectiveness:} 7, 14, 21, 28, 35, 42, 49, 56, 63
\end{enumerate}

Based on our clinical and college samples, we tentatively recommend the following ranges for interpreting SRQ total scores with the 63-item scale:

\begin{itemize}
\tightlist
\item
  \textbf{\textgreater= 239}: High (intact) self-regulation capacity (top quartile)
\item
  \textbf{214 - 238}: Intermediate (moderate) self-regulation capacity (middle quartiles)
\item
  \textbf{\textless= 213}: Low (impaired) self-regulation capacity (bottom quartile)
\end{itemize}

\hypertarget{app-mai}{%
\section{Metacognition}\label{app-mai}}

Metacognitive Awareness Inventory (MAI) proposed by
Schraw \& Dennison (\protect\hyperlink{ref-schraw1994assessing}{1994}) and revised by Terlecki \& McMahon (\protect\hyperlink{ref-terlecki2018call}{2018}).

\emph{Think of yourself as a \textbf{learner}. Read each statement carefully, and
rate it as it generally applies to you when you are in the role of a
learner (student, attending classes, university etc.) Please indicate
how true each reason is for you using the following scale:}

\begin{longtable}[]{@{}ll@{}}
\toprule()
Score & Response \\
\midrule()
\endhead
1 & I \textbf{NEVER} do this \\
2 & I do this \textbf{infrequently} \\
3 & I do this \textbf{inconsistently} \\
4 & I do this \textbf{frequently} \\
5 & I \textbf{ALWAYS} do this \\
\bottomrule()
\end{longtable}

\begin{enumerate}
\def\labelenumi{\arabic{enumi}.}
\tightlist
\item
  I ask myself periodically if I am meeting my goals.
\item
  I consider several alternatives to a problem before I answer.
\item
  I try to use strategies that have worked in the past.
\item
  I pace myself while learning in order to have enough time.
\item
  I understand my intellectual strengths and weaknesses.
\item
  I think about what I really need to learn before I begin a task.
\item
  I know how well I did once I finish a test.
\item
  I set specific goals before I begin a task.
\item
  I slow down when I encounter important information.
\item
  I know what kind of information is most important to learn.
\item
  I ask myself if I have considered all options when solving a problem.
\item
  I am good at organizing information.
\item
  I consciously focus my attention on important information.
\item
  I have a specific purpose for each strategy I use.
\item
  I learn best when I know something about the topic.
\item
  I know what the teacher expects me to learn.
\item
  I am good at remembering information.
\item
  I use different learning strategies depending on the situation.
\item
  I ask myself if there was an easier way to do things after I finish a task.
\item
  I have control over how well I learn.
\item
  I periodically review to help me understand important relationships.
\item
  I ask myself questions about the material before I begin.
\item
  I think of several ways to solve a problem and choose the best one.
\item
  I summarize what I've learned after I finish.
\item
  I ask others for help when I don't understand something.
\item
  I can motivate myself to learn when I need to.
\item
  I am aware of what strategies I use when I study.
\item
  I find myself analyzing the usefulness of strategies while I study.
\item
  I use my intellectual strengths to compensate for my weaknesses.
\item
  I focus on the meaning and significance of new information.
\item
  I create my own examples to make information more meaningful.
\item
  I am a good judge of how well I understand something.
\item
  I find myself using helpful learning strategies automatically.
\item
  I find myself pausing regularly to check my comprehension.
\item
  I know when each strategy I use will be most effective.
\item
  I ask myself how well I accomplish my goals once I'm finished.
\item
  I draw pictures or diagrams to help me understand while learning.
\item
  I ask myself if I have considered all options after I solve a problem.
\item
  I try to translate new information into my own words.
\item
  I change strategies when I fail to understand.
\item
  I use the organizational structure of the text to help me learn.
\item
  I read instructions carefully before I begin a task.
\item
  I ask myself if what I'm reading is related to what I already know.
\item
  I reevaluate my assumptions when I get confused.
\item
  I organize my time to best accomplish my goals.
\item
  I learn more when I am interested in the topic.
\item
  I try to break studying down into smaller steps.
\item
  I focus on overall meaning rather than specifics.
\item
  I ask myself questions about how well I am doing while I am learning something new.
\item
  I ask myself if I learned as much as I could have once I finish a task.
\item
  I stop and go back over new information that is not clear.
\item
  I stop and reread when I get confused.
\end{enumerate}

\textbf{Scoring Directions:} Score each response from 1 (never) to 5
(always), and calculate the following subscale scores by summing the
items on that subscale.

\emph{Knowledge about Cognition:}

\begin{enumerate}
\def\labelenumi{\arabic{enumi}.}
\tightlist
\item
  \emph{Declarative Knowledge:} 5, 10, 12, 16, 17, 20, 32, 46 (score out of \(8\times5 = 40\))
\item
  \emph{Procedural Knowledge:} 3, 14, 27, 33 (score out of \(4\times5 = 20\))
\item
  \emph{Conditional Knowledge:} 15, 18, 26, 29, 35 (score out of \(5\times5 = 25\))
\end{enumerate}

\emph{Regulation of Cognition:}

\begin{enumerate}
\def\labelenumi{\arabic{enumi}.}
\tightlist
\item
  \emph{Planning:} 4, 6, 8, 22, 23, 42, 45 (score out of \(7\times5 = 35\))
\item
  \emph{Information Management Strategies:} 9, 13, 30, 31, 37, 39, 41, 43, 47, 48 (score out of \(10\times5 = 50\))
\item
  \emph{Comprehension Monitoring:} 1, 2, 11, 21, 28, 34, 49 (score out of \(7\times5 = 35\))
\item
  \emph{Debugging Strategies:} 25, 40, 44, 51, 52 (score out of \(5\times5 = 25\))
\item
  \emph{Evaluation:} 7, 19, 24, 36, 38, 50 (score out of \(6\times5 = 30\))
  --\textgreater{}
\end{enumerate}

\hypertarget{app_pre_post_tasks}{%
\chapter{Questionnaires for Initial (SES1) and Final (SES3) Sessions}\label{app_pre_post_tasks}}

\hypertarget{app_midterm_survey}{%
\chapter{SUR2: Midterm Survey}\label{app_midterm_survey}}

\hypertarget{app_final_survey}{%
\chapter{SUR3: Exit Survey}\label{app_final_survey}}

\hypertarget{app_variables}{%
\chapter{Variables and Measures}\label{app_variables}}

\hypertarget{sec_app_ack}{%
\chapter{Acknowledgements - The PhD Journey}\label{sec_app_ack}}

Similar to David Maxwell's thesis.

This section will be fleshed out in more detail after the initial committee-submission on Feb 27, 2023.
For now, I wish to thank the following people and organisations (in no particular order):

\begin{itemize}
\tightlist
\item
  Jacek Gwizdka
\item
  Soo Young Rieh + Funding
\item
  Committee Members
\item
  HEB
\item
  Finland people
\item
  Slovenia People
\item
  Germany People

  \begin{itemize}
  \tightlist
  \item
    Anke, Xiaofei, Michael, Hema, Himanshu, Ambika, Hardik\ldots{}
  \end{itemize}
\item
  India People
\item
  UK People
\item
  USA People
\item
  HCI4SouthAsia People
\item
  ASIST People
\item
  CHIIR People + Conferences
\item
  UT Graduate School Funding
\item
  SALPilot Study People
\item
  I303 People
\item
  DAAD
\item
  ABB
\item
  iSchool Doc Colleagues
\item
  Labmates, Officemates
\item
  LinkedIn people
\item
  Twitter people

  \begin{itemize}
  \tightlist
  \item
    Jason Baldridge
  \end{itemize}
\end{itemize}

\hypertarget{references}{%
\chapter*{References}\label{references}}
\addcontentsline{toc}{chapter}{References}

\markboth{References}{}

\hypertarget{refs}{}
\begin{CSLReferences}{1}{0}
\leavevmode\vadjust pre{\hypertarget{ref-agosti2014evaluation}{}}%
Agosti, M., Fuhr, N., Toms, E., \& Vakkari, P. (2014). Evaluation methodologies in information retrieval dagstuhl seminar 13441. \emph{ACM SIGIR Forum}, \emph{48}, 36--41.

\leavevmode\vadjust pre{\hypertarget{ref-allan2012frontiers}{}}%
Allan, J., Croft, B., Moffat, A., \& Sanderson, M. (2012). Frontiers, challenges, and opportunities for information retrieval: Report from SWIRL 2012 the second strategic workshop on information retrieval in lorne. \emph{ACM SIGIR Forum}, \emph{46}, 2--32.

\leavevmode\vadjust pre{\hypertarget{ref-ambrose2010howa}{}}%
Ambrose, S. A., Bridges, M. W., DiPietro, M., Lovett, M. C., \& Norman, M. K. (2010). \emph{How {Learning Works}: Seven {Research}-{Based Principles} for {Smart Teaching}}. {John Wiley \& Sons}.

\leavevmode\vadjust pre{\hypertarget{ref-amina2017active}{}}%
Amina, T. (2017). Active knowledge making: Epistemic dimensions of e-learning. In \emph{E-learning ecologies} (pp. 65--87). Routledge.

\leavevmode\vadjust pre{\hypertarget{ref-ausubel2012acquisition}{}}%
Ausubel, D. P. (2012). \emph{The acquisition and retention of knowledge: A cognitive view}. Springer Science \& Business Media.

\leavevmode\vadjust pre{\hypertarget{ref-ausubel1968educational}{}}%
Ausubel, D. P., Novak, J. D., Hanesian, H., et al. (1968). \emph{Educational psychology: A cognitive view} (Vol. 6). Holt, Rinehart; Winston New York.

\leavevmode\vadjust pre{\hypertarget{ref-belkin1982ask}{}}%
Belkin, N. J., Oddy, R. N., \& Brooks, H. M. (1982). ASK for information retrieval: Part i. Background and theory. \emph{Journal of Documentation}.

\leavevmode\vadjust pre{\hypertarget{ref-bhattacharya2021longitudinal}{}}%
Bhattacharya, N. (2021). A longitudinal study to understand learning during search. \emph{Proceedings of the 2021 Conference on Human Information Interaction and Retrieval}, 363--366.

\leavevmode\vadjust pre{\hypertarget{ref-blanken2017metacognition}{}}%
Blanken-Webb, J. (2017). Metacognition: Cognitive dimensions of e-learning. In \emph{E-learning ecologies} (pp. 163--182). Routledge.

\leavevmode\vadjust pre{\hypertarget{ref-borlund2013interactive}{}}%
Borlund, P. (2013). Interactive {Information Retrieval}: {An Introduction}. \emph{Journal of Information Science Theory and Practice}, \emph{1}(3), 12--32. \url{https://doi.org/10.1633/JISTAP.2013.1.3.2}

\leavevmode\vadjust pre{\hypertarget{ref-breakstone2018we}{}}%
Breakstone, J., McGrew, S., Smith, M., Ortega, T., \& Wineburg, S. (2018). Why we need a new approach to teaching digital literacy. \emph{Phi Delta Kappan}, \emph{99}(6), 27--32.

\leavevmode\vadjust pre{\hypertarget{ref-breakstone2021students}{}}%
Breakstone, J., Smith, M., Wineburg, S., Rapaport, A., Carle, J., Garland, M., \& Saavedra, A. (2021). Students' {Civic Online Reasoning}: A {National Portrait}. \emph{Educational Researcher}. \url{https://doi.org/10.3102/0013189X211017495}

\leavevmode\vadjust pre{\hypertarget{ref-brennan2016factor}{}}%
Brennan, K., Kelly, D., \& Zhang, Y. (2016). Factor analysis of a search self-efficacy scale. \emph{Proceedings of the 2016 ACM on Conference on Human Information Interaction and Retrieval}, 241--244.

\leavevmode\vadjust pre{\hypertarget{ref-brookes1980foundations}{}}%
Brookes, B. C. (1980). The foundations of information science. Part i. Philosophical aspects. \emph{Journal of Information Science}, \emph{2}(3-4), 125--133.

\leavevmode\vadjust pre{\hypertarget{ref-brown1999self}{}}%
Brown, J. M., Miller, W. R., \& Lawendowski, L. A. (1999). The self-regulation questionnaire. In V. L. \& J. T. L. (Eds.), \emph{Innovations in clinical practice: A sourcebook} (Vol. 17, pp. 281--292). Professional Resource Press/Professional Resource Exchange.

\leavevmode\vadjust pre{\hypertarget{ref-collins2017search}{}}%
Collins-Thompson, K., Hansen, P., \& Hauff, C. (2017). Search as learning (dagstuhl seminar 17092). \emph{Dagstuhl Reports}, \emph{7}.

\leavevmode\vadjust pre{\hypertarget{ref-cope2017elearningc}{}}%
Cope, B., \& Kalantzis, M. (2017). \emph{E-{Learning Ecologies}: Principles for {New Learning} and {Assessment}}. {Taylor \& Francis}.

\leavevmode\vadjust pre{\hypertarget{ref-cope2013new}{}}%
Cope, B., \& Kalantzis, M. (2013). Towards a {New Learning}: The {\emph{Scholar}} {Social Knowledge Workspace}, in {Theory} and {Practice}. \emph{E-Learning and Digital Media}, \emph{10}(4), 332--356. \url{https://doi.org/10.2304/elea.2013.10.4.332}

\leavevmode\vadjust pre{\hypertarget{ref-dervin2010sensemaking}{}}%
Dervin, B., \& Naumer, C. M. (2010). Sense-making. In M. J. Bates \& M. M. N. (Eds.), \emph{Encyclopedia of library and information sciences (3rd ed.)} (pp. 4696-\/-4707). Taylor; Francis.

\leavevmode\vadjust pre{\hypertarget{ref-dicerbo2014impacts}{}}%
DiCerbo, K. E., \& Behrens, J. T. (2014). Impacts of the digital ocean on education. \emph{London: Pearson}, \emph{1}.

\leavevmode\vadjust pre{\hypertarget{ref-egusa2010usingb}{}}%
Egusa, Y., Saito, H., Takaku, M., Terai, H., Miwa, M., \& Kando, N. (2010). Using a {Concept Map} to {Evaluate Exploratory Search}. \emph{Proceedings of the {Third Symposium} on {Information Interaction} in {Context}}, 175--184. \url{https://doi.org/10.1145/1840784.1840810}

\leavevmode\vadjust pre{\hypertarget{ref-egusa2014howd}{}}%
Egusa, Y., Takaku, M., \& Saito, H. (2014a). How {Concept Maps Change} if a {User Does Search} or {Not}? \emph{Proceedings of the 5th {Information Interaction} in {Context Symposium}}, 68--75. \url{https://doi.org/10.1145/2637002.2637012}

\leavevmode\vadjust pre{\hypertarget{ref-egusa2014howe}{}}%
Egusa, Y., Takaku, M., \& Saito, H. (2014b). How to evaluate searching as learning. \emph{Searching as {Learning Workshop} ({IIiX} 2014 Workshop)}. \url{http://www.diigubc.ca/IIIXSAL/program.html}

\leavevmode\vadjust pre{\hypertarget{ref-egusa2017evaluating}{}}%
Egusa, Y., Takaku, M., \& Saito, H. (2017). Evaluating {Complex Interactive Searches Using Concept Maps}. \emph{{SCST}@ {CHIIR}}, 15--17.

\leavevmode\vadjust pre{\hypertarget{ref-eickhoff2017introduction}{}}%
Eickhoff, C., Gwizdka, J., Hauff, C., \& He, J. (2017). Introduction to the special issue on search as learning. \emph{Information Retrieval Journal}, \emph{20}(5), 399--402.

\leavevmode\vadjust pre{\hypertarget{ref-freund2013searching}{}}%
Freund, L., Gwizdka, J., Hansen, P., Kando, N., \& Rieh, S. Y. (2013). From searching to learning. \emph{Evaluation Methodologies in Information Retrieval. Dagstuhl Reports}, \emph{13441}, 102--105.

\leavevmode\vadjust pre{\hypertarget{ref-freund2014searching}{}}%
Freund, L., He, J., Gwizdka, J., Kando, N., Hansen, P., \& Rieh, S. Y. (2014). Searching as learning (SAL) workshop 2014. \emph{Proceedings of the 5th Information Interaction in Context Symposium}, 7--7.

\leavevmode\vadjust pre{\hypertarget{ref-grabowski1996generative}{}}%
Grabowski, B. L. (1996). Generative learning: Past, present, and future. \emph{Handbook of Research for Educational Communications and Technology}, 897--918.

\leavevmode\vadjust pre{\hypertarget{ref-gwizdka2016search}{}}%
Gwizdka, J., Hansen, P., Hauff, C., He, J., \& Kando, N. (2016). Search as learning (SAL) workshop 2016. \emph{Proceedings of the 39th International ACM SIGIR Conference on Research and Development in Information Retrieval}, 1249--1250.

\leavevmode\vadjust pre{\hypertarget{ref-halttunen2005assessing}{}}%
Halttunen, K., \& Jarvelin, K. (2005). Assessing learning outcomes in two information retrieval learning environments. \emph{Information Processing \& Management}, \emph{41}(4), 949--972. \url{https://doi.org/10.1016/j.ipm.2004.02.004}

\leavevmode\vadjust pre{\hypertarget{ref-hansen2016editorial}{}}%
Hansen, P., \& Rieh, S. Y. (2016). Editorial: Recent advances on searching as learning: An introduction to the special issue. \emph{Journal of Information Science}, \emph{42}(1), 3--6. \url{https://doi.org/10.1177/0165551515614473}

\leavevmode\vadjust pre{\hypertarget{ref-kahne2012digital}{}}%
Kahne, J., Lee, N.-J., \& Feezell, J. T. (2012). Digital media literacy education and online civic and political participation. \emph{International Journal of Communication}, \emph{6}, 24.

\leavevmode\vadjust pre{\hypertarget{ref-kalantzis2012newa}{}}%
Kalantzis, M., \& Cope, B. (2012). \emph{New {Learning}: Elements of a {Science} of {Education}}. {Cambridge University Press}.

\leavevmode\vadjust pre{\hypertarget{ref-karapanos2021advances}{}}%
Karapanos, E., Gerken, J., Kjeldskov, J., \& Skov, M. B. (Eds.). (2021). \emph{Advances in {Longitudinal HCI Research}}. {Springer International Publishing}. \url{https://doi.org/10.1007/978-3-030-67322-2}

\leavevmode\vadjust pre{\hypertarget{ref-kelly2006measuring_a}{}}%
Kelly, D. (2006a). Measuring online information seeking context, {Part} 1: Background and method. \emph{Journal of the American Society for Information Science and Technology}, \emph{57}(13), 1729--1739. \url{https://doi.org/10.1002/asi.20483}

\leavevmode\vadjust pre{\hypertarget{ref-kelly2006measuring_b}{}}%
Kelly, D. (2006b). Measuring online information seeking context, {Part} 2: Findings and discussion. \emph{Journal of the American Society for Information Science and Technology}, \emph{57}(14), 1862--1874. \url{https://doi.org/10.1002/asi.20484}

\leavevmode\vadjust pre{\hypertarget{ref-kelly2009evaluation}{}}%
Kelly, D., Dumais, S., \& Pedersen, J. O. (2009). Evaluation challenges and directions for information-seeking support systems. \emph{IEEE Computer}, \emph{42}(3).

\leavevmode\vadjust pre{\hypertarget{ref-ko2021seeking}{}}%
Ko, A. J. (2021). Seeking information. In \emph{Foundations of {Information}}. \url{https://faculty.washington.edu/ajko/books/foundations-of-information/\#/seeking}

\leavevmode\vadjust pre{\hypertarget{ref-HCIUXres81_online}{}}%
Koeman, L. (2020). \emph{HCI/UX research: What methods do we use? -- lisa koeman -- blog}. \url{https://lisakoeman.nl/blog/hci-ux-research-what-methods-do-we-use/}.

\leavevmode\vadjust pre{\hypertarget{ref-kuhlthau2004seeking}{}}%
Kuhlthau, C. C. (2004). \emph{Seeking meaning: A process approach to library and information services} (Vol. 2). Libraries Unlimited Westport, CT.

\leavevmode\vadjust pre{\hypertarget{ref-lei2015effect}{}}%
Lei, P.-L., Sun, C.-T., Lin, S. S., \& Huang, T.-K. (2015). Effect of metacognitive strategies and verbal-imagery cognitive style on biology-based video search and learning performance. \emph{Computers \& Education}, \emph{87}, 326--339.

\leavevmode\vadjust pre{\hypertarget{ref-marchionini1995information}{}}%
Marchionini, G. (1995). \emph{Information {Seeking} in {Electronic Environments}}. {Cambridge University Press}.

\leavevmode\vadjust pre{\hypertarget{ref-marchionini2006toward}{}}%
Marchionini, G. (2006). Toward human-computer information retrieval. \emph{Bulletin of the American Society for Information Science and Technology}, \emph{32}(5), 20--22.

\leavevmode\vadjust pre{\hypertarget{ref-marton1976qualitative_b}{}}%
Marton, F., \& Säaljö, R. (1976). On qualitative differences in learning---ii outcome as a function of the learner's conception of the task. \emph{British Journal of Educational Psychology}, \emph{46}(2), 115--127.

\leavevmode\vadjust pre{\hypertarget{ref-marton1976qualitative_a}{}}%
Marton, F., \& Säljö, R. (1976). On qualitative differences in learning: I---outcome and process. \emph{British Journal of Educational Psychology}, \emph{46}(1), 4--11.

\leavevmode\vadjust pre{\hypertarget{ref-mcgrew2020learning}{}}%
McGrew, S. (2020). Learning to evaluate: An intervention in civic online reasoning. \emph{Computers \& Education}, \emph{145}, 103711.

\leavevmode\vadjust pre{\hypertarget{ref-mcgrew2021skipping}{}}%
McGrew, S. (2021). Skipping the source and checking the contents: An in-depth look at students' approaches to web evaluation. \emph{Computers in the Schools}, \emph{38}(2), 75--97.

\leavevmode\vadjust pre{\hypertarget{ref-mcgrew2018can}{}}%
McGrew, S., Breakstone, J., Ortega, T., Smith, M., \& Wineburg, S. (2018). Can students evaluate online sources? Learning from assessments of civic online reasoning. \emph{Theory \& Research in Social Education}, \emph{46}(2), 165--193.

\leavevmode\vadjust pre{\hypertarget{ref-mcgrew2021click}{}}%
McGrew, S., \& Glass, A. C. (2021). Click {Restraint}: Teaching {Students} to {Analyze Search Results}. \emph{Proceedings of the 14th {International Conference} on {Computer}-{Supported Collaborative Learning}-{CSCL} 2021}.

\leavevmode\vadjust pre{\hypertarget{ref-mcgrew2017challenge}{}}%
McGrew, S., Ortega, T., Breakstone, J., \& Wineburg, S. (2017). The challenge that's bigger than fake news: Civic reasoning in a social media environment. \emph{American Educator}, \emph{41}(3), 4.

\leavevmode\vadjust pre{\hypertarget{ref-mihailidis2013media}{}}%
Mihailidis, P., \& Thevenin, B. (2013). Media literacy as a core competency for engaged citizenship in participatory democracy. \emph{American Behavioral Scientist}, \emph{57}(11), 1611--1622.

\leavevmode\vadjust pre{\hypertarget{ref-council2000how}{}}%
National Research Council. (2000). \emph{How people learn: {Brain}, mind, experience, and school: {Expanded} edition}. {The National Academies Press}. \url{https://doi.org/10.17226/9853}

\leavevmode\vadjust pre{\hypertarget{ref-newlondon1996pedagogy}{}}%
New London Group. (1996). A pedagogy of multiliteracies: Designing social futures. \emph{Harvard Educational Review}, \emph{66}(1), 60--92.

\leavevmode\vadjust pre{\hypertarget{ref-ngss_sensemaking}{}}%
Next Generation Science Standards. (2021). \emph{Task annotation project in science \textbar{} sense-making}. \url{https://www.nextgenscience.org/sites/default/files/TAPS\%20Sense-making.pdf}.

\leavevmode\vadjust pre{\hypertarget{ref-novak2002meaningful}{}}%
Novak, J. D. (2002). Meaningful learning: The essential factor for conceptual change in limited or inappropriate propositional hierarchies leading to empowerment of learners. \emph{Science Education}, \emph{86}(4), 548--571.

\leavevmode\vadjust pre{\hypertarget{ref-novak2010learninga}{}}%
Novak, J. D. (2010). \emph{Learning, creating, and using knowledge: Concept maps as facilitative tools in schools and corporations} (2nd ed). {Routledge}.

\leavevmode\vadjust pre{\hypertarget{ref-novak1984learning}{}}%
Novak, J. D., \& Gowin, D. B. (1984). \emph{Learning how to learn}. Cambridge University Press. \url{https://doi.org/10.1017/CBO9781139173469}

\leavevmode\vadjust pre{\hypertarget{ref-pea2014learning}{}}%
Pea, R., \& Jacks, D. (2014). \emph{The learning analytics workgroup: A report on building the field of learning analytics for personalized learning at scale}. \url{https://ed.stanford.edu/sites/default/files/law_report_complete_09-02-2014.pdf}; Stanford, CA: Stanford University.

\leavevmode\vadjust pre{\hypertarget{ref-piaget1936origins}{}}%
Piaget, J. (1936). \emph{Origins of intelligence in children.}

\leavevmode\vadjust pre{\hypertarget{ref-url_rieh_homepage}{}}%
Rieh, S. Y. (2020). \emph{Research area 1: Searching as learning}. \url{https://rieh.ischool.utexas.edu/research}.

\leavevmode\vadjust pre{\hypertarget{ref-rieh2016searching}{}}%
Rieh, S. Y., Collins-Thompson, K., Hansen, P., \& Lee, H.-J. (2016). Towards searching as a learning process: A review of current perspectives and future directions. \emph{Journal of Information Science}, \emph{42}(1), 19--34. \url{https://doi.org/10.1177/0165551515615841}

\leavevmode\vadjust pre{\hypertarget{ref-roy2021note}{}}%
Roy, N., Torre, M. V., Gadiraju, U., Maxwell, D., \& Hauff, C. (2021). Note the highlight: Incorporating active reading tools in a search as learning environment. \emph{Proceedings of the 2021 Conference on Human Information Interaction and Retrieval}, 229--238.

\leavevmode\vadjust pre{\hypertarget{ref-rumelhart1981accretion}{}}%
Rumelhart, D. E., \& Norman, D. A. (1981). Accretion, tuning and restructuring: Three modes of learning. In J. W. Cotton \& K. R. (Eds.), \emph{Semantic factors in cognition} (pp. 37--90).

\leavevmode\vadjust pre{\hypertarget{ref-rumelhart1977representation}{}}%
Rumelhart, D. E., \& Ortony, A. (1977). The representation of knowledge in memory. In R. C. Anderson, S. R. J., \& M. W. E. (Eds.), \emph{Schooling and the acquisition of knowledge} (pp. 99--135). Hillsdale, NJ: Erlbaum.

\leavevmode\vadjust pre{\hypertarget{ref-ryan1982control}{}}%
Ryan, R. M. (1982). Control and information in the intrapersonal sphere: An extension of cognitive evaluation theory. \emph{Journal of Personality and Social Psychology}, \emph{43}(3), 450.

\leavevmode\vadjust pre{\hypertarget{ref-sawyer2005cambridge}{}}%
Sawyer, R. K. (2005). \emph{The {Cambridge} handbook of the learning sciences}. {Cambridge University Press}.

\leavevmode\vadjust pre{\hypertarget{ref-schraw1994assessing}{}}%
Schraw, G., \& Dennison, R. S. (1994). Assessing {Metacognitive Awareness}. \emph{Contemporary Educational Psychology}, \emph{19}(4), 460--475. \url{https://doi.org/10.1006/ceps.1994.1033}

\leavevmode\vadjust pre{\hypertarget{ref-terlecki2020revising}{}}%
Terlecki, M. (2020). Revising the {Metacognitive Awareness Inventory} ({MAI}) to be {More User}-{Friendly}. In \emph{Improve with Metacognition}. \url{https://www.improvewithmetacognition.com/revising-the-metacognitive-awareness-inventory/}

\leavevmode\vadjust pre{\hypertarget{ref-terlecki2018call}{}}%
Terlecki, M., \& McMahon, A. (2018). A {Call} for {Metacognitive Intervention}: Improvements {Due} to {Curricular Programming} in {Leadership}. \emph{Journal of Leadership Education}, \emph{17}(4), 130--145. \url{https://doi.org/10.12806/V17/I4/R8}

\leavevmode\vadjust pre{\hypertarget{ref-vakkari2016searching}{}}%
Vakkari, P. (2016). Searching as learning: A systematization based on literature. \emph{Journal of Information Science}, \emph{42}(1), 7--18. \url{https://doi.org/10.1177/0165551515615833}

\leavevmode\vadjust pre{\hypertarget{ref-vakkari2001changes}{}}%
Vakkari, P. (2001). Changes in search tactics and relevance judgements when preparing a research proposal a summary of the findings of a longitudinal study. \emph{Information Retrieval}, \emph{4}(3), 295--310.

\leavevmode\vadjust pre{\hypertarget{ref-van2017development}{}}%
Van Der Vaart, R., \& Drossaert, C. (2017). Development of the digital health literacy instrument: Measuring a broad spectrum of health 1.0 and health 2.0 skills. \emph{Journal of Medical Internet Research}, \emph{19}(1), e27.

\leavevmode\vadjust pre{\hypertarget{ref-weber2019informationseeking}{}}%
Weber, H., Becker, D., \& Hillmert, S. (2019). Information-seeking behaviour and academic success in higher education: Which search strategies matter for grade differences among university students and how does this relevance differ by field of study? \emph{Higher Education}, \emph{77}(4), 657--678. \url{https://doi.org/10.1007/s10734-018-0296-4}

\leavevmode\vadjust pre{\hypertarget{ref-weber2018can}{}}%
Weber, H., Hillmert, S., \& Rott, K. J. (2018). Can digital information literacy among undergraduates be improved? Evidence from an experimental study. \emph{Teaching in Higher Education}, \emph{23}(8), 909--926. \url{https://doi.org/10.1080/13562517.2018.1449740}

\leavevmode\vadjust pre{\hypertarget{ref-white2016interactions}{}}%
White, R. (2016a). \emph{Interactions with search systems}. Cambridge University Press.

\leavevmode\vadjust pre{\hypertarget{ref-white_2016_iwss_learning}{}}%
White, R. (2016b). Learning and use. In \emph{Interactions with search systems} (pp. 231--248). {Cambridge University Press}. \url{https://doi.org/10.1017/CBO9781139525305.010}

\leavevmode\vadjust pre{\hypertarget{ref-white2009characterizing}{}}%
White, R., Dumais, S., \& Teevan, J. (2009). Characterizing the influence of domain expertise on web search behavior. \emph{Proceedings of the {Second ACM International Conference} on {Web Search} and {Data Mining} - {WSDM} '09}, 132. \url{https://doi.org/10.1145/1498759.1498819}

\leavevmode\vadjust pre{\hypertarget{ref-wildemuth2004effects}{}}%
Wildemuth, B. M. (2004). The effects of domain knowledge on search tactic formulation. \emph{Journal of the American Society for Information Science and Technology}, \emph{55}(3), 246--258. \url{https://doi.org/10.1002/asi.10367}

\leavevmode\vadjust pre{\hypertarget{ref-wilson2013comparison}{}}%
Wilson, M. J., \& Wilson, M. L. (2013). A comparison of techniques for measuring sensemaking and learning within participant-generated summaries. \emph{Journal of the American Society for Information Science and Technology}, \emph{64}(2), 291--306.

\leavevmode\vadjust pre{\hypertarget{ref-wineburg2016students}{}}%
Wineburg, S., \& McGrew, S. (2016). Why students can't google their way to the truth. \emph{Education Week}, \emph{36}(11), 22--28.

\leavevmode\vadjust pre{\hypertarget{ref-wineburg2017lateral}{}}%
Wineburg, S., \& McGrew, S. (2017). \emph{Lateral reading: Reading less and learning more when evaluating digital information}.

\leavevmode\vadjust pre{\hypertarget{ref-wittrock1989generative}{}}%
Wittrock, M. C. (1989). Generative processes of comprehension. \emph{Educational Psychologist}, \emph{24}(4), 345--376.

\leavevmode\vadjust pre{\hypertarget{ref-zhang2014towards}{}}%
Zhang, P., \& Soergel, D. (2014). Towards a comprehensive model of the cognitive process and mechanisms of individual sensemaking. \emph{Journal of the Association for Information Science and Technology}, \emph{65}(9), 1733--1756. \url{https://doi.org/10.1002/asi.23125}

\leavevmode\vadjust pre{\hypertarget{ref-zlatkin2021students}{}}%
Zlatkin-Troitschanskaia, O., Hartig, J., Goldhammer, F., \& Krstev, J. (2021). Students' online information use and learning progress in higher education \textendash{} {A} critical literature review. \emph{Studies in Higher Education}, 1--26. \url{https://doi.org/10.1080/03075079.2021.1953336}

\end{CSLReferences}

%%%%% REFERENCES


\end{document}
