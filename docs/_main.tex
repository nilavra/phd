%%%%%%%%%%%%%%%%%%%%%%%%%%%%%%%%%%%%%%%%%%%%%%%%%%%%%%%%%%%%%%%
%% U TEXAS THESIS TEMPLATE
%
% Originally by Keith A. Gillow (gillow@maths.ox.ac.uk), 1997
% Modified by Sam Evans (sam@samuelevansresearch.org), 2007
% Modified by John McManigle (john@oxfordechoes.com), 2015
% Modified by Ulrik Lyngs (ulrik.lyngs@cs.ox.ac.uk), 2018-, for use with R Markdown
% Modified by Nilavra Bhattacharya (nilavra@utexas.edu), 2023-, for use with R Markdown
%
% Nilavra Bhattacharya, 24 Jan 2023: Following previous authors: Ulrik Lyngs and John McManigle, broad permissions are granted to use, modify, and distribute this software
% as specified in the MIT License included in this distribution's LICENSE file.
%
% Ulrik Lyngs, 25 Nov 2018: Following John McManigle, broad permissions are granted to use, modify, and distribute this software
% as specified in the MIT License included in this distribution's LICENSE file.
%
% John commented this file extensively, so read through to see how to use the various options.  Remember that in LaTeX,
% any line starting with a % is NOT executed.

%%%%% PAGE LAYOUT
% The most common choices should be below.  You can also do other things, like replace "a4paper" with "letterpaper", etc.

% 'twoside' formats for two-sided binding (ie left and right pages have mirror margins; blank pages inserted where needed):
%\documentclass[a4paper,twoside]{templates/ociamthesis}
% Specifying nothing formats for one-sided binding (ie left margin > right margin; no extra blank pages):
%\documentclass[a4paper]{ociamthesis}
% 'nobind' formats for PDF output (ie equal margins, no extra blank pages):
%\documentclass[a4paper,nobind]{templates/ociamthesis}

% As you can see from the line below, oxforddown uses the a4paper size, 
% and passes in the binding option from the YAML header in index.Rmd:
\documentclass[letterpaper, nobind]{templates/ociamthesis}


%%%%% ADDING LATEX PACKAGES
% add hyperref package with options from YAML %
\usepackage[pdfpagelabels]{hyperref}
% handle long urls
\usepackage{xurl}
% change the default coloring of links to something sensible
\usepackage{xcolor}
% single double and other spacing
\usepackage{setspace}

\usepackage{fontspec}    
\newfontfamily\dispnamefont{kalpurush.ttf}
% don't use Tiro Bangla; messes up juktakkhors

% for bengali fonts
% \babelprovide[import]{bangla}
% \babelfont[bangla]{rm}{kalpurush.ttf}
% \usepackage{fontspec} 
% \usepackage{polyglossia} 
% \setdefaultlanguage{english} 
% \setotherlanguages{bengali}
% \newfontfamily{\bengalifont}[Script=Bengali]{kalpurush.ttf}
% \setotherlanguages{hindi,sanskrit,bengali}
% \setmainfont{Times New Roman} 
% \newfontfamily{\devanagarifont}[Script=Devanagari]{Lohit Devanagari.ttf} 


\definecolor{mylinkcolor}{RGB}{0,0,139}
\definecolor{myurlcolor}{RGB}{0,0,139}
\definecolor{mycitecolor}{RGB}{0,33,71}

\hypersetup{
  hidelinks,
  colorlinks,
  linktocpage=true,
  linkcolor=mylinkcolor,
  urlcolor=myurlcolor,
  citecolor=mycitecolor
}


% add float package to allow manual control of figure positioning %
\usepackage{float}

% enable strikethrough
\usepackage[normalem]{ulem}

% use soul package for correction highlighting
\usepackage{color, soulutf8}
\definecolor{correctioncolor}{HTML}{CCCCFF}
\sethlcolor{correctioncolor}
\newcommand{\ctext}[3][RGB]{%
  \begingroup
  \definecolor{hlcolor}{#1}{#2}\sethlcolor{hlcolor}%
  \hl{#3}%
  \endgroup
}
% stop soul from freaking out when it sees citation commands
\soulregister\ref7
\soulregister\cite7
\soulregister\citet7
\soulregister\autocite7
\soulregister\textcite7
\soulregister\pageref7

%%%%% FIXING / ADDING THINGS THAT'S SPECIAL TO R MARKDOWN'S USE OF LATEX TEMPLATES
% pandoc puts lists in 'tightlist' command when no space between bullet points in Rmd file,
% so we add this command to the template
\providecommand{\tightlist}{%
  \setlength{\itemsep}{0pt}\setlength{\parskip}{0pt}}
 
% allow us to include code blocks in shaded environments

% User-included things with header_includes or in_header will appear here
% kableExtra packages will appear here if you use library(kableExtra)
\usepackage{booktabs}
\usepackage{longtable}
\usepackage{array}
\usepackage{multirow}
\usepackage{wrapfig}
\usepackage{float}
\usepackage{colortbl}
\usepackage{pdflscape}
\usepackage{tabu}
\usepackage{threeparttable}
\usepackage{threeparttablex}
\usepackage[normalem]{ulem}
\usepackage{makecell}
\usepackage{xcolor}


%UL set section header spacing
\usepackage{titlesec}
% 
\titlespacing\subsubsection{0pt}{24pt plus 4pt minus 2pt}{0pt plus 2pt minus 2pt}


%UL set whitespace around verbatim environments
\usepackage{etoolbox}
\makeatletter
\preto{\@verbatim}{\topsep=0pt \partopsep=0pt }
\makeatother


%%%%%%% PAGE HEADERS AND FOOTERS %%%%%%%%%
\usepackage{fancyhdr}
\setlength{\headheight}{15pt}
\fancyhf{} % clear the header and footers
\pagestyle{fancy}
\renewcommand{\chaptermark}[1]{\markboth{\thechapter. #1}{\thechapter. #1}}
\renewcommand{\sectionmark}[1]{\markright{\thesection. #1}} 
\renewcommand{\headrulewidth}{0pt}

\fancyhead[LO]{\emph{\leftmark}} 
\fancyhead[RE]{\emph{\rightmark}} 




% UL page number position 
\fancyfoot[C]{\emph{\thepage}} %regular pages
\fancypagestyle{plain}{\fancyhf{}\fancyfoot[C]{\emph{\thepage}}} %chapter pages




%%%%% SELECT YOUR DRAFT OPTIONS
% This adds a "DRAFT" footer to every normal page.  (The first page of each chapter is not a "normal" page.)
\fancyhead[R]{\emph{Draft \today}}

% IP feb 2021: option to include line numbers in PDF

% for line wrapping in code blocks
\usepackage{fancyvrb}
\usepackage{fvextra}
\DefineVerbatimEnvironment{Highlighting}{Verbatim}{breaklines=true, breakanywhere=true, commandchars=\\\{\}}

% This highlights (in blue) corrections marked with (for words) \mccorrect{blah} or (for whole
% paragraphs) \begin{mccorrection} . . . \end{mccorrection}.  This can be useful for sending a PDF of
% your corrected thesis to your examiners for review.  Turn it off, and the blue disappears.
\correctionstrue


%%%%% BIBLIOGRAPHY SETUP
% Note that your bibliography will require some tweaking depending on your department, preferred format, etc.
% If you've not used LaTeX before, I recommend just using pandoc for citations -- this is what's used unless you specific e.g. "citation_package: natbib" in index.Rmd
% If you're already a LaTeX pro and are used to natbib or something, modify as necessary.

% this allows the latex template to handle pandoc citations
\newlength{\cslhangindent}
\setlength{\cslhangindent}{1.5em}
\newlength{\csllabelwidth}
\setlength{\csllabelwidth}{3em}
\newlength{\cslentryspacingunit} % times entry-spacing
\setlength{\cslentryspacingunit}{\parskip}
\newenvironment{CSLReferences}[2] % #1 hanging-ident, #2 entry spacing
 {% don't indent paragraphs
  \setlength{\parindent}{0pt}
  % turn on hanging indent if param 1 is 1
  \ifodd #1
  \let\oldpar\par
  \def\par{\hangindent=\cslhangindent\oldpar}
  \fi
  % set entry spacing
  \setlength{\parskip}{1mm}
  \setlength{\baselineskip}{6mm}
 }%
 {}
\usepackage{calc}
\newcommand{\CSLBlock}[1]{#1\hfill\break}
\newcommand{\CSLLeftMargin}[1]{\parbox[t]{\csllabelwidth}{#1}}
\newcommand{\CSLRightInline}[1]{\parbox[t]{\linewidth - \csllabelwidth}{#1}\break}
\newcommand{\CSLIndent}[1]{\hspace{\cslhangindent}#1}




% Uncomment this if you want equation numbers per section (2.3.12), instead of per chapter (2.18):
%\numberwithin{equation}{subsection}


%%%%% THESIS / TITLE PAGE INFORMATION
% Everybody needs to complete the following:
\title{LongSAL: A Longitudinal Search as Learning Study With University Students {[}Draft{]}}
\author{Nilavra Bhattacharya}


% Master's candidates who require the alternate title page (with candidate number and word count)
% must also un-comment and complete the following three lines:

% Uncomment the following line if your degree also includes exams (eg most masters):
%\renewcommand{\submittedtext}{Submitted in partial completion of the}
% Your full degree name.  (But remember that DPhils aren't "in" anything.  They're just DPhils.)
\degree{Doctor of Philosophy}

% Term and year of submission, or date if your board requires (eg most masters)



%%%%% YOUR OWN PERSONAL MACROS
% This is a good place to dump your own LaTeX macros as they come up.

% To make text superscripts shortcuts
\renewcommand{\th}{\textsuperscript{th}} % ex: I won 4\th place
\newcommand{\nd}{\textsuperscript{nd}}
\renewcommand{\st}{\textsuperscript{st}}
\newcommand{\rd}{\textsuperscript{rd}}

%%%%% THE ACTUAL DOCUMENT STARTS HERE
\begin{document}

%%%%% CHOOSE YOUR LINE SPACING HERE
% This is the official option.  Use it for your submission copy and library copy:
\setlength{\textbaselineskip}{22pt plus2pt}
% This is closer spacing (about 1.5-spaced) that you might prefer for your personal copies:
%\setlength{\textbaselineskip}{18pt plus2pt minus1pt}

% You can set the spacing here for the roman-numbered pages (acknowledgements, table of contents, etc.)
\setlength{\frontmatterbaselineskip}{17pt plus1pt minus1pt}

% UL: You can set the line and paragraph spacing here for the separate abstract page to be handed in to Examination schools
\setlength{\abstractseparatelineskip}{13pt plus1pt minus1pt}
\setlength{\abstractseparateparskip}{0pt plus 1pt}

% UL: You can set the general paragraph spacing here - I've set it to 2pt (was 0) so
% it's less claustrophobic
\setlength{\parskip}{2pt plus 1pt}

%
% Customise title page
%
\def\crest{}
\renewcommand{\university}{The University of Texas at Austin}
\renewcommand{\submittedtext}{}
\renewcommand{\thesistitlesize}{\fontsize{22pt}{28pt}\selectfont}
\renewcommand{\gapbeforecrest}{25mm}
\renewcommand{\gapaftercrest}{25mm
}


% Leave this line alone; it gets things started for the real document.
\setlength{\baselineskip}{\textbaselineskip}


%%%%% CHOOSE YOUR SECTION NUMBERING DEPTH HERE
% You have two choices.  First, how far down are sections numbered?  (Below that, they're named but
% don't get numbers.)  Second, what level of section appears in the table of contents?  These don't have
% to match: you can have numbered sections that don't show up in the ToC, or unnumbered sections that
% do.  Throughout, 0 = chapter; 1 = section; 2 = subsection; 3 = subsubsection, 4 = paragraph...

% The level that gets a number:
\setcounter{secnumdepth}{4}
% The level that shows up in the ToC:
\setcounter{tocdepth}{4}


%%%%% ABSTRACT SEPARATE
% This is used to create the separate, one-page abstract that you are required to hand into the Exam
% Schools.  You can comment it out to generate a PDF for printing or whatnot.

% JEM: Pages are roman numbered from here, though page numbers are invisible until ToC.  This is in
% keeping with most typesetting conventions.
\begin{romanpages}

% Title page is created here
% original code:
% \maketitle

% --------- start: UTexas Frontmatter (committee membership, title page) -------

% ~~~~ committee membership page ~~~~
\thispagestyle{empty} % to not show roman page numbers
\begin{center}
  The Dissertation Committee for Nilavra Bhattacharya\\
  certifies that this is the approved version of the following Dissertation:\\
  \vspace*{30pt}
  \begin{spacing}{2}
    {\Large{\textbf{LongSAL: A Longitudinal Search as Learning Study With University Students {[}Draft{]}}}}
  \end{spacing}
\end{center}

\vspace*{55pt}

\phantom{x}\hspace{45ex} {\large{\textbf{Committee:}}}\\
% \vspace*{12pt}

\begin{flushright}
  Jacek Gwizdka, Supervisor\\
  \vspace*{24pt}
  Soo-Young Rieh\\
  \vspace*{24pt}
  Matthew Lease\\
  \vspace*{24pt}
  Robert Capra
\end{flushright}


% ~~~~ title page ~~~~
\newpage
\thispagestyle{empty} % to not show roman page numbers
\begin{center}
  
  \begin{spacing}{2}
    {\Large{\textbf{LongSAL: A Longitudinal Search as Learning Study With University Students {[}Draft{]}}}}
  \end{spacing}
  
  \vspace*{24pt}

  \begin{spacing}{1.4}
    by\\
    
    \vspace*{24pt}
    
    {\Large{\textbf{
      Nilavra Bhattacharya\\
      \vspace*{10pt}
      {\Huge \dispnamefont নীলাভ্র ভট্টাচার্য্য}
    }}}\\
    
    \vspace*{72pt}
    
    {\Large{\textbf{Dissertation}}}\\
    
    \vspace*{24pt}
    
    Presented to the Faculty of the Graduate School of\\
    The University of Texas at Austin\\
    in Partial Fulfillment\\
    of the Requirements\\
    for the Degree of\\
    
    \vspace*{30pt}
    
    {\Large{\textbf{Doctor of Philosophy}}}\\
    
    \vfill

    {\large{The University of Texas at Austin\\
    May 2023}}

  \end{spacing}
\end{center}


% --------- end: UTexas Frontmatter (committee membership, title page) -------


%%%%% DEDICATION

%%%%% ACKNOWLEDGEMENTS


\begin{acknowledgements}
 	This section will be fleshed out in more detail after the initial committee-submission on Feb 27, 2023.
 For now, I wish to thank the following people :

 \begin{itemize}
 \tightlist
 \item
   Committee Members: Jacek Gwizdka, Soo Young Rieh, Matt Lease, Rob Capra
 \end{itemize}
\end{acknowledgements}



%%%%% ABSTRACT



% --------- start: UTexas abstract -------
\begin{center}
  \textbf{Abstract}\\
  
  \vspace{18pt}
  
  % title
  \begin{spacing}{2}
    {\Large{\textbf{LongSAL: A Longitudinal Search as Learning Study With University Students {[}Draft{]}}}}
  \end{spacing}

  \vspace{18pt}

  % author + supervisor
  \begin{spacing}{1.4}
    % Nilavra Bhattacharya নীলাভ্র ভট্টাচার্য্য, PhD TBD\\
    Nilavra Bhattacharya {\dispnamefont নীলাভ্র ভট্টাচার্য্য}, PhD TBD\\
    The University of Texas at Austin, 2023\\
    \vspace{18pt}
    Supervisor: Jacek Gwizdka
  \end{spacing}

\end{center}

\begin{spacing}{1.5}
  \indent
  % \indent
  Learning today is about navigation, discernment, induction, and synthesis of the wide body of information on the Internet present ubiquitously at every student's fingertips.
  Learning, or addressing a gap in one's knowledge, has been well established as an important motivator behind information-seeking activities.
  The Search as Learning research community advocates that online information search systems should be reconfigured to become educational platforms to foster learning and sensemaking.
  Modern search systems have yet to adapt to support this function.
  An important step to foster learning during online search is to identify behavioural patterns that distinguish searchers gaining more vs.~less knowledge during search.
  Previous efforts have primarily studied searchers in the short term, typically during a single lab session.
  Researchers have expressed concerns over this ephemeral approach, as learning takes place over time, and is not fleeting.
  In this dissertation, an exploratory longitudinal study was conducted to analyse the long-term searching behaviour of students enrolled in a university course, over the span of a semester.
  Our research aims are to identify if and how students' searching behaviour changes over time, as they gain new knowledge on a subject; and how do processes like motivation, metacognition, self-regulation, and other individual differences moderate their `searching as learning' behaviour.
  Findings from this exploratory longitudinal study will help to build improved search systems that foster human learning and sensemaking, and are more equitable in the face of learner diversity.
\end{spacing}

% --------- end: UTexas abstract -------



%%%%% MINI TABLES
% This lays the groundwork for per-chapter, mini tables of contents.  Comment the following line
% (and remove \minitoc from the chapter files) if you don't want this.  Un-comment either of the
% next two lines if you want a per-chapter list of figures or tables.
\dominitoc % include a mini table of contents

% This aligns the bottom of the text of each page.  It generally makes things look better.
\flushbottom

% This is where the whole-document ToC appears:
\tableofcontents

\listoffigures
	\mtcaddchapter
  	% \mtcaddchapter is needed when adding a non-chapter (but chapter-like) entity to avoid confusing minitoc

% Uncomment to generate a list of tables:
\listoftables
  \mtcaddchapter
%%%%% LIST OF ABBREVIATIONS
% This example includes a list of abbreviations.  Look at text/abbreviations.tex to see how that file is
% formatted.  The template can handle any kind of list though, so this might be a good place for a
% glossary, etc.

% The Roman pages, like the Roman Empire, must come to its inevitable close.
\end{romanpages}

%%%%% CHAPTERS
% Add or remove any chapters you'd like here, by file name (excluding '.tex'):
\flushbottom

% all your chapters and appendices will appear here
\hypertarget{introduction}{%
\chapter{Introduction}\label{introduction}}

\hypertarget{sec-intro-overview}{%
\section{Searching as Learning: Overview}\label{sec-intro-overview}}

Searching for information is a fundamental human activity. In the modern world, it is frequently conducted by users interacting with online search systems (e.g., web search engines), or more formally, \textbf{Information Retrieval} (IR) systems.
As early as in 1980, Bertam Brookes, in his `fundamental equation' of information and knowledge, had stated that an information searcher's current state of knowledge is changed to a new knowledge structure by exposure to information (\protect\hyperlink{ref-brookes1980foundations}{Brookes, 1980, p. 131}).
This indicates that searchers acquire new knowledge in the search process, and the same information will have different effects on different searchers' knowledge states.
Fifteen years later, Marchionini (\protect\hyperlink{ref-marchionini1995information}{1995}) described information seeking as ``a process, in which humans purposefully engage in order to change
their state of knowledge''.
Thus, we have known for quite a while that search is driven by higher-level human needs, and IR systems are a means to an end, and not the end in itself.
\textbf{Interactive information retrieval} (IIR), a.k.a. human-computer information retrieval (HCIR) (\protect\hyperlink{ref-marchionini2006toward}{Marchionini, 2006}) refers to the study and evaluation of users' interaction with IR systems and users' satisfaction with the retrieved information (\protect\hyperlink{ref-borlund2013interactive}{Borlund, 2013}).

Despite their technological marvels, modern IR systems falls short in several aspects of fully satisfying the higher level human need for information.
In essence, IR systems are software that take, as input, some query, and return as output some ranked list of resources.

\begin{quote}
\emph{Within the context of information seeking, (search engines and IR systems) \textbf{feel} like they play a prominent role in our lives, when in actuality, they only play a small role: the \textbf{retrieval} part of information \ldots{}}

\begin{itemize}
\item
  \emph{Search engines \textbf{don't help us identify what we need} -- that's up to us; search engines don't question what we ask for, though they do recommend queries that use similar words.}
\item
  \emph{Search engines \textbf{don't help us choose a source} -- though they are themselves a source, and a heavily marketed one, so we are certainly compelled to choose search engines over other sources, even when other sources might have better information.}
\item
  \emph{Search engines \textbf{don't help us express our query} accurately or precisely -- though they will help with minor spelling corrections.}
\item
  \emph{Search engines do help retrieve information---this is the primary part that they automate.}
\item
  \emph{Search engines \textbf{don't help us evaluate the answers we retrieve} -- it's up to us to decide whether the results are relevant, credible, true; Google doesn't view those as their responsibility.}
\item
  \emph{Search engines \textbf{don't help us sensemake} -- we have to use our minds to integrate what we've found into our knowledge.}
\end{itemize}

\hfill --- Ko (\protect\hyperlink{ref-ko2021seeking}{2021})
\end{quote}

In recent years, the IIR research community has been actively promoting the \textbf{Search as Learning} (SAL) research direction.
This fast-growing community of researchers propose that search environments should be augmented and reconfigured to foster learning, sensemaking, and long-term knowledge-gain.
Various workshops and seminars have been organized to develop research agendas at the interaction of IIR and the Learning Sciences (\protect\hyperlink{ref-agosti2014evaluation}{Agosti et al., 2014}; \protect\hyperlink{ref-allan2012frontiers}{Allan et al., 2012}; \protect\hyperlink{ref-collins2017search}{Collins-Thompson et al., 2017}; \protect\hyperlink{ref-freund2013searching}{Freund et al., 2013}, \protect\hyperlink{ref-freund2014searching}{2014}; \protect\hyperlink{ref-gwizdka2016search}{Gwizdka et al., 2016}).
Additionally, special issues on Search as Learning have also been published in the \emph{Journal of Information Science} (\protect\hyperlink{ref-hansen2016editorial}{Hansen \& Rieh, 2016}) and in the \emph{Information Retrieval Journal} (\protect\hyperlink{ref-eickhoff2017introduction}{Eickhoff et al., 2017}).
Articles in these special issued presented landmark literature reviews (\protect\hyperlink{ref-rieh2016searching}{Rieh et al., 2016}; \protect\hyperlink{ref-vakkari2016searching}{Vakkari, 2016}), research agendas, and ideas
in this direction.
Overall, these works generally advocate that future research in this domain should aim to:

\begin{itemize}
\tightlist
\item
  understand the contexts in which people search to learn
\item
  understand factors that can influence learning outcomes
\item
  understand how search behaviours can predict learning outcomes
\item
  develop search systems to better support learning and sensemaking
\item
  help searchers be more critical consumers of information
\item
  understand the cognitive biases fostered by existing search systems
\item
  develop search engine ranking algorithms and interface tools that foster long term knowledge gain
\end{itemize}

Parallelly, the Educational Science and the Learning Science research communities have also been organizing workshops and formulating research
agendas to conceptualize forms of `new learning' (\protect\hyperlink{ref-cope2013new}{Cope \& Kalantzis, 2013}; \protect\hyperlink{ref-kalantzis2012newa}{Kalantzis \& Cope, 2012}; \protect\hyperlink{ref-newlondon1996pedagogy}{New London Group, 1996}) that are afforded by innovations in digital technologies and e-learning ecologies (\protect\hyperlink{ref-cope2017elearningc}{Cope \& Kalantzis, 2017}).
Higher education researchers have been increasingly studying how students' information search and information use behaviour affect and support their learning (\protect\hyperlink{ref-weber2018can}{Weber et al., 2018}, \protect\hyperlink{ref-weber2019informationseeking}{2019}; \protect\hyperlink{ref-zlatkin2021students}{Zlatkin-Troitschanskaia et al., 2021}).
Efforts are underway to conceptualize a theoretical framework around new forms of e-Learning that is aided and afforded by digital technologies (\protect\hyperlink{ref-amina2017active}{Amina, 2017}; \protect\hyperlink{ref-cope2017elearningc}{Cope \& Kalantzis, 2017}).
In the community's own words: ``learning today is more about navigation, discernment, induction, and synthesis'' of the wide body of information present ubiquitously at every student's fingertips (\protect\hyperlink{ref-amina2017active}{Amina, 2017}).
Therefore, ``knowing the source, finding the source, and using the information aptly is important to learn and know now more than ever before'' (\protect\hyperlink{ref-cope2013new}{Cope \& Kalantzis, 2013}).
All of these interests in the intersection of searching and learning goes to emphasize that understanding learning during search is critical to
improve human-information interaction.

\hypertarget{sec-intro-problem-statement}{%
\section{Problem Statement}\label{sec-intro-problem-statement}}

A major limitation in the area of Search as Learning, Interactive IR (IIR), and more broadly, in Human-Computer Interaction (HCI) research is
that, the user is examined in the short-term, typically over the course of a single experimental session in a lab
(\protect\hyperlink{ref-karapanos2021advances}{Karapanos et al., 2021}; \protect\hyperlink{ref-kelly2009evaluation}{Kelly et al., 2009}; \protect\hyperlink{ref-koeman2020hciux}{Koeman, 2020}; \protect\hyperlink{ref-zlatkin2021students}{Zlatkin-Troitschanskaia et al., 2021}).
Very few studies exist in the search-as-learning domain that have observed \emph{the same participant} over a longer period of time than a single search session (\protect\hyperlink{ref-kelly2006measuring-a}{Kelly, 2006a}, \protect\hyperlink{ref-kelly2006measuring-b}{2006b}; \protect\hyperlink{ref-kuhlthau2004seeking}{Kuhlthau, 2004}; \protect\hyperlink{ref-vakkari2001changes}{Vakkari, 2001a}; \protect\hyperlink{ref-white2009characterizing}{White et al., 2009}; \protect\hyperlink{ref-wildemuth2004effects}{Wildemuth, 2004}).
This ephemeral approach has acute implications in any domain where learning is involved because ``learning is a \emph{process} that leads to \emph{change} in knowledge \ldots{} (which) unfolds over time'' (\protect\hyperlink{ref-ambrose2010howa}{Ambrose et al., 2010}), and ``\ldots does not happen all at once'' (\protect\hyperlink{ref-white-2016-iwss-learning}{White, 2016b}).

\textbf{To the best of the author's knowledge, almost no new longitudinal studies were reported in major search-as-learning literature in the last five years, that systematically studied students' information search behaviour and information-use over the long term, in their \emph{in-situ} naturalistic environment and contexts, and linked those behaviours quantitatively to the students' learning outcomes and individual differences.}

Higher education students are increasingly using the Internet as their main learning environment and source of information when studying. Yet, the short term nature of research in this domain creates significant gaps in our knowledge regarding how students' information search behaviour and information use develop over time, and how it affects their learning (\protect\hyperlink{ref-zlatkin2021students}{Zlatkin-Troitschanskaia et al., 2021}).

\begin{quote}
\emph{When research in this area relies so heavily on (short-term) lab studies, can we realistically say we are comprehensively studying human-tech interactions -- when many of those interactions take place over long periods of time in real-world contexts? \ldots{} An over-reliance on short studies risks inaccurate findings, potentially resulting in prematurely embracing or disregarding new concepts.}

\hfill --- Koeman (\protect\hyperlink{ref-koeman2020hciux}{2020})
\end{quote}

Current search engines and information retrieval systems ``do not help us know what we want to know, \ldots do not help us know if what we've found is relevant or true; and they do not help us make sense of the retrieved information.
All they do is quickly retrieve what other people on the internet have shared'' (\protect\hyperlink{ref-ko2021seeking}{Ko, 2021}).
Unless we have more long-term understanding of the nature of knowledge gain during search, the limitations of current search systems will continue to persist.
Increased knowledge and understanding of students', and more broadly searchers', information searching and learning behaviour over time will help us to overcome the limitations of current IR systems, and transform them into rich learning spaces where ``search experiences and learning experiences are intertwined and even synergized'' (\protect\hyperlink{ref-url-rieh-homepage}{Rieh, 2020}).
The internet and digital educational technologies offer great opportunities to transform learning and the education experience.
Enabled by our increased comprehension of the longitudinal searching-as-learning process, improved and validated by empirical data, we can create a new wave of fundamentally transformative educational technologies and ``e-learning ecologies, that will be more engaging for learners, more effective (than traditional classroom practices), more resource efficient, and more equitable in the face of learner diversity'' (\protect\hyperlink{ref-cope2017elearningc}{Cope \& Kalantzis, 2017}).

\hypertarget{sec-intro-purpose}{%
\section{Purpose of this Dissertation Proposal}\label{sec-intro-purpose}}

To address the gaps in our knowledge of how information searching influences students' learning process over time, this dissertation proposal proposes to conduct a semester-long longitudinal study (approx. 16 weeks) with university student participants.
The overarching research aim is to identify how students' online searching behaviour correlate with their learning outcomes for a particular university course.
Building upon principles from the Learning Sciences (\protect\hyperlink{ref-ambrose2010howa}{Ambrose et al., 2010}; \protect\hyperlink{ref-council2000how}{National Research Council, 2000}; \protect\hyperlink{ref-novak2010learninga}{Novak, 2010}; \protect\hyperlink{ref-sawyer2005cambridge}{Sawyer, 2005}),
and empirical evidences from the Information Sciences (\protect\hyperlink{ref-rieh2016searching}{Rieh et al., 2016}; \protect\hyperlink{ref-vakkari2016searching}{Vakkari, 2016}; \protect\hyperlink{ref-white2016interactions}{White, 2016a}),
this dissertation proposal aims to:

\begin{itemize}
\tightlist
\item
  situate students as learners in their naturalistic contexts, and characterized by their individual differences
\item
  measure students' information search and information use behaviour over time
\item
  correlate the information search behaviour with the learning outcomes for the university course
\end{itemize}

Learning, or addressing a gap in one's knowledge, has been well established as an important motivator behind information-seeking activities
Section \ref{sec-intro-overview}.
Therefore, search systems that support rapid learning across a number of searchers, and a range of tasks, can be considered as more effective search systems (\protect\hyperlink{ref-white2016interactions}{White, 2016a, p. 310}).
This dissertation proposal takes a step in this direction.
``It opens great expectations for many-sided, great contribution to our knowledge on the relations between search process and learning outcomes'' (anonymous reviewer for \protect\hyperlink{ref-bhattacharya2021longitudinal}{Bhattacharya, 2021}).

\hypertarget{sec-intro-outline}{%
\section{Outline}\label{sec-intro-outline}}

This dissertation proposal document is structured as follows.
First, principles of learning and relevant background from the domain of Educational Sciences are presented in Chapter 2.
Next, relevant empirical evidences from the Information Searching Literature are discussed in Chapter 3.
Chapter 4 presents the research questions, the overarching hypotheses, and discusses their rationale in the context of
the existing research gaps.
Chapter 5 describes the research methods, including the longitudinal study design, experimental procedures, data collection and analyses plans, anticipated limitations, and expected schedule to complete the dissertation.

\hypertarget{ch-bg-learn}{%
\chapter{Background: Knowledge and Learning}\label{ch-bg-learn}}

This first chapter on background literature discusses relevant concepts
from the disciplines of Education and Learning Sciences. First, we
introduce some relevant terminology, and the concepts of deep or
meaningful learning. Then we discuss several research backed principles
that have been shown to lead to meaningful learning. Next, we discuss
how learning, sensemaking, and searching for information are related,
and how modern technologies provide affordances for new forms of
learning and knowledge work in the 21st century. We also discuss some
concepts about individual differences of learners as well as techniques
that can promote better learning. In the last section, we state what
implications these findings have for shaping the proposed study in this
dissertation proposal.

\hypertarget{sec-bg-learn-terminology}{%
\section{Terminology}\label{sec-bg-learn-terminology}}

The Webster dictionary\footnote{\url{https://www.merriam-webster.com/dictionary/knowledge}} defines \textbf{knowledge} in two ways. The first
definition is ``the range of one's information or understanding''.
Vakkari (\protect\hyperlink{ref-vakkari2016searching}{2016}) says it is ``the totality what a person knows,
that is, a \textbf{personal knowledge} or \textbf{belief system}. It may include
both justified, true beliefs and less justified, not so true beliefs,
which the person more or less thinks hold true.'' Webster's second
definition of knowledge is ``the sum of what is known: the body of truth,
information, and principles acquired by humankind''. We can regard this
as \textbf{universal knowledge}.

\textbf{Learning} is a \emph{process}, that leads to a \emph{change} in (personal)
knowledge, beliefs, behaviours, and attitudes (\protect\hyperlink{ref-ambrose2010howa}{Ambrose et al., 2010}). Thus,
learning always aims to increase one's personal knowledge, and can often
draw from the body of universal knowledge. In some cases, the change in
personal knowledge can also lead to change in universal knowledge, such
as when new discoveries are made, or new philosophies are proposed.
Human learning is an innate capacity. It is longitudinal and unfolds
over time. Learning is lifelong and life-wide, and has a lasting impact
on how humans think and act (\protect\hyperlink{ref-ambrose2010howa}{Ambrose et al., 2010}; \protect\hyperlink{ref-kalantzis2012newa}{Kalantzis \& Cope, 2012}).
Learning can be informal or formal. \textbf{Informal learning} is the casual
learning taking place in everyday life, and is incidental to the
everyday life experience. \textbf{Formal learning} is the deliberate,
conscious, systematic, and explicit acquiring of knowledge
(\protect\hyperlink{ref-kalantzis2012newa}{Kalantzis \& Cope, 2012}).

\textbf{Education} is a form of formal learning. It is the systematic
acquiring of knowledge. In today's world, the institutions of education
are formally constructed places (classrooms), times (of the day and of
life) and social relations (teachers and students); for instance,
schools, colleges, and universities. The scientific discipline of
Education concerns itself with the systematic investigation of the ways
in which humans know and learn. It is the science of ``coming to know''
(\protect\hyperlink{ref-kalantzis2012newa}{Kalantzis \& Cope, 2012}).

\textbf{Pedagogy} describes small sequences of learner activities that
promote learning in educational settings (\protect\hyperlink{ref-kalantzis2012newa}{Kalantzis \& Cope, 2012}).
Traditional approaches to (classroom) pedagogy, especially the \emph{didactic
pedagogy}, primarily involves a teacher telling, and a learner
listening. The teacher is in command of the knowledge, and their mission
is to transmit this knowledge to the learners, in a one-way flow. It is
hoped that the learners will dutifully absorb the knowledge laid before
them by the teacher. The balance of agency weighs heavily towards the
teacher. ``There is a special focus on long-term memory, or retention,
measurable by the ritual of closed-book, summative examination''
(\protect\hyperlink{ref-cope2017elearningc}{Cope \& Kalantzis, 2017}).

Cognitive scientists had discovered that learners retain material
better, and are able to generalize and apply it to a broader range of
contexts, when they learn \textbf{deep knowledge} rather than \textbf{surface
knowledge}, and when they learn how to use that knowledge in real-world
social and practical settings (\protect\hyperlink{ref-sawyer2005cambridge}{Sawyer, 2005}). Deep learning \footnote{
  of the human kind}
takes place when ``the learner chooses conscientiously to integrate new
knowledge to knowledge that the learner already possesses'' and involves
``substantive, non-arbitrary incorporations of concepts into cognitive
structure'' (\protect\hyperlink{ref-novak2002meaningful}{Novak, 2002, p. 549}) and may eventually lead to the
development of transferable knowledge and skills. A parallel terminology
for deep learning (\protect\hyperlink{ref-marton1976qualitative-b}{Marton \& Säaljö, 1976}; \protect\hyperlink{ref-marton1976qualitative-a}{Marton \& Säljö, 1976})
is \textbf{meaningful learning}
(\protect\hyperlink{ref-ausubel1968educational}{Ausubel et al., 1968}; \protect\hyperlink{ref-novak2002meaningful}{Novak, 2002}), and they are often
contrasted with \emph{surface learning} or \emph{rote learning}.
Table
discusses some more
details on deep or meaningful learning, and the limitations of
traditional classroom practices to promote deep learning.
Figure
describes (using a concept map) how
meaningful learning can be achieved and sustained, and our annotations
highlight how Search-as-learning systems can foster the same.

\hypertarget{sec-bg-learn-principles}{%
\section{Principles of Meaningful Learning}\label{sec-bg-learn-principles}}

Ambrose et al. (\protect\hyperlink{ref-ambrose2010howa}{2010}) have proposed several principles of (student)
learning that lead to creation of deeper knowledge in learners, and help
educators understand why certain teaching approaches may help or hinder
learning. These principles are based on research and literature from a
range of disciplines in psychology, education, and anthropology, and the
authors claim they are domain independent, experience independent, and
cross-culturally relevant.

\begin{enumerate}
\def\labelenumi{\arabic{enumi}.}
\tightlist
\item
  Students' \textbf{prior knowledge} can help or hinder learning.
\item
  How students \textbf{organize knowledge} influences how they learn and apply what they know.
\item
  Students' \textbf{motivation} determines, directs, and sustains what they do to learn.
\item
  Goal-directed practice coupled with \textbf{targeted feedback} enhances the quality of students' learning.
\item
  Students' current level of development interacts with the social, emotional, and intellectual \textbf{context} around the student to impact learning.
\item
  To become \textbf{self-directed} learners, students must learn to \textbf{monitor and adjust} their approaches to learning.
\end{enumerate}

In line with the above, the US National Research Council identified
several key principles about \textbf{experts' knowledge} (\protect\hyperlink{ref-council2000how}{National Research Council, 2000}),
that illustrate the outcome of successful learning:

\begin{enumerate}
\def\labelenumi{\arabic{enumi}.}
\item
  Experts notice features and \textbf{meaningful patterns} of information
  that are not noticed by novices.
\item
  Experts have acquired a great deal of content knowledge that is
  \textbf{organized} in ways that reflect a deep understanding of their
  subject matter.
\item
  Experts' knowledge cannot be reduced to sets of isolated facts or
  propositions but, instead, reflects contexts of \textbf{applicability}:
  that is, the knowledge is `conditionalized' on a set of
  circumstances.
\item
  Experts are able to \textbf{flexibly retrieve} important aspects of their
  knowledge with little attentional effort.
\item
  Though experts know their disciplines thoroughly, this does not
  guarantee that they are able to teach others.
\item
  Experts have varying levels of flexibility in their approach to new
  situations.
\end{enumerate}

The principles of learning illustrate that both the \emph{context} of
learning, and the \emph{individual differences} of learners moderate the
learning process. The findings about expert knowledge suggests that
\emph{incorporating new information into existing knowledge structures} in a
meaningful manner is a key aspect of learning. We discuss these concepts
in more detail in the following sections.

\hypertarget{sec-bg-learn-sensemaking}{%
\section{Meaningful Learning as Sensemaking}\label{sec-bg-learn-sensemaking}}

In this section, we discuss how meaningful learning can be further
qualified using the concepts of sensemaking.
\textbf{Sensemaking}\footnote{
  ``Brenda Dervin, one of the originators of the sense-making methodology, prefers the spelling with a hyphen, while the community in computer science and more technical people in information science (e.g., SIGCHI) use sensemaking without a hyphen'' (\protect\hyperlink{ref-zhang2014towards}{P. Zhang \& Soergel, 2014}).} is a
process that occurs when learners \emph{connect} their \emph{previously developed}
knowledge, ideas, abilities, and experiences together to address the
uncertainty presented by a newly introduced phenomenon, problem, or
piece of information (\protect\hyperlink{ref-ngss-sensemaking}{Next Generation Science Standards, 2021}). A significant portion of
learning is sensemaking, especially those which use recorded information
or systematic discovery to learn concepts, ideas, theories, and facts in
a domain (such as science or history) (\protect\hyperlink{ref-zhang2014towards}{P. Zhang \& Soergel, 2014}). The phrase
``figure something out'' is often synonymous with sensemaking. Sensemaking
is generally about actively trying to figure out the way the world
works, and/or exploring how to create or alter things to achieve desired
goals (\protect\hyperlink{ref-ngss-sensemaking}{Next Generation Science Standards, 2021}). (\protect\hyperlink{ref-dervin2010sensemaking}{Dervin \& Naumer, 2010}) distinguish work on
sensemaking in four fields: ``Human Computer Interaction (HCI) (Russell's
sensemaking); Cognitive Systems Engineering (Klein's sensemaking);
Organizational Communication (Weick's sensemaking; Kurtz and Snowden's
sense-making); and Library and Information Science (Dervin's
sense-making)''.

Many theories of learning and sensemaking revolve around the concept of
fitting new information into an existing or adapted knowledge structure
(\protect\hyperlink{ref-zhang2014towards}{P. Zhang \& Soergel, 2014}). The central idea is that knowledge is stored in
human memory as \emph{structures} or \emph{schemas}, which comprise interconnected
concepts and relationships. When new information is encountered or
acquired, the learner or sensemaker needs to actively construct a
revised or entirely new knowledge structure. Examples of some such
theories include: the \emph{assimilation theory (theory of meaningful
learning)}
(\protect\hyperlink{ref-ausubel1968educational}{Ausubel et al., 1968}; \protect\hyperlink{ref-ausubel2012acquisition}{Ausubel, 2012}; \protect\hyperlink{ref-novak2002meaningful}{Novak, 2002}; \protect\hyperlink{ref-novak2010learninga}{Novak, 2010});
the \emph{schema theory}
(\protect\hyperlink{ref-rumelhart1981accretion}{Rumelhart \& Norman, 1981}; \protect\hyperlink{ref-rumelhart1977representation}{Rumelhart \& Ortony, 1977}); and the
\emph{generative learning theory}
(\protect\hyperlink{ref-grabowski1996generative}{Grabowski, 1996}; \protect\hyperlink{ref-wittrock1989generative}{Wittrock, 1989}); all of which have
their foundations in the Piagetian concepts of \emph{assimilation} and
\emph{accommodation} (\protect\hyperlink{ref-piaget1936origins}{Piaget, 1936}).

\textbf{Assimilation} means addition of new information into an existing
knowledge structure. A ``synonym'' (\protect\hyperlink{ref-vakkari2016searching}{Vakkari, 2016}) for
assimilation is \textbf{accretion}, which is the gradual addition of factual
information to an existing knowledge structure, without structural
changes. Accretion does not change concepts and their relations in the
structure, but may populate a concept with new instances or facts.
\textbf{Accommodation} means modifying or changing existing knowledge
structures, by adding or removing concepts and their connections in the
knowledge structure. Accommodation is subdivided into \emph{tuning} /
\emph{weak-revision}, and \emph{restructuring}, based on the degree of structural
changes (\protect\hyperlink{ref-zhang2014towards}{P. Zhang \& Soergel, 2014}). \textbf{Tuning} or \textbf{weak revision} does not
include replacing concepts or connections between concepts in the
structure, but tuning of the scope and meaning of concepts and their
connections. This may include, for example, generalizing or specifying a
concept. \textbf{Restructuring} means radically changing and replacing
concepts and their connections in the existing knowledge structure, or
creating of new structures. Such radical changes often take place when
prior knowledge conflicts with new information. New structures are
constructed either to reinterpret old information or to account for new
information (\protect\hyperlink{ref-vakkari2016searching}{Vakkari, 2016}; \protect\hyperlink{ref-zhang2014towards}{P. Zhang \& Soergel, 2014}). A comparison of
these types of conceptual changes can be found in (\protect\hyperlink{ref-zhang2014towards}{P. Zhang \& Soergel, 2014} Table 3).

\hypertarget{sec-bg-concept-maps}{%
\subsection{Concept Maps to enhance Sensemaking}\label{sec-bg-concept-maps}}

As we saw in the previous section, deep learning / meaningful learning /
sensemaking is a process in which new information is connected to a
relevant area of a learner's existing knowledge structure. However, the
\emph{learner must choose} to do this, and must actively seek a way to
integrate the new information with existing relevant information in
their cognitive structure
(\protect\hyperlink{ref-ausubel1968educational}{Ausubel et al., 1968}; \protect\hyperlink{ref-novak2010learninga}{Novak, 2010}). Learning facilitators
(e.g., teachers) can encourage this choice by using the concept mapping
technique.

A \textbf{concept-map} is a two-dimensional, hierarchical node-link diagram
(a \emph{graph} in Computer Science parlance) that depicts the structure of
knowledge within a discipline, as viewed by a student, an instructor, or
an expert in a field or sub-field. The map is composed of concept
labels, each enclosed in a box (graph \emph{nodes}); a series of labelled
linking lines (\emph{labelled edges}); and an inclusive, general-to-specific
organization (\protect\hyperlink{ref-halttunen2005assessing}{Halttunen \& Jarvelin, 2005}). Concept-maps assess how well
students see the `'big picture'', and where there are knowledge-gaps and
misconceptions. A \emph{mind map} is a diagram similar to a concept map,
comprising nodes and links between nodes. However, mind maps emerge from
a single centre, and have a more hierarchical, tree like structure.
Concept maps are more free-form, allowing multiple hubs and clusters.
Also, mind-maps have unlabelled links, and are subjective to the
creator. There are no ``correct'' relationships between nodes in a mind
map. Figure
shows the key features of a concept
map, with the help of a concept map.

\textbf{Concept maps are therefore, arguably the most suited mechanism to
represent the cognitive knowledge structures, connections, and patterns
in a learner's mind}. Conventional tests, such as multiple choice
questions, are best at assessing students' recall of facts and guessing
skills. Their format treats information as distinct and separate items,
rather than interconnected pieces of a bigger picture. Concept maps on
the other hand, encourage learners to identify and make connections
between concepts that they know, and concepts that are new to them.
Concept maps have been used for over 50 years to provide a useful and
visually appealing way of illustrating and assessing learners'
conceptual knowledge
(\protect\hyperlink{ref-egusa2010usingb}{Egusa et al., 2010}, \protect\hyperlink{ref-egusa2014howd}{2014a}, \protect\hyperlink{ref-egusa2014howe}{2014b}, \protect\hyperlink{ref-egusa2017evaluating}{2017}; \protect\hyperlink{ref-halttunen2005assessing}{Halttunen \& Jarvelin, 2005}; \protect\hyperlink{ref-novak2010learninga}{Novak, 2010}; \protect\hyperlink{ref-novak1984learning}{Novak \& Gowin, 1984}).

Analysis of concept maps can reveal interesting patterns of learning and
thinking. Some of these measures that have been used by
(\protect\hyperlink{ref-halttunen2005assessing}{Halttunen \& Jarvelin, 2005}) are: addition, deletion, and differences in
top-level concept-nodes; depths of hierarchy; and number of concepts
that were ignored or changed fundamentally. In this regard,
(\protect\hyperlink{ref-novak1984learning}{Novak \& Gowin, 1984}) have presented well-established scoring schemes to
evaluate concept-maps: 1 point is awarded for each correct relationship
(i.e.~concept--concept linkage); 5 points for each valid level of
hierarchy; 10 points for each valid and significant cross-link; and 1
point for each example.

Having discussed how deep learning / meaningful learning / sensemaking
involves creation of knowledge structures in the learner's mind, and
suitably adding new pieces of information in the knowledge structure, we
now discuss how these processes are influenced in the 21st century with
the presence of new media, digital technologies, and information
retrieval systems.

\hypertarget{sec-bg-learn-active-knowledge-multiliteracy}{%
\section{`New' Learning as Online Information Searching}\label{sec-bg-learn-active-knowledge-multiliteracy}}

Digital media technologies and e-learning `ecologies' can enable new
forms and models of learning, that are fundamentally different from the
traditional classroom practices of didactic pedagogy
(\protect\hyperlink{ref-cope2017elearningc}{Cope \& Kalantzis, 2017}). Some key concepts associated with these forms of
`new learning' are described below. These concepts from the Educational
Sciences domain tie back strongly to the issues, challenges, and
research agenda being investigated by researchers in the Search as
Learning and Information Retrieval domain (Section \ref{sec-intro-overview}.

\hypertarget{sec-bg-learn-active-knowledge-making}{%
\subsection{Active Knowledge Making}\label{sec-bg-learn-active-knowledge-making}}

The Internet and new forms of media provide us the opportunity to create
learning environments where learners are no longer mainly \emph{consumers} of
knowledge, but also \emph{modifiers}, \emph{producers}, and \emph{exchangers} of
knowledge. In \textbf{active knowledge making}, learners can, and often need
to, find information on their own using online resources. They are not
restricted to the textbook alone. The Internet is often a definitive
resource for information on any given topic. A learner can search the
web (to learn) at any time, from anywhere, on any web-enabled device.

As knowledge producers, learners search and analyze multiple sources
with differing and contradictory perspectives, and develop their own
observations and conclusions. In this process, they become researchers
themselves and learn to collaborate with peers in knowledge production.
Collaboration gives learners the opportunity to work with others as
coauthors of knowledge, peer reviewers, and discussants to completed
works. Because learners bring their own views, outlooks, and
experiences, the knowledge artefact they create is often uniquely voiced
instead of a templated ``correct'' response (\protect\hyperlink{ref-amina2017active}{Amina, 2017}).

\begin{quote}
\emph{Learners become \textbf{active knowledge producers} (for instance, project-based learning, using multiple knowledge sources, and research based knowledge making), and not merely knowledge consumers (as exemplified in the `transmission' pedagogies of traditional textbook learning or e-learning focused on video or e-textbook delivery). Active knowledge making practices underpin contemporary emphases on innovation, creativity and problem solving, which are quintessential `knowledge economy' and `knowledge society' attributes.}
\hfill --- Cope \& Kalantzis (\protect\hyperlink{ref-cope2017elearningc}{2017})
\end{quote}

\hypertarget{sec-bg-learn-artefact}{%
\subsection{Artefacts for Learning Assessment}\label{sec-bg-learn-artefact}}

Traditionally, the focus of learning outcomes has been long term memory.
Students and learners were expected to remember a collection of facts,
definitions, proofs, equations, and other associated details. For a
significant amount of modern knowledge-work today, \textbf{memory is actually
less important}. Information is so readily accessible now that it is no
longer necessary to remember the information. Because of the
technological phenomenon, the mass of information is available
ubiquitously \footnote{
  as long as there is internet connection} to a learner (or a knowledge worker), in every moment
of learning. Empirical details such as facts, definitions, proofs, or
equations do not need to be remembered today, because they can always be
looked up again (\protect\hyperlink{ref-amina2017active}{Amina, 2017}; \protect\hyperlink{ref-cope2017elearningc}{Cope \& Kalantzis, 2017}).

This creates an interesting shift in the focus of learning and knowledge
work today: \emph{``if we are not going to measure and value long-term memory in education, what are we going to assess?''}
Cope \& Kalantzis (\protect\hyperlink{ref-cope2017elearningc}{2017}) suggest that \textbf{we assess the knowledge artefacts} that learners
produce. In active knowledge making, the final work \footnote{
  be it a project report, poster, presentation, video, software,research paper, website, etc.} can be proof of
the learning outcome and represent a learner's ability to use the
resources that are available (\protect\hyperlink{ref-amina2017active}{Amina, 2017}). \textbf{Measure of learning can be measure of information quality and information use in artefacts.} This shows a shift in pedagogy and assessment and an
increase in personalization and individualization of learning
(\protect\hyperlink{ref-pea2014learning}{Pea \& Jacks, 2014}). Memorizing the information on a topic is less
important, compared to the writing, synthesizing, analyzing, and
\textbf{sensemaking} of the available information that has been referenced in
the work. This shifts the focus of assessment to the quality of the
artefacts and the processes of their construction. Moreover, as
technology increases the ability to capture detailed data from formal
and informal learning activities, it can give us a new view of how
learners progress in acquiring knowledge, skills, and attributes
(\protect\hyperlink{ref-dicerbo2014impacts}{DiCerbo \& Behrens, 2014}). Because learning is a continuous, longitudinal
process, these advanced, technologically enhanced assessments are more
useful in understanding the learning process and knowledge development
(\protect\hyperlink{ref-amina2017active}{Amina, 2017}).

Assessing open-ended artefacts does come with its challenges and
limitations. First, assessing and grading artefacts requires the
development of detailed qualitative coding guides
(\protect\hyperlink{ref-wilson2013comparison}{M. J. Wilson \& Wilson, 2013}). This process involves defining grading criteria
and measuring inter-coder agreement to ensure that the coding guide is
reliable. Prior studies have scored summaries along dimensions such as
the inclusion of facts, relationships between facts, and evaluative
statements (\protect\hyperlink{ref-lei2015effect}{Lei et al., 2015}; \protect\hyperlink{ref-roy2021note}{Roy et al., 2021}; \protect\hyperlink{ref-wilson2013comparison}{M. J. Wilson \& Wilson, 2013}).
Second, the quality of responses may be difficult to compare across
learners. Since this type of assessment imposes very few constraints on
the learners' responses, it may cause some learners to \emph{satisfice}, and
not convey everything that was learned. Additionally, writing skills are
likely to vary across learners, and some may not be able to effectively
articulate everything that was learnt.

\hypertarget{sec-bg-learn-info-eval}{%
\subsection{`Information Search and Evaluation' as and for Learning}\label{sec-bg-learn-info-eval}}

Learning today is more about \textbf{navigation, discernment, induction, and
synthesis}, and less about memory and deduction (\protect\hyperlink{ref-cope2013new}{Cope \& Kalantzis, 2013}).
However, knowing the source, finding the source, and using the
information critically is important to learn and know now more than ever
before (\protect\hyperlink{ref-amina2017active}{Amina, 2017}). Learners must know the social sources of
knowledge and understand and correctly use quotations, paraphrases,
remixes, links, citations, and the like in the works that they develop.
Searching and sourcing from the web entails a process of developing and
completing a work that inevitably makes learners \textbf{knowledge
producers}, as long as they can navigate and critically discern the
value of multiple sources. This is a skill that must be learned, as many
sources of information are not valid, reliable, or authentic
(\protect\hyperlink{ref-mcgrew2018can}{McGrew et al., 2018}; \protect\hyperlink{ref-wineburg2016students}{Wineburg \& McGrew, 2016}). Understanding the different
sources and identifying the more reliable ones are essential for
effective teaching and learning
(\protect\hyperlink{ref-mcgrew2017challenge}{McGrew et al., 2017}; \protect\hyperlink{ref-mcgrew2021skipping}{McGrew, 2021}). This is a critical aspect
because the inability to cite properly or to use reliable resources
provides learners with misconstrued information and ideas
(\protect\hyperlink{ref-amina2017active}{Amina, 2017}; \protect\hyperlink{ref-breakstone2021students}{Breakstone et al., 2021}; \protect\hyperlink{ref-mcgrew2017challenge}{McGrew et al., 2017}).

The Stanford History Education Group (SHEG) conceptualised the \textbf{Civic
Online Reasoning} (COR) curriculum \footnote{\url{https://cor.stanford.edu}} to enable students to
effectively search for and evaluate online information
(\protect\hyperlink{ref-breakstone2018we}{Breakstone et al., 2018}; \protect\hyperlink{ref-breakstone2021students}{Breakstone et al., 2021}; \protect\hyperlink{ref-mcgrew2020learning}{McGrew, 2020}). The
curriculum centres on asking three questions of any digital content:
\emph{(i)} who is behind a piece of information? \emph{(ii)} what is the evidence
for a claim? \emph{(iii)} what do other sources say? The curriculum has
lessons and assessments for information evaluation skills such as
lateral reading (\protect\hyperlink{ref-wineburg2017lateral}{Wineburg \& McGrew, 2017}), identifying news versus
opinions, checking domain names, identifying sponsored content,
evaluating evidence, and practising click restraint (\protect\hyperlink{ref-mcgrew2021click}{McGrew \& Glass, 2021}).
The lessons were developed and piloted by the Stanford History Education
Group (\protect\hyperlink{ref-mcgrew2018can}{McGrew et al., 2018}; \protect\hyperlink{ref-mcgrew2020learning}{McGrew, 2020}; \protect\hyperlink{ref-mcgrew2021click}{McGrew \& Glass, 2021}). Taken
together, these strategies will allow academics and students to better
evaluate digital content, from the perspectives of professional fact
checkers.

The purview of the \emph{Civic Online Reasoning} curriculum is more targeted
than the expansive fields of media and digital literacy \footnote{
  \emph{``Digital literacy describes a holistic approach to cultivating skills that allow people to participate meaningfully in online communities, interpret the changing digital landscape, understand the relationships between systemic -isms and information, and unlock the power of digital tools for good. This includes media literacy. Terms like critical media literacy, media literacy, news literacy, and more are not necessarily interchangeable.''} -- Collins (\protect\hyperlink{ref-collins2021reimagining}{2021})}, (which can embrace topics ranging from cyberbullying to identity theft).
Civic Online Reasoning focuses squarely on how to sort fact from fiction
online, a prerequisite for responsible civic engagement in the
twenty-first century (\protect\hyperlink{ref-breakstone2021students}{Breakstone et al., 2021}; \protect\hyperlink{ref-kahne2012digital}{Kahne et al., 2012}; \protect\hyperlink{ref-mihailidis2013media}{Mihailidis \& Thevenin, 2013}).

\hypertarget{sec-bg-learn-promoting-learning}{%
\section{Promoting Better Learning}\label{sec-bg-learn-promoting-learning}}

\begin{quote}
\emph{It is not the technology that makes a difference; it is the pedagogy.}

\hfill --- Cope \& Kalantzis (\protect\hyperlink{ref-cope2017elearningc}{2017})
\end{quote}

Having discussed how meaningful learning takes place, and how it is
influenced by the presence of digital media and the mass of information
on the Internet, let us now look deeper into the learners as persons
themselves. In this section, we discuss how different cognitive and
metacognitive practices and aspects of learners can promote better
learning. These phenomena have important implications for any digital
systems that aim to foster learning.

\hypertarget{sec-bg-learn-articulation}{%
\subsection{Externalization and Articulation}\label{sec-bg-learn-articulation}}

The learning sciences have discovered that when learners externalize and
articulate their developing knowledge, they learn more effectively
(\protect\hyperlink{ref-council2000how}{National Research Council, 2000}). Best learning takes place when learners articulate
their unformed and still developing understanding, and continue to
articulate it throughout the process of learning. This phenomenon was
first studied in the 1920s by Russian psychologist Lev Vygotsky.
Articulating and learning go hand in hand, in a mutually reinforcing
feedback loop. Often learners do not actually learn something until they
start to articulate it. While thinking out loud, they learn more rapidly
and deeply than while studying quietly (\protect\hyperlink{ref-sawyer2005cambridge}{Sawyer, 2005}). The
learning sciences community is actively researching how to support
students in their ongoing process of articulation, and which forms of
articulation are the most beneficial to learning. Articulation is more
effective if it is scaffolded -- channelled so that certain kinds of
knowledge are articulated, and in a certain form that is most likely to
result in useful reflection (\protect\hyperlink{ref-sawyer2005cambridge}{Sawyer, 2005}). Students need help
in articulating their developing understandings, as they do not yet know
how to think about thinking, or talk about thinking; their knowledge
state is \emph{anomalous} (\protect\hyperlink{ref-belkin1982ask}{Belkin et al., 1982}).

\hypertarget{sec-bg-learn-metacognition}{%
\subsection{Metacognition and Reflection}\label{sec-bg-learn-metacognition}}

One of the reasons that articulation is so helpful to learning is that
it promotes \emph{reflection} or \emph{metacognition}. \textbf{Metacognition}, commonly
referred to as thinking about thinking, involves thinking at a higher
level of abstraction, which in turn improves thinking and learning
(\protect\hyperlink{ref-blanken2017metacognition}{Blanken-Webb, 2017}). It is ``the process of reflecting on and
directing one's own thinking'' (\protect\hyperlink{ref-council2000how}{National Research Council, 2000, p. 78}), and involves
thinking about the process of learning, and thinking about knowledge.
This ties forward to the self-regulation that effective learners exhibit
(Section \ref{sec-bg-learn-self-regulation}). Effective learners are aware
of their learning process, and can measure how efficiently they are
learning as they study.

The literature on metacognition broadly identifies two fundamental
components of metacognition: knowledge about cognition, and regulation
of cognition. \textbf{Knowledge about cognition} includes three subprocesses
that facilitate the \emph{reflective} aspect of metacognition: declarative
knowledge (knowledge about self and about strategies), procedural
knowledge (knowledge about how to use strategies), and conditional
knowledge (knowledge about when and why to use strategies). \textbf{Regulation
of cognition} include a number of subprocesses that facilitate the
\emph{control} aspect of learning. Five component skills of regulation have
been discussed extensively in the literature, including planning,
information management strategies, comprehension monitoring, debugging
strategies, and evaluation. The operational definitions of these
components are described in Table
.
Schraw \& Dennison (\protect\hyperlink{ref-schraw1994assessing}{1994})
developed the \textbf{Metacognitive Awareness Inventory} (MAI) survey and a
scoring guide to measure these self-reported components and subprocesses
of metacognition. The original survey consists of 52 true/false
questions (Appendix \ref{app-mai}), such as ``\emph{I consider several alternatives to a problem before I answer}'', ``\emph{I understand my intellectual strengths and weaknesses}'', ``\emph{I have control over how well I learn'', and ''I change strategies when I fail to understand}''.
The instrument has been widely used in research, and has its reliability and validity measures available. Later, Terlecki \& McMahon (\protect\hyperlink{ref-terlecki2018call}{2018}) proposed a revised version of the MAI, using five-point Likert-scales, ranging from ``\emph{I never do this}'' to ``\emph{I do this always}''.
They argue that when measuring change in metacognition over time, the Likert-scale based `how often' questions are more effective than dichotomous `Yes/No' questions (\protect\hyperlink{ref-terlecki2020revising}{Terlecki, 2020}; \protect\hyperlink{ref-terlecki2018call}{Terlecki \& McMahon, 2018}).

\hypertarget{motivation}{%
\subsection{Motivation}\label{motivation}}

\textbf{Motivation} is the process that initiates, guides, and maintains
goal-oriented behaviours (\protect\hyperlink{ref-cherry2020what}{Cherry, 2020}). The \textbf{Self-Determination
Theory} (SDT) represents a broad framework for the study of human
motivation and personality (\protect\hyperlink{ref-ryan2017self}{Ryan \& Deci, 2017}). SDT differentiates the types
of motivation based on the reasons that give rise to behaviour:
intrinsic motivation and extrinsic motivation. \textbf{Intrinsic motivation}
is engaging in a task or behaviour for the rewards \emph{inside} the task or
behaviour, such the pleasure, enjoyment and satisfaction that the
behaviour provides. It is a stable form of motivation. \textbf{Extrinsic
motivation} is engaging in a task or behaviour for the rewards
\emph{outside} the task or behaviour, such as receiving rewards, avoidance of
punishment, gaining social approval, or achievement of a valued result.
Extrinsic motivation is on a continuum from less stable to more stable,
as illustrated in Figure \ref{fig-sdt-components}.
Extrinsic motivation does not last
unless the rewards and punishments are explicitly visible
(\protect\hyperlink{ref-deci2013intrinsic}{Deci \& Ryan, 2013}; \protect\hyperlink{ref-ryan2000self}{Ryan \& Deci, 2000}; \protect\hyperlink{ref-tahamtan2019effect}{Tahamtan, 2019}).

(\protect\hyperlink{ref-ryan1982control}{Ryan, 1982}) proposed the \textbf{Intrinsic Motivation Inventory} (IMI)
(Appendix \ref{app-imi}, a multidimensional questionnaire intended to
assess participants' subjective experience related to a target activity
in laboratory experiments. The instrument assesses participants'
interest/enjoyment, perceived competence, effort, value/usefulness, felt
pressure and tension, and perceived choice while performing a given
activity, yielding six subscale scores. The \emph{interest/enjoyment}
subscale is considered the most indicative self-report measure of
intrinsic motivation. The \emph{perceived choice} and \emph{perceived competence}
concepts are theorized to be positive predictors of both self-report and
behavioral measures of intrinsic motivation. The \emph{pressure/tension} is
theorized to be a negative predictor of intrinsic motivation. \emph{Effort}
is a separate variable that is relevant to some motivation questions, so
it is used if its relevant. The \emph{value/usefulness} subscale is used to
measure internalization, with the idea being that people internalize and
become self-regulating with respect to activities that they experience
as useful or valuable for themselves.

\hypertarget{sec-bg-learn-self-regulation}{%
\subsection{Self-regulation}\label{sec-bg-learn-self-regulation}}

\textbf{Self-regulation} is the ability to develop, implement, and flexibly
maintain planned behaviour in order to achieve one's goals.
Self-regulation, and more broadly, self-direction, are critical to being
an effective ``lifelong'' learner. Self-regulation becomes increasingly
important at higher levels of education and in professional life, as
people take on more complex tasks and greater responsibilities for their
own learning. However, these metacognitive skills tend to fall outside
the content area of most courses, and therefore, often neglected in
instruction (\protect\hyperlink{ref-ambrose2010howa}{Ambrose et al., 2010, p. 191}). Building on the foundational work
of (\protect\hyperlink{ref-kanfer1970self-a}{Kanfer, 1970b}, \protect\hyperlink{ref-kanfer1970self-b}{1970a}), Miller and Brown formulated a
seven-step model of self-regulation (\protect\hyperlink{ref-brown1998self}{J. Brown, 1998}; \protect\hyperlink{ref-miller1991self}{W. R. Miller \& Brown, 1991}).
In this model, behavioural self-regulation may falter because of failure
or deficits at any of these seven steps: \emph{(i)} receiving relevant
information, \emph{(ii)} evaluating the information and comparing it to
norms, \emph{(iiii)} triggering change, \emph{(iv)} searching for options, \emph{(v)}
formulating a plan, \emph{(vi)} implementing the plan, and \emph{(vii)} assessing
the plan's effectiveness (which recycles to steps \emph{(i)} and \emph{(ii)}).
Although this model was developed specifically to study addictive
behaviours, the self-regulatory processes it describes are meant to be
general principles of behavioural self-control. (\protect\hyperlink{ref-brown1999self}{J. M. Brown et al., 1999})
developed the \textbf{Self-Regulation Questionnaire} (SRQ) (Appendix
\ref{app-srq}) to
assess these self-regulatory processes through self-report. The items
were developed to mark each of the seven sub-processes of the
(\protect\hyperlink{ref-miller1991self}{W. R. Miller \& Brown, 1991}) model, forming seven subscales of the SRQ. The 63-item
scale elicits responses in the form of 5-point Likert scale, ranging
from strongly disagree to strongly agree. Based on clinical and college
samples, the authors tentatively recommend a score of 239 and above as
high (intact) self-regulation capacity (top quartile), 214-238 as
intermediate (moderate) self-regulation capacity (middle quartiles), and
213 and below as low (impaired) self-regulation capacity (bottom
quartile).

\hypertarget{self-directed-and-self-regulated-learning}{%
\subsubsection{Self-directed and Self-regulated Learning}\label{self-directed-and-self-regulated-learning}}

As we saw in the previous sections, self-regulation, motivation, and
metacognition are key concepts that moderate the learning process. These
terms are couched in the concepts of self-regulated learning and
self-directed learning.

\textbf{Self-directed learning} (SDL) is a ``process in which individuals take
the initiative, with or without the help from others, in diagnosing
their learning needs, formulating goals, identifying human and material
resources, choosing and implementing appropriate learning strategies,
and evaluating learning outcomes''(\protect\hyperlink{ref-knowles1975self}{Knowles, 1975, p. 18}).
\textbf{Self-regulated learning} (SRL) can be described as the degree to
which students are ``metacognitively, motivationally, and behaviourally
active participants in their own learning process'' (\protect\hyperlink{ref-zimmerman1989social}{Zimmerman, 1989, p. 329}).

Often used interchangeably, self-directed learning (SDL) and
self-regulated learning (SRL) have some important similarities and
differences (Figure \ref{fig-sdl-v-srl}) (\protect\hyperlink{ref-saks2014distinguishing}{Saks \& Leijen, 2014}). SDL, originating
from adult education, is a broader, macro-level construct, and is
usually practised outside the traditional school environment. The
self-directed learner is free to design their own learning environment,
and free to plan and set their own learning goals. SRL, on the other
hand, is a narrower, micro-level construct, originating from educational
and cognitive psychology, and is mostly utilized in the school
environment. Learners do not have as much freedom as in SDL. The
instructor or facilitator often defines the learning task and the
learning goals. Self-directed learning may include self-regulated
learning, but the converse is not true
(\protect\hyperlink{ref-jossberger2010challenge}{Jossberger et al., 2010}; \protect\hyperlink{ref-loyens2008selfdirected}{Loyens et al., 2008}). In other words, \emph{``a self-directed learner is supposed to self-regulate, but a self-regulated learner may not self-direct''} (\protect\hyperlink{ref-saks2014distinguishing}{Saks \& Leijen, 2014}). Despite their
differences, SDL and SRL share key similarities
(\protect\hyperlink{ref-saks2014distinguishing}{Saks \& Leijen, 2014}).
First, both can be seen in two dimensions:
\emph{(i)} \emph{external} to the learner, as a process or series of events, and
\emph{(ii)} \emph{internal} to the learner, arising from the learner's personality, aptitude, and individual differences.
Second, both the
learning processes have four key phases:
\emph{(i)} defining tasks,
\emph{(ii)} setting goals and planning,
\emph{(iii)} enacting strategies, and
\emph{(iv)} monitoring and reflecting.
Third, both SDL and SRL require active
participation, goal-directed behaviour, metacognition, and intrinsic
motivation.

In summary, metacognition is monitoring and controlling what is in the
learner's head; self-regulation is monitoring and controlling how the
learner interacts with their environment; self-regulated learning is the
application of metacognition and self-regulation to learning
(\protect\hyperlink{ref-mannion2020metacognition}{Mannion, 2020}); and the whole learning process is sustained
by motivation, which is desirable to be intrinsic.

\hypertarget{sec-bg-learn-summary}{%
\section{Summary and Implications for this Proposal}\label{sec-bg-learn-summary}}

In this first chapter of the background literature review, we discussed
\emph{(i)} what is meaningful learning, a.k.a. deep learning, or sensemaking;
\emph{(ii)} how meaningful learning updates the learner's cognitive knowledge
structure; \emph{(iii)} how the learning process is influenced by digital
technologies, mass of information on the Internet, and IR systems; and
\emph{(iv)} what principles and practices learners and educators must realize
and follow to promote meaningful learning. These findings are from the
domains of Educational Sciences, Learning Sciences and Cognitive
Sciences. We argue that these are important aspects to be considered
when designing future IR or educational information systems that aim to
combine and improve the searching and learning experience.

Guided by these findings, we make some important decision choices for
the proposed longitudinal study in this dissertation proposal. We aim to
situate learners in their context, and incorporate their individual
differences using metacognition, motivation, and self-regulation
characteristics. Additionally, we aim to assess learning using artefacts
and concept maps. We choose not use traditional tests like
question-answers, and multiple choice assignments, since they are often
not the preferred choice of knowledge-work output in real world
scenarios. Concept maps are better suited to represent the learning and
sensemaking process, and artefacts are better able to demonstrate a
learner's knowledge work.

In the next chapter, we look at relevant literature from the Information
Sciences and Interactive Information Retrieval disciplines.

\hypertarget{ch-bg-search}{%
\chapter{Background: Information Searching}\label{ch-bg-search}}

This second chapter on background literature discusses relevant concepts
from the disciplines of Information Sciences, and more specifically
Interaction Information Retrieval. First, we introduce some terminology
around information behaviour, information need, and information
relevance. Then we discuss relevant findings various empirical studies,
from the lens of three-stage interactions in the information search
process. Then we discuss some overall generic characteristics of
information search behaviour, and how they are linked to expertise and
working memory. Next we discuss how learning has been assessed in recent
search-as-learning studies. We also discuss some limitations of current
search systems to foster learning, including the lack of sufficient
number of longitudinal studies. In the last section, we state what
implications these findings have for shaping the proposed study in this
dissertation proposal.

\hypertarget{sec-bg-search-terminology}{%
\section{Terminology}\label{sec-bg-search-terminology}}

\textbf{Information retrieval} (IR) is the process of obtaining \emph{information
objects}, that are \emph{relevant} to an \emph{information need}, from a
collection of those objects (Wikipedia). \textbf{Information objects} are
entities that can potentially convey information. They can take many
forms, such as documents, webpages, facts, music, spoken words, images,
videos, artefacts, and other forms of human expression. Areas where
information retrieval techniques are employed include search engines,
such as web search, social search, and desktop search; media search, as
in image, music, video; digital libraries and recommender systems, as
well as domain specific applications like geographical information
systems, e-Commerce websites, legal information search, and others.

Multiple perspectives exist around how users interact with information,
and IR systems. In the \textbf{Search Engine application view}, the
interactions are restricted to the search engine interface. In the
\textbf{Human-computer interaction} (HCI) view, interactions are between a
person and a system; but the system can go \emph{beyond} supporting only
retrieval, to supporting more complex tasks. In the \textbf{cognitive view of
IR}, which is the broadest, the interactions for obtaining information
can be between a person and a system, as well as between people, for
retrieval of information.

People's behaviour around information can be modelled as a nested Venn
diagram as proposed by (\protect\hyperlink{ref-wilson1999models}{T. D. Wilson, 1999}) (Figure
\ref{fig-wilson-info-behaviour}). \textbf{Information behaviour} is
the more general field of investigation. \textbf{Information-seeking
behaviour} can be seen as a sub-set of the field, particularly
concerned with the variety of methods people employ to discover, and
gain access to information objects. \textbf{Information search behaviour} is
yet a sub-set of information-seeking, concerned with the interactions
between the user and computer-based information systems. In this
dissertation, we focus on information search rather than the other two
higher hierarchical concepts. This is because online IR systems, such as
search engines or digital libraries, have become the primary source for
people to obtain information in modern times, and web search is becoming
ever more pervasive and ubiquitious in our day to day lives.

The field of \textbf{interactive information retrieval} (IIR) posits that IR
systems should operate in the way that good libraries do. Good libraries
provide both the information a visitor needs, as well as a \emph{partner} in
the learning process --- the information professional --- to navigate
that information, make sense of it, preserve it, and turn it into
knowledge. As early as in 1980, Bertram Brookes stated that searchers
acquire new knowledge in the information seeking process
(\protect\hyperlink{ref-brookes1980foundations}{Brookes, 1980}). Fifteen years later, Gary Marchionini
described information seeking, as \emph{``a process, in which humans
purposefully engage in order to change their state of knowledge''}
(\protect\hyperlink{ref-marchionini1995information}{Marchionini, 1995}). So we have known for quite a while that
search is driven by the higher-level human need to gain knowledge.
Information Retrieval is thus a means to an end, and not the end in
itself. Thus, the ideal IR system should not only help users to locate
information, but also help them to \textbf{bridge the gap between information
and knowledge}.

This brings us to the concept of information need. \textbf{Information Need}
is the desire to locate and obtain information to satisfy a conscious or
unconscious human need. Most search systems of today assume that the
search query is an accurate representation of a user's information need.
However, (\protect\hyperlink{ref-belkin1982ask}{Belkin et al., 1982}) observed that in many cases, users of search
systems are unable to precisely formulate what they need. They miss some
vital knowledge to formulate their queries. As humans, we have
difficulty in asking questions about what we do not know. Belkin called
this phenomenon as \textbf{Anomalous State of Knowledge}, or ASK. Later,
(\protect\hyperlink{ref-huang2013relevance}{Huang \& Soergel, 2013}) identified an exhaustive set of criteria that
should be considered in order to ideally represent a user's information
need. These criteria for information need are highly dependent on the
user context: user attributes, tasks or goals, as well as the situation
the user is embedded in. This brings us to another closely related
concept: information relevance.

\textbf{Relevance} is a fundamental concept of Information Science and
Information Retrieval, and perhaps the most celebrated work in this area
has been done by Tefko Saracevic
(\protect\hyperlink{ref-saracevic1975relevance}{Saracevic, 1975}, \protect\hyperlink{ref-saracevic2007relevance}{2007a}, \protect\hyperlink{ref-saracevic2007relevancea}{2007b}, \protect\hyperlink{ref-saracevic2016notion}{2016}).
Webster dictionary define relevance as ``a relation to the matter at
hand''. In most circumstances, relevance is a ``y'know'' notion. People
apply it effortlessly, without anybody having to define for them what
``relevance'' is. This creates one of the most fascinating challenges in
the information field: humans understand relevance intuitively, while it
is an open research problem to represent relevance effectively for use
by algorithmic systems. The situation becomes more interesting because
relevance always depends on context, and the context is ever dynamic, as
the matter at hand changes.

\hypertarget{sec-bg-search-3-stage}{%
\section{Three-stage Interactions with Online Search Systems}\label{sec-bg-search-3-stage}}

As we saw in the previous section, information search behaviour is the
(study of) interactions between a user, and digital Information
Retrieval (IR) systems. The field of Information Science/Studies has
developed multiple models explaining how information search works
(\protect\hyperlink{ref-wilson1999models}{T. D. Wilson, 1999}). A few of them are presented in Figure
\ref{fig-info-search-models}. Across many of these models, we
observe that most major Information Retrieval (IR) systems have three
fundamental ways of letting users interact with the system, and the
underlying information: \emph{(1)} an interface for entering search
\textbf{queries}; \emph{(2)} an interface for viewing and evaluating a \textbf{list} of
retrieved information-objects, or search results; \emph{(3)} an interface for
viewing and evaluating \textbf{individual information-objects}. For instance,
(\protect\hyperlink{ref-marchionini1995information}{Marchionini, 1995})'s ISP model hints at these three
interfaces in the fourth, sixth and seventh stages, namely ``formulate
query'', ``examine results'', and ``extract info''. (\protect\hyperlink{ref-spink1997study}{Spink, 1997})'s model
of the IR interaction process consists of sequential steps or cycles,
and each cycle comprises one or more interactive feedback occurrences of
user input (query), IR system output (list), and user interpretation and
judgement (of individual information-objects). Consequently, findings
from the large body of empirical research in interactive IR (especially
those with web based search systems) can be grouped around these thee
stages of interactions with search systems:

\begin{enumerate}
\def\labelenumi{\arabic{enumi}.}
\tightlist
\item
  \emph{Stage 1:} search query formulation / reformulation
\item
  \emph{Stage 2:} search results evaluation (or source selection)
\item
  \emph{Stage 3:} content page evaluation (or, interacting with sources)
\end{enumerate}

The discussions in the following subsections are based around these
three stages of interactions. The empirical studies discussed below
generally follow some common principles of user studies in Interactive
IR (IIR) (\protect\hyperlink{ref-borlund2013interactive}{Borlund, 2013}; \protect\hyperlink{ref-kelly2009methods}{Kelly, 2009}): participants are
presented with a search task or search topic, and then they are asked to
search the internet (or a simulation of the open web) for information.
During the search, the various interactions (queries, clicks, webpages
opened etc.) are recorded, and these are analysed and correlated with
other sources of data to answer research questions.

\hypertarget{sec-bg-search-query}{%
\subsection{Stage 1: Query (Re)formulation}\label{sec-bg-search-query}}

\emph{How do users behave when submitting search queries (to an IR system)?}

\textbf{Query formulation} is the process of composing a search query that
describes the information need of a searcher. \textbf{Query reformulation}
refers to the act of either modifying a previous query, or creating a
new query. Query reformulation typically occurs due to a searcher's
improved understanding of how to better translate their information need
into a search query. The relationship between two successively issued
queries have been classified in a number of ways. These classifications
are called \emph{Query Reformulation Types}, or QRTs. Amongst many other,
Boldi et al. (\protect\hyperlink{ref-boldi2009dango}{2009}) used cognitive aspects of the searchers issuing the
query to propose a taxonomy of QRTs, while C. Liu et al. (\protect\hyperlink{ref-liu2010analysis}{2010}) proposed a
similar taxonomy focusing more on the linguistic properties of the two
successive queries. These are compared and contrasted in
Table~\ref{tab-res-Q-QRT-txnmy}.

Task-type, task-topic, task-goal, and domain-expertise were found to
influence query reformulation patterns of searchers (\protect\hyperlink{ref-127}{Eickhoff et al., 2015}; \protect\hyperlink{ref-126}{Jiang et al., 2014}; \protect\hyperlink{ref-133}{Mao et al., 2018}).
At first glance, a significant portion of the query reformulation terms
(\(\sim86\%\)) seemed to be coming from the task-description itself
(\protect\hyperlink{ref-126}{Jiang et al., 2014}; \protect\hyperlink{ref-133}{Mao et al., 2018}). This was characterized by significantly more fixations on
the task-description, rather than other SERP elements. Jiang et al. (\protect\hyperlink{ref-126}{2014}) and Mao et al. (\protect\hyperlink{ref-133}{2018})
investigated this phenomenon further. Jiang et al. (\protect\hyperlink{ref-126}{2014}) controlled for the task-type
and task-goal, using the faceted-framework by Li \& Belkin (\protect\hyperlink{ref-li2008faceted}{2008}). Mao et al. (\protect\hyperlink{ref-133}{2018})
controlled for the task-topic and the domain-expertise of the searchers.

If search tasks had \emph{factual} goals, searchers relied heavily on the
task-description for reformulating their queries (\protect\hyperlink{ref-126}{Jiang et al., 2014}). For
\emph{interpretive} tasks (intellectual tasks with specific goals), users
spent more time reading search result surrogates, before reformulating
their queries. This was observed by increased eye-fixations (indicative
of visual attention) and dwell time on search result snippets
(surrogates). For exploratory tasks, searchers fixated the longest on
query-autocompletion (QAC) suggestions, indicating that they were
possibly looking for help and suggestion based on their specific query,
as the search-task had non-specific (amorphous) goals.

Searchers also relied on the task-description for reformulating queries,
when the search-task was outside their domain of expertise (\protect\hyperlink{ref-133}{Mao et al., 2018}). For
in-domain tasks, they used query terms from their own knowledge, that
were not fixated on in visited SERPs and content pages. Eickhoff et al. (\protect\hyperlink{ref-127}{2015}) reported
that a significant share of new query terms came from visited SERPs and
content pages, and query reformulation (specialization) often did not
literally re-use previously encountered terms, but highly related
ones \footnote{
  measured using Leacock-Chodorow semantic similarity metric (\protect\hyperlink{ref-leacock1998combining}{Leacock \& Chodorow, 1998})} instead. These observations can possibly be explained by Mao et al. (\protect\hyperlink{ref-133}{2018})'s
findings: when exploring a new domain, the searcher may accumulate
vocabulary and learn how to query during the search; when performing
in-domain search-tasks, the searcher may have enough prior knowledge to
come up with effective query terms. It was also seen that searchers from
medicine domain used more unread query terms for their in-domain
search-tasks, compared to politics and environment domains (\protect\hyperlink{ref-133}{Mao et al., 2018}). This
suggested that domain knowledge and expertise is more important for
formulating good search queries in highly technical disciplines (e.g.,
medicine), compared to less technical domains (e.g., politics).

\textbf{Query Auto Completion (QAC)} is a technological feature that suggests
possible queries to web search users from the moment they start typing a
query. It is nearly ubiquitous in modern search systems, and is thought
to reduce physical and cognitive effort when formulating a query. QAC
suggestions are usually displayed as a list
(Figure~\ref{fig-int-Q}(b)
and (c)), and users interact in a variety of ways with the list. Hofmann et al. (\protect\hyperlink{ref-125}{2014})
observed a strong position bias among searchers who examined the QAC
list: the top suggestions received the highest visual attention, even
when the ordering of the suggestions were randomized. Average fixation
time decreased consistently on suggested items from top to bottom. Even
when the ranking of suggestions were randomized, time taken to formulate
queries did not significantly differ.

Search topics were found to have a large effect on QAC usage
(\protect\hyperlink{ref-126}{Jiang et al., 2014}; \protect\hyperlink{ref-129}{Smith et al., 2016}). Search was easiest for the topics with the highest QAC
usage. Total eye-gaze duration was longest when visual attention was
shared between the QAC suggestions and the actual search query input
box. Some additional time was probably due to decision making on whether
to use a QAC suggestion. Typing was faster when a QAC was not used.
However, the IR system's retrieval performance (measured using \href{mailto:NDCG@3}{\nolinkurl{NDCG@3}}),
was greater when QAC was used. So Smith et al. (\protect\hyperlink{ref-129}{2016}) speculated that the value of
using QAC suggestions was realized later in the search session by users,
when they saw a reduction in the number of additional queries needed, or
an increase in the value of the information found.

Several user behavioural profiles were identified by exploring
associations between visual attention from eye-tracking, search
interactions from mouse and keyboard activity, and the use of QAC
suggestions (\protect\hyperlink{ref-125}{Hofmann et al., 2014}; \protect\hyperlink{ref-129}{Smith et al., 2016}). These profiles are described in
Table~\ref{tab-res-Q-QAC-profiles}. An interesting, yet common-sense
observation was that participants' touch-typing ability greatly
influenced their interactions with QAC suggestions.

The native language of searchers was found to influence their overall
querying and searching behaviour. Ling et al. (\protect\hyperlink{ref-132}{2018}) explored this space using four
variations of a multi-lingual search interface. They observed that
participants strongly preferred to issue queries in their first or
native language. A second or non-native language was the next preferred
choice. Mixing of first and second-languages occurred very rarely. In
80\% of the total 300 tasks (25 users \(\times\) 4 interfaces \(\times\) 3
task-types), participants used a single language for querying. In the
rest 20\% of the tasks, participants switched languages for querying,
with a transition from first language to second language being the most
common.

\hypertarget{sec-bg-search-list}{%
\subsection{Stage 2: Search Results Evaluation / List Item Selection}\label{sec-bg-search-list}}

\emph{How do users behave when examining a list of information-objects
(returned by an IR system)?}

After a user submits a query to an IR system, the next action they
generally perform is examining and evaluating the list of search results
returned by the IR system. In this section, we discuss empirical studies
which investigated information-searching behaviour around a list of
information-objects, or a representation of information-objects (also
called \emph{surrogates}). We identified some common themes in the research
questions investigated. The discussion below is grouped along these
themes, as relationships between search behaviour and: \emph{(i)} ranking of
search results; \emph{(ii)} information shown in search results; \emph{(iii)}
individual user characteristics; and \emph{(iv)} relevance judgement and
feedback.

\hypertarget{ranking-of-search-results}{%
\subsubsection{Ranking of search results}\label{ranking-of-search-results}}

Most search engines display results in a rank ordered list, with the
highest \emph{algorithmically} relevant results placed at the top, and others
results ordered below. Granka et al. (\protect\hyperlink{ref-101}{2004}; \protect\hyperlink{ref-108}{Lorigo et al., 2008}) studied eye-movement behaviour of
searchers examining SERPs, and reported observations from three user
studies. They saw that in 96\% of the queries, participants looked at
only the first result page, containing the top 10 results. No
participant looked beyond the third result page for a given query.
Participants looked primarily at the first few results, with nearly
equal attention (dwell time) given to the first and the second results.
However, despite equal attention, the first result was clicked 42\% of
the time, while the second was clicked only 8\% of the time. If none of
the top three results appeared to be relevant, then users chose not to
explore further results, but issued a reformulated query instead. When
the ranking of the search results were reversed (i.e.~placing less
relevant results in the higher ranked positions), participants spent
considerably more time scrutinizing and comparing results (more
fixations and regressions) before making a decision to click or
reformulate.

Some effects of gender were found to influence SERP examination (\protect\hyperlink{ref-108}{Lorigo et al., 2008}).
Females clicked on the second result twice as often, and made more
regressions or repeat viewings of already visited abstracts, compared to
males. Males were more likely to click on lower ranked results, from
entries 7 through 10, and also look beyond the first 10 results
significantly more often than women. Males were also more linear in
their scanning patterns, with less regressions. Pupil dilation did not
differ significantly between gender groups.

Effects of task-type and task-goals also influenced SERP examination
behaviour. Guan \& Cutrell (\protect\hyperlink{ref-105}{2007}) used Broder (\protect\hyperlink{ref-broder2002taxonomy}{2002})'s taxonomy of navigational vs.
informational searches. The authors reported that when users could not
find the target results for navigational searches, they either selected
the first result, or switched to a new query. However, for informational
searches, users rarely issued a new query and were more likely to try
out the top-ranked results, even when those results had lower relevance
to the task. This illustrated possible strong confidence of searchers in
the search engine's relevance ranking, even though searchers clearly saw
target results at lower positions. Thus, people were more likely to
deprecate their own sense of objective relevance and obeyed the ranking
determined by the search engine. Jiang et al. (\protect\hyperlink{ref-126}{2014}) used Li \& Belkin (\protect\hyperlink{ref-li2008faceted}{2008})'s framework of
search-tasks, and saw that in tasks having specific goals, searchers
fixated more on lower ranked results after some time. On the other hand,
for tasks having amorphous goals, there was a wider breadth in viewing
the SERP, and less effort spent in viewing the content pages. Fixations
tended to decrease as search session progressed, indicating decreased
interest and increasing mental effort, which could demonstrate
\emph{satisficing} behaviour (\protect\hyperlink{ref-simon1956rational}{Simon, 1956}). A comprehensive overview
of various behavioural traits associated with task-types and task-goals
can be found in (\protect\hyperlink{ref-126}{Jiang et al., 2014} Table 8).

\hypertarget{information-shown-in-search-results-surrogates}{%
\subsubsection{Information Shown in Search Results (Surrogates)}\label{information-shown-in-search-results-surrogates}}

The amount and quality of different kinds of information shown on SERPs
also affected user's information searching behaviour. Cutrell \& Guan (\protect\hyperlink{ref-104}{2007}) saw that as
the length of the surrogate information (result snippets) was increased,
user's search performance improved for informational tasks, but degraded
for navigational tasks (\protect\hyperlink{ref-broder2002taxonomy}{Broder, 2002}). Analyzing eye-tracking
data, they posited that the difference in performance was due to users
paying more attention to the snippet, and less attention to the URL
located at the bottom of the search result. This led to performance
deterioration in navigational searches. Buscher et al. (\protect\hyperlink{ref-115}{2010}) studied the effects of the
quality of advertisements placed in the SERPs (Figure \ref{fig-int-L-serp}(b)). Similar to findings discussed above, a
strong position bias of visual attention was found towards the top few
organic result entries --- the well known F-shaped pattern of visual
attention --- which was stronger for informational than for navigational
tasks. However, a strong bias \emph{against} sponsored links was observed in
general. Even for informational tasks, where participants generally had
a harder time finding a solution, the ads did not receive any additional
attention from the participants. Lorigo et al. (\protect\hyperlink{ref-108}{2008}) compared the visual attention
patterns of searchers using two different search engines: Google, and
Yahoo!. Behavioural trends followed similar patterns for both search
engines, even though Google was rated as the primary search engine of
all but one of the participants. They found slight variations in some
eye-tracking measures (reading time of surrogates, time to click
results, and query reformulation time), and some self-reported measures
(perceived ease of use, perceived satisfaction, and success rate).
However, none of these differences were statistically significant.

The novel query-preview interface by Qvarfordt et al. (\protect\hyperlink{ref-121}{2013}) was discussed in
Section~\ref{sec-bg-search-query} and in
Figure~\ref{fig-int-Q}(a).
The authors also reported several observations about user behaviour on
SERPs. They saw that the presence of the preview visualization enabled
participants to look deeper into the results lists. Participants tried
to use the preview as a navigation tool, although it was not designed as
such. The tool increased the rates at which participants examined
documents at middle ranks in query results, and thus helped discover
more useful documents in those middle ranks than without the preview
widget. The preview tool also helped to increase the diversity of
documents found in a search session, which could in turn lead to better
performance in terms of recall and precision. Thus, the tool helped
searchers overcome the strong position bias towards top-ranked results,
as observed by other studies discussed previously.

\hypertarget{individual-user-characteristics}{%
\subsubsection{Individual User Characteristics}\label{individual-user-characteristics}}

Individual traits of searchers also influence their pattern of
interactions with a SERP, and these patterns can be revealed by
analyzing eye-tracking data. For instance, searchers have been
classified as \emph{economic} vs.~\emph{exhaustive}, based on their style of
evaluating SERPs (\protect\hyperlink{ref-102}{Aula et al., 2005}). \emph{Economic} searchers were found to scan less
than half (three) of the displayed results above the fold, before making
their first action (query re-formulation, or following a link).
\emph{Exhaustive} searchers evaluated more than half of the visible results
above the fold, or even scrolled the results page to view all of the
results, before performing the first action. Thus, economic searchers
demonstrated depth-first search strategy, while exhaustive users
favoured the breadth-first approach
(Figure~\ref{fig-res-L-serp-user-chars}(a)). Dumais et al. (\protect\hyperlink{ref-117}{2010}) demonstrated the use of
unsupervised clustering to re-identify the \emph{economic}-\emph{exhaustive} user
groups, based on differences in total fixation impact \footnote{
  a measure derived from eye fixation durations, proposed by Buscher et al. (\protect\hyperlink{ref-110}{2009})}, scanpaths,
task outcomes, and questionnaire data. The \emph{economic} cluster was
further broken down by users who looked primarily at results
(\emph{economic-results} cluster), and users who viewed both results and ads
(\emph{economic-ads} cluster). All three groups spent the highest amount of
time on the first three results, with the \emph{exhaustive} group being
substantially slower than the other two groups. The \emph{exhaustive} and
\emph{economic-results} groups spent the second-highest amount of time on
results four through six, while the \emph{economic-eds} group spent this time
on the main advertisements. This group spent more than twice as much
time on the main ads as the \emph{economic-results} group, and even more time
on main ads than the \emph{exhaustive group}. This observation is incongruent
to Buscher et al. (\protect\hyperlink{ref-115}{2010})`s findings, as they observed a generally strong bias \emph{against}
viewing sponsored links. Abualsaud \& Smucker (\protect\hyperlink{ref-135}{2019}) conducted further analysis using these
user types, and, in general, reconfirmed the previous findings. They
found that the results above the fold, especially, \textbf{\emph{the first three
search results are special}}, more so for economic users. On submitting
a 'weak' query, if economic users did not find a correct result within
the first three results, they abandoned examination, and reformulated
their query.

Age of searchers also influence SERP evaluation behaviour. Gossen et al. (\protect\hyperlink{ref-124}{2014})
demonstrated differences in SERP evaluation for children and adults
(Figure~\ref{fig-res-L-serp-user-chars}(b)). When answers were not found
within the top search results, the adults reformulated the query
starting a new search, while young users exhaustively explored all the
ten results, and used the navigation buttons between results pages to
continue further examination. Children also paid more attention to
thumbnails and embedded media, and focused less on textual snippets.
Children saw the query suggestions at the bottom of the Google SERP
(because they navigated to the bottom), while the adults did not. Bilal \& Gwizdka (\protect\hyperlink{ref-139}{2016}; \protect\hyperlink{ref-140}{Gwizdka \& Bilal, 2017}) investigated this phenomenon further, and observed that even
within children, age plays a role in SERP evaluation behaviour. Younger
children (grade six, age 11) clicked more often on results in
lower-ranked positions than older children (grade eight, age 13). Older
children's clicking behaviour was based more often on reading result
snippets, and not just on the ranked position of a result in a SERP.
Whereas, younger children made less deliberate choices in choosing which
result to click, and were more exhaustive in the exploration of results.
Thus, using Aula et al. (\protect\hyperlink{ref-102}{2005})'s classification and Dumais et al. (\protect\hyperlink{ref-117}{2010})'s observations, it can be
posited that (younger) children start out as \emph{exhaustive} searchers.
With increase in age and maturity, older children and adults evolve into
\emph{economic} searchers. Interestingly, very similar behaviour patterns as
with children (scrolling further down on SERPs, exhaustive exploration,
etc.) were also observed recently for searchers with dyslexia
(\protect\hyperlink{ref-palani2020eye}{Palani et al., 2020}).

Searcher's native language also influenced SERP interaction behaviour
(\protect\hyperlink{ref-132}{Ling et al., 2018}) (Figure~\ref{fig-int-L-serp}(c)). We discussed in
Section~\ref{sec-bg-search-query} that users strongly preferred issuing
queries in a single language, especially their native language. However,
while examining SERPs, they marked search results in both their first
language and second language to be relevant, to an equal degree. This
confirms the usefulness of search result pages that integrate results
from multiple languages. However, a clear separation in the language of
the search results was strongly preferred, and an `interleaved'
presentation (e.g.~odd numbered results in one language and even
numbered results in another language) was least preferred.

\hypertarget{relevance-judgement}{%
\subsubsection{Relevance Judgement}\label{relevance-judgement}}

Balatsoukas \& Ruthven (\protect\hyperlink{ref-119}{2012}), Balatsoukas \& Ruthven (\protect\hyperlink{ref-114}{2010}) proposed a list of `relevance criteria' for understanding
how searchers evaluate search results, or perform \emph{relevance judgement}.
These criteria were developed based on literature reviews and their
empirical findings from eye-tracking studies. The final list contains 15
relevance criteria (e.g., \emph{topicality}, \emph{quality}, \emph{recency}, \emph{scope},
\emph{availability}, etc.) and can be found in (\protect\hyperlink{ref-119}{Balatsoukas \& Ruthven, 2012} Appendix B).

Search engines are increasingly adding different modalities of
information on the SERP, besides the ``ten blue links''. These include
images, videos, encyclopaedic information, and maps
(Figure~\ref{fig-int-L-serp-new-vertical}). Z. Liu et al. (\protect\hyperlink{ref-128}{2015}) studied the influence of
these different forms of SERP information -- called `verticals' -- on
searcher's relevance judgements. A general observation was that if
verticals were present in a SERP, they created strong attraction biases.
The attraction effect was influenced by the type of verticals, while the
vertical quality (relevant or not) did not have a major impact. For
instance, `images' and `software download' verticals had higher visual
attention, while news verticals had equal attention as the ``ten blue
links'' search results.

\hypertarget{sec-bg-search-content-page}{%
\subsection{Stage 3: Content Page Evaluation / Item Examination}\label{sec-bg-search-content-page}}

\begin{quote}
\emph{How do users behave when examining a single information-object (e.g., a
a non-search-engine webpage, aka content page) obtained from an IR
system?}
\end{quote}

In online information searching, searchers repeatedly interact with
individual webpages, a.k.a. `content pages' in IR terminology. These
webpages can be visited by following links from a search engine,
following links between different webpages, or directly typing the URL
in the browser.

The first group of papers we discuss investigated users' \textbf{visual
attention} and \textbf{reading behaviour} on webpages. Pan et al. (\protect\hyperlink{ref-pan2004determinants}{2004})
studied whether eye-tracking scanpaths on webpages varied based on
task-type, webpage type (business, news, search, or shopping), viewing
order of webpages, and gender of users. The found significant
differences for all factors, except for task-type, which seemed to have
no effect on scanpaths. They used weak task-types: remembering what was
on a webpage vs.~no specific task. In a later work on using
informational vs.~navigational search-tasks, they again saw limited
effect of task-type on visual attention (\protect\hyperlink{ref-lorigo2006influence}{Lorigo et al., 2006}). Findings
from Josephson \& Holmes (\protect\hyperlink{ref-josephson2002visual}{2002})'s study suggested that users possibly follow
habitually preferred scanpaths on a webpage, which can be influenced by
factors like webpage characteristics and memory. However, they used only
three webpages, making the findings difficult to generalize.
Goldberg et al. (\protect\hyperlink{ref-goldberg2002eye}{2002}) studied eye movements on Web portals during
search-tasks, and saw that header bars were typically not viewed before
focusing the main part of the page. So they suggested placing navigation
bars on the left side of a page. Beymer et al. (\protect\hyperlink{ref-beymer2007eye}{2007}) focused on a very
specific feature on webpages: images that are placed next to text
content and how they influence eye movements during a reading task. They
found significant influence on fixation location and duration. Those
influences were dependent on how the image contents related to the text
contents (i.e., whether they showed ads or text-related images). Buscher et al. (\protect\hyperlink{ref-110}{2009})
presented findings from a large scale study where users performed
information-foraging and page-recognition tasks. They observed that in
the first few moments, users quickly scanned the top left of the page,
presumably looking for clues about the content, provenance, type of
information, etc. for that page. The elements that were normally
displayed in the upper left third of webpages (e.g., logos, headlines,
titles or perhaps an important picture related to the content) seemed to
be important for recognizing and categorizing a page. After these
initial moments, influence of task-type set in. For page-recognition
tasks, the attention remained in the top-left corner of the webpage.
However, for information-foraging tasks, fixations moved to the
center-left region of the webpage, where the user was possibly trying to
find task-specific information. The right third of webpages attracted
almost no visual attention during the first one-second of each page
view. Afterwards as well, most users seemed to entirely ignore this
region, or only occasionally look at it. This suggested that users had
low expectations of information-content or general relevance on the
right side of most webpages. As many webpages display advertisements on
the right side, this was a plausible observation, and are in line with
the observed ``F-shaped-patterns'' \footnote{
  \url{https://www.nngroup.com/articles/f-shaped-pattern-reading-web-content}} on webpages.

Buscher et al. (\protect\hyperlink{ref-110}{2009}) also proposed an eye-tracking measure called \emph{fixation impact}.
This measure first appends a circular Gaussian distribution around each
fixation on a webpage element, to create a fuzzy area of interest. This
is called the \emph{distance impact} value. If a webpage element completely
covers the fixation circle (Gaussian distribution), it gets a \emph{distance
impact} value of 1. If the element partially covers the fixation circle,
its \emph{distance impact} value is smaller. Multiplying the \emph{distance
impact} value with the fixation duration gives the fixation impact for
the given webpage element. Thus, an element that completely covers the
fixation circle gets the full fixation duration as \emph{fixation impact}
value. Elements which are partially inside the circle get a value
proportional to the Gaussian distribution. The authors posited that the
rationale behind creating the fixation impact measure was motivated by
observations from human vision research, which indicates that fixation
duration correlates with the amount of visual information processed; the
longer a fixation, the more information is processed around the fixation
centre. Using the fixation impact measure, Buscher et al. (\protect\hyperlink{ref-110}{2009}) proposed a model for
predicting the amount visual attention that individual webpage elements
may receive (i.e.~visual salience).

Another group of studies investigated how users judged \textbf{relevance of
webpages} w.r.t. an assigned search-task or information need.
(\protect\hyperlink{ref-74}{Gwizdka, 2018}; \protect\hyperlink{ref-47}{Gwizdka \& Zhang, 2015a}, \protect\hyperlink{ref-48}{2015b}) observed that when relevant pages were revisited, the
webpages were read more carefully. Pupil dilations were significantly
larger on visits and revisits to relevant pages, and just before
relevance judgements were made. Certain conditions of visits and
revisits also showed significant differences in EEG alpha frequency band
power, and EEG-derived attention levels. Relevance of individual webpage
elements were also assessed as \emph{click-intention}: whether users would
click on an element they were looking at. Slanzi et al. (\protect\hyperlink{ref-69}{2017}) used pupillometry and EEG
signals to predict whether a mouse click was present for each eye
fixation. EEG features included simple statistical features of signals
(mean, SD, power, etc.), as well as sophisticated mathematical features
(Hjorth features, Fractal Dimensions, Entropy, etc.). A battery of
classifier models were tested. However, the results were not promising.
Logistic Regression had the highest accuracy (71\%), but very low F1
score (0.33), while neural network based classifiers the had highest F1
score (0.4). The authors suspected that the low sampling rate of their
instruments (30 Hz eye-tracker and 128 Hz 14-channel EEG) impacted their
classifier performances. González-Ibáñez et al. (\protect\hyperlink{ref-81}{2019}) compared relevance prediction performances
in the presence and absence of eye-tracking data, and argued that when
eye-tracking data collection is not feasible, mouse left-clicks can be
used a good alternative indicator of relevance.

The `\emph{Competition for Attention}' theory states that items in our visual
field compete for our attention (\protect\hyperlink{ref-desimone1995neural}{Desimone \& Duncan, 1995}). Djamasbi et al. (\protect\hyperlink{ref-30}{2013}) studied web
search and browsing from the perspective of this theory. Theoretical
models suggest that in goal-directed searches, information-salience
and/or information-relevance drives search behaviour (i.e.~competition
for attention does not hold true), whereas exploratory search behaviour
is influenced by competition among stimuli that attracts a user's
attention (i.e.~competition for attention holds true). However, in
practice, information search behaviour often becomes a combination of
both types of visual search activities (\protect\hyperlink{ref-groner1984looking}{Groner et al., 1984}). Djamasbi et al. (\protect\hyperlink{ref-30}{2013}) found
that, despite the goal directed nature of their search-task (finding the
best snack place in Boston to take their friends) \emph{competition for
attention} had some effect at the content page level. Some of the users'
attention was diverted to non-focal areas on content pages. However,
there was little effect of \emph{competition for attention} on how the
results were viewed on SERPs. Users exhibited the familiar top-to-bottom
pattern of viewing
(Section~\ref{sec-bg-search-list}), paying the most attention to the top
two entries.

\hypertarget{sec-bg-search-expertise}{%
\section{Effects of Expertise and Working Memory on Search Behaviour}\label{sec-bg-search-expertise}}

Our focus of discussion in this proposal is information searching and
learning. As we saw in Chapter \ref{ch-bg-learn}, learning and expertise are closely connected:
expertise is an evolving characteristic of users that reflects learning
over time, rather than being a static property
(\protect\hyperlink{ref-rieh2016searching}{Rieh et al., 2016}; \protect\hyperlink{ref-sawyer2005cambridge}{Sawyer, 2005}). (\protect\hyperlink{ref-white2016interactions}{White, 2016a, Chapter 7}) considers three types of expertise, that are relevant in
information seeking settings: \emph{(i)} domain or subject-matter expertise;
\emph{(ii)} search expertise; and \emph{(iii)} task expertise. \textbf{Domain or
subject-matter expertise} describes people's knowledge in a specialised
subject area such as a domain of interest. \textbf{Search expertise} refers
to people's skill level at performing information-seeking activities,
both in a Web search setting and in other settings such as specialised
domains. \textbf{Task expertise} describes people's expertise in performing
particular search tasks, potentially independent of domain. Although
considered distinctly, the boundaries between these expertise types are
quite blurred, and therefore difficult to estimate at the time of
search, and model it in a way that can be consumed by search systems.

Previous work on domain knowledge and expertise have linked \footnote{and continue to link} domain
expertise and search behaviour in terms of metrics, behavioural
patterns, and criteria
(\protect\hyperlink{ref-cole2013inferring}{M. J. Cole et al., 2013}; \protect\hyperlink{ref-133}{Mao et al., 2018}; \protect\hyperlink{ref-o2020role}{O'Brien et al., 2020}; \protect\hyperlink{ref-white2009characterizing}{White et al., 2009}). A
representative summary is presented in Table \ref{tab-search-behaviours}, and is adapted from literature
reviews by (\protect\hyperlink{ref-rieh2016searching}{Rieh et al., 2016}) and (\protect\hyperlink{ref-vakkari2016searching}{Vakkari, 2016}). Briefly,
(\protect\hyperlink{ref-wildemuth2004effects}{Wildemuth, 2004}) showed that novices converge toward the same
search patterns as experts, as they are exposed to a topic and learn
more about it. (\protect\hyperlink{ref-zhang2011predicting}{X. Zhang et al., 2011}) found that features such as
document retention, query length, and the average rank of results
selected could be predictive of domain expertise. (\protect\hyperlink{ref-cole2013inferring}{M. J. Cole et al., 2013})
showed that eye-gaze patterns could be used to predict an individual's
level of domain expertise using estimates of cognitive effort associated
with reading. (\protect\hyperlink{ref-white2009characterizing}{White et al., 2009}) showed that measures such as
diverse website visitation, more narrow topical focus, less diversity
(or entropy), more `branchiness' of search sessions, less dwell time,
and higher query and session complexity are indicative of expert
knoweldge and/or search behaviour.

As a stark contrast, (\protect\hyperlink{ref-zlatkin2021students}{Zlatkin-Troitschanskaia et al., 2021}) reviewed literature on
higher education \textbf{students' information search behaviour}. Students
can be considered as novices in all three respects:
domain/subject-matter, search skills, and task. The authors report that
across literature, higher education students' information search
behaviour tends to follow some general general patterns: \emph{(i) foraging:}
no explicit (task-specific) research plan and little understanding of
the differences (pros/cons) between various IR systems; \emph{(ii) Google
dependence:} no intention to use any search tool other than Google,
causing students to struggle to understand library information
structures and engage with scholarly literature effectively; \emph{(iii)
rudimentary search heuristic:} reliance on one and the same simple
search strategy, regardless of search context; \emph{(iv) habitual topic
changing:} students change the search topic after rather superficial
skimming, and before evaluating all search results; and \emph{(v) overuse of
natural language:} students type questions into the search box that are
phrased as if posing them to a person. Highly ranked online sources
accessed via a well-known search engine were perceived as trustworthy.

Effects of memory span and working memory capacity have also been found
to influence search effort and search behaviour
(\protect\hyperlink{ref-arguello2019effects}{Arguello \& Choi, 2019}; \protect\hyperlink{ref-CHIIR19}{Bhattacharya \& Gwizdka, 2019a}; \protect\hyperlink{ref-cole2020more}{L. Cole et al., 2020}; \protect\hyperlink{ref-gwizdka2013effects}{Gwizdka, 2013}, \protect\hyperlink{ref-gwizdka2017can}{2017}).
\textbf{Working memory} (WM) is considered a core executive function is
defined as someone's ability to hold information in short-term memory
when it is no longer perceptually present
(\protect\hyperlink{ref-diamond2013executive}{Diamond, 2013}; \protect\hyperlink{ref-miller1956magical}{G. A. Miller, 1956}). (\protect\hyperlink{ref-bailey2011amount}{Bailey \& Kelly, 2011}) showed
that the amount of effort was a good indicator of user success on search
tasks. (\protect\hyperlink{ref-smith2008user}{Smith \& Kantor, 2008}) studied searcher adaptation to poorly performing
systems and found that searchers changed their search behaviors between
difficult and easy topics in a way that could indicate that users are
satisficing. Differences in search effort between different types
systems (higher effort invested in searching library database vs.~web)
were found by (\protect\hyperlink{ref-rieh2012amount}{Rieh et al., 2012}). A couple of studies showed that mental
effort involved in judging document relevance is lower for irrelevant
and higher for relevant documents (\protect\hyperlink{ref-37}{Gwizdka, 2014}; \protect\hyperlink{ref-villa2013relevance}{Villa \& Halvey, 2013}).
(\protect\hyperlink{ref-gwizdka2017can}{Gwizdka, 2017}) found that that higher WM searchers perform more
actions and that most significant differences are in time spent on
reading results pages. Behaviour of high and low WM searchers were also
found to change differently in the course of a search task performance.

\hypertarget{assessing-learning-during-search}{%
\section{Assessing Learning during Search}\label{assessing-learning-during-search}}

In order for IR systems to foster user-learning at scale, while
respecting individual differences of searchers, there is a need for
measures to represent, assess, and evaluate the learning process,
possibly in an automated fashion. Consequently, a variety of assessment
tools have been used in prior studies. These include self reports, close
ended factual questions (multiple choice), open ended questions (short
answers, summaries, essays, free recall, sentence generation), and
visual mapping techniques using concept maps or mind maps. Each approach
has its own associated advantages and limitations. \textbf{Self-report} asks
searchers to rate their self-perceived pre-search and post-search
knowledge levels (\protect\hyperlink{ref-ghosh2018SearchingLearningExploring}{Ghosh et al., 2018}; \protect\hyperlink{ref-o2020role}{O'Brien et al., 2020}).
This approach is the easiest to construct, and can be generalised over
any search topic. However, self-perceptions may not objectively
represent true learning. \textbf{Closed ended questions} test searchers'
knowledge using factual multiple choice questions (MCQs). The answer
options can be a mixture of fact-based responses (\emph{TRUE}, \emph{FALSE}, or \emph{I
DON'T KNOW}),
(\protect\hyperlink{ref-gadiraju2018AnalyzingKnowledgeGain}{Gadiraju et al., 2018}; \protect\hyperlink{ref-xu2020does}{Xu et al., 2020}; \protect\hyperlink{ref-yu2018PredictingUserKnowledgea}{Yu et al., 2018})
or recall-based responses (\emph{I remember / don't remember seeing this
information}) (\protect\hyperlink{ref-kruikemeier2018learning}{Kruikemeier et al., 2018}; \protect\hyperlink{ref-roy2020exploring}{Roy et al., 2020}).
Constructing topic-dependant MCQs may take time and effort, since they
are topic dependant. Recent work on automatic question generation may be
leveraged to overcome this limitation (\protect\hyperlink{ref-syed2020improving}{Syed et al., 2020}). Evaluating
close ended questions is the easiest, and generally automated in various
online learning platforms. Multiple choice questions, however, suffer
from a limitation: they allow respondents to answer correctly by
guesswork. \textbf{Open ended questions} assess learning by letting searchers
write natural language summaries or short answers
(\protect\hyperlink{ref-bhattacharya2018relating}{Bhattacharya \& Gwizdka, 2018}; \protect\hyperlink{ref-o2020role}{O'Brien et al., 2020}; \protect\hyperlink{ref-roy2021note}{Roy et al., 2021}). Depending on
experimental design, prompts for writing such responses can be generic
(least effort) (\protect\hyperlink{ref-bhattacharya2018relating}{Bhattacharya \& Gwizdka, 2018}, \protect\hyperlink{ref-bhattacharya2019measuring}{2019b}),
or topic-specific (some effort) (\protect\hyperlink{ref-syed2020improving}{Syed et al., 2020}). While this
approach can provide the richest information about the searcher's
knowledge state, evaluating such responses is the most challenging, and
requires extensive human intervention
(\protect\hyperlink{ref-kanniainen2021assessing}{Kanniainen et al., 2021}; \protect\hyperlink{ref-leu2015new}{Leu et al., 2015}; \protect\hyperlink{ref-wilson2013comparison}{M. J. Wilson \& Wilson, 2013}) (as
discussed in Section \ref{sec-bg-learn-artefact}). \textbf{Visual mapping} techniques such
as mind maps and concept maps have also been used to assess learning
during search
(\protect\hyperlink{ref-egusa2010usingb}{Egusa et al., 2010}, \protect\hyperlink{ref-egusa2014howd}{2014a}, \protect\hyperlink{ref-egusa2014howe}{2014b}, \protect\hyperlink{ref-egusa2017evaluating}{2017}; \protect\hyperlink{ref-halttunen2005assessing}{Halttunen \& Jarvelin, 2005}).
Concept maps have been discussed at length in Section \ref{sec-bg-concept-maps}. Learning has also been measured in
\textbf{other ways}, such as user's familiarity with concepts and
relationships between concepts (\protect\hyperlink{ref-pirolli1996scatter}{Pirolli et al., 1996}), gains in user's
understanding of the topic structure, e.g., via conceptual changes
described in pre-defined taxonomies (\protect\hyperlink{ref-zhang2016process}{P. Zhang \& Soergel, 2016}), and user's
ability to formulate more effective queries
(\protect\hyperlink{ref-chen2020understanding}{Chen et al., 2020}; \protect\hyperlink{ref-pirolli1996scatter}{Pirolli et al., 1996}).

\hypertarget{limitations-of-current-search-systems-in-fostering-learning}{%
\section{Limitations of Current Search Systems in Fostering Learning}\label{limitations-of-current-search-systems-in-fostering-learning}}

\hypertarget{sec-bg-search-longitudinal-studies}{%
\subsection{Longitudinal studies}\label{sec-bg-search-longitudinal-studies}}

Learning is a longitudinal process, occurring gradually over time
(Sections \ref{sec-bg-learn-sensemaking} and \ref{sec-bg-learn-principles}). Therefore, information
researchers have studied participant's search behaviour in prior,
\textbf{albeit few}, longitudinal studies. Examples include studies by
(\protect\hyperlink{ref-kelly2006measuring-a}{Kelly, 2006a}, \protect\hyperlink{ref-kelly2006measuring-b}{2006b}; \protect\hyperlink{ref-kuhlthau2004seeking}{Kuhlthau, 2004}; \protect\hyperlink{ref-vakkari2001changes}{Vakkari, 2001a}; \protect\hyperlink{ref-white2009characterizing}{White et al., 2009}; \protect\hyperlink{ref-wildemuth2004effects}{Wildemuth, 2004}).

(\protect\hyperlink{ref-wildemuth2004effects}{Wildemuth, 2004}) examined the search behaviour of medical
students in microbiology. In this experiment, students were observed at
three points of time (at the beginning of the course, at the end of the
course, and six months after the course), under the assumption that
domain expertise changes during a semester. Some search strategies, most
notably the gradual narrowing of the results through iterative query
modification, were the same throughout the observation period. Other
strategies varied over time as individuals gained domains knowledge.
Novices were less efficient in selecting concepts to include in search
and less accurate in their tactics for modifying searches.
(\protect\hyperlink{ref-pennanen2003students}{Pennanen \& Vakkari, 2003}; \protect\hyperlink{ref-vakkari2000cognition}{Vakkari, 2000}, \protect\hyperlink{ref-vakkari2001changes}{2001a}, \protect\hyperlink{ref-vakkari2001theory}{2001b})
also examined students at multiple points in time, as they were
developing their thesis proposal. One important change in behaviour was
the use of a more varied and more specific vocabulary as students
learned more about their research topic. (\protect\hyperlink{ref-weber2019informationseeking}{Weber et al., 2019})
examined a large sample of German students from all academic fields in a
two wave study and found that the more advanced they are in their
studies, the more students show a more advanced search behaviour (e.g.,
using more English queries and accessing academic databases more
frequently). \textbf{Advanced search behaviour predicted better university
grades.} (\protect\hyperlink{ref-weber2018can}{Weber et al., 2018}) also provide mixed evidence on the potential
long-term effects of such interventions, as some of their participants
reverted to their previous habits two weeks after the study and
therefore exhibited only short-term changes in their information-seeking
behaviour.

Overall, results regarding the promotion of user' search and evaluation
skills are encouraging. But there is a clear need for more longitudinal
studies. The general body of search-as-learning literature examines the
learner in the short-term, typically over the course of a single lab
session (\protect\hyperlink{ref-kelly2009evaluation}{Kelly et al., 2009}; \protect\hyperlink{ref-zlatkin2021students}{Zlatkin-Troitschanskaia et al., 2021}). The trend is
similar in other Human-Computer Interaction (HCI) research venues. A
meta-analysis of 1014 user studies reported in the ACM CHI 2020
conference revealed that more than 85\% of the studies observed
participants for a day or less. To this day, ``longitudinal studies are
the exception rather than the norm'' (\protect\hyperlink{ref-koeman2020hciux}{Koeman, 2020}). ``An
over-reliance on short studies risks inaccurate findings, potentially
resulting in prematurely embracing or disregarding new concepts''
(\protect\hyperlink{ref-koeman2020hciux}{Koeman, 2020}).

\hypertarget{sec-bg-search-sensemaking}{%
\subsection{Supporting sensemaking and reflection}\label{sec-bg-search-sensemaking}}

As we saw in Section \ref{sec-bg-learn-sensemaking}, learning \emph{is} sensemaking. Yet,
modern search systems are still quite far from supporting sensemaking
and learning, and rather, at best are good \emph{locators} of information.
(\protect\hyperlink{ref-rieh2016searching}{Rieh et al., 2016}) says that modern search systems should support
sensemaking by offering more interactive functions, such as tagging for
annotation, or tracking individuals' search history, so that a learner
could return to a particular learning point. In addition, a system could
provide new features that allow learners to reflect upon their own
learning process and search outcomes, thus facilitating the development
of critical thinking skills.

\begin{quote}
\emph{It's easy to be impressed by the scientific and engineering feats that have produced web search engines. They are, unquestionably, one of the most impactful and disruptive information technologies of our time. However, it's critical to remember their many limitations: they do not help us \textbf{know what we want to know}; they do not help us \textbf{choose the right words} to find it; they do not help us know if what we've found is \textbf{relevant or true}; and they do not help us \textbf{make sense of it}. All they do is quickly retrieve what other people on the internet have shared. While this is a great feat, all of the content on the internet is far from everything we know, and quite often a poor substitute for expertise.}

\hfill --- Ko (\protect\hyperlink{ref-ko2021seeking}{2021}) (emphasis our own)
\end{quote}

\hypertarget{sec-bg-search-summary}{%
\section{Summary}\label{sec-bg-search-summary}}

In this second chapter of the background literature review, we discussed
\emph{(i)} how searchers interact with three stages / interfaces of modern
information retrieval system: query formulation, search results
evaluation, and content page evaluation; \emph{(ii)} how expertise and
working memory influence overall search behaviour; \emph{(iii)} how learning
or knowledge gain during search has been assessed in recent search as
learning literature; and \emph{(iv)} what are the limitations of current
search systems to foster learning, including gaps in literature about
long term search behaviour and learning outcomes, as well as lack of
support for sensemaking.

We saw that while we have a plethora of studies investigating search
behaviour searchers in the short term, we have merely a handful of
studies observing the same participant for more than a day. To the best
of the author's knowledge, most of these studies were conducted over a
decade ago. Thus, while we have excellent knowledge of short term nature
of influence of searching on learning, we do not know what are the
longer term effects. Furthermore, we we have gaps in our knowledge of
\emph{(i)} how practices like articulation and externalization, and user
attributes like metacognition, motivation, and self regulation moderate
the searching as learning process; \emph{(ii)} how these moderator variables
change over time; and \emph{(iii)} what these phenomena collectively entail
for the design of future learning-centric IR systems. In the next
chapter, we take these gaps in knowledge and use them to inform our
research questions and hypotheses.

\hypertarget{research-questions-and-hypotheses}{%
\chapter{Research Questions and Hypotheses}\label{research-questions-and-hypotheses}}

\hypertarget{methods-longitudinal-study}{%
\chapter{Methods: Longitudinal Study}\label{methods-longitudinal-study}}

\hypertarget{sec-method-exp-design}{%
\section{Study Design}\label{sec-method-exp-design}}

\hypertarget{apparatus}{%
\section{Apparatus}\label{apparatus}}

\hypertarget{yasbil-browsing-logger}{%
\subsection{YASBIL Browsing Logger}\label{yasbil-browsing-logger}}

\hypertarget{qualtrics-survey-software}{%
\subsection{Qualtrics Survey Software}\label{qualtrics-survey-software}}

\hypertarget{zoom-video-conferencing-software}{%
\subsection{Zoom Video-conferencing Software}\label{zoom-video-conferencing-software}}

\hypertarget{sec-method-search-task-template}{%
\section{Search Task Template}\label{sec-method-search-task-template}}

\hypertarget{sec-method-procedure}{%
\section{Procedure}\label{sec-method-procedure}}

Insert diagram and check how it looks

(cap-fig-study-proc-diss) Very very very very very very very very long caption.

\begin{figure}

{\centering \includegraphics[width=1\linewidth]{figs/fig-study-proc-diss} 

}

\caption[Short Caption for LoF]{(cap-fig-study-proc-diss)}\label{fig:fig-study-proc-diss}
\end{figure}

Reference it

\hypertarget{sec-method-sur1}{%
\subsection{SUR1: Entry Survey}\label{sec-method-sur1}}

\hypertarget{ses1-initial-session}{%
\subsection{SES1: Initial Session}\label{ses1-initial-session}}

\hypertarget{ses2a---ses2d-longitudinal-tracking-sessions}{%
\subsection{SES2a - SES2d: Longitudinal Tracking Sessions}\label{ses2a---ses2d-longitudinal-tracking-sessions}}

\hypertarget{sec-method-sur2}{%
\subsection{SUR2: Mid-Term Survey}\label{sec-method-sur2}}

\hypertarget{ses3-final-session}{%
\subsection{SES3: Final Session}\label{ses3-final-session}}

\hypertarget{sur3-exit-survey}{%
\subsection{SUR3: Exit Survey}\label{sur3-exit-survey}}

\hypertarget{data-analysis}{%
\chapter{Data Analysis}\label{data-analysis}}

Note about pronouns:
all participants are referred to using gender-neutral they/them pronouns.

Final feedback:
P022Pisa said
\textgreater{} \emph{It is great to be able to participate in the research this semester. Using the extension somehow brings me postive feedback and that helps me in study I303. So I wanna say thank you}
\textgreater{} - P022Pisa

\hypertarget{data-cleaning-and-processing}{%
\section{Data Cleaning and Processing}\label{data-cleaning-and-processing}}

see crescenzi thesis

\hypertarget{url-categorization}{%
\section{URL Categorization}\label{url-categorization}}

\begin{itemize}
\tightlist
\item
  peer-reviewed publications are \texttt{PUB}s
\item
  others are \texttt{ARTICLE}s (e.g.~Wikipedia)
\item
  if no other info, then \texttt{WEB}
\item
  fuzzy between \texttt{WEB} and \texttt{ARTICLE} (when classified manually)
\item
  \texttt{ARTICLE} if there is a clear author

  \begin{itemize}
  \tightlist
  \item
    except \texttt{WIKIPEDIA}, due to common parlance
  \item
    encyclopedias
  \end{itemize}
\item
  journal homepages are \texttt{WEB}
\item
  list of chapters in a book are \texttt{L.PUB}.

  \begin{itemize}
  \tightlist
  \item
    e.g.~in detail view of
  \end{itemize}
\item
  book chapter is \texttt{PUB}
\end{itemize}

\hypertarget{user-characteristics-latent-profiles}{%
\section{User characteristics: Latent Profiles}\label{user-characteristics-latent-profiles}}

\begin{itemize}
\tightlist
\item
  feature sets for profiling (toggle on and off)

  \begin{itemize}
  \tightlist
  \item
    \texttt{IMI}
  \item
    \texttt{MAI}
  \item
    \texttt{SRQ}
  \item
    \texttt{WMC} / memory span:

    \begin{itemize}
    \tightlist
    \item
      scaling by dividing by 10 (?) (that's the max Coglab would show)
    \end{itemize}
  \end{itemize}
\end{itemize}

Memory span values normalized by 10, because ``\emph{The maximum memory span measurable with this experiment is ten}'' as per CogLab output

\begin{itemize}
\tightlist
\item
  \st{Use LIME / SHAP and counterfactual explanations to understand which components contribute to change in Profile Membership}
\item
  no no; a simple examination of what feature values changed between timepoints will be enough
\item
  use 2 groups! 2 groups is always better. easier to explain; easier to write.
\end{itemize}

\hypertarget{search-behaviour-data-analysis-approach}{%
\section{Search Behaviour Data Analysis Approach}\label{search-behaviour-data-analysis-approach}}

From White (\protect\hyperlink{ref-white2016interactions}{2016a}), Table 2.1 (adapted from Bates, 1989):

\begin{itemize}
\tightlist
\item
  \textbf{Level 1: Move}

  \begin{itemize}
  \tightlist
  \item
    Atomic search event -- for example, a query or click
    (\emph{An identifiable thought or action that is part of information searching.})
  \end{itemize}
\item
  \textbf{Level 2: Tactic}

  \begin{itemize}
  \tightlist
  \item
    Goal or task, including query or click chain
    (\emph{One or several moves made to further a search})
  \end{itemize}
\item
  \textbf{Level 3: Statagem}

  \begin{itemize}
  \tightlist
  \item
    Mission or session
    (\emph{A larger, more complex set of thoughts and/or actions than the tactic; a stratagem consists of multiple tactics and/or moves, all of which are designed to exploit a particular search domain that is thought to contain the desired information})
  \end{itemize}
\item
  \textbf{Level 4: Strategy}

  \begin{itemize}
  \tightlist
  \item
    Session or cross-session search task
    (\emph{A plan, which may contain moves, tactics, and/or stratagems, for an entire information search.})
  \end{itemize}
\end{itemize}

\hypertarget{level-1-moves---query-reformulation}{%
\section{Level 1: Moves - Query Reformulation}\label{level-1-moves---query-reformulation}}

\begin{itemize}
\tightlist
\item
  Yung Sheng's Dissertation
\item
  (\protect\hyperlink{ref-hassan2014struggling}{Hassan et al., 2014}) Table 1
\end{itemize}

Measures:

\begin{itemize}
\tightlist
\item
  Number of terms per Query
\item
  Query length (characters?)
\item
  Number of (unique) queries per search
\item
  Number of reformulated query types
\item
  Abandoned Queries (Percentage of queries with no clicks)
\item
  query similarity (\protect\hyperlink{ref-hassan2014struggling}{Hassan et al., 2014})

  \begin{itemize}
  \tightlist
  \item
    average similarity between all queries to the first query in every session
  \item
    exact match, approx match, lemma match, semantic match
  \end{itemize}
\end{itemize}

\hypertarget{level-1-moves---dwell-time-dt}{%
\section{Level 1: Moves - Dwell Time (DT)}\label{level-1-moves---dwell-time-dt}}

Dwell Time categories:

\begin{itemize}
\tightlist
\item
  \texttt{short}

  \begin{itemize}
  \tightlist
  \item
    1 \textless= DT \textless= 5s
  \item
    ``\emph{A time span of less than 5 seconds is a too short period for being able to read a summary and extract information}'' (\protect\hyperlink{ref-he2016beyond}{He et al., 2016}).
  \end{itemize}
\item
  \texttt{med}

  \begin{itemize}
  \tightlist
  \item
    5 \textless{} DT \textless= 30s
  \end{itemize}
\item
  \texttt{long}

  \begin{itemize}
  \tightlist
  \item
    DT \textgreater{} 30s
  \item
    from Handehawa's thesis, and related Shah references
  \end{itemize}
\end{itemize}

Other Ideas

\begin{itemize}
\tightlist
\item
  do analysis of \texttt{LIB} and \texttt{LIBGUIDE} type URLs - library websites
\item
  Dwell Time:

  \begin{itemize}
  \tightlist
  \item
    overall
  \item
    per-task
  \item
    per timepoint?
  \end{itemize}
\item
  sig diff in DT between

  \begin{itemize}
  \tightlist
  \item
    \texttt{SES1} and \texttt{SES3}
  \item
    \texttt{SES2a} and \texttt{SES2d}?
  \end{itemize}
\item
  checkout: \texttt{sns.pairplot()} -- pairwise relationships in a dataset

  \begin{itemize}
  \tightlist
  \item
    \url{https://seaborn.pydata.org/generated/seaborn.pairplot.html}
  \end{itemize}
\item
  checkout seaborn's plotting capabilites!
\end{itemize}

\hypertarget{level-1-moves---other-behaviour}{%
\section{Level 1: Moves - Other Behaviour}\label{level-1-moves---other-behaviour}}

\begin{itemize}
\tightlist
\item
  \texttt{TAB}: Parallel Browsing Events

  \begin{itemize}
  \tightlist
  \item
    \texttt{open}
  \item
    \texttt{switch}
  \item
    \texttt{close}
  \end{itemize}
\item
  \texttt{SESSION}: from YASBIL events and \texttt{task\_id}

  \begin{itemize}
  \tightlist
  \item
    \texttt{start}
  \item
    \texttt{end}
  \end{itemize}
\item
  \texttt{IDLE}: user stays idle for 1 minute (\protect\hyperlink{ref-taramigkou2018leveraging}{Taramigkou et al., 2018})
\end{itemize}

\hypertarget{level-2-implicit-features-for-exploratory-search-process}{%
\section{Level 2: Implicit Features for Exploratory Search Process}\label{level-2-implicit-features-for-exploratory-search-process}}

From Hendahewa (\protect\hyperlink{ref-hendahewa2016implicit}{2016}) and related papers.

\begin{itemize}
\tightlist
\item
  \textbf{Creativity} --\textgreater{} Information Novelty

  \begin{itemize}
  \tightlist
  \item
    \textbf{Unique Coverage:} Unique web pages visited
  \item
    \textbf{Likelihood of Discovery:} Measurement of difficulty to find certain information
  \end{itemize}
\item
  \textbf{Exploration}

  \begin{itemize}
  \tightlist
  \item
    \textbf{Total Coverage:} Total number of content pages visited
  \item
    \textbf{Distinct Queries:} Total number of different queries issued
  \item
    \textbf{Query diversity:} Measurement of similarity between queries issued
  \end{itemize}
\item
  \textbf{Knowledge Discovery} --\textgreater{} Finding useful and relevant information

  \begin{itemize}
  \tightlist
  \item
    \textbf{Useful Coverage:} Number of pages where users spend a considerable amount of time
  \item
    \textbf{Relevant Coverage:} Number of pages that users denote as relevant to the task
  \end{itemize}
\end{itemize}

\url{https://rucore.libraries.rutgers.edu/rutgers-lib/49207/}

\hypertarget{level-2-search-tactics-and-strategies}{%
\section{Level 2+: Search Tactics and Strategies}\label{level-2-search-tactics-and-strategies}}

Pernilla Q et al's paper: Jiyin He, Pernilla Qvarfort, Martin Halvey, Gene Golovchinsky. Beyond actions: Exploring the discovery of tactics from user logs. In Information Processing \& Management, vol.~52, issue 6, Nov.~2016, pp.~1200--1226.

\begin{itemize}
\tightlist
\item
  \url{http://dx.doi.org/10.1016/j.ipm.2016.05.007}
\item
  \url{https://www.pernillaq.com/exploratory-search}
\item
  \href{https://scholar.google.com/scholar?cites=16001618163231710088\&as_sdt=5,44\&sciodt=0,44\&hl=en}{checkout the forward citations of this paper on automated log analyses}
\item
  like process mining!
\item
  ``\emph{Since modern search systems may allow user interactions beyond the tactics defined in the literature the tactics may need to be extended}''
\end{itemize}

Behaviours from \emph{Search Patterns} book

\begin{itemize}
\tightlist
\item
  quit
\item
  narrow
\item
  expand
\item
  pearl growing / citation mining / snowballing
\item
  pogo sticking
\item
  thrashing
\end{itemize}

\hypertarget{strategies-from-dwell-time-dt}{%
\subsection{Strategies from Dwell Time (DT)}\label{strategies-from-dwell-time-dt}}

Tactics

\begin{itemize}
\tightlist
\item
  Q.\{types of reformulations\}
\item
  L.\{DT categories\}
\item
  L.click (only hyperlink clicks)
\item
  I.\{DT categories\}
\item
  I.click
\item
  OTHER.\{DT categories\}
\item
  TAB.\{tab events\}
\item
  IDLE
\item
  SESSION.\{session events\}
\end{itemize}

\hypertarget{strategies-from-webpage-l2-categories}{%
\subsection{Strategies from webpage L2 categories}\label{strategies-from-webpage-l2-categories}}

Tactics

\begin{itemize}
\tightlist
\item
  Q.\{types of reformulations\}
\item
  L.\{\texttt{PUB} / \texttt{WEB} / \texttt{X}\}
\item
  L.click (only hyperlink clicks)
\item
  I.\{\texttt{PUB} / \texttt{WEB} / \texttt{ARTICLE} / X\}
\item
  I.click
\item
  OTHER.\{\texttt{X}\}
\item
  TAB.\{tab events\}
\item
  IDLE
\item
  SESSION.\{session events\}
\end{itemize}

\hypertarget{strategies-from-lam2007session}{%
\subsection{\texorpdfstring{Strategies from Lam et al. (\protect\hyperlink{ref-lam2007session}{2007})}{Strategies from Lam et al. (2007)}}\label{strategies-from-lam2007session}}

\begin{itemize}
\tightlist
\item
  As in (\protect\hyperlink{ref-lam2007session}{Lam et al., 2007}) Sec 6 (\texttt{S} to denote a Search event and \texttt{X} to denote a non-search engine event):

  \begin{itemize}
  \tightlist
  \item
    Short Navigation: S(Start) --\textgreater{} X (End), with the S event limits to the first session events and the X event to the last events.
  \item
    Topic Exploration: S --\textgreater{} X --\textgreater{} X --\textgreater{} X --\textgreater{} X --\textgreater{} \ldots{}
  \item
    Methodical Results Exploration: S --\textgreater{} X --\textgreater{} S --\textgreater{} X --\textgreater{} S \ldots{}
  \item
    Query Refinement: S --\textgreater{} S --\textgreater{} S --\textgreater{} S \ldots{}
  \end{itemize}
\item
  Using \texttt{WebNavigation} events and tab switches
\item
  can do sequential pattern mining as in (\protect\hyperlink{ref-ibanez2022comparison}{Ibáñez \& Simperl, 2022})

  \begin{itemize}
  \tightlist
  \item
    maximal sequential pattern
  \item
    etc.
  \end{itemize}
\item
  Other search patters:

  \begin{itemize}
  \tightlist
  \item
    \texttt{SS}: Search-engine Searches
  \item
    \texttt{TS}: Third-party Searches using third-party online sites as search engines
  \item
    \texttt{TE}: True Explorations of search results
  \end{itemize}
\end{itemize}

\hypertarget{struggling-vs.-exploring}{%
\subsection{Struggling vs.~Exploring}\label{struggling-vs.-exploring}}

Indicators predictive of struggling (\protect\hyperlink{ref-hassan2014struggling}{Hassan et al., 2014}):

\begin{itemize}
\tightlist
\item
  low amount of similarity between consecutive queries
\item
  more clicks per query
\item
  differences in the nature of the reformulation patterns: less query term substitution and more addition/removal with exploring
\end{itemize}

\hypertarget{navigators-vs.-explorers}{%
\subsection{Navigators vs.~Explorers}\label{navigators-vs.-explorers}}

From IWSS book
(later, low priority)

\hypertarget{transition-analysis-of-search-tactics-strategies}{%
\subsection{Transition analysis of Search Tactics / Strategies}\label{transition-analysis-of-search-tactics-strategies}}

Transition analysis can be applied to:

\begin{itemize}
\tightlist
\item
  search tactics
\item
  tabs: parallel browsing behaviour (think hard\ldots{} do we need a constant number of tabs for this to work?)
\item
  combined (opening tab and closing tabs are tactics / events in Markov process)
\end{itemize}

Help with analyses:

\begin{itemize}
\tightlist
\item
  Chen and Cooper (2002) used a Chi-square test to compare the distribution of transitions in search tactic transition matrices.
\item
  (\protect\hyperlink{ref-he2016beyond}{He et al., 2016, p. 1220} sec 6.1) for calculation formulas for entropy of transitions
\item
  (\protect\hyperlink{ref-krejtz2014entropy}{Krejtz et al., 2014}) sec 4 for calculation formulas
\item
  (\protect\hyperlink{ref-he2016beyond}{He et al., 2016, p. 1220} sec 6.2) for hypotheses
\end{itemize}

\hypertarget{correlation-analysis}{%
\section{Correlation analysis}\label{correlation-analysis}}

Correlation between user profiles and search tactics (Table 9 \protect\hyperlink{ref-taramigkou2018leveraging}{Taramigkou et al., 2018})

\hypertarget{results-and-discussion}{%
\chapter{Results and Discussion}\label{results-and-discussion}}

What about learning??
What are the measures of learning?

\begin{itemize}
\tightlist
\item
  Also see Yung Sheng's Dissertation
\item
  think hard about which data component has not been touched / analysed
\end{itemize}

Hypotheses from He et al. (\protect\hyperlink{ref-he2016beyond}{2016}):

The second set (H2) compares two different user groups, experts and novices, using one of the search systems in two different conditions. The H2 hypotheses illustrate how a focus on search tactics provides a different lens to view search logs.

\begin{itemize}
\tightlist
\item
  H2.1: Search experts are likely to be more predictable in their choice of search tactics compared to novices
\item
  H2.2: Search experts have developed a set of search tactics they prefer over others, while novices use search tactics more uniformly.
\item
  H2.3: While working with a search system novices will find a preferred method of transitioning from one search tactic to another. In other words, their search tactics transitions will become more predictable over time.
\item
  H2.4: While working with a search systems novices will find preferred search tactics to use. In other words, their distribution of search tactics will become less uniform over time.
\end{itemize}

\hypertarget{descriptive-statistics}{%
\section{Descriptive statistics}\label{descriptive-statistics}}

\hypertarget{profile-transitions}{%
\subsection{profile transitions}\label{profile-transitions}}

\begin{itemize}
\tightlist
\item
  how the following changed over time

  \begin{itemize}
  \tightlist
  \item
    motivation, metacognition, self regulation
  \item
    perceived learning, perceived search outcome
  \end{itemize}
\end{itemize}

\hypertarget{q---query-reformulation}{%
\section{Q - query (re)formulation}\label{q---query-reformulation}}

\hypertarget{l---source-selection-item-selection}{%
\section{L - source selection / Item Selection}\label{l---source-selection-item-selection}}

\begin{itemize}
\tightlist
\item
  dwell times
\item
  ``Item'' selection as in IWSS
\end{itemize}

\hypertarget{i---interacting-with-sources}{%
\section{I - interacting with sources}\label{i---interacting-with-sources}}

\begin{itemize}
\tightlist
\item
  dwell times
\end{itemize}

\begin{quote}
\emph{If the dwell time is long, i.e.~≥ 5 seconds, it is more likely that a user is reading the search results summary (ER) rather than only skimming it (EI). The 5 second threshold was determined based on reading research using eye tracking (Rayner, 1998) and the size of the summaries in Querium. A time span of less than 5 seconds is a too short period for being able to read a summary and extract information.}

\hfill --- He et al. (\protect\hyperlink{ref-he2016beyond}{2016})
\end{quote}

\hypertarget{overall-search-behaviour}{%
\section{Overall search behaviour}\label{overall-search-behaviour}}

\hypertarget{search-tactics}{%
\subsection{search tactics}\label{search-tactics}}

\hypertarget{sheg-tasks---information-evaluation-capabilities}{%
\section{SHEG tasks - information evaluation capabilities}\label{sheg-tasks---information-evaluation-capabilities}}

\begin{quote}
\emph{We've confused young people's ability to operate digital devices with the sophistication they need to discern whether the information those devices yield is something that can be relied upon}

\url{https://twitter.com/suzettelohmeyer/status/1617909351766757376}
\url{https://www.grid.news/story/misinformation/2023/01/23/will-information-literacy-in-schools-fix-our-misinformation-problem/}
\end{quote}

\hypertarget{qualitative-free-text-results}{%
\section{Qualitative / Free Text Results}\label{qualitative-free-text-results}}

Note-taking strategies

\begin{itemize}
\tightlist
\item
  how do you organize your notes
\item
  how long do you store your notes
\item
  how do you search for a bit of info in the notes
\end{itemize}

Other surveys

\hypertarget{discussion---research-questions}{%
\section{Discussion - Research Questions}\label{discussion---research-questions}}

\hypertarget{rq1---search-behaviours}{%
\subsection{RQ1: - search behaviours?}\label{rq1---search-behaviours}}

\hypertarget{rq2-mention-here}{%
\subsection{RQ2: mention here}\label{rq2-mention-here}}

\hypertarget{rq3-mention-here}{%
\subsection{RQ3: mention here}\label{rq3-mention-here}}

\hypertarget{rq4-mention-here}{%
\subsection{RQ4: mention here}\label{rq4-mention-here}}

\hypertarget{conclusions-contributions-and-future-work}{%
\chapter{Conclusions, Contributions, and Future Work}\label{conclusions-contributions-and-future-work}}

see Jacek's thesis

\hypertarget{research-summary}{%
\section{Research Summary}\label{research-summary}}

\hypertarget{summary-of-results}{%
\section{Summary of Results}\label{summary-of-results}}

\hypertarget{methodology}{%
\section{Methodology}\label{methodology}}

\hypertarget{contributions}{%
\section{Contributions}\label{contributions}}

From ASIST award session

FOLLOW UP WITH \ldots{}

\begin{itemize}
\tightlist
\item
  Change of the self, self-reflection
\item
  Look at anthropology perspective
\end{itemize}

ROB:

\begin{itemize}
\tightlist
\item
  Look at qualitative data. More interesting
\end{itemize}

HEATHER:

\begin{itemize}
\tightlist
\item
  what can we share back to the teachers of the course, librarians and others
\end{itemize}

\hypertarget{limitations}{%
\section{Limitations}\label{limitations}}

\begin{itemize}
\tightlist
\item
  No PDF
\item
  N=16 to N=10
\item
  Also check anticipated limitations section from proposal
\end{itemize}

\hypertarget{future-work}{%
\section{Future Work}\label{future-work}}

\startappendices

\hypertarget{app-pilot-study}{%
\chapter{Prior Work: Pilot Study}\label{app-pilot-study}}

\hypertarget{ses1-initial-session-1}{%
\section{SES1: Initial Session}\label{ses1-initial-session-1}}

\hypertarget{app-signup-survey}{%
\chapter{SUR1: Entry Survey}\label{app-signup-survey}}

\hypertarget{app-demographics}{%
\section{Demographics}\label{app-demographics}}

\begin{enumerate}
\def\labelenumi{\arabic{enumi}.}
\tightlist
\item
  Please select the degree level/name of the program you are in.
\item
  Please state which year of the program you are in.
\item
  Please state your major(s)
\item
  Do you have native-level familiarity with English language? Yes / No
  / Other:
\item
  Please state your age (in years)
\item
  Please state your gender
\item
  With which ethnicities do you identify? Please check all that apply:

  \begin{itemize}
  \tightlist
  \item
    African
  \item
    African American / Black
  \item
    Asian - East
  \item
    Asian - South East
  \item
    Asian - South
  \item
    Asian - Middle East
  \item
    Caucasian / White
  \item
    Hispanic / Latinx
  \item
    Native American
  \item
    Pacific Islander
  \item
    Mixed
  \item
    Other: \ldots\ldots{}
  \end{itemize}
\item
  Are you an international student? Yes / No; If Yes, where are you originally from?
\item
  Please enter an email address that you check regularly. We will send communications and compensation information to this email address.
\item
  Your name as you would like us to address you.
\end{enumerate}

\hypertarget{app-search-it-proficiency}{%
\section{Search and IT Proficiency}\label{app-search-it-proficiency}}

\begin{enumerate}
\def\labelenumi{\arabic{enumi}.}
\tightlist
\item
  Which device(s) and browser(s) do you normally use to surf the internet?
\end{enumerate}

\begin{longtable}[]{@{}
  >{\raggedright\arraybackslash}p{(\columnwidth - 14\tabcolsep) * \real{0.1250}}
  >{\raggedright\arraybackslash}p{(\columnwidth - 14\tabcolsep) * \real{0.1250}}
  >{\raggedright\arraybackslash}p{(\columnwidth - 14\tabcolsep) * \real{0.1250}}
  >{\raggedright\arraybackslash}p{(\columnwidth - 14\tabcolsep) * \real{0.1250}}
  >{\raggedright\arraybackslash}p{(\columnwidth - 14\tabcolsep) * \real{0.1250}}
  >{\raggedright\arraybackslash}p{(\columnwidth - 14\tabcolsep) * \real{0.1250}}
  >{\raggedright\arraybackslash}p{(\columnwidth - 14\tabcolsep) * \real{0.1250}}
  >{\raggedright\arraybackslash}p{(\columnwidth - 14\tabcolsep) * \real{0.1250}}@{}}
\toprule\noalign{}
\begin{minipage}[b]{\linewidth}\raggedright
Choice
\end{minipage} & \begin{minipage}[b]{\linewidth}\raggedright
Chrome (1)
\end{minipage} & \begin{minipage}[b]{\linewidth}\raggedright
Safari (2)
\end{minipage} & \begin{minipage}[b]{\linewidth}\raggedright
Firefox (3)
\end{minipage} & \begin{minipage}[b]{\linewidth}\raggedright
Edge (4)
\end{minipage} & \begin{minipage}[b]{\linewidth}\raggedright
Opera (5)
\end{minipage} & \begin{minipage}[b]{\linewidth}\raggedright
Other (6)
\end{minipage} & \begin{minipage}[b]{\linewidth}\raggedright
None (7)
\end{minipage} \\
\midrule\noalign{}
\endhead
\bottomrule\noalign{}
\endlastfoot
Desktop & & & & & & & \\
Laptop & & & & & & & \\
Tablet & & & & & & & \\
Smartphone & & & & & & & \\
& & & & & & & \\
\end{longtable}

\begin{enumerate}
\def\labelenumi{\arabic{enumi}.}
\setcounter{enumi}{1}
\item
  How comfortable are you with using Mozilla Firefox to search information on the internet?

  \begin{enumerate}
  \def\labelenumii{\arabic{enumii}.}
  \tightlist
  \item
    I do not know how to use Mozilla Firefox.
  \item
    I have never used Mozilla Firefox.
  \item
    I feel very uncomfortable to use Mozilla Firefox.
  \item
    I feel uncomfortable to use Mozilla Firefox.
  \item
    I feel neither comfortable nor uncomfortable to use Mozilla Firefox.
  \item
    I feel comfortable to use Mozilla Firefox.
  \item
    I feel very comfortable to use Mozilla Firefox.
  \item
    Other: \ldots\ldots{}
  \end{enumerate}
\item
  Which search engines do you normally use?

  \begin{enumerate}
  \def\labelenumii{\arabic{enumii}.}
  \tightlist
  \item
    Google
  \item
    Bing
  \item
    Baidu
  \item
    Yahoo!
  \item
    Yandex
  \item
    DuckDuckGo
  \item
    Other:
  \end{enumerate}
\end{enumerate}

The following items are adapted from the \textbf{Digital Health Literacy Instrument (DHLI)} by Van Der Vaart \& Drossaert (\protect\hyperlink{ref-van2017development}{2017}).

\emph{On a scale of 1 to 5 \ldots{}\\
(1) Very difficult / Very seldom -- Difficult / Seldom -- Neutral -- Easy / Often -- Very easy / Very often (5)}

\emph{How easy or difficult is it for you to\ldots{}}

\begin{enumerate}
\def\labelenumi{\arabic{enumi}.}
\setcounter{enumi}{3}
\tightlist
\item
  Use the keyboard of a computer (e.g., to type words)?
\item
  Use the mouse (e.g., to put the cursor in the right field or to click)?
\item
  Use the buttons or links and hyperlinks on websites?
\end{enumerate}

\emph{When you search the Internet for information, how easy or difficult is it for you to \ldots{}}

\begin{enumerate}
\def\labelenumi{\arabic{enumi}.}
\setcounter{enumi}{6}
\tightlist
\item
  Make a choice from all the information you find?
\item
  Use the proper words or search query to find the information you are looking for
\item
  Find the exact information you are looking for?
\item
  Decide whether the information is reliable or not?
\item
  Decide whether the information is written with commercial interests (e.g., by people trying to sell a product)?
\item
  Check different websites to see whether they provide the same information?
\item
  Decide if the information you found is applicable to your situation?
\item
  Apply the information you found in your daily life?
\item
  Use the information you found to make decisions about your life
\end{enumerate}

\emph{When you search the Internet for information, how often does it happen that\ldots{}}

\begin{enumerate}
\def\labelenumi{\arabic{enumi}.}
\setcounter{enumi}{15}
\tightlist
\item
  You lose track of where you are on a website or the Internet?
\item
  You do not know how to return to a previous page?
\item
  You click on something and get to see something different than you expected?
\end{enumerate}

The following items are adapted from the \textbf{Search Self-Efficacy Scale (SSE)} by Brennan et al. (\protect\hyperlink{ref-brennan2016factor}{2016}).

\emph{On a scale of 1 to 5, how confident are you that you can \ldots{}\\
(1) Not at all confident -- Neither confident nor unconfident -- Totally confident (5)}

\begin{enumerate}
\def\labelenumi{\arabic{enumi}.}
\setcounter{enumi}{18}
\tightlist
\item
  Identify the major requirements of the search from the initial statement of the topic.
\item
  Correctly develop search queries to reflect my requirements.
\item
  Use special syntax in advanced searching (e.g., AND, OR, NOT).
\item
  Evaluate the resulting list to monitor the success of my approach.
\item
  Develop a search query which will retrieve a large number of appropriate articles.
\item
  Find an adequate number of articles.
\item
  Find articles similar in quality to those obtained by a professional searcher.
\item
  Devise a query which will result in a very small percentage of irrelevant items on my list.
\item
  Efficiently structure my time to complete the task.
\item
  Develop a focused search query that will retrieve a small number of appropriate articles.
\item
  Distinguish between relevant and irrelevant articles.
\item
  Complete the search competently and effectively.
\item
  Complete the individual steps of the search with little difficulty.
\item
  Structure my time effectively so that I will finish the search in the allocated time.
\end{enumerate}

\hypertarget{app-course-load}{%
\section{Course Load and Other Engagements}\label{app-course-load}}

\begin{enumerate}
\def\labelenumi{\arabic{enumi}.}
\tightlist
\item
  How many total weekly hours of coursework are you registered for this semester?
\item
  How many weekly hours do you anticipate putting in for studying this course?
\item
  What are your other time commitments, as hours per week? (enter 0 if not applicable)

  \begin{itemize}
  \tightlist
  \item
    jobs
  \item
    extra-curriculars
  \item
    other
  \end{itemize}
\item
  Do you hold a position of responsibility (officer / committee member) in any (student) organisation? Yes / No
\end{enumerate}

\hypertarget{app-note-taking-strategies}{%
\section{Note-taking Strategies}\label{app-note-taking-strategies}}

Adapted from \emph{Listening and Note Taking Survey} by
(\protect\hyperlink{ref-note-taking-survey-penn-state}{Penn State Learning, 2021}), and \emph{Note Taking Strategies Inventory}
by (\protect\hyperlink{ref-note-taking-strategies-umass}{UMass Amherst Student Success, 2021}).

\emph{For each question, choose the response that best describes your actions
(not the one that describes what you think you should be doing). There
are no right or wrong answers. In general (not specifically for this
course)}

\begin{enumerate}
\def\labelenumi{\arabic{enumi}.}
\tightlist
\item
  I take notes using (check all that apply)

  \begin{itemize}
  \tightlist
  \item
    Paper and Pen / Pencil
  \item
    Laptop / Desktop
  \item
    Tablet with Keyboard
  \item
    Tablet with Stylus / Digital Pen
  \end{itemize}
\item
  When taking notes on the laptop, I minimize distractions by:
\end{enumerate}

\begin{center}\rule{0.5\linewidth}{0.5pt}\end{center}

\emph{On a scale of 1 to 5 \ldots{}\\
(1) Never -- Rarely -- Sometimes -- Often -- Always (5)}

\begin{enumerate}
\def\labelenumi{\arabic{enumi}.}
\setcounter{enumi}{2}
\tightlist
\item
  I read my assignments before I go to lecture.
\item
  I find lectures interesting and/or challenging.
\item
  My lecture notes are well organized.
\item
  I recognize main ideas in lectures.
\item
  I recognize supporting details of main ideas.
\item
  I recognize patterns in lectures, e.g., cause-effect, concept-example.
\item
  My lecture notes are complete.
\item
  I recognize relationships between lecture and readings.
\item
  I integrate my lecture notes with my reading notes.
\item
  I summarize my notes, both lecture and reading, in my own words.
\item
  I review my notes immediately after class.
\item
  I conduct weekly reviews of my notes.
\item
  I edit my notes within 24 hours after class.
\item
  I take notes
\item
  I put dates on my notes
\item
  I makes notes in the margins of the text when I read (on paper / digital medium, e.g.~iPad and Apple Pencil)
\item
  I pause periodically while reviewing notes to summarize or paraphrase the information.
\item
  I use diagrams in my notes
\item
  I use different colours when writing notes
\item
  I create outlines, concept maps or organizational charts of how ideas fit together.
\item
  I write down questions I want to ask the instructor
\item
  I reorganize and fill in notes I took in class
\item
  I put things in my own words
\item
  I rewrite my notes
\item
  I use abbreviations in my notes
\item
  I write out my own descriptions of the main concepts
\item
  I keep track of things I do not understand and note when they finally become clear and what made that happen
\item
  I understand my notes
\item
  I refer back to my notes
\end{enumerate}

\begin{center}\rule{0.5\linewidth}{0.5pt}\end{center}

\begin{enumerate}
\def\labelenumi{\arabic{enumi}.}
\setcounter{enumi}{31}
\tightlist
\item
  How do you organise your notes? \ldots\ldots{}
\item
  Have you ever wished that you had written better notes? Why?

  \begin{itemize}
  \tightlist
  \item
    Yes: \ldots\ldots{}
  \item
    No: \ldots\ldots{}
  \end{itemize}
\item
  How long do you store your notes for?

  \begin{enumerate}
  \def\labelenumii{\arabic{enumii}.}
  \tightlist
  \item
    Till the end of the semester
  \item
    End of academic year
  \item
    End of college
  \item
    Lifelong
  \item
    Other: \ldots\ldots{}
  \end{enumerate}
\item
  How do you search for a bit of information in your notes?
\end{enumerate}

\hypertarget{app-imi}{%
\section{Motivation}\label{app-imi}}

Adapted from Intrinsic Motivation Inventory (IMI) (\protect\hyperlink{ref-ryan1982control}{Ryan, 1982}).
Items will be randomly ordered.

\textbf{Scoring directions:}
Score each response from 1 (not at all true) to 5 (very true).
Then reverse score the items marked with \textbf{(R)}.
To do that, subtract the item response from 6, and use the resulting number as the item score.
Then, calculate subscale scores by averaging across all the items on that subscale.
The subscale scores are then used in the analyses of relevant research questions.

\emph{For each of the following statements, please indicate how true it is
for you, using the following scale:}\\
\emph{(1) not at all true --- somewhat true --- very true (5)}

\hypertarget{interestenjoyment}{%
\subsection{Interest/Enjoyment}\label{interestenjoyment}}

\begin{enumerate}
\def\labelenumi{\arabic{enumi}.}
\tightlist
\item
  I will enjoy taking this course very much.
\item
  This course will be fun to do.
\item
  I think this will be a boring course. \textbf{(R)}
\item
  This course will not hold my attention at all. \textbf{(R)}
\item
  I would describe this course as very interesting.
\item
  I think this course will be quite enjoyable.
\end{enumerate}

\hypertarget{perceived-competence}{%
\subsection{Perceived Competence}\label{perceived-competence}}

\begin{enumerate}
\def\labelenumi{\arabic{enumi}.}
\tightlist
\item
  I think I will be pretty good at this course.
\item
  I think I will be doing pretty well at this course, compared to other students.
\item
  After working at this course for awhile, I will feel pretty competent.
\item
  I think I will be satisfied with my performance in this course.
\item
  I think I am pretty skilled at this course.
\item
  This is a course that I think would not be able to do very well. \textbf{(R)}
\end{enumerate}

\hypertarget{effortimportance}{%
\subsection{Effort/Importance}\label{effortimportance}}

\begin{enumerate}
\def\labelenumi{\arabic{enumi}.}
\tightlist
\item
  I plan to put a lot of effort into this course.
\item
  I don't think I will try very hard to do well at this course. \textbf{(R)}
\item
  I will try very hard on this course.
\item
  It is important to me to do well in this course.
\item
  I do not plan to put much energy into this course. \textbf{(R)}
\end{enumerate}

\hypertarget{valueusefulness}{%
\subsection{Value/Usefulness}\label{valueusefulness}}

\begin{enumerate}
\def\labelenumi{\arabic{enumi}.}
\tightlist
\item
  I believe the course and the final project activities could be of some value to me.
\item
  I think that doing the final project activities is useful for me.
\item
  I think the final project is important activity to do because it can equip me with skills that are necessary for making ethical decisions in my adult and professional life.
\item
  I would be willing to do research on the final project topic again because it has some value to me.
\item
  I think doing the final project activities will help me in my adult and professional life
\item
  I believe doing the final project activities will be beneficial to me.
\item
  I think this is an important course.
\end{enumerate}

\hypertarget{app-srq}{%
\section{Self-regulation}\label{app-srq}}

Self-Regulation Questionnaire (SRQ) by (\protect\hyperlink{ref-brown1999self}{J. M. Brown et al., 1999}).

\emph{Please answer the following questions by selecting the option that best
describes how you are. There are no right or wrong answers. Work quickly
and don't think too long about your answers.\\
\strut \\
(1) Strongly Disagree -- Disagree -- Neutral -- Agree -- Strongly Agree (5)}

\begin{enumerate}
\def\labelenumi{\arabic{enumi}.}
\tightlist
\item
  I usually keep track of my progress toward my goals.
\item
  My behavior is not that different from other people's. \textbf{(R)}
\item
  Others tell me that I keep on with things too long. \textbf{(R)}
\item
  I doubt I could change even if I wanted to. \textbf{(R)}
\item
  I have trouble making up my mind about things. \textbf{(R)}
\item
  I get easily distracted from my plans. \textbf{(R)}
\item
  I reward myself for progress toward my goals.
\item
  I don't notice the effects of my actions until it's too late. \textbf{(R)}
\item
  My behavior is similar to that of my friends. Evaluating
\item
  It's hard for me to see anything helpful about changing my ways. \textbf{(R)}
\item
  I am able to accomplish goals I set for myself.
\item
  I put off making decisions. \textbf{(R)}
\item
  I have so many plans that it's hard for me to focus on any one of them. \textbf{(R)}
\item
  I change the way I do things when I see a problem with how things are going.
\item
  It's hard for me to notice when I've ``had enough'' (alcohol, food, sweets, internet, social media) \textbf{(R)}
\item
  I think a lot about what other people think of me.
\item
  I am willing to consider other ways of doing things.
\item
  If I wanted to change, I am confident that I could do it.
\item
  When it comes to deciding about a change, I feel overwhelmed by the choices. \textbf{(R)}
\item
  I have trouble following through with things once I've made up my mind to do something. \textbf{(R)}
\item
  I don't seem to learn from my mistakes. \textbf{(R)}
\item
  I'm usually careful not to overdo it when working, eating, drinking, or being on social media.
\item
  I tend to compare myself with other people.
\item
  I enjoy a routine, and like things to stay the same. \textbf{(R)}
\item
  I have sought out advice or information about changing.
\item
  I can come up with lots of ways to change, but it's hard for me to decide which one to use. \textbf{(R)}
\item
  I can stick to a plan that's working well.
\item
  I usually only have to make a mistake one time in order to learn from it.
\item
  I don't learn well from punishment. \textbf{(R)}
\item
  I have personal standards, and try to live up to them.
\item
  I am set in my ways. \textbf{(R)}
\item
  As soon as I see a problem or challenge, I start looking for possible solutions.
\item
  I have a hard time setting goals for myself. \textbf{(R)}
\item
  I have a lot of willpower.
\item
  When I'm trying to change something, I pay a lot of attention to how I'm doing.
\item
  I usually judge what I'm doing by the consequences of my actions.
\item
  I don't care if I'm different from most people. \textbf{(R)}
\item
  As soon as I see things aren't going right I want to do something about it.
\item
  There is usually more than one way to accomplish something.
\item
  I have trouble making plans to help me reach my goals. \textbf{(R)}
\item
  I am able to resist temptation.
\item
  I set goals for myself and keep track of my progress.
\item
  Most of the time I don't pay attention to what I'm doing. \textbf{(R)}
\item
  I try to be like people around me.
\item
  I tend to keep doing the same thing, even when it doesn't work. \textbf{(R)}
\item
  I can usually find several different possibilities when I want to change something.
\item
  Once I have a goal, I can usually plan how to reach it.
\item
  I have rules that I stick by no matter what.
\item
  If I make a resolution to change something, I pay a lot of attention to how I'm doing.
\item
  Often I don't notice what I'm doing until someone calls it to my attention. \textbf{(R)}
\item
  I think a lot about how I'm doing.
\item
  Usually I see the need to change before others do.
\item
  I'm good at finding different ways to get what I want.
\item
  I usually think before I act.
\item
  Little problems or distractions throw me off course. \textbf{(R)}
\item
  I feel bad when I don't meet my goals.
\item
  I learn from my mistakes.
\item
  I know how I want to be.
\item
  It bothers me when things aren't the way I want them.
\item
  I call in others for help when I need it.
\item
  Before making a decision, I consider what is likely to happen if I do one thing or another.
\item
  I give up quickly. \textbf{(R)}
\item
  I usually decide to change and hope for the best. \textbf{(R)}
\end{enumerate}

\textbf{Scoring Directions:}
Score each response from 1 (strongly disagree) to 5 (strongly agree), and calculate the following seven subscale scores by
summing the items on that subscale.
Items marked \textbf{(R)} are reverse-coded (i.e.~1 = strongly agree and 5 = strongly disagree).
To do that, subtract the item response from 6, and use the resulting number as the item score.

\begin{enumerate}
\def\labelenumi{\arabic{enumi}.}
\tightlist
\item
  \emph{Receiving relevant information:} 1, 8, 15, 22, 29, 36, 43, 50, 57
\item
  \emph{Evaluating the information and comparing it to norms:} 2, 9, 16, 23, 30, 37, 44, 51, 58
\item
  \emph{Triggering change:} 3, 10, 17, 24, 31, 38, 45, 52, 59
\item
  \emph{Searching for options:} 4, 11, 18, 25, 32, 39, 46, 53, 60
\item
  \emph{Formulating a plan:} 5, 12, 19, 26, 33, 40, 47, 54, 61
\item
  \emph{Implementing the plan:} 6, 13, 20, 27, 34, 41, 48, 55, 62
\item
  \emph{Assessing the plan's effectiveness:} 7, 14, 21, 28, 35, 42, 49, 56, 63
\end{enumerate}

Based on our clinical and college samples, we tentatively recommend the following ranges for interpreting SRQ total scores with the 63-item scale:

\begin{itemize}
\tightlist
\item
  \textbf{\textgreater= 239}: High (intact) self-regulation capacity (top quartile)
\item
  \textbf{214 - 238}: Intermediate (moderate) self-regulation capacity (middle quartiles)
\item
  \textbf{\textless= 213}: Low (impaired) self-regulation capacity (bottom quartile)
\end{itemize}

\hypertarget{app-mai}{%
\section{Metacognition}\label{app-mai}}

Metacognitive Awareness Inventory (MAI) proposed by
Schraw \& Dennison (\protect\hyperlink{ref-schraw1994assessing}{1994}) and revised by Terlecki \& McMahon (\protect\hyperlink{ref-terlecki2018call}{2018}).

\emph{Think of yourself as a \textbf{learner}. Read each statement carefully, and
rate it as it generally applies to you when you are in the role of a
learner (student, attending classes, university etc.) Please indicate
how true each reason is for you using the following scale:}

\begin{longtable}[]{@{}ll@{}}
\toprule\noalign{}
Score & Response \\
\midrule\noalign{}
\endhead
\bottomrule\noalign{}
\endlastfoot
1 & I \textbf{NEVER} do this \\
2 & I do this \textbf{infrequently} \\
3 & I do this \textbf{inconsistently} \\
4 & I do this \textbf{frequently} \\
5 & I \textbf{ALWAYS} do this \\
\end{longtable}

\begin{enumerate}
\def\labelenumi{\arabic{enumi}.}
\tightlist
\item
  I ask myself periodically if I am meeting my goals.
\item
  I consider several alternatives to a problem before I answer.
\item
  I try to use strategies that have worked in the past.
\item
  I pace myself while learning in order to have enough time.
\item
  I understand my intellectual strengths and weaknesses.
\item
  I think about what I really need to learn before I begin a task.
\item
  I know how well I did once I finish a test.
\item
  I set specific goals before I begin a task.
\item
  I slow down when I encounter important information.
\item
  I know what kind of information is most important to learn.
\item
  I ask myself if I have considered all options when solving a problem.
\item
  I am good at organizing information.
\item
  I consciously focus my attention on important information.
\item
  I have a specific purpose for each strategy I use.
\item
  I learn best when I know something about the topic.
\item
  I know what the teacher expects me to learn.
\item
  I am good at remembering information.
\item
  I use different learning strategies depending on the situation.
\item
  I ask myself if there was an easier way to do things after I finish a task.
\item
  I have control over how well I learn.
\item
  I periodically review to help me understand important relationships.
\item
  I ask myself questions about the material before I begin.
\item
  I think of several ways to solve a problem and choose the best one.
\item
  I summarize what I've learned after I finish.
\item
  I ask others for help when I don't understand something.
\item
  I can motivate myself to learn when I need to.
\item
  I am aware of what strategies I use when I study.
\item
  I find myself analyzing the usefulness of strategies while I study.
\item
  I use my intellectual strengths to compensate for my weaknesses.
\item
  I focus on the meaning and significance of new information.
\item
  I create my own examples to make information more meaningful.
\item
  I am a good judge of how well I understand something.
\item
  I find myself using helpful learning strategies automatically.
\item
  I find myself pausing regularly to check my comprehension.
\item
  I know when each strategy I use will be most effective.
\item
  I ask myself how well I accomplish my goals once I'm finished.
\item
  I draw pictures or diagrams to help me understand while learning.
\item
  I ask myself if I have considered all options after I solve a problem.
\item
  I try to translate new information into my own words.
\item
  I change strategies when I fail to understand.
\item
  I use the organizational structure of the text to help me learn.
\item
  I read instructions carefully before I begin a task.
\item
  I ask myself if what I'm reading is related to what I already know.
\item
  I reevaluate my assumptions when I get confused.
\item
  I organize my time to best accomplish my goals.
\item
  I learn more when I am interested in the topic.
\item
  I try to break studying down into smaller steps.
\item
  I focus on overall meaning rather than specifics.
\item
  I ask myself questions about how well I am doing while I am learning something new.
\item
  I ask myself if I learned as much as I could have once I finish a task.
\item
  I stop and go back over new information that is not clear.
\item
  I stop and reread when I get confused.
\end{enumerate}

\textbf{Scoring Directions:} Score each response from 1 (never) to 5
(always), and calculate the following subscale scores by summing the
items on that subscale.

\emph{Knowledge about Cognition:}

\begin{enumerate}
\def\labelenumi{\arabic{enumi}.}
\tightlist
\item
  \emph{Declarative Knowledge:} 5, 10, 12, 16, 17, 20, 32, 46 (score out of \(8\times5 = 40\))
\item
  \emph{Procedural Knowledge:} 3, 14, 27, 33 (score out of \(4\times5 = 20\))
\item
  \emph{Conditional Knowledge:} 15, 18, 26, 29, 35 (score out of \(5\times5 = 25\))
\end{enumerate}

\emph{Regulation of Cognition:}

\begin{enumerate}
\def\labelenumi{\arabic{enumi}.}
\tightlist
\item
  \emph{Planning:} 4, 6, 8, 22, 23, 42, 45 (score out of \(7\times5 = 35\))
\item
  \emph{Information Management Strategies:} 9, 13, 30, 31, 37, 39, 41, 43, 47, 48 (score out of \(10\times5 = 50\))
\item
  \emph{Comprehension Monitoring:} 1, 2, 11, 21, 28, 34, 49 (score out of \(7\times5 = 35\))
\item
  \emph{Debugging Strategies:} 25, 40, 44, 51, 52 (score out of \(5\times5 = 25\))
\item
  \emph{Evaluation:} 7, 19, 24, 36, 38, 50 (score out of \(6\times5 = 30\))
  --\textgreater{}
\end{enumerate}

\hypertarget{app-pre-post-tasks}{%
\chapter{Questionnaires for Initial (SES1) and Final (SES3) Sessions}\label{app-pre-post-tasks}}

\hypertarget{app-midterm-survey}{%
\chapter{SUR2: Midterm Survey}\label{app-midterm-survey}}

\hypertarget{app-final-survey}{%
\chapter{SUR3: Exit Survey}\label{app-final-survey}}

\hypertarget{app-variables}{%
\chapter{Variables and Measures}\label{app-variables}}

\hypertarget{sec-app-ack}{%
\chapter{Acknowledgements - The PhD Journey}\label{sec-app-ack}}

Similar to David Maxwell's thesis.

This section will be fleshed out in more detail after the initial committee-submission on Feb 27, 2023.
For now, I wish to thank the following people and organisations (in no particular order):

\begin{itemize}
\tightlist
\item
  Jacek Gwizdka
\item
  Soo Young Rieh + Funding
\item
  Committee Members
\item
  HEB
\item
  Finland people
\item
  Slovenia People
\item
  Germany People

  \begin{itemize}
  \tightlist
  \item
    Anke, Xiaofei, Michael, Hema, Himanshu, Ambika, Hardik\ldots{}
  \end{itemize}
\item
  India People
\item
  UK People
\item
  USA People
\item
  HCI4SouthAsia People
\item
  ASIST People
\item
  CHIIR People + Conferences
\item
  UT Graduate School Funding
\item
  SALPilot Study People
\item
  I303 People
\item
  DAAD
\item
  ABB
\item
  iSchool Doc Colleagues
\item
  Labmates, Officemates
\item
  LinkedIn people
\item
  Twitter people

  \begin{itemize}
  \tightlist
  \item
    Jason Baldridge
  \end{itemize}
\end{itemize}

\hypertarget{references}{%
\chapter*{References}\label{references}}
\addcontentsline{toc}{chapter}{References}

\markboth{References}{}

\hypertarget{refs}{}
\begin{CSLReferences}{1}{0}
\leavevmode\vadjust pre{\hypertarget{ref-135}{}}%
Abualsaud, M., \& Smucker, M. D. (2019). Patterns of search result examination: {Query} to first action. \emph{Proceedings of the 28th {ACM} International Conference on Information and Knowledge Management}, 1833--1842. \url{https://doi.org/10.1145/3357384.3358041}

\leavevmode\vadjust pre{\hypertarget{ref-agosti2014evaluation}{}}%
Agosti, M., Fuhr, N., Toms, E., \& Vakkari, P. (2014). Evaluation methodologies in information retrieval dagstuhl seminar 13441. \emph{ACM SIGIR Forum}, \emph{48}, 36--41.

\leavevmode\vadjust pre{\hypertarget{ref-allan2012frontiers}{}}%
Allan, J., Croft, B., Moffat, A., \& Sanderson, M. (2012). Frontiers, challenges, and opportunities for information retrieval: Report from SWIRL 2012 the second strategic workshop on information retrieval in lorne. \emph{ACM SIGIR Forum}, \emph{46}, 2--32.

\leavevmode\vadjust pre{\hypertarget{ref-ambrose2010howa}{}}%
Ambrose, S. A., Bridges, M. W., DiPietro, M., Lovett, M. C., \& Norman, M. K. (2010). \emph{How {Learning Works}: Seven {Research}-{Based Principles} for {Smart Teaching}}. {John Wiley \& Sons}.

\leavevmode\vadjust pre{\hypertarget{ref-amina2017active}{}}%
Amina, T. (2017). Active knowledge making: Epistemic dimensions of e-learning. In \emph{E-learning ecologies} (pp. 65--87). Routledge.

\leavevmode\vadjust pre{\hypertarget{ref-arguello2019effects}{}}%
Arguello, J., \& Choi, B. (2019). The effects of working memory, perceptual speed, and inhibition in aggregated search. \emph{ACM Transactions on Information Systems}, \emph{37}(3). \url{https://doi.org/10.1145/3322128}

\leavevmode\vadjust pre{\hypertarget{ref-102}{}}%
Aula, A., Majaranta, P., \& Räihä, K.-J. (2005). Eye-tracking reveals the personal styles for search result evaluation. In M. F. Costabile \& F. Paternò (Eds.), \emph{Human-computer interaction - {INTERACT} 2005} (pp. 1058--1061). {Springer Berlin Heidelberg}.

\leavevmode\vadjust pre{\hypertarget{ref-ausubel2012acquisition}{}}%
Ausubel, D. P. (2012). \emph{The acquisition and retention of knowledge: A cognitive view}. Springer Science \& Business Media.

\leavevmode\vadjust pre{\hypertarget{ref-ausubel1968educational}{}}%
Ausubel, D. P., Novak, J. D., Hanesian, H., et al. (1968). \emph{Educational psychology: A cognitive view} (Vol. 6). Holt, Rinehart; Winston New York.

\leavevmode\vadjust pre{\hypertarget{ref-bailey2011amount}{}}%
Bailey, E., \& Kelly, D. (2011). Is amount of effort a better predictor of search success than use of specific search tactics? \emph{Proceedings of the American Society for Information Science and Technology}, \emph{48}(1), 1--10.

\leavevmode\vadjust pre{\hypertarget{ref-114}{}}%
Balatsoukas, P., \& Ruthven, I. (2010). The use of relevance criteria during predictive judgment: {An} eye tracking approach. \emph{Proceedings of the American Society for Information Science and Technology}, \emph{47}(1), 1--10. \url{https://doi.org/10.1002/meet.14504701145}

\leavevmode\vadjust pre{\hypertarget{ref-119}{}}%
Balatsoukas, P., \& Ruthven, I. (2012). An eye-tracking approach to the analysis of relevance judgments on the {Web}: {The} case of {Google} search engine. \emph{Journal of the American Society for Information Science and Technology}, \emph{63}(9), 1728--1746. \url{https://doi.org/10.1002/asi.22707}

\leavevmode\vadjust pre{\hypertarget{ref-belkin1982ask}{}}%
Belkin, N. J., Oddy, R. N., \& Brooks, H. M. (1982). ASK for information retrieval: Part i. Background and theory. \emph{Journal of Documentation}.

\leavevmode\vadjust pre{\hypertarget{ref-beymer2007eye}{}}%
Beymer, D., Orton, P. Z., \& Russell, D. M. (2007). An eye tracking study of how pictures influence online reading. \emph{IFIP Conference on Human-Computer Interaction}, 456--460.

\leavevmode\vadjust pre{\hypertarget{ref-bhattacharya2021longitudinal}{}}%
Bhattacharya, N. (2021). A longitudinal study to understand learning during search. \emph{Proceedings of the 2021 Conference on Human Information Interaction and Retrieval}, 363--366.

\leavevmode\vadjust pre{\hypertarget{ref-bhattacharya2018relating}{}}%
Bhattacharya, N., \& Gwizdka, J. (2018). Relating eye-tracking measures with changes in knowledge on search tasks. \emph{Symposium on Eye Tracking Research \& Applications (ETRA)}.

\leavevmode\vadjust pre{\hypertarget{ref-bhattacharya2019measuring}{}}%
Bhattacharya, N., \& Gwizdka, J. (2019b). Measuring learning during search: Differences in interactions, eye-gaze, and semantic similarity to expert knowledge. \emph{Proceedings of the 2019 Conference on Human Information Interaction and Retrieval}, 63--71.

\leavevmode\vadjust pre{\hypertarget{ref-CHIIR19}{}}%
Bhattacharya, N., \& Gwizdka, J. (2019a). Measuring learning during search: Differences in interactions, eye-gaze, and semantic similarity to expert knowledge. \emph{CHIIR'19}.

\leavevmode\vadjust pre{\hypertarget{ref-139}{}}%
Bilal, D., \& Gwizdka, J. (2016). Children's {Eye}-fixations on {Google Search Results}. \emph{Proceedings of the 79th {ASIS}\&{T Annual Meeting}}, \emph{79}, 89:1--89:6. \url{https://doi.org/10.1002/pra2.2016.14505301089}

\leavevmode\vadjust pre{\hypertarget{ref-blanken2017metacognition}{}}%
Blanken-Webb, J. (2017). Metacognition: Cognitive dimensions of e-learning. In \emph{E-learning ecologies} (pp. 163--182). Routledge.

\leavevmode\vadjust pre{\hypertarget{ref-boldi2009dango}{}}%
Boldi, P., Bonchi, F., Castillo, C., \& Vigna, S. (2009). From" dango" to" japanese cakes": Query reformulation models and patterns. \emph{2009 IEEE/WIC/ACM International Joint Conference on Web Intelligence and Intelligent Agent Technology}, \emph{1}, 183--190.

\leavevmode\vadjust pre{\hypertarget{ref-borlund2013interactive}{}}%
Borlund, P. (2013). Interactive {Information Retrieval}: {An Introduction}. \emph{Journal of Information Science Theory and Practice}, \emph{1}(3), 12--32. \url{https://doi.org/10.1633/JISTAP.2013.1.3.2}

\leavevmode\vadjust pre{\hypertarget{ref-breakstone2018we}{}}%
Breakstone, J., McGrew, S., Smith, M., Ortega, T., \& Wineburg, S. (2018). Why we need a new approach to teaching digital literacy. \emph{Phi Delta Kappan}, \emph{99}(6), 27--32.

\leavevmode\vadjust pre{\hypertarget{ref-breakstone2021students}{}}%
Breakstone, J., Smith, M., Wineburg, S., Rapaport, A., Carle, J., Garland, M., \& Saavedra, A. (2021). Students' {Civic Online Reasoning}: A {National Portrait}. \emph{Educational Researcher}. \url{https://doi.org/10.3102/0013189X211017495}

\leavevmode\vadjust pre{\hypertarget{ref-brennan2016factor}{}}%
Brennan, K., Kelly, D., \& Zhang, Y. (2016). Factor analysis of a search self-efficacy scale. \emph{Proceedings of the 2016 ACM on Conference on Human Information Interaction and Retrieval}, 241--244.

\leavevmode\vadjust pre{\hypertarget{ref-broder2002taxonomy}{}}%
Broder, A. (2002). A taxonomy of web search. \emph{SIGIR Forum}, \emph{36}(2), 3--10. \url{https://doi.org/10.1145/792550.792552}

\leavevmode\vadjust pre{\hypertarget{ref-brookes1980foundations}{}}%
Brookes, B. C. (1980). The foundations of information science. Part i. Philosophical aspects. \emph{Journal of Information Science}, \emph{2}(3-4), 125--133.

\leavevmode\vadjust pre{\hypertarget{ref-brown1998self}{}}%
Brown, J. (1998). \emph{Self-regulation and the addictive behaviours}. New York: Plenum Press.

\leavevmode\vadjust pre{\hypertarget{ref-brown1999self}{}}%
Brown, J. M., Miller, W. R., \& Lawendowski, L. A. (1999). The self-regulation questionnaire. In V. L. \& J. T. L. (Eds.), \emph{Innovations in clinical practice: A sourcebook} (Vol. 17, pp. 281--292). Professional Resource Press/Professional Resource Exchange.

\leavevmode\vadjust pre{\hypertarget{ref-110}{}}%
Buscher, G., Cutrell, E., \& Morris, M. R. (2009). What {Do You See When You}'re {Surfing}? {Using Eye Tracking} to {Predict Salient Regions} of {Web Pages}. \emph{Proceedings of the SIGCHI Conference on Human Factors in Computing Systems}, 10.

\leavevmode\vadjust pre{\hypertarget{ref-115}{}}%
Buscher, G., Dumais, S. T., \& Cutrell, E. (2010). The good, the bad, and the random: {An} eye-tracking study of ad quality in web search. \emph{Proceedings of the 33rd International {ACM SIGIR} Conference on Research and Development in Information Retrieval}, 42--49. \url{https://doi.org/10.1145/1835449.1835459}

\leavevmode\vadjust pre{\hypertarget{ref-chen2020understanding}{}}%
Chen, Y., Zhao, Y., \& Wang, Z. (2020). Understanding online health information consumers' search as a learning process. \emph{Library Hi Tech}.

\leavevmode\vadjust pre{\hypertarget{ref-cherry2020what}{}}%
Cherry, K. (2020). What {Is Motivation}? In \emph{Verywell Mind}. \url{https://www.verywellmind.com/what-is-motivation-2795378}

\leavevmode\vadjust pre{\hypertarget{ref-cole2020more}{}}%
Cole, L., MacFarlane, A., \& Makri, S. (2020). More than words: The impact of memory on how undergraduates with dyslexia interact with information. \emph{Proceedings of the 2020 Conference on Human Information Interaction and Retrieval}, 353--357. \url{https://doi.org/10.1145/3343413.3378005}

\leavevmode\vadjust pre{\hypertarget{ref-cole2013inferring}{}}%
Cole, M. J., Gwizdka, J., Liu, C., Belkin, N. J., \& Zhang, X. (2013). Inferring user knowledge level from eye movement patterns. \emph{Information Processing \& Management}, \emph{49}(5), 1075--1091.

\leavevmode\vadjust pre{\hypertarget{ref-collins2021reimagining}{}}%
Collins, C. (2021). Reimagining {Digital Literacy Education} to {Save Ourselves}. \emph{Learning for Justice}, \emph{Fall 2021}. \url{https://www.learningforjustice.org/magazine/fall-2021/reimagining-digital-literacy-education-to-save-ourselves}

\leavevmode\vadjust pre{\hypertarget{ref-collins2017search}{}}%
Collins-Thompson, K., Hansen, P., \& Hauff, C. (2017). Search as learning (dagstuhl seminar 17092). \emph{Dagstuhl Reports}, \emph{7}.

\leavevmode\vadjust pre{\hypertarget{ref-cope2013new}{}}%
Cope, B., \& Kalantzis, M. (2013). Towards a {New Learning}: The {\emph{Scholar}} {Social Knowledge Workspace}, in {Theory} and {Practice}. \emph{E-Learning and Digital Media}, \emph{10}(4), 332--356. \url{https://doi.org/10.2304/elea.2013.10.4.332}

\leavevmode\vadjust pre{\hypertarget{ref-cope2017elearningc}{}}%
Cope, B., \& Kalantzis, M. (2017). \emph{E-{Learning Ecologies}: Principles for {New Learning} and {Assessment}}. {Taylor \& Francis}.

\leavevmode\vadjust pre{\hypertarget{ref-104}{}}%
Cutrell, E., \& Guan, Z. (2007). What are you looking for? {An} eye-tracking study of information usage in web search. \emph{Proceedings of the {SIGCHI} Conference on Human Factors in Computing Systems}, 407--416. \url{https://doi.org/10.1145/1240624.1240690}

\leavevmode\vadjust pre{\hypertarget{ref-deci2013intrinsic}{}}%
Deci, E. L., \& Ryan, R. M. (2013). \emph{Intrinsic motivation and self-determination in human behavior}. Springer Science \& Business Media.

\leavevmode\vadjust pre{\hypertarget{ref-dervin2010sensemaking}{}}%
Dervin, B., \& Naumer, C. M. (2010). Sense-making. In M. J. Bates \& M. M. N. (Eds.), \emph{Encyclopedia of library and information sciences (3rd ed.)} (pp. 4696-\/-4707). Taylor; Francis.

\leavevmode\vadjust pre{\hypertarget{ref-desimone1995neural}{}}%
Desimone, R., \& Duncan, J. (1995). Neural mechanisms of selective visual attention. \emph{Annual Review of Neuroscience}, \emph{18}(1), 193--222.

\leavevmode\vadjust pre{\hypertarget{ref-diamond2013executive}{}}%
Diamond, A. (2013). Executive functions. \emph{Annual Review of Psychology}, \emph{64}, 135--168.

\leavevmode\vadjust pre{\hypertarget{ref-dicerbo2014impacts}{}}%
DiCerbo, K. E., \& Behrens, J. T. (2014). Impacts of the digital ocean on education. \emph{London: Pearson}, \emph{1}.

\leavevmode\vadjust pre{\hypertarget{ref-30}{}}%
Djamasbi, S., Hall-Phillips, A., \& Yang, R. (Rachel). (2013). Search {Results Pages} and {Competition} for {Attention Theory}: {An Exploratory Eye}-{Tracking Study}. In S. Yamamoto (Ed.), \emph{Human {Interface} and the {Management} of {Information}. {Information} and {Interaction Design}} (pp. 576--583). {Springer Berlin Heidelberg}. \url{http://link.springer.com.ezproxy.lib.utexas.edu/chapter/10.1007/978-3-642-39209-2-64}

\leavevmode\vadjust pre{\hypertarget{ref-117}{}}%
Dumais, S. T., Buscher, G., \& Cutrell, E. (2010). Individual differences in gaze patterns for web search. \emph{Proceedings of the Third Symposium on Information Interaction in Context}, 185--194. \url{https://doi.org/10.1145/1840784.1840812}

\leavevmode\vadjust pre{\hypertarget{ref-egusa2010usingb}{}}%
Egusa, Y., Saito, H., Takaku, M., Terai, H., Miwa, M., \& Kando, N. (2010). Using a {Concept Map} to {Evaluate Exploratory Search}. \emph{Proceedings of the {Third Symposium} on {Information Interaction} in {Context}}, 175--184. \url{https://doi.org/10.1145/1840784.1840810}

\leavevmode\vadjust pre{\hypertarget{ref-egusa2014howd}{}}%
Egusa, Y., Takaku, M., \& Saito, H. (2014a). How {Concept Maps Change} if a {User Does Search} or {Not}? \emph{Proceedings of the 5th {Information Interaction} in {Context Symposium}}, 68--75. \url{https://doi.org/10.1145/2637002.2637012}

\leavevmode\vadjust pre{\hypertarget{ref-egusa2014howe}{}}%
Egusa, Y., Takaku, M., \& Saito, H. (2014b). How to evaluate searching as learning. \emph{Searching as {Learning Workshop} ({IIiX} 2014 Workshop)}. \url{http://www.diigubc.ca/IIIXSAL/program.html}

\leavevmode\vadjust pre{\hypertarget{ref-egusa2017evaluating}{}}%
Egusa, Y., Takaku, M., \& Saito, H. (2017). Evaluating {Complex Interactive Searches Using Concept Maps}. \emph{{SCST}@ {CHIIR}}, 15--17.

\leavevmode\vadjust pre{\hypertarget{ref-127}{}}%
Eickhoff, C., Dungs, S., \& Tran, V. (2015). An eye-tracking study of query reformulation. \emph{Proceedings of the 38th International {ACM SIGIR} Conference on Research and Development in Information Retrieval}, 13--22. \url{https://doi.org/10.1145/2766462.2767703}

\leavevmode\vadjust pre{\hypertarget{ref-eickhoff2017introduction}{}}%
Eickhoff, C., Gwizdka, J., Hauff, C., \& He, J. (2017). Introduction to the special issue on search as learning. \emph{Information Retrieval Journal}, \emph{20}(5), 399--402.

\leavevmode\vadjust pre{\hypertarget{ref-freund2013searching}{}}%
Freund, L., Gwizdka, J., Hansen, P., Kando, N., \& Rieh, S. Y. (2013). From searching to learning. \emph{Evaluation Methodologies in Information Retrieval. Dagstuhl Reports}, \emph{13441}, 102--105.

\leavevmode\vadjust pre{\hypertarget{ref-freund2014searching}{}}%
Freund, L., He, J., Gwizdka, J., Kando, N., Hansen, P., \& Rieh, S. Y. (2014). Searching as learning (SAL) workshop 2014. \emph{Proceedings of the 5th Information Interaction in Context Symposium}, 7--7.

\leavevmode\vadjust pre{\hypertarget{ref-gadiraju2018AnalyzingKnowledgeGain}{}}%
Gadiraju, U., Yu, R., Dietze, S., \& Holtz, P. (2018). Analyzing knowledge gain of users in informational search sessions on the web. \emph{Conference on Human Information Interaction \& Retrieval (CHIIR)}.

\leavevmode\vadjust pre{\hypertarget{ref-ghosh2018SearchingLearningExploring}{}}%
Ghosh, S., Rath, M., \& Shah, C. (2018). Searching as learning: Exploring search behavior and learning outcomes in learning-related tasks. \emph{Conference on Human Information Interaction \& Retrieval (CHIIR)}.

\leavevmode\vadjust pre{\hypertarget{ref-goldberg2002eye}{}}%
Goldberg, J. H., Stimson, M. J., Lewenstein, M., Scott, N., \& Wichansky, A. M. (2002). Eye tracking in web search tasks: Design implications. \emph{Proceedings of the 2002 Symposium on Eye Tracking Research \& Applications}, 51--58.

\leavevmode\vadjust pre{\hypertarget{ref-81}{}}%
González-Ibáñez, R., Esparza-Villamán, A., Vargas-Godoy, J. C., \& Shah, C. (2019). A comparison of unimodal and multimodal models for implicit detection of relevance in interactive {IR}. \emph{Journal of the Association for Information Science and Technology}, \emph{0}(0). \url{https://doi.org/10.1002/asi.24202}

\leavevmode\vadjust pre{\hypertarget{ref-124}{}}%
Gossen, T., Höbel, J., \& Nürnberger, A. (2014). A comparative study about children's and adults' perception of targeted web search engines. \emph{Proceedings of the {SIGCHI} Conference on Human Factors in Computing Systems}, 1821--1824. \url{https://doi.org/10.1145/2556288.2557031}

\leavevmode\vadjust pre{\hypertarget{ref-grabowski1996generative}{}}%
Grabowski, B. L. (1996). Generative learning: Past, present, and future. \emph{Handbook of Research for Educational Communications and Technology}, 897--918.

\leavevmode\vadjust pre{\hypertarget{ref-101}{}}%
Granka, L. A., Joachims, T., \& Gay, G. (2004). Eye-tracking analysis of user behavior in {WWW} search. \emph{Proceedings of the 27th Annual International {ACM SIGIR} Conference on Research and Development in Information Retrieval}, 478--479. \url{https://doi.org/10.1145/1008992.1009079}

\leavevmode\vadjust pre{\hypertarget{ref-groner1984looking}{}}%
Groner, R., Walder, F., \& Groner, M. (1984). Looking at faces: Local and global aspects of scanpaths. In \emph{Advances in psychology} (Vol. 22, pp. 523--533). Elsevier.

\leavevmode\vadjust pre{\hypertarget{ref-105}{}}%
Guan, Z., \& Cutrell, E. (2007). An eye tracking study of the effect of target rank on web search. \emph{Proceedings of the {SIGCHI} Conference on Human Factors in Computing Systems}, 417--420. \url{https://doi.org/10.1145/1240624.1240691}

\leavevmode\vadjust pre{\hypertarget{ref-gwizdka2013effects}{}}%
Gwizdka, J. (2013). Effects of working memory capacity on users' search effort. \emph{Proceedings of the {International Conference} on {Multimedia}, {Interaction}, {Design} and {Innovation}}, 11:1--11:8. \url{https://doi.org/10.1145/2500342.2500358}

\leavevmode\vadjust pre{\hypertarget{ref-37}{}}%
Gwizdka, J. (2014). Characterizing {Relevance} with {Eye}-tracking {Measures}. \emph{Proceedings of the 5th {Information Interaction} in {Context Symposium}}, 58--67. \url{https://doi.org/10.1145/2637002.2637011}

\leavevmode\vadjust pre{\hypertarget{ref-gwizdka2017can}{}}%
Gwizdka, J. (2017). I {Can} and {So I Search More}: Effects {Of Memory Span On Search Behavior}. \emph{Proceedings of the 2017 {Conference} on {Conference Human Information Interaction} and {Retrieval}}, 341--344. \url{https://doi.org/10.1145/3020165.3022148}

\leavevmode\vadjust pre{\hypertarget{ref-74}{}}%
Gwizdka, J. (2018). Inferring {Web Page Relevance Using Pupillometry} and {Single Channel EEG}. In F. D. Davis, R. Riedl, J. vom Brocke, P.-M. Léger, \& A. B. Randolph (Eds.), \emph{Information {Systems} and {Neuroscience}} (pp. 175--183). {Springer International Publishing}. \url{https://doi.org/10.1007/978-3-319-67431-5-20}

\leavevmode\vadjust pre{\hypertarget{ref-140}{}}%
Gwizdka, J., \& Bilal, D. (2017). Analysis of {Children}'s {Queries} and {Click Behavior} on {Ranked Results} and {Their Thought Processes} in {Google Search}. \emph{Proceedings of the 2017 {Conference} on {Conference Human Information Interaction} and {Retrieval}}, 377--380. \url{https://doi.org/10.1145/3020165.3022157}

\leavevmode\vadjust pre{\hypertarget{ref-gwizdka2016search}{}}%
Gwizdka, J., Hansen, P., Hauff, C., He, J., \& Kando, N. (2016). Search as learning (SAL) workshop 2016. \emph{Proceedings of the 39th International ACM SIGIR Conference on Research and Development in Information Retrieval}, 1249--1250.

\leavevmode\vadjust pre{\hypertarget{ref-47}{}}%
Gwizdka, J., \& Zhang, Y. (2015a). Differences in {Eye}-{Tracking Measures Between Visits} and {Revisits} to {Relevant} and {Irrelevant Web Pages}. \emph{Proceedings of the 38th {International ACM SIGIR Conference} on {Research} and {Development} in {Information Retrieval}}, 811--814. \url{https://doi.org/10.1145/2766462.2767795}

\leavevmode\vadjust pre{\hypertarget{ref-48}{}}%
Gwizdka, J., \& Zhang, Y. (2015b). Towards {Inferring Web Page Relevance} \textendash{} {An Eye}-{Tracking Study}. \emph{Proceedings of {iConference}'2015}, 5. \url{https://www.ideals.illinois.edu/handle/2142/73709}

\leavevmode\vadjust pre{\hypertarget{ref-halttunen2005assessing}{}}%
Halttunen, K., \& Jarvelin, K. (2005). Assessing learning outcomes in two information retrieval learning environments. \emph{Information Processing \& Management}, \emph{41}(4), 949--972. \url{https://doi.org/10.1016/j.ipm.2004.02.004}

\leavevmode\vadjust pre{\hypertarget{ref-hansen2016editorial}{}}%
Hansen, P., \& Rieh, S. Y. (2016). Editorial: Recent advances on searching as learning: An introduction to the special issue. \emph{Journal of Information Science}, \emph{42}(1), 3--6. \url{https://doi.org/10.1177/0165551515614473}

\leavevmode\vadjust pre{\hypertarget{ref-hassan2014struggling}{}}%
Hassan, A., White, R. W., Dumais, S. T., \& Wang, Y.-M. (2014). Struggling or exploring? Disambiguating long search sessions. \emph{Proceedings of the 7th ACM International Conference on Web Search and Data Mining}, 53--62.

\leavevmode\vadjust pre{\hypertarget{ref-he2016beyond}{}}%
He, J., Qvarfordt, P., Halvey, M., \& Golovchinsky, G. (2016). Beyond actions: Exploring the discovery of tactics from user logs. \emph{Information Processing \& Management}, \emph{52}(6), 1200--1226.

\leavevmode\vadjust pre{\hypertarget{ref-hendahewa2016implicit}{}}%
Hendahewa, C. H. (2016). \emph{Implicit search feature based approach to assist users in exploratory search tasks} {[}PhD thesis{]}. Rutgers, The State University of New Jersey.

\leavevmode\vadjust pre{\hypertarget{ref-125}{}}%
Hofmann, K., Mitra, B., Radlinski, F., \& Shokouhi, M. (2014). An eye-tracking study of user interactions with query auto completion. \emph{Proceedings of the 23rd {ACM} International Conference on Conference on Information and Knowledge Management}, 549--558. \url{https://doi.org/10.1145/2661829.2661922}

\leavevmode\vadjust pre{\hypertarget{ref-huang2013relevance}{}}%
Huang, X., \& Soergel, D. (2013). Relevance: {An} improved framework for explicating the notion. \emph{Journal of the American Society for Information Science and Technology}, \emph{64}(1), 18--35. \url{https://doi.org/10.1002/asi.22811}

\leavevmode\vadjust pre{\hypertarget{ref-ibanez2022comparison}{}}%
Ibáñez, L.-D., \& Simperl, E. (2022). A comparison of dataset search behaviour of internal versus search engine referred sessions. \emph{Acm Sigir Conference on Human Information Interaction and Retrieval}, 158--168.

\leavevmode\vadjust pre{\hypertarget{ref-126}{}}%
Jiang, J., He, D., \& Allan, J. (2014). Searching, browsing, and clicking in a search session: {Changes} in user behavior by task and over time. \emph{Proceedings of the 37th International {ACM SIGIR} Conference on Research \& Development in Information Retrieval}, 607--616. \url{https://doi.org/10.1145/2600428.2609633}

\leavevmode\vadjust pre{\hypertarget{ref-josephson2002visual}{}}%
Josephson, S., \& Holmes, M. E. (2002). Visual attention to repeated internet images: Testing the scanpath theory on the world wide web. \emph{Proceedings of the 2002 Symposium on Eye Tracking Research \& Applications}, 43--49.

\leavevmode\vadjust pre{\hypertarget{ref-jossberger2010challenge}{}}%
Jossberger, H., Brand-Gruwel, S., Boshuizen, H., \& Van de Wiel, M. (2010). The challenge of self-directed and self-regulated learning in vocational education: A theoretical analysis and synthesis of requirements. \emph{Journal of Vocational Education and Training}, \emph{62}(4), 415--440.

\leavevmode\vadjust pre{\hypertarget{ref-kahne2012digital}{}}%
Kahne, J., Lee, N.-J., \& Feezell, J. T. (2012). Digital media literacy education and online civic and political participation. \emph{International Journal of Communication}, \emph{6}, 24.

\leavevmode\vadjust pre{\hypertarget{ref-kalantzis2012newa}{}}%
Kalantzis, M., \& Cope, B. (2012). \emph{New {Learning}: Elements of a {Science} of {Education}}. {Cambridge University Press}.

\leavevmode\vadjust pre{\hypertarget{ref-kanfer1970self-b}{}}%
Kanfer, F. H. (1970a). \emph{Self-monitoring: Methodological limitations and clinical applications.}

\leavevmode\vadjust pre{\hypertarget{ref-kanfer1970self-a}{}}%
Kanfer, F. H. (1970b). Self-regulation: Research, issues, and speculations. \emph{Behavior Modification in Clinical Psychology}, \emph{74}, 178--220.

\leavevmode\vadjust pre{\hypertarget{ref-kanniainen2021assessing}{}}%
Kanniainen, L., Kiili, C., Tolvanen, A., Aro, M., Anmarkrud, Ø., \& Leppänen, P. H. T. (2021). Assessing reading and online research comprehension: Do difficulties in attention and executive function matter? \emph{Learning and Individual Differences}, \emph{87}, 101985. \url{https://doi.org/10.1016/j.lindif.2021.101985}

\leavevmode\vadjust pre{\hypertarget{ref-karapanos2021advances}{}}%
Karapanos, E., Gerken, J., Kjeldskov, J., \& Skov, M. B. (Eds.). (2021). \emph{Advances in {Longitudinal HCI Research}}. {Springer International Publishing}. \url{https://doi.org/10.1007/978-3-030-67322-2}

\leavevmode\vadjust pre{\hypertarget{ref-kelly2006measuring-a}{}}%
Kelly, D. (2006a). Measuring online information seeking context, {Part} 1: Background and method. \emph{Journal of the American Society for Information Science and Technology}, \emph{57}(13), 1729--1739. \url{https://doi.org/10.1002/asi.20483}

\leavevmode\vadjust pre{\hypertarget{ref-kelly2006measuring-b}{}}%
Kelly, D. (2006b). Measuring online information seeking context, {Part} 2: Findings and discussion. \emph{Journal of the American Society for Information Science and Technology}, \emph{57}(14), 1862--1874. \url{https://doi.org/10.1002/asi.20484}

\leavevmode\vadjust pre{\hypertarget{ref-kelly2009methods}{}}%
Kelly, D. (2009). Methods for evaluating interactive information retrieval systems with users. \emph{Foundations and Trends in Information Retrieval}, \emph{3}(1---2), 1--224.

\leavevmode\vadjust pre{\hypertarget{ref-kelly2009evaluation}{}}%
Kelly, D., Dumais, S., \& Pedersen, J. O. (2009). Evaluation challenges and directions for information-seeking support systems. \emph{IEEE Computer}, \emph{42}(3).

\leavevmode\vadjust pre{\hypertarget{ref-knowles1975self}{}}%
Knowles, M. S. (1975). \emph{Self-directed learning: A guide for learners and teachers.} New York: Association press.

\leavevmode\vadjust pre{\hypertarget{ref-ko2021seeking}{}}%
Ko, A. J. (2021). Seeking information. In \emph{Foundations of {Information}}. \url{https://faculty.washington.edu/ajko/books/foundations-of-information/\#/seeking}

\leavevmode\vadjust pre{\hypertarget{ref-koeman2020hciux}{}}%
Koeman, L. (2020). \emph{HCI/UX research: What methods do we use? -- lisa koeman -- blog}. \url{https://lisakoeman.nl/blog/hci-ux-research-what-methods-do-we-use/}.

\leavevmode\vadjust pre{\hypertarget{ref-krejtz2014entropy}{}}%
Krejtz, K., Szmidt, T., Duchowski, A. T., \& Krejtz, I. (2014). Entropy-based statistical analysis of eye movement transitions. \emph{Proceedings of the Symposium on Eye Tracking Research and Applications}, 159--166.

\leavevmode\vadjust pre{\hypertarget{ref-kruikemeier2018learning}{}}%
Kruikemeier, S., Lecheler, S., \& Boyer, M. M. (2018). Learning from news on different media platforms: An eye-tracking experiment. \emph{Political Communication}, \emph{35}(1), 75--96.

\leavevmode\vadjust pre{\hypertarget{ref-kuhlthau2004seeking}{}}%
Kuhlthau, C. C. (2004). \emph{Seeking meaning: A process approach to library and information services} (Vol. 2). Libraries Unlimited Westport, CT.

\leavevmode\vadjust pre{\hypertarget{ref-lam2007session}{}}%
Lam, H., Russell, D., Tang, D., \& Munzner, T. (2007). Session viewer: Visual exploratory analysis of web session logs. \emph{2007 IEEE Symposium on Visual Analytics Science and Technology}, 147--154.

\leavevmode\vadjust pre{\hypertarget{ref-leacock1998combining}{}}%
Leacock, C., \& Chodorow, M. (1998). Combining local context and WordNet similarity for word sense identification. \emph{WordNet: An Electronic Lexical Database}, \emph{49}(2), 265--283.

\leavevmode\vadjust pre{\hypertarget{ref-lei2015effect}{}}%
Lei, P.-L., Sun, C.-T., Lin, S. S., \& Huang, T.-K. (2015). Effect of metacognitive strategies and verbal-imagery cognitive style on biology-based video search and learning performance. \emph{Computers \& Education}, \emph{87}, 326--339.

\leavevmode\vadjust pre{\hypertarget{ref-leu2015new}{}}%
Leu, D. J., Forzani, E., Rhoads, C., Maykel, C., Kennedy, C., \& Timbrell, N. (2015). The {New Literacies} of {Online Research} and {Comprehension}: Rethinking the {Reading Achievement Gap}. \emph{Reading Research Quarterly}, \emph{50}(1), 37--59. \url{https://doi.org/10.1002/rrq.85}

\leavevmode\vadjust pre{\hypertarget{ref-li2008faceted}{}}%
Li, Y., \& Belkin, N. J. (2008). A faceted approach to conceptualizing tasks in information seeking. \emph{Information Processing \& Management}, \emph{44}(6), 1822--1837.

\leavevmode\vadjust pre{\hypertarget{ref-132}{}}%
Ling, C., Steichen, B., \& Choulos, A. G. (2018). A comparative user study of interactive multilingual search interfaces. \emph{Proceedings of the 2018 Conference on Human Information Interaction \& Retrieval}, 211--220. \url{https://doi.org/10.1145/3176349.3176383}

\leavevmode\vadjust pre{\hypertarget{ref-liu2010analysis}{}}%
Liu, C., Gwizdka, J., Liu, J., Xu, T., \& Belkin, N. J. (2010). Analysis and evaluation of query reformulations in different task types. \emph{Proceedings of the American Society for Information Science and Technology}, \emph{47}(1), 1--9.

\leavevmode\vadjust pre{\hypertarget{ref-128}{}}%
Liu, Z., Liu, Y., Zhou, K., Zhang, M., \& Ma, S. (2015). Influence of vertical result in web search examination. \emph{Proceedings of the 38th International {ACM SIGIR} Conference on Research and Development in Information Retrieval}, 193--202. \url{https://doi.org/10.1145/2766462.2767714}

\leavevmode\vadjust pre{\hypertarget{ref-108}{}}%
Lorigo, L., Haridasan, M., Brynjarsdóttir, H., Xia, L., Joachims, T., Gay, G., Granka, L., Pellacini, F., \& Pan, B. (2008). Eye tracking and online search: {Lessons} learned and challenges ahead. \emph{Journal of the American Society for Information Science and Technology}, \emph{59}(7), 1041--1052. \url{https://doi.org/10.1002/asi.20794}

\leavevmode\vadjust pre{\hypertarget{ref-lorigo2006influence}{}}%
Lorigo, L., Pan, B., Hembrooke, H., Joachims, T., Granka, L., \& Gay, G. (2006). The influence of task and gender on search and evaluation behavior using google. \emph{Information Processing \& Management}, \emph{42}(4), 1123--1131.

\leavevmode\vadjust pre{\hypertarget{ref-loyens2008selfdirected}{}}%
Loyens, S. M. M., Magda, J., \& Rikers, R. M. J. P. (2008). Self-{Directed Learning} in {Problem}-{Based Learning} and its {Relationships} with {Self}-{Regulated Learning}. \emph{Educational Psychology Review}, \emph{20}(4), 411--427. \url{https://doi.org/10.1007/s10648-008-9082-7}

\leavevmode\vadjust pre{\hypertarget{ref-mannion2020metacognition}{}}%
Mannion, J. (2020). Metacognition, self-regulation and self-regulated learning: What's the difference? In \emph{impact.chartered.college}. \url{https://impact.chartered.college/article/metacognition-self-regulation-regulated-learning-difference/}

\leavevmode\vadjust pre{\hypertarget{ref-133}{}}%
Mao, J., Liu, Y., Kando, N., Zhang, M., \& Ma, S. (2018). How does domain expertise affect users' search interaction and outcome in exploratory search? \emph{ACM Transactions on Information Systems}, \emph{36}.

\leavevmode\vadjust pre{\hypertarget{ref-marchionini1995information}{}}%
Marchionini, G. (1995). \emph{Information {Seeking} in {Electronic Environments}}. {Cambridge University Press}.

\leavevmode\vadjust pre{\hypertarget{ref-marchionini2006toward}{}}%
Marchionini, G. (2006). Toward human-computer information retrieval. \emph{Bulletin of the American Society for Information Science and Technology}, \emph{32}(5), 20--22.

\leavevmode\vadjust pre{\hypertarget{ref-marton1976qualitative-b}{}}%
Marton, F., \& Säaljö, R. (1976). On qualitative differences in learning---ii outcome as a function of the learner's conception of the task. \emph{British Journal of Educational Psychology}, \emph{46}(2), 115--127.

\leavevmode\vadjust pre{\hypertarget{ref-marton1976qualitative-a}{}}%
Marton, F., \& Säljö, R. (1976). On qualitative differences in learning: I---outcome and process. \emph{British Journal of Educational Psychology}, \emph{46}(1), 4--11.

\leavevmode\vadjust pre{\hypertarget{ref-mcgrew2020learning}{}}%
McGrew, S. (2020). Learning to evaluate: An intervention in civic online reasoning. \emph{Computers \& Education}, \emph{145}, 103711.

\leavevmode\vadjust pre{\hypertarget{ref-mcgrew2021skipping}{}}%
McGrew, S. (2021). Skipping the source and checking the contents: An in-depth look at students' approaches to web evaluation. \emph{Computers in the Schools}, \emph{38}(2), 75--97.

\leavevmode\vadjust pre{\hypertarget{ref-mcgrew2018can}{}}%
McGrew, S., Breakstone, J., Ortega, T., Smith, M., \& Wineburg, S. (2018). Can students evaluate online sources? Learning from assessments of civic online reasoning. \emph{Theory \& Research in Social Education}, \emph{46}(2), 165--193.

\leavevmode\vadjust pre{\hypertarget{ref-mcgrew2021click}{}}%
McGrew, S., \& Glass, A. C. (2021). Click {Restraint}: Teaching {Students} to {Analyze Search Results}. \emph{Proceedings of the 14th {International Conference} on {Computer}-{Supported Collaborative Learning}-{CSCL} 2021}.

\leavevmode\vadjust pre{\hypertarget{ref-mcgrew2017challenge}{}}%
McGrew, S., Ortega, T., Breakstone, J., \& Wineburg, S. (2017). The challenge that's bigger than fake news: Civic reasoning in a social media environment. \emph{American Educator}, \emph{41}(3), 4.

\leavevmode\vadjust pre{\hypertarget{ref-mihailidis2013media}{}}%
Mihailidis, P., \& Thevenin, B. (2013). Media literacy as a core competency for engaged citizenship in participatory democracy. \emph{American Behavioral Scientist}, \emph{57}(11), 1611--1622.

\leavevmode\vadjust pre{\hypertarget{ref-miller1956magical}{}}%
Miller, G. A. (1956). The magical number seven, plus or minus two: Some limits on our capacity for processing information. \emph{Psychological Review}, \emph{63}(2), 81.

\leavevmode\vadjust pre{\hypertarget{ref-miller1991self}{}}%
Miller, W. R., \& Brown, J. M. (1991). Self-regulation as a conceptual basis for the prevention and treatment of addictive behaviours. \emph{Self-Control and the Addictive Behaviours}, 3--79.

\leavevmode\vadjust pre{\hypertarget{ref-council2000how}{}}%
National Research Council. (2000). \emph{How people learn: {Brain}, mind, experience, and school: {Expanded} edition}. {The National Academies Press}. \url{https://doi.org/10.17226/9853}

\leavevmode\vadjust pre{\hypertarget{ref-newlondon1996pedagogy}{}}%
New London Group. (1996). A pedagogy of multiliteracies: Designing social futures. \emph{Harvard Educational Review}, \emph{66}(1), 60--92.

\leavevmode\vadjust pre{\hypertarget{ref-ngss-sensemaking}{}}%
Next Generation Science Standards. (2021). \emph{Task annotation project in science \textbar{} sense-making}. \url{https://www.nextgenscience.org/sites/default/files/TAPS\%20Sense-making.pdf}.

\leavevmode\vadjust pre{\hypertarget{ref-novak2002meaningful}{}}%
Novak, J. D. (2002). Meaningful learning: The essential factor for conceptual change in limited or inappropriate propositional hierarchies leading to empowerment of learners. \emph{Science Education}, \emph{86}(4), 548--571.

\leavevmode\vadjust pre{\hypertarget{ref-novak2010learninga}{}}%
Novak, J. D. (2010). \emph{Learning, creating, and using knowledge: Concept maps as facilitative tools in schools and corporations} (2nd ed). {Routledge}.

\leavevmode\vadjust pre{\hypertarget{ref-novak1984learning}{}}%
Novak, J. D., \& Gowin, D. B. (1984). \emph{Learning how to learn}. Cambridge University Press. \url{https://doi.org/10.1017/CBO9781139173469}

\leavevmode\vadjust pre{\hypertarget{ref-o2020role}{}}%
O'Brien, H. L., Kampen, A., Cole, A. W., \& Brennan, K. (2020). The role of domain knowledge in search as learning. \emph{Conference on Human Information Interaction and Retrieval (CHIIR)}.

\leavevmode\vadjust pre{\hypertarget{ref-palani2020eye}{}}%
Palani, S., Fourney, A., Williams, S., Larson, K., Spiridonova, I., \& Morris, M. R. (2020). An eye tracking study of web search by people with and without dyslexia. \emph{Proceedings of the 43rd International ACM SIGIR Conference on Research and Development in Information Retrieval}, 729--738. \url{https://doi.org/10.1145/3397271.3401103}

\leavevmode\vadjust pre{\hypertarget{ref-pan2004determinants}{}}%
Pan, B., Hembrooke, H. A., Gay, G. K., Granka, L. A., Feusner, M. K., \& Newman, J. K. (2004). The determinants of web page viewing behavior: An eye-tracking study. \emph{Proceedings of the 2004 Symposium on Eye Tracking Research \& Applications}, 147--154.

\leavevmode\vadjust pre{\hypertarget{ref-pea2014learning}{}}%
Pea, R., \& Jacks, D. (2014). \emph{The learning analytics workgroup: A report on building the field of learning analytics for personalized learning at scale}. \url{https://ed.stanford.edu/sites/default/files/law-report-complete-09-02-2014.pdf}; Stanford, CA: Stanford University.

\leavevmode\vadjust pre{\hypertarget{ref-note-taking-survey-penn-state}{}}%
Penn State Learning. (2021). \emph{Listening and note taking survey \textbar{} penn state learning}. \url{https://pennstatelearning.psu.edu/listening-and-note-taking-survey}; Pennsylvania State University.

\leavevmode\vadjust pre{\hypertarget{ref-pennanen2003students}{}}%
Pennanen, M., \& Vakkari, P. (2003). Students' conceptual structure, search process, and outcome while preparing a research proposal: A longitudinal case study. \emph{Journal of the American Society for Information Science and Technology}, \emph{54}(8), 759--770.

\leavevmode\vadjust pre{\hypertarget{ref-piaget1936origins}{}}%
Piaget, J. (1936). \emph{Origins of intelligence in children.}

\leavevmode\vadjust pre{\hypertarget{ref-pirolli1996scatter}{}}%
Pirolli, P., Schank, P., Hearst, M., \& Diehl, C. (1996). Scatter/gather browsing communicates the topic structure of a very large text collection. \emph{Conference on Human Factors in Computing Systems (CHI'96)}.

\leavevmode\vadjust pre{\hypertarget{ref-121}{}}%
Qvarfordt, P., Golovchinsky, G., Dunnigan, T., \& Agapie, E. (2013). Looking ahead: {Query} preview in exploratory search. \emph{Proceedings of the 36th International {ACM SIGIR} Conference on Research and Development in Information Retrieval}, 243--252. \url{https://doi.org/10.1145/2484028.2484084}

\leavevmode\vadjust pre{\hypertarget{ref-url-rieh-homepage}{}}%
Rieh, S. Y. (2020). \emph{Research area 1: Searching as learning}. \url{https://rieh.ischool.utexas.edu/research}.

\leavevmode\vadjust pre{\hypertarget{ref-rieh2016searching}{}}%
Rieh, S. Y., Collins-Thompson, K., Hansen, P., \& Lee, H.-J. (2016). Towards searching as a learning process: A review of current perspectives and future directions. \emph{Journal of Information Science}, \emph{42}(1), 19--34. \url{https://doi.org/10.1177/0165551515615841}

\leavevmode\vadjust pre{\hypertarget{ref-rieh2012amount}{}}%
Rieh, S. Y., Kim, Y.-M., \& Markey, K. (2012). Amount of invested mental effort (AIME) in online searching. \emph{Information Processing \& Management}, \emph{48}(6), 1136--1150.

\leavevmode\vadjust pre{\hypertarget{ref-roy2020exploring}{}}%
Roy, N., Moraes, F., \& Hauff, C. (2020). Exploring users' learning gains within search sessions. \emph{Conference on Human Information Interaction and Retrieval (CHIIR)}.

\leavevmode\vadjust pre{\hypertarget{ref-roy2021note}{}}%
Roy, N., Torre, M. V., Gadiraju, U., Maxwell, D., \& Hauff, C. (2021). Note the highlight: Incorporating active reading tools in a search as learning environment. \emph{Proceedings of the 2021 Conference on Human Information Interaction and Retrieval}, 229--238.

\leavevmode\vadjust pre{\hypertarget{ref-rumelhart1981accretion}{}}%
Rumelhart, D. E., \& Norman, D. A. (1981). Accretion, tuning and restructuring: Three modes of learning. In J. W. Cotton \& K. R. (Eds.), \emph{Semantic factors in cognition} (pp. 37--90).

\leavevmode\vadjust pre{\hypertarget{ref-rumelhart1977representation}{}}%
Rumelhart, D. E., \& Ortony, A. (1977). The representation of knowledge in memory. In R. C. Anderson, S. R. J., \& M. W. E. (Eds.), \emph{Schooling and the acquisition of knowledge} (pp. 99--135). Hillsdale, NJ: Erlbaum.

\leavevmode\vadjust pre{\hypertarget{ref-ryan1982control}{}}%
Ryan, R. M. (1982). Control and information in the intrapersonal sphere: An extension of cognitive evaluation theory. \emph{Journal of Personality and Social Psychology}, \emph{43}(3), 450.

\leavevmode\vadjust pre{\hypertarget{ref-ryan2000self}{}}%
Ryan, R. M., \& Deci, E. L. (2000). Self-determination theory and the facilitation of intrinsic motivation, social development, and well-being. \emph{American Psychologist}, \emph{55}(1), 68.

\leavevmode\vadjust pre{\hypertarget{ref-ryan2017self}{}}%
Ryan, R. M., \& Deci, E. L. (2017). \emph{Self-determination theory: Basic psychological needs in motivation, development, and wellness}. Guilford Publications.

\leavevmode\vadjust pre{\hypertarget{ref-saks2014distinguishing}{}}%
Saks, K., \& Leijen, Ä. (2014). Distinguishing {Self}-directed and {Self}-regulated {Learning} and {Measuring} them in the {E}-learning {Context}. \emph{Procedia - Social and Behavioral Sciences}, \emph{112}, 190--198. \url{https://doi.org/10.1016/j.sbspro.2014.01.1155}

\leavevmode\vadjust pre{\hypertarget{ref-saracevic1975relevance}{}}%
Saracevic, T. (1975). Relevance: A review of and a framework for the thinking on the notion in information science. \emph{Journal of the American Society for Information Science}, \emph{26}(6), 321--343.

\leavevmode\vadjust pre{\hypertarget{ref-saracevic2007relevance}{}}%
Saracevic, T. (2007a). Relevance: A review of the literature and a framework for thinking on the notion in information science. {Part II}: Nature and manifestations of relevance. \emph{Journal of the American Society for Information Science and Technology}, \emph{58}(13), 1915--1933. \url{https://doi.org/10.1002/asi.20682}

\leavevmode\vadjust pre{\hypertarget{ref-saracevic2007relevancea}{}}%
Saracevic, T. (2007b). Relevance: A review of the literature and a framework for thinking on the notion in information science. {Part III}: Behavior and effects of relevance. \emph{Journal of the American Society for Information Science and Technology}, \emph{58}(13), 2126--2144.

\leavevmode\vadjust pre{\hypertarget{ref-saracevic2016notion}{}}%
Saracevic, T. (2016). The {Notion} of {Relevance} in {Information Science}: {Everybody} knows what relevance is. {But}, what is it really? \emph{Synthesis Lectures on Information Concepts, Retrieval, and Services}.

\leavevmode\vadjust pre{\hypertarget{ref-sawyer2005cambridge}{}}%
Sawyer, R. K. (2005). \emph{The {Cambridge} handbook of the learning sciences}. {Cambridge University Press}.

\leavevmode\vadjust pre{\hypertarget{ref-schraw1994assessing}{}}%
Schraw, G., \& Dennison, R. S. (1994). Assessing {Metacognitive Awareness}. \emph{Contemporary Educational Psychology}, \emph{19}(4), 460--475. \url{https://doi.org/10.1006/ceps.1994.1033}

\leavevmode\vadjust pre{\hypertarget{ref-simon1956rational}{}}%
Simon, H. A. (1956). Rational choice and the structure of the environment. \emph{Psychological Review}, \emph{63}(2), 129.

\leavevmode\vadjust pre{\hypertarget{ref-69}{}}%
Slanzi, G., Balazs, J. A., \& Velásquez, J. D. (2017). Combining eye tracking, pupil dilation and {EEG} analysis for predicting web users click intention. \emph{Information Fusion}, \emph{35}, 51--57. \url{https://doi.org/10.1016/j.inffus.2016.09.003}

\leavevmode\vadjust pre{\hypertarget{ref-129}{}}%
Smith, C. L., Gwizdka, J., \& Feild, H. (2016). Exploring the use of query auto completion: {Search} behavior and query entry profiles. \emph{Proceedings of the 2016 {ACM} on Conference on Human Information Interaction and Retrieval}, 101--110. \url{https://doi.org/10.1145/2854946.2854975}

\leavevmode\vadjust pre{\hypertarget{ref-smith2008user}{}}%
Smith, C. L., \& Kantor, P. B. (2008). User adaptation: Good results from poor systems. \emph{Proceedings of the 31st Annual International ACM SIGIR Conference on Research and Development in Information Retrieval}, 147--154.

\leavevmode\vadjust pre{\hypertarget{ref-spink1997study}{}}%
Spink, A. (1997). Study of interactive feedback during mediated information retrieval. \emph{Journal of the American Society for Information Science}.

\leavevmode\vadjust pre{\hypertarget{ref-syed2020improving}{}}%
Syed, R., Collins-Thompson, K., Bennett, P. N., Teng, M., Williams, S., Tay, D. W. W., \& Iqbal, S. (2020). Improving learning outcomes with gaze tracking and automatic question generation. \emph{The Web Conference (WWW)}.

\leavevmode\vadjust pre{\hypertarget{ref-tahamtan2019effect}{}}%
Tahamtan, I. (2019). The effect of motivation on web search behaviors of health consumers. \emph{Proceedings of the 2019 Conference on Human Information Interaction and Retrieval}, 401--404.

\leavevmode\vadjust pre{\hypertarget{ref-taramigkou2018leveraging}{}}%
Taramigkou, M., Apostolou, D., \& Mentzas, G. (2018). Leveraging exploratory search with personality traits and interactional context. \emph{Information Processing \& Management}, \emph{54}(4), 609--629.

\leavevmode\vadjust pre{\hypertarget{ref-terlecki2020revising}{}}%
Terlecki, M. (2020). Revising the {Metacognitive Awareness Inventory} ({MAI}) to be {More User}-{Friendly}. In \emph{Improve with Metacognition}. \url{https://www.improvewithmetacognition.com/revising-the-metacognitive-awareness-inventory/}

\leavevmode\vadjust pre{\hypertarget{ref-terlecki2018call}{}}%
Terlecki, M., \& McMahon, A. (2018). A {Call} for {Metacognitive Intervention}: Improvements {Due} to {Curricular Programming} in {Leadership}. \emph{Journal of Leadership Education}, \emph{17}(4), 130--145. \url{https://doi.org/10.12806/V17/I4/R8}

\leavevmode\vadjust pre{\hypertarget{ref-note-taking-strategies-umass}{}}%
UMass Amherst Student Success. (2021). \emph{Note taking strategies inventory \textbar{} success@UMass}. \url{https://www.umass.edu/studentsuccess/sites/default/files/inline-files/note-taking-strategies-0.pdf}.

\leavevmode\vadjust pre{\hypertarget{ref-vakkari2000cognition}{}}%
Vakkari, P. (2000). Cognition and changes of search terms and tactics during task performance: A longitudinal case study. In \emph{Content-based multimedia information access-volume 1} (pp. 894--907).

\leavevmode\vadjust pre{\hypertarget{ref-vakkari2001changes}{}}%
Vakkari, P. (2001a). Changes in search tactics and relevance judgements when preparing a research proposal a summary of the findings of a longitudinal study. \emph{Information Retrieval}, \emph{4}(3), 295--310.

\leavevmode\vadjust pre{\hypertarget{ref-vakkari2001theory}{}}%
Vakkari, P. (2001b). A theory of the task-based information retrieval process: A summary and generalisation of a longitudinal study. \emph{Journal of Documentation}, \emph{57}(1), 44--60. \url{https://doi.org/10.1108/EUM0000000007075}

\leavevmode\vadjust pre{\hypertarget{ref-vakkari2016searching}{}}%
Vakkari, P. (2016). Searching as learning: A systematization based on literature. \emph{Journal of Information Science}, \emph{42}(1), 7--18. \url{https://doi.org/10.1177/0165551515615833}

\leavevmode\vadjust pre{\hypertarget{ref-van2017development}{}}%
Van Der Vaart, R., \& Drossaert, C. (2017). Development of the digital health literacy instrument: Measuring a broad spectrum of health 1.0 and health 2.0 skills. \emph{Journal of Medical Internet Research}, \emph{19}(1), e27.

\leavevmode\vadjust pre{\hypertarget{ref-villa2013relevance}{}}%
Villa, R., \& Halvey, M. (2013). Is relevance hard work? Evaluating the effort of making relevant assessments. \emph{Proceedings of the 36th International ACM SIGIR Conference on Research and Development in Information Retrieval}, 765--768.

\leavevmode\vadjust pre{\hypertarget{ref-weber2019informationseeking}{}}%
Weber, H., Becker, D., \& Hillmert, S. (2019). Information-seeking behaviour and academic success in higher education: Which search strategies matter for grade differences among university students and how does this relevance differ by field of study? \emph{Higher Education}, \emph{77}(4), 657--678. \url{https://doi.org/10.1007/s10734-018-0296-4}

\leavevmode\vadjust pre{\hypertarget{ref-weber2018can}{}}%
Weber, H., Hillmert, S., \& Rott, K. J. (2018). Can digital information literacy among undergraduates be improved? Evidence from an experimental study. \emph{Teaching in Higher Education}, \emph{23}(8), 909--926. \url{https://doi.org/10.1080/13562517.2018.1449740}

\leavevmode\vadjust pre{\hypertarget{ref-white2016interactions}{}}%
White, R. (2016a). \emph{Interactions with search systems}. Cambridge University Press.

\leavevmode\vadjust pre{\hypertarget{ref-white-2016-iwss-learning}{}}%
White, R. (2016b). Learning and use. In \emph{Interactions with search systems} (pp. 231--248). {Cambridge University Press}. \url{https://doi.org/10.1017/CBO9781139525305.010}

\leavevmode\vadjust pre{\hypertarget{ref-white2009characterizing}{}}%
White, R., Dumais, S., \& Teevan, J. (2009). Characterizing the influence of domain expertise on web search behavior. \emph{Proceedings of the {Second ACM International Conference} on {Web Search} and {Data Mining} - {WSDM} '09}, 132. \url{https://doi.org/10.1145/1498759.1498819}

\leavevmode\vadjust pre{\hypertarget{ref-wildemuth2004effects}{}}%
Wildemuth, B. M. (2004). The effects of domain knowledge on search tactic formulation. \emph{Journal of the American Society for Information Science and Technology}, \emph{55}(3), 246--258. \url{https://doi.org/10.1002/asi.10367}

\leavevmode\vadjust pre{\hypertarget{ref-wilson2013comparison}{}}%
Wilson, M. J., \& Wilson, M. L. (2013). A comparison of techniques for measuring sensemaking and learning within participant-generated summaries. \emph{Journal of the American Society for Information Science and Technology}, \emph{64}(2), 291--306.

\leavevmode\vadjust pre{\hypertarget{ref-wilson1999models}{}}%
Wilson, T. D. (1999). Models in information behaviour research. \emph{Journal of Documentation}, \emph{55}(3), 249--270.

\leavevmode\vadjust pre{\hypertarget{ref-wineburg2016students}{}}%
Wineburg, S., \& McGrew, S. (2016). Why students can't google their way to the truth. \emph{Education Week}, \emph{36}(11), 22--28.

\leavevmode\vadjust pre{\hypertarget{ref-wineburg2017lateral}{}}%
Wineburg, S., \& McGrew, S. (2017). \emph{Lateral reading: Reading less and learning more when evaluating digital information}.

\leavevmode\vadjust pre{\hypertarget{ref-wittrock1989generative}{}}%
Wittrock, M. C. (1989). Generative processes of comprehension. \emph{Educational Psychologist}, \emph{24}(4), 345--376.

\leavevmode\vadjust pre{\hypertarget{ref-xu2020does}{}}%
Xu, L., Zhou, X., \& Gadiraju, U. (2020). How does team composition affect knowledge gain of users in collaborative web search? \emph{Conference on Hypertext and Social Media (HT)}.

\leavevmode\vadjust pre{\hypertarget{ref-yu2018PredictingUserKnowledgea}{}}%
Yu, R., Gadiraju, U., Holtz, P., Rokicki, M., Kemkes, P., \& Dietze, S. (2018). Predicting {User Knowledge Gain} in {Informational Search Sessions}. \emph{The 41st {International ACM SIGIR Conference} on {Research} \& {Development} in {Information Retrieval}}, 75--84. \url{https://doi.org/10.1145/3209978.3210064}

\leavevmode\vadjust pre{\hypertarget{ref-zhang2014towards}{}}%
Zhang, P., \& Soergel, D. (2014). Towards a comprehensive model of the cognitive process and mechanisms of individual sensemaking. \emph{Journal of the Association for Information Science and Technology}, \emph{65}(9), 1733--1756. \url{https://doi.org/10.1002/asi.23125}

\leavevmode\vadjust pre{\hypertarget{ref-zhang2016process}{}}%
Zhang, P., \& Soergel, D. (2016). Process patterns and conceptual changes in knowledge representations during information seeking and sensemaking: A qualitative user study. \emph{Journal of Information Science}, \emph{42}(1), 59--78.

\leavevmode\vadjust pre{\hypertarget{ref-zhang2011predicting}{}}%
Zhang, X., Cole, M., \& Belkin, N. (2011). Predicting {Users}' {Domain Knowledge} from {Search Behaviors}. \emph{Proceedings of the 34th {International ACM SIGIR Conference} on {Research} and {Development} in {Information Retrieval}}, 1225--1226. \url{https://doi.org/10.1145/2009916.2010131}

\leavevmode\vadjust pre{\hypertarget{ref-zimmerman1989social}{}}%
Zimmerman, B. J. (1989). A social cognitive view of self-regulated academic learning. \emph{Journal of Educational Psychology}, \emph{81}(3), 329.

\leavevmode\vadjust pre{\hypertarget{ref-zlatkin2021students}{}}%
Zlatkin-Troitschanskaia, O., Hartig, J., Goldhammer, F., \& Krstev, J. (2021). Students' online information use and learning progress in higher education \textendash{} {A} critical literature review. \emph{Studies in Higher Education}, 1--26. \url{https://doi.org/10.1080/03075079.2021.1953336}

\end{CSLReferences}

%%%%% REFERENCES


\end{document}
