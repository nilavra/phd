%%%%%%%%%%%%%%%%%%%%%%%%%%%%%%%%%%%%%%%%%%%%%%%%%%%%%%%%%%%%%%%
%% OXFORD THESIS TEMPLATE

% Use this template to produce a standard thesis that meets the Oxford University requirements for DPhil submission
%
% Originally by Keith A. Gillow (gillow@maths.ox.ac.uk), 1997
% Modified by Sam Evans (sam@samuelevansresearch.org), 2007
% Modified by John McManigle (john@oxfordechoes.com), 2015
% Modified by Ulrik Lyngs (ulrik.lyngs@cs.ox.ac.uk), 2018-, for use with R Markdown
%
% Ulrik Lyngs, 25 Nov 2018: Following John McManigle, broad permissions are granted to use, modify, and distribute this software
% as specified in the MIT License included in this distribution's LICENSE file.
%
% John commented this file extensively, so read through to see how to use the various options.  Remember that in LaTeX,
% any line starting with a % is NOT executed.

%%%%% PAGE LAYOUT
% The most common choices should be below.  You can also do other things, like replace "a4paper" with "letterpaper", etc.

% 'twoside' formats for two-sided binding (ie left and right pages have mirror margins; blank pages inserted where needed):
%\documentclass[a4paper,twoside]{templates/ociamthesis}
% Specifying nothing formats for one-sided binding (ie left margin > right margin; no extra blank pages):
%\documentclass[a4paper]{ociamthesis}
% 'nobind' formats for PDF output (ie equal margins, no extra blank pages):
%\documentclass[a4paper,nobind]{templates/ociamthesis}

% As you can see from the line below, oxforddown uses the a4paper size, 
% and passes in the binding option from the YAML header in index.Rmd:
\documentclass[a4paper, nobind]{templates/ociamthesis}


%%%%% ADDING LATEX PACKAGES
% add hyperref package with options from YAML %
\usepackage[pdfpagelabels]{hyperref}
% handle long urls
\usepackage{xurl}
% change the default coloring of links to something sensible
\usepackage{xcolor}

\definecolor{mylinkcolor}{RGB}{0,0,139}
\definecolor{myurlcolor}{RGB}{0,0,139}
\definecolor{mycitecolor}{RGB}{0,33,71}

\hypersetup{
  hidelinks,
  colorlinks,
  linktocpage=true,
  linkcolor=mylinkcolor,
  urlcolor=myurlcolor,
  citecolor=mycitecolor
}


% add float package to allow manual control of figure positioning %
\usepackage{float}

% enable strikethrough
\usepackage[normalem]{ulem}

% use soul package for correction highlighting
\usepackage{color, soulutf8}
\definecolor{correctioncolor}{HTML}{CCCCFF}
\sethlcolor{correctioncolor}
\newcommand{\ctext}[3][RGB]{%
  \begingroup
  \definecolor{hlcolor}{#1}{#2}\sethlcolor{hlcolor}%
  \hl{#3}%
  \endgroup
}
% stop soul from freaking out when it sees citation commands
\soulregister\ref7
\soulregister\cite7
\soulregister\citet7
\soulregister\autocite7
\soulregister\textcite7
\soulregister\pageref7

%%%%% FIXING / ADDING THINGS THAT'S SPECIAL TO R MARKDOWN'S USE OF LATEX TEMPLATES
% pandoc puts lists in 'tightlist' command when no space between bullet points in Rmd file,
% so we add this command to the template
\providecommand{\tightlist}{%
  \setlength{\itemsep}{0pt}\setlength{\parskip}{0pt}}
 
% allow us to include code blocks in shaded environments

% User-included things with header_includes or in_header will appear here
% kableExtra packages will appear here if you use library(kableExtra)
\usepackage{booktabs}
\usepackage{longtable}
\usepackage{array}
\usepackage{multirow}
\usepackage{wrapfig}
\usepackage{float}
\usepackage{colortbl}
\usepackage{pdflscape}
\usepackage{tabu}
\usepackage{threeparttable}
\usepackage{threeparttablex}
\usepackage[normalem]{ulem}
\usepackage{makecell}
\usepackage{xcolor}


%UL set section header spacing
\usepackage{titlesec}
% 
\titlespacing\subsubsection{0pt}{24pt plus 4pt minus 2pt}{0pt plus 2pt minus 2pt}


%UL set whitespace around verbatim environments
\usepackage{etoolbox}
\makeatletter
\preto{\@verbatim}{\topsep=0pt \partopsep=0pt }
\makeatother


%%%%%%% PAGE HEADERS AND FOOTERS %%%%%%%%%
\usepackage{fancyhdr}
\setlength{\headheight}{15pt}
\fancyhf{} % clear the header and footers
\pagestyle{fancy}
\renewcommand{\chaptermark}[1]{\markboth{\thechapter. #1}{\thechapter. #1}}
\renewcommand{\sectionmark}[1]{\markright{\thesection. #1}} 
\renewcommand{\headrulewidth}{0pt}

\fancyhead[LO]{\emph{\leftmark}} 
\fancyhead[RE]{\emph{\rightmark}} 




% UL page number position 
\fancyfoot[C]{\emph{\thepage}} %regular pages
\fancypagestyle{plain}{\fancyhf{}\fancyfoot[C]{\emph{\thepage}}} %chapter pages




%%%%% SELECT YOUR DRAFT OPTIONS
% This adds a "DRAFT" footer to every normal page.  (The first page of each chapter is not a "normal" page.)

% IP feb 2021: option to include line numbers in PDF

% for line wrapping in code blocks
\usepackage{fancyvrb}
\usepackage{fvextra}
\DefineVerbatimEnvironment{Highlighting}{Verbatim}{breaklines=true, breakanywhere=true, commandchars=\\\{\}}

% This highlights (in blue) corrections marked with (for words) \mccorrect{blah} or (for whole
% paragraphs) \begin{mccorrection} . . . \end{mccorrection}.  This can be useful for sending a PDF of
% your corrected thesis to your examiners for review.  Turn it off, and the blue disappears.
\correctionstrue


%%%%% BIBLIOGRAPHY SETUP
% Note that your bibliography will require some tweaking depending on your department, preferred format, etc.
% If you've not used LaTeX before, I recommend just using pandoc for citations -- this is what's used unless you specific e.g. "citation_package: natbib" in index.Rmd
% If you're already a LaTeX pro and are used to natbib or something, modify as necessary.

% this allows the latex template to handle pandoc citations
\newlength{\cslhangindent}
\setlength{\cslhangindent}{1.5em}
\newlength{\csllabelwidth}
\setlength{\csllabelwidth}{3em}
\newlength{\cslentryspacingunit} % times entry-spacing
\setlength{\cslentryspacingunit}{\parskip}
\newenvironment{CSLReferences}[2] % #1 hanging-ident, #2 entry spacing
 {% don't indent paragraphs
  \setlength{\parindent}{0pt}
  % turn on hanging indent if param 1 is 1
  \ifodd #1
  \let\oldpar\par
  \def\par{\hangindent=\cslhangindent\oldpar}
  \fi
  % set entry spacing
  \setlength{\parskip}{1mm}
  \setlength{\baselineskip}{6mm}
 }%
 {}
\usepackage{calc}
\newcommand{\CSLBlock}[1]{#1\hfill\break}
\newcommand{\CSLLeftMargin}[1]{\parbox[t]{\csllabelwidth}{#1}}
\newcommand{\CSLRightInline}[1]{\parbox[t]{\linewidth - \csllabelwidth}{#1}\break}
\newcommand{\CSLIndent}[1]{\hspace{\cslhangindent}#1}




% Uncomment this if you want equation numbers per section (2.3.12), instead of per chapter (2.18):
%\numberwithin{equation}{subsection}


%%%%% THESIS / TITLE PAGE INFORMATION
% Everybody needs to complete the following:
\title{LongSAL: A Longitudinal Study to Understand\\
University Students' Learning During Search}
\author{Nilavra Bhattacharya}
\college{School of Information}

% Master's candidates who require the alternate title page (with candidate number and word count)
% must also un-comment and complete the following three lines:

% Uncomment the following line if your degree also includes exams (eg most masters):
%\renewcommand{\submittedtext}{Submitted in partial completion of the}
% Your full degree name.  (But remember that DPhils aren't "in" anything.  They're just DPhils.)
\degree{Doctor of Philosophy}

% Term and year of submission, or date if your board requires (eg most masters)
\degreedate{May 2023}


%%%%% YOUR OWN PERSONAL MACROS
% This is a good place to dump your own LaTeX macros as they come up.

% To make text superscripts shortcuts
\renewcommand{\th}{\textsuperscript{th}} % ex: I won 4\th place
\newcommand{\nd}{\textsuperscript{nd}}
\renewcommand{\st}{\textsuperscript{st}}
\newcommand{\rd}{\textsuperscript{rd}}

%%%%% THE ACTUAL DOCUMENT STARTS HERE
\begin{document}

%%%%% CHOOSE YOUR LINE SPACING HERE
% This is the official option.  Use it for your submission copy and library copy:
\setlength{\textbaselineskip}{22pt plus2pt}
% This is closer spacing (about 1.5-spaced) that you might prefer for your personal copies:
%\setlength{\textbaselineskip}{18pt plus2pt minus1pt}

% You can set the spacing here for the roman-numbered pages (acknowledgements, table of contents, etc.)
\setlength{\frontmatterbaselineskip}{17pt plus1pt minus1pt}

% UL: You can set the line and paragraph spacing here for the separate abstract page to be handed in to Examination schools
\setlength{\abstractseparatelineskip}{13pt plus1pt minus1pt}
\setlength{\abstractseparateparskip}{0pt plus 1pt}

% UL: You can set the general paragraph spacing here - I've set it to 2pt (was 0) so
% it's less claustrophobic
\setlength{\parskip}{2pt plus 1pt}

%
% Customise title page
%
\def\crest{{\includegraphics[width=5cm]{templates/beltcrest.pdf}}}
\renewcommand{\university}{University of Texas at Austin}
\renewcommand{\submittedtext}{A thesis submitted for the degree of}
\renewcommand{\thesistitlesize}{\fontsize{22pt}{28pt}\selectfont}
\renewcommand{\gapbeforecrest}{25mm}
\renewcommand{\gapaftercrest}{25mm
}


% Leave this line alone; it gets things started for the real document.
\setlength{\baselineskip}{\textbaselineskip}


%%%%% CHOOSE YOUR SECTION NUMBERING DEPTH HERE
% You have two choices.  First, how far down are sections numbered?  (Below that, they're named but
% don't get numbers.)  Second, what level of section appears in the table of contents?  These don't have
% to match: you can have numbered sections that don't show up in the ToC, or unnumbered sections that
% do.  Throughout, 0 = chapter; 1 = section; 2 = subsection; 3 = subsubsection, 4 = paragraph...

% The level that gets a number:
\setcounter{secnumdepth}{2}
% The level that shows up in the ToC:
\setcounter{tocdepth}{1}


%%%%% ABSTRACT SEPARATE
% This is used to create the separate, one-page abstract that you are required to hand into the Exam
% Schools.  You can comment it out to generate a PDF for printing or whatnot.

% JEM: Pages are roman numbered from here, though page numbers are invisible until ToC.  This is in
% keeping with most typesetting conventions.
\begin{romanpages}

% Title page is created here
\maketitle

%%%%% DEDICATION
\begin{dedication}
  For Yihui Xie
\end{dedication}

%%%%% ACKNOWLEDGEMENTS


\begin{acknowledgements}
 	This is where you will normally thank your advisor, colleagues, family and friends, as well as funding and institutional support. In our case, we will give our praises to the people who developed the ideas and tools that allow us to push open science a little step forward by writing plain-text, transparent, and reproducible theses in R Markdown.

 We must be grateful to John Gruber for inventing the original version of Markdown, to John MacFarlane for creating Pandoc (\url{http://pandoc.org}) which converts Markdown to a large number of output formats, and to Yihui Xie for creating \texttt{knitr} which introduced R Markdown as a way of embedding code in Markdown documents, and \texttt{bookdown} which added tools for technical and longer-form writing.

 Special thanks to \href{http://chester.rbind.io}{Chester Ismay}, who created the \texttt{thesisdown} package that helped many a PhD student write their theses in R Markdown. And a very special thanks to John McManigle, whose adaption of Sam Evans' adaptation of Keith Gillow's original maths template for writing an Oxford University DPhil thesis in LaTeX provided the template that I in turn adapted for R Markdown.

 Finally, profuse thanks to JJ Allaire, the founder and CEO of \href{http://rstudio.com}{RStudio}, and Hadley Wickham, the mastermind of the tidyverse without whom we'd all just given up and done data science in Python instead. Thanks for making data science easier, more accessible, and more fun for us all.

 \begin{flushright}
 Ulrik Lyngs \\
 Linacre College, Oxford \\
 2 December 2018
 \end{flushright}
\end{acknowledgements}



%%%%% ABSTRACT


\renewcommand{\abstracttitle}{Abstract}
\begin{abstract}
	This \emph{R Markdown} template is for writing an Oxford University thesis. The template is built using Yihui Xie's \texttt{bookdown} package, with heavy inspiration from Chester Ismay's \texttt{thesisdown} and the \texttt{OxThesis} \LaTeX~template (most recently adapted by John McManigle).

This template's sample content include illustrations of how to write a thesis in R Markdown, and largely follows the structure from \href{https://ulyngs.github.io/rmarkdown-workshop-2019/}{this R Markdown workshop}.

Congratulations for taking a step further into the lands of open, reproducible science by writing your thesis using a tool that allows you to transparently include tables and dynamically generated plots directly from the underlying data. Hip hooray!
\end{abstract}



%%%%% MINI TABLES
% This lays the groundwork for per-chapter, mini tables of contents.  Comment the following line
% (and remove \minitoc from the chapter files) if you don't want this.  Un-comment either of the
% next two lines if you want a per-chapter list of figures or tables.
\dominitoc % include a mini table of contents

% This aligns the bottom of the text of each page.  It generally makes things look better.
\flushbottom

% This is where the whole-document ToC appears:
\tableofcontents

\listoffigures
	\mtcaddchapter
  	% \mtcaddchapter is needed when adding a non-chapter (but chapter-like) entity to avoid confusing minitoc

% Uncomment to generate a list of tables:
\listoftables
  \mtcaddchapter
%%%%% LIST OF ABBREVIATIONS
% This example includes a list of abbreviations.  Look at text/abbreviations.tex to see how that file is
% formatted.  The template can handle any kind of list though, so this might be a good place for a
% glossary, etc.
% First parameter can be changed eg to "Glossary" or something.
% Second parameter is the max length of bold terms.
\begin{mclistof}{List of Abbreviations}{3.2cm}

\item[1-D, 2-D]

One- or two-dimensional, referring \textbf{in this thesis} to spatial dimensions in an image.

\item[Otter]

One of the finest of water mammals.

\item[Hedgehog]

Quite a nice prickly friend.

\end{mclistof} 


% The Roman pages, like the Roman Empire, must come to its inevitable close.
\end{romanpages}

%%%%% CHAPTERS
% Add or remove any chapters you'd like here, by file name (excluding '.tex'):
\flushbottom

% all your chapters and appendices will appear here
\hypertarget{introduction}{%
\chapter{Introduction}\label{introduction}}

\hypertarget{sec:intro_overview}{%
\section{Search as Learning: Overview}\label{sec:intro_overview}}

Searching for information is a fundamental human activity. In the modern
world, it is frequently conducted by users interacting with online
search systems (e.g., web search engines), or more formally,
\textbf{Information Retrieval} (IR) systems. As early as in 1980, Bertam
Brookes, in his `fundamental equation' of information and knowledge, had
stated that an information searcher's current state of knowledge is
changed to a new knowledge structure by exposure to information
(\protect\hyperlink{ref-brookes1980foundations}{Brookes, 1980, p. 131}). This indicates that searchers acquire
new knowledge in the search process, and the same information will have
different effects on different searchers' knowledge states. Fifteen
years later, (\protect\hyperlink{ref-marchionini1995information}{Marchionini, 1995}) described information seeking
as ``a process, in which humans purposefully engage in order to change
their state of knowledge''. Thus, we have known for quite a while that
search is driven by higher-level human needs, and IR systems are a means
to an end, and not the end in itself. \textbf{Interactive information
retrieval} (IIR), a.k.a. human-computer information retrieval (HCIR)
(\protect\hyperlink{ref-marchionini2006toward}{Marchionini, 2006}) refers to the study and evaluation of users'
interaction with IR systems and users' satisfaction with the retrieved
information (\protect\hyperlink{ref-borlund2013interactive}{Borlund, 2013}).

Despite their technological marvels, modern IR systems falls short in
several aspects of fully satisfying the higher level human need for
information. In essence, IR systems are software that take, as input,
some query, and return as output some ranked list of resources.

``Within the context of information seeking, (search engines and IR
systems) \textbf{feel} like they play a prominent role in our lives, when in
actuality, they only play a small role: the \textbf{retrieval} part of
\[information\] \ldots{}

\begin{itemize}
\item
  Search engines \textbf{don't help us identify what we need} -- that's up
  to us; search engines don't question what we ask for, though they do
  recommend queries that use similar words.
\item
  Search engines \textbf{don't help us choose a source} -- though they are
  themselves a source, and a heavily marketed one, so we are certainly
  compelled to choose search engines over other sources, even when
  other sources might have better information.
\item
  Search engines \textbf{don't help us express our query} accurately or
  precisely -- though they will help with minor spelling corrections.
\item
  Search engines do help retrieve information---this is the primary
  part that they automate.
\item
  Search engines \textbf{don't help us evaluate the answers we retrieve} --
  it's up to us to decide whether the results are relevant, credible,
  true; Google doesn't view those as their responsibility.
\item
  Search engines \textbf{don't help us sensemake} -- we have to use our
  minds to integrate what we've found into our knowledge.''
\end{itemize}

-- (\protect\hyperlink{ref-ko2021seeking}{Ko, 2021})

In recent years, the IIR research community has been actively promoting
the \textbf{Search as Learning} (SAL) research direction. This fast growing
community of researchers propose that search environments should be
augmented and reconfigured to foster learning, sensemaking, and
long-term knowledge-gain. Various workshops and seminars have been
organized to develop research agendas at the interaction of IIR and the
Learning Sciences
(\protect\hyperlink{ref-agosti2014evaluation}{Agosti et al., 2014}; \protect\hyperlink{ref-allan2012frontiers}{Allan et al., 2012}; \protect\hyperlink{ref-collins2017search}{Collins-Thompson et al., 2017}; \protect\hyperlink{ref-freund2013searching}{Freund et al., 2013}, \protect\hyperlink{ref-freund2014searching}{2014}; \protect\hyperlink{ref-gwizdka2016search}{Gwizdka et al., 2016})
Additionally, special issues on Search as Learning have also been
published in the \emph{Journal of Information Science} (\protect\hyperlink{ref-hansen2016editorial}{Hansen \& Rieh, 2016})
and in the \emph{Information Retrieval Journal} (\protect\hyperlink{ref-eickhoff2017introduction}{Eickhoff et al., 2017}).
Articles in these special issued presented landmark literature reviews
(\protect\hyperlink{ref-rieh2016searching}{Rieh et al., 2016}; \protect\hyperlink{ref-vakkari2016searching}{Vakkari, 2016}), research agendas, and ideas
in this direction. Overall, these works generally advocate that future
research in this domain should aim to:

\begin{itemize}
\item
  understand the contexts in which people search to learn
\item
  understand factors that can influence learning outcomes
\item
  understand how search behaviours can predict learning outcomes
\item
  develop search systems to better support learning and sensemaking
\item
  help searchers be more critical consumers of information
\item
  understand the cognitive biases fostered by existing search systems
\item
  develop search engine ranking algorithms and interface tools that
  foster long term knowledge gain
\end{itemize}

Parallelly, the Educational Science and the Learning Science research
communities have also been organizing workshops and formulating research
agendas to conceptualize forms of `new learning'
(\protect\hyperlink{ref-cope2013new}{Cope \& Kalantzis, 2013}; \protect\hyperlink{ref-kalantzis2012newa}{Kalantzis \& Cope, 2012}; \protect\hyperlink{ref-newlondon1996pedagogy}{New London Group, 1996}) that are
afforded by innovations in digital technologies and e-learning ecologies
(\protect\hyperlink{ref-cope2017elearningc}{Cope \& Kalantzis, 2017}). Higher education researchers have been
increasingly studying how students' information search and information
use behaviour affect and support their learning
(\protect\hyperlink{ref-weber2019informationseeking}{Weber et al., 2019}, \protect\hyperlink{ref-weber2018can}{2018}; \protect\hyperlink{ref-zlatkin2021students}{Zlatkin-Troitschanskaia et al., 2021}).
Efforts are underway to conceptualize a theoretical framework around new
forms of e-Learning that is aided and afforded by digital technologies
(\protect\hyperlink{ref-amina2017active}{Amina, 2017}; \protect\hyperlink{ref-cope2017elearningc}{Cope \& Kalantzis, 2017}). In the community's own words:
``learning today is more about navigation, discernment, induction, and
synthesis'' of the wide body of information present ubiquitously at every
student's fingertips (\protect\hyperlink{ref-amina2017active}{Amina, 2017}). Therefore ``knowing the source,
finding the source, and using the information aptly is important to
learn and know now more than ever before'' (\protect\hyperlink{ref-cope2013new}{Cope \& Kalantzis, 2013}). All of these
interests in the intersection of searching and learning goes to
emphasise that understanding learning during search is critical to
improve human-information interaction.

\hypertarget{sec:intro_problem_statement}{%
\section{Problem Statement}\label{sec:intro_problem_statement}}

A major limitation in the area of Search as Learning, Interactive IR
(IIR), and more broadly, in Human-Computer Interaction (HCI) research is
that, the user is examined in the short-term, typically over the course
of a single experimental session in a lab
(\protect\hyperlink{ref-karapanos2021advances}{Karapanos et al., 2021}; \protect\hyperlink{ref-kelly2009evaluation}{Kelly et al., 2009}; \protect\hyperlink{ref-HCIUXres81_online}{Koeman, 2020}; \protect\hyperlink{ref-zlatkin2021students}{Zlatkin-Troitschanskaia et al., 2021}).
Very few studies exist in the search-as-learning domain that have
observed \emph{the same participant} over a longer period of time than a
single search session
(\protect\hyperlink{ref-kelly2006measuring_a}{Kelly, 2006a}, \protect\hyperlink{ref-kelly2006measuring_b}{2006b}; \protect\hyperlink{ref-kuhlthau2004seeking}{Kuhlthau, 2004}; \protect\hyperlink{ref-vakkari2001changes}{Vakkari, 2001b}; \protect\hyperlink{ref-white2009characterizing}{White et al., 2009}; \protect\hyperlink{ref-wildemuth2004effects}{Wildemuth, 2004}).
This ephemeral approach has acute implications in any domain where
learning is involved because ``learning is a \emph{process} that leads to
\emph{change} in knowledge \ldots{} (which) unfolds over time'' (\protect\hyperlink{ref-ambrose2010howa}{Ambrose et al., 2010}),
and ``\ldots does not happen all at once''(\protect\hyperlink{ref-white_2016_iwss_learning}{White, 2016b}).

\textbf{To the best of the author's knowledge, almost no new longitudinal
studies were reported in major search-as-learning literature in the last
five years, that systematically studied students' information search
behaviour and information-use over the long term, in their \emph{in-situ}
naturalistic environment and contexts, and linked those behaviours
quantitatively to the students' learning outcomes and individual
differences.} Higher education students are increasingly using the
Internet as their main learning environment and source of information
when studying. Yet, the short term nature of research in this domain
creates significant gaps in our knowledge regarding how students'
information search behaviour and information use develop over time, and
how it affects their learning (\protect\hyperlink{ref-zlatkin2021students}{Zlatkin-Troitschanskaia et al., 2021}). When research in
this area ``relies so heavily on (short-term) lab studies, can we
realistically say we are comprehensively studying human-tech
interactions -- when many of those interactions take place over long
periods of time in real-world contexts? \ldots An over-reliance on short
studies risks inaccurate findings, potentially resulting in prematurely
embracing or disregarding new concepts.'' (\protect\hyperlink{ref-HCIUXres81_online}{Koeman, 2020}).

Current search engines and information retrieval systems ``do not help us
know what we want to know, \ldots do not help us know if what we've found is
relevant or true; and they do not help us make sense of \[the retrieved
information\]. All they do is quickly retrieve what other people on the
internet have shared'' (\protect\hyperlink{ref-ko2021seeking}{Ko, 2021}). Unless we have more long-term
understanding of the nature of knowledge gain during search, the
limitations of current search systems will continue to persist.
Increased knowledge and understanding of students', and more broadly
searchers', information searching and learning behaviour over time will
help us to overcome the limitations of current IR systems, and transform
them into rich learning spaces where ``search experiences and learning
experiences are intertwined and even synergized'' (\protect\hyperlink{ref-url_rieh_homepage}{Rieh, 2020}).
The internet and digital educational technologies offer great
opportunities to transform learning and the education experience.
Enabled by our increased comprehension of the longitudinal
searching-as-learning process, improved and validated by empirical data,
we can create a new wave of fundamentally transformative educational
technologies and ``e-learning ecologies, that will be more engaging for
learners, more effective (than traditional classroom practices), more
resource efficient, and more equitable in the face of learner diversity''
(\protect\hyperlink{ref-cope2017elearningc}{Cope \& Kalantzis, 2017}).

\hypertarget{sec:intro_purpose}{%
\section{Purpose of this Dissertation Proposal}\label{sec:intro_purpose}}

To address the gaps in our knowledge of how information searching
influences students' learning process over time, this dissertation
proposal proposes to conduct a semester-long longitudinal study (approx.
16 weeks) with university student participants. The overarching research
aim is to identify how students' online searching behaviour correlate
with their learning outcomes for a particular university course.
Building upon principles from the Learning Sciences
(\protect\hyperlink{ref-ambrose2010howa}{Ambrose et al., 2010}; \protect\hyperlink{ref-council2000how}{National Research Council, 2000}; \protect\hyperlink{ref-novak2010learninga}{Novak, 2010}; \protect\hyperlink{ref-sawyer2005cambridge}{Sawyer, 2005}),
and empirical evidences from the Information Sciences
(\protect\hyperlink{ref-rieh2016searching}{Rieh et al., 2016}; \protect\hyperlink{ref-vakkari2016searching}{Vakkari, 2016}; \protect\hyperlink{ref-white2016interactions}{White, 2016a}),
this dissertation proposal aims to

\begin{itemize}
\item
  situate students as learners in their naturalistic contexts, and
  characterised by their individual differences,
\item
  measure students' information search and information use behaviour
  over time, and
\item
  correlate the information search behaviour with the learning
  outcomes for the university course.
\end{itemize}

Learning, or addressing a gap in one's knowledge, has been well
established as an important motivator behind information-seeking
activities (Section \protect\hyperlink{sec:intro_overview}{1.1}). Therefore, search systems that support
rapid learning across a number of searchers, and a range of tasks, can
be considered as more effective search systems (\protect\hyperlink{ref-white2016interactions}{White, 2016a, p. 310}). This dissertation proposal takes a step in this direction. ``It
opens great expectations for many-sided, great contribution to our
knowledge on the relations between search process and learning outcomes''
(\protect\hyperlink{ref-bhattacharya2021longitudinal}{Bhattacharya, 2021} anonymous reviewer).

\hypertarget{sec:intro_outline}{%
\section{Outline}\label{sec:intro_outline}}

This dissertation proposal document is structured as follows. First,
principles of learning and relevant background from the domain of
Educational Sciences are presented in Chapter 2. Next, relevant
empirical evidences from the Information Searching Literature are
discussed in Chapter 3. Chapter 4 presents the research questions, the
overarching hypotheses, and discusses their rationale in the context of
the existing research gaps. Chapter 5 describes the research methods,
including the longitudinal study design, experimental procedures, data
collection and analyses plans, anticipated limitations, and expected
schedule to complete the dissertation.

\hypertarget{ch:bg_learn}{%
\chapter{Background: Knowledge and Learning}\label{ch:bg_learn}}

This first chapter on background literature discusses relevant concepts
from the disciplines of Education and Learning Sciences. First, we
introduce some relevant terminology, and the concepts of deep or
meaningful learning. Then we discuss several research backed principles
that have been shown to lead to meaningful learning. Next, we discuss
how learning, sensemaking, and searching for information are related,
and how modern technologies provide affordances for new forms of
learning and knowledge work in the 21st century. We also discuss some
concepts about individual differences of learners as well as techniques
that can promote better learning. In the last section, we state what
implications these findings have for shaping the proposed study in this
dissertation proposal.

\hypertarget{sec:bg_learn_terminology}{%
\section{Terminology}\label{sec:bg_learn_terminology}}

The Webster dictionary\footnote{\url{https://developer.chrome.com/docs/extensions/reference/history/\#transition-types}} defines \textbf{knowledge} in two ways. The first
definition is ``the range of one's information or understanding''.
(\protect\hyperlink{ref-vakkari2016searching}{Vakkari, 2016}) says it is ``the totality what a person knows,
that is, a \textbf{personal knowledge} or \textbf{belief system}. It may include
both justified, true beliefs and less justified, not so true beliefs,
which the person more or less thinks hold true.'' Webster's second
definition of knowledge is ``the sum of what is known: the body of truth,
information, and principles acquired by humankind''. We can regard this
as \textbf{universal knowledge}.

\textbf{Learning} is a \emph{process}, that leads to a \emph{change} in (personal)
knowledge, beliefs, behaviours, and attitudes (\protect\hyperlink{ref-ambrose2010howa}{Ambrose et al., 2010}). Thus,
learning always aims to increase one's personal knowledge, and can often
draw from the body of universal knowledge. In some cases, the change in
personal knowledge can also lead to change in universal knowledge, such
as when new discoveries are made, or new philosophies are proposed.
Human learning is an innate capacity. It is longitudinal and unfolds
over time. Learning is lifelong and life-wide, and has a lasting impact
on how humans think and act (\protect\hyperlink{ref-ambrose2010howa}{Ambrose et al., 2010}; \protect\hyperlink{ref-kalantzis2012newa}{Kalantzis \& Cope, 2012}).
Learning can be informal or formal. \textbf{Informal learning} is the casual
learning taking place in everyday life, and is incidental to the
everyday life experience. \textbf{Formal learning} is the deliberate,
conscious, systematic, and explicit acquiring of knowledge
(\protect\hyperlink{ref-kalantzis2012newa}{Kalantzis \& Cope, 2012}).

\textbf{Education} is a form of formal learning. It is the systematic
acquiring of knowledge. In today's world, the institutions of education
are formally constructed places (classrooms), times (of the day and of
life) and social relations (teachers and students); for instance,
schools, colleges, and universities. The scientific discipline of
Education concerns itself with the systematic investigation of the ways
in which humans know and learn. It is the science of ``coming to know''
(\protect\hyperlink{ref-kalantzis2012newa}{Kalantzis \& Cope, 2012}).

\textbf{Pedagogy} describes small sequences of learner activities that
promote learning in educational settings (\protect\hyperlink{ref-kalantzis2012newa}{Kalantzis \& Cope, 2012}).
Traditional approaches to (classroom) pedagogy, especially the \emph{didactic
pedagogy}, primarily involves a teacher telling, and a learner
listening. The teacher is in command of the knowledge, and their mission
is to transmit this knowledge to the learners, in a one-way flow. It is
hoped that the learners will dutifully absorb the knowledge laid before
them by the teacher. The balance of agency weighs heavily towards the
teacher. ``There is a special focus on long-term memory, or retention,
measurable by the ritual of closed-book, summative examination''
(\protect\hyperlink{ref-cope2017elearningc}{Cope \& Kalantzis, 2017}).

\begin{figure}
\centering
\includegraphics{figs/deep_learning_surface_learning.pdf}
\caption{image}
\end{figure}

\begin{figure}
\hypertarget{fig_meaningful_learning}{%
\centering
\includegraphics{figs/meaningful_learning.pdf}
\caption{Meaningful learning (ake deep learning) as explained by
Novak (\protect\hyperlink{ref-novak2010learninga}{2010, fig. 5.3}) (annotations our own).}\label{fig_meaningful_learning}
}
\end{figure}

Cognitive scientists had discovered that learners retain material
better, and are able to generalize and apply it to a broader range of
contexts, when they learn \textbf{deep knowledge} rather than \textbf{surface
knowledge}, and when they learn how to use that knowledge in real-world
social and practical settings (\protect\hyperlink{ref-sawyer2005cambridge}{Sawyer, 2005}). Deep learning\footnote{\url{https://youtu.be/RkBUZ4At8Qg}}
takes place when ``the learner chooses conscientiously to integrate new
knowledge to knowledge that the learner already possesses'' and involves
``substantive, non-arbitrary incorporations of concepts into cognitive
structure'' (\protect\hyperlink{ref-novak2002meaningful}{Novak, 2002, p. 549}) and may eventually lead to the
development of transferable knowledge and skills. A parallel terminology
for deep learning (\protect\hyperlink{ref-marton1976qualitative_b}{Marton \& Säaljö, 1976}; \protect\hyperlink{ref-marton1976qualitative_a}{Marton \& Säljö, 1976})
is \textbf{meaningful learning}
(\protect\hyperlink{ref-ausubel1968educational}{Ausubel et al., 1968}; \protect\hyperlink{ref-novak2002meaningful}{Novak, 2002}), and they are often
contrasted with \emph{surface learning} or \emph{rote learning}. Table
\protect\hyperlink{tab_deep_learning_surface_learning}{\[tab_deep_learning_surface_learning\]} discusses some more
details on deep or meaningful learning, and the limitations of
traditional classroom practices to promote deep learning. Figure
\protect\hyperlink{fig_meaningful_learning}{1.1} describes (using a concept map) how
meaningful learning can be achieved and sustained, and our annotations
highlight how Search-as-learning systems can foster the same.

\hypertarget{sec:bg_learn_principles}{%
\section{Principles of Meaningful Learning}\label{sec:bg_learn_principles}}

(\protect\hyperlink{ref-ambrose2010howa}{Ambrose et al., 2010}) have proposed several principles of (student)
learning that lead to creation of deeper knowledge in learners, and help
educators understand why certain teaching approaches may help or hinder
learning. These principles are based on research and literature from a
range of disciplines in psychology, education, and anthropology, and the
authors claim they are domain independent, experience independent, and
cross-culturally relevant.

\begin{enumerate}
\def\labelenumi{\arabic{enumi}.}
\item
  Students' \textbf{prior knowledge} can help or hinder learning.
\item
  How students \textbf{organize knowledge} influences how they learn and
  apply what they know.
\item
  Students' \textbf{motivation} determines, directs, and sustains what they
  do to learn.
\item
  Goal-directed practice coupled with \textbf{targeted feedback} enhances
  the quality of students' learning.
\item
  Students' current level of development interacts with the social,
  emotional, and intellectual \textbf{context} around the student to impact
  learning.
\item
  To become \textbf{self-directed} learners, students must learn to
  \textbf{monitor and adjust} their approaches to learning.
\end{enumerate}

In line with the above, the US National Research Council identified
several key principles about \textbf{experts' knowledge} (\protect\hyperlink{ref-council2000how}{National Research Council, 2000}),
that illustrate the outcome of successful learning:

\begin{enumerate}
\def\labelenumi{\arabic{enumi}.}
\item
  Experts notice features and \textbf{meaningful patterns} of information
  that are not noticed by novices.
\item
  Experts have acquired a great deal of content knowledge that is
  \textbf{organized} in ways that reflect a deep understanding of their
  subject matter.
\item
  Experts' knowledge cannot be reduced to sets of isolated facts or
  propositions but, instead, reflects contexts of \textbf{applicability}:
  that is, the knowledge is `conditionalized' on a set of
  circumstances.
\item
  Experts are able to \textbf{flexibly retrieve} important aspects of their
  knowledge with little attentional effort.
\item
  Though experts know their disciplines thoroughly, this does not
  guarantee that they are able to teach others.
\item
  Experts have varying levels of flexibility in their approach to new
  situations.
\end{enumerate}

The principles of learning illustrate that both the \emph{context} of
learning, and the \emph{individual differences} of learners moderate the
learning process. The findings about expert knowledge suggests that
\emph{incorporating new information into existing knowledge structures} in a
meaningful manner is a key aspect of learning. We discuss these concepts
in more detail in the following sections.

\hypertarget{sec:bg_learn_sensemaking}{%
\section{Meaningful Learning as Sensemaking}\label{sec:bg_learn_sensemaking}}

In this section, we discuss how meaningful learning can be further
qualified using the concepts of sensemaking. \textbf{Sensemaking}\footnote{\url{https://www.nngroup.com/articles/f-shaped-pattern-reading-web-content}} is a
process that occurs when learners \emph{connect} their \emph{previously developed}
knowledge, ideas, abilities, and experiences together to address the
uncertainty presented by a newly introduced phenomenon, problem, or
piece of information (\protect\hyperlink{ref-ngss_sensemaking}{Next Generation Science Standards, 2021}). A significant portion of
learning is sensemaking, especially those which use recorded information
or systematic discovery to learn concepts, ideas, theories, and facts in
a domain (such as science or history) (\protect\hyperlink{ref-zhang2014towards}{P. Zhang \& Soergel, 2014}). The phrase
``figure something out'' is often synonymous with sensemaking. Sensemaking
is generally about actively trying to figure out the way the world
works, and/or exploring how to create or alter things to achieve desired
goals (\protect\hyperlink{ref-ngss_sensemaking}{Next Generation Science Standards, 2021}). (\protect\hyperlink{ref-dervin2010sensemaking}{Dervin \& Naumer, 2010}) distinguish work on
sensemaking in four fields: ``Human Computer Interaction (HCI) (Russell's
sensemaking); Cognitive Systems Engineering (Klein's sensemaking);
Organizational Communication (Weick's sensemaking; Kurtz and Snowden's
sense-making); and Library and Information Science (Dervin's
sense-making)''.

Many theories of learning and sensemaking revolve around the concept of
fitting new information into an existing or adapted knowledge structure
(\protect\hyperlink{ref-zhang2014towards}{P. Zhang \& Soergel, 2014}). The central idea is that knowledge is stored in
human memory as \emph{structures} or \emph{schemas}, which comprise interconnected
concepts and relationships. When new information is encountered or
acquired, the learner or sensemaker needs to actively construct a
revised or entirely new knowledge structure. Examples of some such
theories include: the \emph{assimilation theory (theory of meaningful
learning)}
(\protect\hyperlink{ref-ausubel1968educational}{Ausubel et al., 1968}; \protect\hyperlink{ref-ausubel2012acquisition}{Ausubel, 2012}; \protect\hyperlink{ref-novak2002meaningful}{Novak, 2002}; \protect\hyperlink{ref-novak2010learninga}{Novak, 2010});
the \emph{schema theory}
(\protect\hyperlink{ref-rumelhart1981accretion}{Rumelhart \& Norman, 1981}; \protect\hyperlink{ref-rumelhart1977representation}{Rumelhart \& Ortony, 1977}); and the
\emph{generative learning theory}
(\protect\hyperlink{ref-grabowski1996generative}{Grabowski, 1996}; \protect\hyperlink{ref-wittrock1989generative}{Wittrock, 1989}); all of which have
their foundations in the Piagetian concepts of \emph{assimilation} and
\emph{accommodation} (\protect\hyperlink{ref-piaget1936origins}{Piaget, 1936}).

\textbf{Assimilation} means addition of new information into an existing
knowledge structure. A ``synonym'' (\protect\hyperlink{ref-vakkari2016searching}{Vakkari, 2016}) for
assimilation is \textbf{accretion}, which is the gradual addition of factual
information to an existing knowledge structure, without structural
changes. Accretion does not change concepts and their relations in the
structure, but may populate a concept with new instances or facts.
\textbf{Accommodation} means modifying or changing existing knowledge
structures, by adding or removing concepts and their connections in the
knowledge structure. Accommodation is subdivided into \emph{tuning} /
\emph{weak-revision}, and \emph{restructuring}, based on the degree of structural
changes (\protect\hyperlink{ref-zhang2014towards}{P. Zhang \& Soergel, 2014}). \textbf{Tuning} or \textbf{weak revision} does not
include replacing concepts or connections between concepts in the
structure, but tuning of the scope and meaning of concepts and their
connections. This may include, for example, generalizing or specifying a
concept. \textbf{Restructuring} means radically changing and replacing
concepts and their connections in the existing knowledge structure, or
creating of new structures. Such radical changes often take place when
prior knowledge conflicts with new information. New structures are
constructed either to reinterpret old information or to account for new
information (\protect\hyperlink{ref-vakkari2016searching}{Vakkari, 2016}; \protect\hyperlink{ref-zhang2014towards}{P. Zhang \& Soergel, 2014}). A comparison of
these types of conceptual changes can be found in (\protect\hyperlink{ref-zhang2014towards}{P. Zhang \& Soergel, 2014} Table 3).

\hypertarget{sec:bg_concept_maps}{%
\subsection{Concept Maps to enhance Sensemaking}\label{sec:bg_concept_maps}}

\begin{figure}
\hypertarget{fig_concept_map_desc}{%
\centering
\includegraphics{figs/concept_map_desc.pdf}
\caption{Concept Map showing the key features of Concept Maps
(\protect\hyperlink{ref-moon2011applieda}{Moon et al., 2011, fig. 1.1}).}\label{fig_concept_map_desc}
}
\end{figure}

As we saw in the previous section, deep learning / meaningful learning /
sensemaking is a process in which new information is connected to a
relevant area of a learner's existing knowledge structure. However, the
\emph{learner must choose} to do this, and must actively seek a way to
integrate the new information with existing relevant information in
their cognitive structure
(\protect\hyperlink{ref-ausubel1968educational}{Ausubel et al., 1968}; \protect\hyperlink{ref-novak2010learninga}{Novak, 2010}). Learning facilitators
(e.g., teachers) can encourage this choice by using the concept mapping
technique.

A \textbf{concept-map} is a two-dimensional, hierarchical node-link diagram
(a \emph{graph} in Computer Science parlance) that depicts the structure of
knowledge within a discipline, as viewed by a student, an instructor, or
an expert in a field or sub-field. The map is composed of concept
labels, each enclosed in a box (graph \emph{nodes}); a series of labelled
linking lines (\emph{labelled edges}); and an inclusive, general-to-specific
organization (\protect\hyperlink{ref-halttunen2005assessing}{Halttunen \& Jarvelin, 2005}). Concept-maps assess how well
students see the `'big picture'', and where there are knowledge-gaps and
misconceptions. A \emph{mind map} is a diagram similar to a concept map,
comprising nodes and links between nodes. However, mind maps emerge from
a single centre, and have a more hierarchical, tree like structure.
Concept maps are more free-form, allowing multiple hubs and clusters.
Also, mind-maps have unlabelled links, and are subjective to the
creator. There are no ``correct'' relationships between nodes in a mind
map. Figure ~\protect\hyperlink{fig_concept_map_desc}{1.2} shows the key features of a concept
map, with the help of a concept map.

\textbf{Concept maps are therefore, arguably the most suited mechanism to
represent the cognitive knowledge structures, connections, and patterns
in a learner's mind}. Conventional tests, such as multiple choice
questions, are best at assessing students' recall of facts and guessing
skills. Their format treats information as distinct and separate items,
rather than interconnected pieces of a bigger picture. Concept maps on
the other hand, encourage learners to identify and make connections
between concepts that they know, and concepts that are new to them.
Concept maps have been used for over 50 years to provide a useful and
visually appealing way of illustrating and assessing learners'
conceptual knowledge
(\protect\hyperlink{ref-egusa2010usingb}{Egusa et al., 2010}, \protect\hyperlink{ref-egusa2014howd}{2014a}, \protect\hyperlink{ref-egusa2014howe}{2014b}, \protect\hyperlink{ref-egusa2017evaluating}{2017}; \protect\hyperlink{ref-halttunen2005assessing}{Halttunen \& Jarvelin, 2005}; \protect\hyperlink{ref-novak2010learninga}{Novak, 2010}; \protect\hyperlink{ref-novak1984learning}{Novak \& Gowin, 1984}).

Analysis of concept maps can reveal interesting patterns of learning and
thinking. Some of these measures that have been used by
(\protect\hyperlink{ref-halttunen2005assessing}{Halttunen \& Jarvelin, 2005}) are: addition, deletion, and differences in
top-level concept-nodes; depths of hierarchy; and number of concepts
that were ignored or changed fundamentally. In this regard,
(\protect\hyperlink{ref-novak1984learning}{Novak \& Gowin, 1984}) have presented well-established scoring schemes to
evaluate concept-maps: 1 point is awarded for each correct relationship
(i.e.~concept--concept linkage); 5 points for each valid level of
hierarchy; 10 points for each valid and significant cross-link; and 1
point for each example.

Having discussed how deep learning / meaningful learning / sensemaking
involves creation of knowledge structures in the learner's mind, and
suitably adding new pieces of information in the knowledge structure, we
now discuss how these processes are influenced in the 21st century with
the presence of new media, digital technologies, and information
retrieval systems.

\hypertarget{sec:bg_learn_active_knowledge_multiliteracy}{%
\section{`New' Learning as Online Information Searching}\label{sec:bg_learn_active_knowledge_multiliteracy}}

Digital media technologies and e-learning `ecologies' can enable new
forms and models of learning, that are fundamentally different from the
traditional classroom practices of didactic pedagogy
(\protect\hyperlink{ref-cope2017elearningc}{Cope \& Kalantzis, 2017}). Some key concepts associated with these forms of
`new learning' are described below. These concepts from the Educational
Sciences domain tie back strongly to the issues, challenges, and
research agenda being investigated by researchers in the Search as
Learning and Information Retrieval domain (Section
\protect\hyperlink{sec:intro_overview}{\[sec:intro_overview\]}).

\hypertarget{sec:bg_learn_active_knowledge_making}{%
\subsection{Active Knowledge Making}\label{sec:bg_learn_active_knowledge_making}}

The Internet and new forms of media provide us the opportunity to create
learning environments where learners are no longer mainly \emph{consumers} of
knowledge, but also \emph{modifiers}, \emph{producers}, and \emph{exchangers} of
knowledge. In \textbf{active knowledge making}, learners can, and often need
to, find information on their own using online resources. They are not
restricted to the textbook alone. The Internet is often a definitive
resource for information on any given topic. A learner can search the
web (to learn) at any time, from anywhere, on any web-enabled device.

As knowledge producers, learners search and analyze multiple sources
with differing and contradictory perspectives, and develop their own
observations and conclusions. In this process, they become researchers
themselves and learn to collaborate with peers in knowledge production.
Collaboration gives learners the opportunity to work with others as
coauthors of knowledge, peer reviewers, and discussants to completed
works. Because learners bring their own views, outlooks, and
experiences, the knowledge artefact they create is often uniquely voiced
instead of a templated ``correct'' response (\protect\hyperlink{ref-amina2017active}{Amina, 2017}).

Learners become \textbf{active knowledge producers} (for instance,
project-based learning, using multiple knowledge sources, and research
based knowledge making), and not merely knowledge consumers (as
exemplified in the `transmission' pedagogies of traditional textbook
learning or e-learning focused on video or e-textbook delivery). Active
knowledge making practices underpin contemporary emphases on innovation,
creativity and problem solving, which are quintessential `knowledge
economy' and `knowledge society' attributes. -- (\protect\hyperlink{ref-cope2017elearningc}{Cope \& Kalantzis, 2017})

\hypertarget{sec:bg_learn_artefact}{%
\subsection{Artefacts for Learning Assessment}\label{sec:bg_learn_artefact}}

Traditionally, the focus of learning outcomes has been long term memory.
Students and learners were expected to remember a collection of facts,
definitions, proofs, equations, and other associated details. For a
significant amount of modern knowledge-work today, \textbf{memory is actually
less important}. Information is so readily accessible now that it is no
longer necessary to remember the information. Because of the
technological phenomenon, the mass of information is available
ubiquitously\footnote{and continue to link} to a learner (or a knowledge worker), in every moment
of learning. Empirical details such as facts, definitions, proofs, or
equations do not need to be remembered today, because they can always be
looked up again (\protect\hyperlink{ref-amina2017active}{Amina, 2017}; \protect\hyperlink{ref-cope2017elearningc}{Cope \& Kalantzis, 2017}).

This creates an interesting shift in the focus of learning and knowledge
work today: \emph{``if we are not going to measure and value long-term memory
in education, what are we going to assess?''} (\protect\hyperlink{ref-cope2017elearningc}{Cope \& Kalantzis, 2017})
suggest that \textbf{we assess the knowledge artefacts} that learners
produce. In active knowledge making, the final work\footnote{be it a project report, poster, presentation, video, software,
  research paper, website, etc.} can be proof of
the learning outcome and represent a learner's ability to use the
resources that are available (\protect\hyperlink{ref-amina2017active}{Amina, 2017}). \textbf{Measure of learning
can be measure of information quality and information use in
artefacts.} This shows a shift in pedagogy and assessment and an
increase in personalization and individualization of learning
(\protect\hyperlink{ref-pea2014learning}{Pea \& Jacks, 2014}). Memorizing the information on a topic is less
important, compared to the writing, synthesizing, analyzing, and
\textbf{sensemaking} of the available information that has been referenced in
the work. This shifts the focus of assessment to the quality of the
artefacts and the processes of their construction. Moreover, as
technology increases the ability to capture detailed data from formal
and informal learning activities, it can give us a new view of how
learners progress in acquiring knowledge, skills, and attributes
(\protect\hyperlink{ref-dicerbo2014impacts}{DiCerbo \& Behrens, 2014}). Because learning is a continuous, longitudinal
process, these advanced, technologically enhanced assessments are more
useful in understanding the learning process and knowledge development
(\protect\hyperlink{ref-amina2017active}{Amina, 2017}).

Assessing open-ended artefacts does come with its challenges and
limitations. First, assessing and grading artefacts requires the
development of detailed qualitative coding guides
(\protect\hyperlink{ref-wilson2013comparison}{M. J. Wilson \& Wilson, 2013}). This process involves defining grading criteria
and measuring inter-coder agreement to ensure that the coding guide is
reliable. Prior studies have scored summaries along dimensions such as
the inclusion of facts, relationships between facts, and evaluative
statements (\protect\hyperlink{ref-lei2015effect}{Lei et al., 2015}; \protect\hyperlink{ref-roy2021note}{Roy et al., 2021}; \protect\hyperlink{ref-wilson2013comparison}{M. J. Wilson \& Wilson, 2013}).
Second, the quality of responses may be difficult to compare across
learners. Since this type of assessment imposes very few constraints on
the learners' responses, it may cause some learners to \emph{satisfice}, and
not convey everything that was learned. Additionally, writing skills are
likely to vary across learners, and some may not be able to effectively
articulate everything that was learnt.

\hypertarget{sec:bg_learn_info_eval}{%
\subsection{`Information Search and Evaluation' as and for Learning}\label{sec:bg_learn_info_eval}}

Learning today is more about \textbf{navigation, discernment, induction, and
synthesis}, and less about memory and deduction (\protect\hyperlink{ref-cope2013new}{Cope \& Kalantzis, 2013}).
However, knowing the source, finding the source, and using the
information critically is important to learn and know now more than ever
before (\protect\hyperlink{ref-amina2017active}{Amina, 2017}). Learners must know the social sources of
knowledge and understand and correctly use quotations, paraphrases,
remixes, links, citations, and the like in the works that they develop.
Searching and sourcing from the web entails a process of developing and
completing a work that inevitably makes learners \textbf{knowledge
producers}, as long as they can navigate and critically discern the
value of multiple sources. This is a skill that must be learned, as many
sources of information are not valid, reliable, or authentic
(\protect\hyperlink{ref-mcgrew2018can}{McGrew et al., 2018}; \protect\hyperlink{ref-wineburg2016students}{Wineburg \& McGrew, 2016}). Understanding the different
sources and identifying the more reliable ones are essential for
effective teaching and learning
(\protect\hyperlink{ref-mcgrew2017challenge}{McGrew et al., 2017}; \protect\hyperlink{ref-mcgrew2021skipping}{McGrew, 2021}). This is a critical aspect
because the inability to cite properly or to use reliable resources
provides learners with misconstrued information and ideas
(\protect\hyperlink{ref-amina2017active}{Amina, 2017}; \protect\hyperlink{ref-breakstone2021students}{Breakstone et al., 2021}; \protect\hyperlink{ref-mcgrew2017challenge}{McGrew et al., 2017}).

The Stanford History Education Group (SHEG) conceptualised the \textbf{Civic
Online Reasoning} (COR) curriculum\footnote{\url{https://cor.stanford.edu}} to enable students to
effectively search for and evaluate online information
(\protect\hyperlink{ref-breakstone2021students}{Breakstone et al., 2021}; \protect\hyperlink{ref-breakstone2018we}{Breakstone et al., 2018}; \protect\hyperlink{ref-mcgrew2020learning}{McGrew, 2020}). The
curriculum centres on asking three questions of any digital content:
\emph{(i)} who is behind a piece of information? \emph{(ii)} what is the evidence
for a claim? \emph{(iii)} what do other sources say? The curriculum has
lessons and assessments for information evaluation skills such as
lateral reading (\protect\hyperlink{ref-wineburg2017lateral}{Wineburg \& McGrew, 2017}), identifying news versus
opinions, checking domain names, identifying sponsored content,
evaluating evidence, and practising click restraint (\protect\hyperlink{ref-mcgrew2021click}{McGrew \& Glass, 2021}).
The lessons were developed and piloted by the Stanford History Education
Group (\protect\hyperlink{ref-mcgrew2018can}{McGrew et al., 2018}; \protect\hyperlink{ref-mcgrew2020learning}{McGrew, 2020}; \protect\hyperlink{ref-mcgrew2021click}{McGrew \& Glass, 2021}). Taken
together, these strategies will allow academics and students to better
evaluate digital content, from the perspectives of professional fact
checkers.

The purview of the \emph{Civic Online Reasoning} curriculum is more targeted
than the expansive fields of media and digital literacy\footnote{``Digital literacy describes a holistic approach to cultivating
  skills that allow people to participate meaningfully in online
  communities, interpret the changing digital landscape, understand
  the relationships between systemic -isms and information, and unlock
  the power of digital tools for good. This includes media literacy.
  Terms like critical media literacy, media literacy, news literacy,
  and more are not necessarily interchangeable.'' --
  (\protect\hyperlink{ref-collins2021reimagining}{Collins, 2021})}, (which can
embrace topics ranging from cyberbullying to identity theft). Civic
Online Reasoning focuses squarely on how to sort fact from fiction
online, a prerequisite for responsible civic engagement in the
twenty-first century
(\protect\hyperlink{ref-breakstone2021students}{Breakstone et al., 2021}; \protect\hyperlink{ref-kahne2012digital}{Kahne et al., 2012}; \protect\hyperlink{ref-mihailidis2013media}{Mihailidis \& Thevenin, 2013}).

\hypertarget{sec:bg_learn_promoting_learning}{%
\section{Promoting Better Learning}\label{sec:bg_learn_promoting_learning}}

``It is not the technology that makes a difference; it is the pedagogy.''
~\\
-- (\protect\hyperlink{ref-cope2017elearningc}{Cope \& Kalantzis, 2017})

Having discussed how meaningful learning takes place, and how it is
influenced by the presence of digital media and the mass of information
on the Internet, let us now look deeper into the learners as persons
themselves. In this section, we discuss how different cognitive and
metacognitive practices and aspects of learners can promote better
learning. These phenomena have important implications for any digital
systems that aim to foster learning.

\hypertarget{sec:bg_learn_articulation}{%
\subsection{Externalization and Articulation}\label{sec:bg_learn_articulation}}

The learning sciences have discovered that when learners externalize and
articulate their developing knowledge, they learn more effectively
(\protect\hyperlink{ref-council2000how}{National Research Council, 2000}). Best learning takes place when learners articulate
their unformed and still developing understanding, and continue to
articulate it throughout the process of learning. This phenomenon was
first studied in the 1920s by Russian psychologist Lev Vygotsky.
Articulating and learning go hand in hand, in a mutually reinforcing
feedback loop. Often learners do not actually learn something until they
start to articulate it. While thinking out loud, they learn more rapidly
and deeply than while studying quietly (\protect\hyperlink{ref-sawyer2005cambridge}{Sawyer, 2005}). The
learning sciences community is actively researching how to support
students in their ongoing process of articulation, and which forms of
articulation are the most beneficial to learning. Articulation is more
effective if it is scaffolded -- channelled so that certain kinds of
knowledge are articulated, and in a certain form that is most likely to
result in useful reflection (\protect\hyperlink{ref-sawyer2005cambridge}{Sawyer, 2005}). Students need help
in articulating their developing understandings, as they do not yet know
how to think about thinking, or talk about thinking; their knowledge
state is \emph{anomalous} (\protect\hyperlink{ref-belkin1982ask}{Belkin et al., 1982}).

\hypertarget{sec:bg_learn_metacognition}{%
\subsection{Metacognition and Reflection}\label{sec:bg_learn_metacognition}}

\begin{figure}
\centering
\includegraphics{figs/metacognition_components.pdf}
\caption{image}
\end{figure}

One of the reasons that articulation is so helpful to learning is that
it promotes \emph{reflection} or \emph{metacognition}. \textbf{Metacognition}, commonly
referred to as thinking about thinking, involves thinking at a higher
level of abstraction, which in turn improves thinking and learning
(\protect\hyperlink{ref-blanken2017metacognition}{Blanken-Webb, 2017}). It is ``the process of reflecting on and
directing one's own thinking'' (\protect\hyperlink{ref-council2000how}{National Research Council, 2000, p. 78}), and involves
thinking about the process of learning, and thinking about knowledge.
This ties forward to the self-regulation that effective learners exhibit
(Section \protect\hyperlink{sec:bg_learn_self_regulation}{1.5.4}. Effective learners are aware
of their learning process, and can measure how efficiently they are
learning as they study.

The literature on metacognition broadly identifies two fundamental
components of metacognition: knowledge about cognition, and regulation
of cognition. \textbf{Knowledge about cognition} includes three subprocesses
that facilitate the \emph{reflective} aspect of metacognition: declarative
knowledge (knowledge about self and about strategies), procedural
knowledge (knowledge about how to use strategies), and conditional
knowledge (knowledge about when and why to use strategies). \textbf{Regulation
of cognition} include a number of subprocesses that facilitate the
\emph{control} aspect of learning. Five component skills of regulation have
been discussed extensively in the literature, including planning,
information management strategies, comprehension monitoring, debugging
strategies, and evaluation. The operational definitions of these
components are described in Table
\protect\hyperlink{tab_metacognition_components}{\[tab_metacognition_components\]}. (\protect\hyperlink{ref-schraw1994assessing}{Schraw \& Dennison, 1994})
developed the \textbf{Metacognitive Awareness Inventory} (MAI) survey and a
scoring guide to measure these self-reported components and subprocesses
of metacognition. The original survey consists of 52 true/false
questions (Appendix \protect\hyperlink{app:mai}{\[app:mai\]}), such as ``I consider several alternatives to a
problem before I answer'', ``I understand my intellectual strengths and
weaknesses'', ``I have control over how well I learn'', and ``I change
strategies when I fail to understand''. The instrument has been widely
used in research, and has its reliability and validity measures
available. Later, (\protect\hyperlink{ref-terlecki2018call}{Terlecki \& McMahon, 2018}) proposed a revised version of the
MAI, using five-point Likert-scales, ranging from ``I never do this'' to
``I do this always''. They argue that when measuring change in
metacognition over time, the Likert-scale based `how often' questions
are more effective than dichotomous `Yes/No' questions
(\protect\hyperlink{ref-terlecki2020revising}{Terlecki, 2020}; \protect\hyperlink{ref-terlecki2018call}{Terlecki \& McMahon, 2018}).

\hypertarget{motivation}{%
\subsection{Motivation}\label{motivation}}

\begin{figure}
\hypertarget{fig_sdt_components}{%
\centering
\includegraphics{figs/sdt_components.pdf}
\caption{The motivation and self-determination continuum, as proposed by the
Self-Determination Theory (SDT). Figure adapted from
(\protect\hyperlink{ref-guyan2013improving}{Guyan, 2013}; \protect\hyperlink{ref-ryan2000intrinsic}{Ryan \& Deci, 2000a}, \protect\hyperlink{ref-ryan2000self}{2000b}).}\label{fig_sdt_components}
}
\end{figure}

\textbf{Motivation} is the process that initiates, guides, and maintains
goal-oriented behaviours (\protect\hyperlink{ref-cherry2020what}{Cherry, 2020}). The \textbf{Self-Determination
Theory} (SDT) represents a broad framework for the study of human
motivation and personality (\protect\hyperlink{ref-ryan2017self}{Ryan \& Deci, 2017}). SDT differentiates the types
of motivation based on the reasons that give rise to behaviour:
intrinsic motivation and extrinsic motivation. \textbf{Intrinsic motivation}
is engaging in a task or behaviour for the rewards \emph{inside} the task or
behaviour, such the pleasure, enjoyment and satisfaction that the
behaviour provides. It is a stable form of motivation. \textbf{Extrinsic
motivation} is engaging in a task or behaviour for the rewards
\emph{outside} the task or behaviour, such as receiving rewards, avoidance of
punishment, gaining social approval, or achievement of a valued result.
Extrinsic motivation is on a continuum from less stable to more stable,
as illustrated in Figure \protect\hyperlink{fig_sdt_components}{1.3}. Extrinsic motivation does not last
unless the rewards and punishments are explicitly visible
(\protect\hyperlink{ref-deci2013intrinsic}{Deci \& Ryan, 2013}; \protect\hyperlink{ref-ryan2000self}{Ryan \& Deci, 2000b}; \protect\hyperlink{ref-tahamtan2019effect}{Tahamtan, 2019}).

(\protect\hyperlink{ref-ryan1982control}{Ryan, 1982}) proposed the \textbf{Intrinsic Motivation Inventory} (IMI)
(Appendix \protect\hyperlink{app:imi}{\[app:imi\]}), a multidimensional questionnaire intended to
assess participants' subjective experience related to a target activity
in laboratory experiments. The instrument assesses participants'
interest/enjoyment, perceived competence, effort, value/usefulness, felt
pressure and tension, and perceived choice while performing a given
activity, yielding six subscale scores. The \emph{interest/enjoyment}
subscale is considered the most indicative self-report measure of
intrinsic motivation. The \emph{perceived choice} and \emph{perceived competence}
concepts are theorized to be positive predictors of both self-report and
behavioral measures of intrinsic motivation. The \emph{pressure/tension} is
theorized to be a negative predictor of intrinsic motivation. \emph{Effort}
is a separate variable that is relevant to some motivation questions, so
it is used if its relevant. The \emph{value/usefulness} subscale is used to
measure internalization, with the idea being that people internalize and
become self-regulating with respect to activities that they experience
as useful or valuable for themselves.

\hypertarget{sec:bg_learn_self_regulation}{%
\subsection{Self-regulation}\label{sec:bg_learn_self_regulation}}

\textbf{Self-regulation} is the ability to develop, implement, and flexibly
maintain planned behaviour in order to achieve one's goals.
Self-regulation, and more broadly, self-direction, are critical to being
an effective ``lifelong'' learner. Self-regulation becomes increasingly
important at higher levels of education and in professional life, as
people take on more complex tasks and greater responsibilities for their
own learning. However, these metacognitive skills tend to fall outside
the content area of most courses, and therefore, often neglected in
instruction (\protect\hyperlink{ref-ambrose2010howa}{Ambrose et al., 2010, p. 191}). Building on the foundational work
of (\protect\hyperlink{ref-kanfer1970self_a}{Kanfer, 1970b}, \protect\hyperlink{ref-kanfer1970self_b}{1970a}), Miller and Brown formulated a
seven-step model of self-regulation (\protect\hyperlink{ref-brown1998self}{J. Brown, 1998}; \protect\hyperlink{ref-miller1991self}{W. R. Miller \& Brown, 1991}).
In this model, behavioural self-regulation may falter because of failure
or deficits at any of these seven steps: \emph{(i)} receiving relevant
information, \emph{(ii)} evaluating the information and comparing it to
norms, \emph{(iiii)} triggering change, \emph{(iv)} searching for options, \emph{(v)}
formulating a plan, \emph{(vi)} implementing the plan, and \emph{(vii)} assessing
the plan's effectiveness (which recycles to steps \emph{(i)} and \emph{(ii)}).
Although this model was developed specifically to study addictive
behaviours, the self-regulatory processes it describes are meant to be
general principles of behavioural self-control. (\protect\hyperlink{ref-brown1999self}{J. M. Brown et al., 1999})
developed the \textbf{Self-Regulation Questionnaire} (SRQ) (Appendix
\protect\hyperlink{app:srq}{\[app:srq\]}) to
assess these self-regulatory processes through self-report. The items
were developed to mark each of the seven sub-processes of the
(\protect\hyperlink{ref-miller1991self}{W. R. Miller \& Brown, 1991}) model, forming seven subscales of the SRQ. The 63-item
scale elicits responses in the form of 5-point Likert scale, ranging
from strongly disagree to strongly agree. Based on clinical and college
samples, the authors tentatively recommend a score of 239 and above as
high (intact) self-regulation capacity (top quartile), 214-238 as
intermediate (moderate) self-regulation capacity (middle quartiles), and
213 and below as low (impaired) self-regulation capacity (bottom
quartile).

\hypertarget{self-directed-and-self-regulated-learning}{%
\subsubsection{Self-directed and Self-regulated Learning}\label{self-directed-and-self-regulated-learning}}

As we saw in the previous sections, self-regulation, motivation, and
metacognition are key concepts that moderate the learning process. These
terms are couched in the concepts of self-regulated learning and
self-directed learning.

\textbf{Self-directed learning} (SDL) is a ``process in which individuals take
the initiative, with or without the help from others, in diagnosing
their learning needs, formulating goals, identifying human and material
resources, choosing and implementing appropriate learning strategies,
and evaluating learning outcomes''(\protect\hyperlink{ref-knowles1975self}{Knowles, 1975, p. 18}).
\textbf{Self-regulated learning} (SRL) can be described as the degree to
which students are ``metacognitively, motivationally, and behaviourally
active participants in their own learning process'' (\protect\hyperlink{ref-zimmerman1989social}{Zimmerman, 1989, p. 329}).

\begin{figure}
\hypertarget{fig_sdl_v_srl}{%
\centering
\includegraphics{figs/sdl_v_srl.pdf}
\caption{Self-directed learning vs.~self-regulated learning, as illustrated by
(\protect\hyperlink{ref-saks2014distinguishing}{Saks \& Leijen, 2014}).}\label{fig_sdl_v_srl}
}
\end{figure}

Often used interchangeably, self-directed learning (SDL) and
self-regulated learning (SRL) have some important similarities and
differences (Figure \protect\hyperlink{fig_sdl_v_srl}{1.4}) (\protect\hyperlink{ref-saks2014distinguishing}{Saks \& Leijen, 2014}). SDL, originating
from adult education, is a broader, macro-level construct, and is
usually practised outside the traditional school environment. The
self-directed learner is free to design their own learning environment,
and free to plan and set their own learning goals. SRL, on the other
hand, is a narrower, micro-level construct, originating from educational
and cognitive psychology, and is mostly utilized in the school
environment. Learners do not have as much freedom as in SDL. The
instructor or facilitator often defines the learning task and the
learning goals. Self-directed learning may include self-regulated
learning, but the converse is not true
(\protect\hyperlink{ref-jossberger2010challenge}{Jossberger et al., 2010}; \protect\hyperlink{ref-loyens2008selfdirected}{Loyens et al., 2008}). In other words, \emph{``a
self-directed learner is supposed to self-regulate, but a self-regulated
learner may not self-direct''} (\protect\hyperlink{ref-saks2014distinguishing}{Saks \& Leijen, 2014}). Despite their
differences, SDL and SRL share key similarities
(\protect\hyperlink{ref-saks2014distinguishing}{Saks \& Leijen, 2014}). First, both can be seen in two dimensions:
\emph{(i)} \emph{external} to the learner, as a process or series of events, and
\emph{(ii)} \emph{internal} to the learner, arising from the learner's
personality, aptitude, and individual differences. Second, both the
learning processes have four key phases: \emph{(i)} defining tasks, \emph{(ii)}
setting goals and planning, \emph{(iii)} enacting strategies, and \emph{(iv)}
monitoring and reflecting. Third, both SDL and SRL require active
participation, goal-directed behaviour, metacognition, and intrinsic
motivation.

In summary, metacognition is monitoring and controlling what is in the
learner's head; self-regulation is monitoring and controlling how the
learner interacts with their environment; self-regulated learning is the
application of metacognition and self-regulation to learning
(\protect\hyperlink{ref-mannion2020metacognition}{Mannion, 2020}); and the whole learning process is sustained
by motivation, which is desirable to be intrinsic.

\hypertarget{sec:bg_learn_summary}{%
\section{Summary and Implications for this Proposal}\label{sec:bg_learn_summary}}

In this first chapter of the background literature review, we discussed
\emph{(i)} what is meaningful learning, a.k.a. deep learning, or sensemaking;
\emph{(ii)} how meaningful learning updates the learner's cognitive knowledge
structure; \emph{(iii)} how the learning process is influenced by digital
technologies, mass of information on the Internet, and IR systems; and
\emph{(iv)} what principles and practices learners and educators must realize
and follow to promote meaningful learning. These findings are from the
domains of Educational Sciences, Learning Sciences and Cognitive
Sciences. We argue that these are important aspects to be considered
when designing future IR or educational information systems that aim to
combine and improve the searching and learning experience.

Guided by these findings, we make some important decision choices for
the proposed longitudinal study in this dissertation proposal. We aim to
situate learners in their context, and incorporate their individual
differences using metacognition, motivation, and self-regulation
characteristics. Additionally, we aim to assess learning using artefacts
and concept maps. We choose not use traditional tests like
question-answers, and multiple choice assignments, since they are often
not the preferred choice of knowledge-work output in real world
scenarios. Concept maps are better suited to represent the learning and
sensemaking process, and artefacts are better able to demonstrate a
learner's knowledge work.

In the next chapter, we look at relevant literature from the Information
Sciences and Interactive Information Retrieval disciplines.

\hypertarget{ch:bg_search}{%
\chapter{Background: Information Searching}\label{ch:bg_search}}

This second chapter on background literature discusses relevant concepts
from the disciplines of Information Sciences, and more specifically
Interaction Information Retrieval. First, we introduce some terminology
around information behaviour, information need, and information
relevance. Then we discuss relevant findings various empirical studies,
from the lens of three-stage interactions in the information search
process. Then we discuss some overall generic characteristics of
information search behaviour, and how they are linked to expertise and
working memory. Next we discuss how learning has been assessed in recent
search-as-learning studies. We also discuss some limitations of current
search systems to foster learning, including the lack of sufficient
number of longitudinal studies. In the last section, we state what
implications these findings have for shaping the proposed study in this
dissertation proposal.

\hypertarget{sec:bg_search_terminology}{%
\section{Terminology}\label{sec:bg_search_terminology}}

\textbf{Information retrieval} (IR) is the process of obtaining \emph{information
objects}, that are \emph{relevant} to an \emph{information need}, from a
collection of those objects (Wikipedia). \textbf{Information objects} are
entities that can potentially convey information. They can take many
forms, such as documents, webpages, facts, music, spoken words, images,
videos, artefacts, and other forms of human expression. Areas where
information retrieval techniques are employed include search engines,
such as web search, social search, and desktop search; media search, as
in image, music, video; digital libraries and recommender systems, as
well as domain specific applications like geographical information
systems, e-Commerce websites, legal information search, and others.

Multiple perspectives exist around how users interact with information,
and IR systems. In the \textbf{Search Engine application view}, the
interactions are restricted to the search engine interface. In the
\textbf{Human-computer interaction} (HCI) view, interactions are between a
person and a system; but the system can go \emph{beyond} supporting only
retrieval, to supporting more complex tasks. In the \textbf{cognitive view of
IR}, which is the broadest, the interactions for obtaining information
can be between a person and a system, as well as between people, for
retrieval of information.

\begin{figure}
\hypertarget{fig_wilson_info_behaviour}{%
\centering
\includegraphics{figs/wilson_info_behaviour.pdf}
\caption{Nested model of information behaviour by (\protect\hyperlink{ref-wilson1999models}{T. D. Wilson, 1999}).}\label{fig_wilson_info_behaviour}
}
\end{figure}

People's behaviour around information can be modelled as a nested Venn
diagram as proposed by (\protect\hyperlink{ref-wilson1999models}{T. D. Wilson, 1999}) (Figure
\protect\hyperlink{fig_wilson_info_behaviour}{1.1}). \textbf{Information behaviour} is
the more general field of investigation. \textbf{Information-seeking
behaviour} can be seen as a sub-set of the field, particularly
concerned with the variety of methods people employ to discover, and
gain access to information objects. \textbf{Information search behaviour} is
yet a sub-set of information-seeking, concerned with the interactions
between the user and computer-based information systems. In this
dissertation, we focus on information search rather than the other two
higher hierarchical concepts. This is because online IR systems, such as
search engines or digital libraries, have become the primary source for
people to obtain information in modern times, and web search is becoming
ever more pervasive and ubiquitious in our day to day lives.

The field of \textbf{interactive information retrieval} (IIR) posits that IR
systems should operate in the way that good libraries do. Good libraries
provide both the information a visitor needs, as well as a \emph{partner} in
the learning process --- the information professional --- to navigate
that information, make sense of it, preserve it, and turn it into
knowledge. As early as in 1980, Bertram Brookes stated that searchers
acquire new knowledge in the information seeking process
(\protect\hyperlink{ref-brookes1980foundations}{Brookes, 1980}). Fifteen years later, Gary Marchionini
described information seeking, as \emph{``a process, in which humans
purposefully engage in order to change their state of knowledge''}
(\protect\hyperlink{ref-marchionini1995information}{Marchionini, 1995}). So we have known for quite a while that
search is driven by the higher-level human need to gain knowledge.
Information Retrieval is thus a means to an end, and not the end in
itself. Thus, the ideal IR system should not only help users to locate
information, but also help them to \textbf{bridge the gap between information
and knowledge}.

This brings us to the concept of information need. \textbf{Information Need}
is the desire to locate and obtain information to satisfy a conscious or
unconscious human need. Most search systems of today assume that the
search query is an accurate representation of a user's information need.
However, (\protect\hyperlink{ref-belkin1982ask}{Belkin et al., 1982}) observed that in many cases, users of search
systems are unable to precisely formulate what they need. They miss some
vital knowledge to formulate their queries. As humans, we have
difficulty in asking questions about what we do not know. Belkin called
this phenomenon as \textbf{Anomalous State of Knowledge}, or ASK. Later,
(\protect\hyperlink{ref-huang2013relevance}{X. Huang \& Soergel, 2013}) identified an exhaustive set of criteria that
should be considered in order to ideally represent a user's information
need. These criteria for information need are highly dependent on the
user context: user attributes, tasks or goals, as well as the situation
the user is embedded in. This brings us to another closely related
concept: information relevance.

\textbf{Relevance} is a fundamental concept of Information Science and
Information Retrieval, and perhaps the most celebrated work in this area
has been done by Tefko Saracevic
(\protect\hyperlink{ref-saracevic1975relevance}{Saracevic, 1975}, \protect\hyperlink{ref-saracevic2007relevance}{2007a}, \protect\hyperlink{ref-saracevic2007relevancea}{2007b}, \protect\hyperlink{ref-saracevic2016notion}{2016}).
Webster dictionary define relevance as ``a relation to the matter at
hand''. In most circumstances, relevance is a ``y'know'' notion. People
apply it effortlessly, without anybody having to define for them what
``relevance'' is. This creates one of the most fascinating challenges in
the information field: humans understand relevance intuitively, while it
is an open research problem to represent relevance effectively for use
by algorithmic systems. The situation becomes more interesting because
relevance always depends on context, and the context is ever dynamic, as
the matter at hand changes.

\hypertarget{sec:bg_search_3_stage}{%
\section{Three-stage Interactions with Online Search Systems}\label{sec:bg_search_3_stage}}

\begin{figure}
\hypertarget{fig_info_search_models}{%
\centering
\includegraphics{figs/info_search_models.pdf}
\caption{Models of information search process, with our coloured annotations
identifying the three stages.}\label{fig_info_search_models}
}
\end{figure}

As we saw in the previous section, information search behaviour is the
(study of) interactions between a user, and digital Information
Retrieval (IR) systems. The field of Information Science/Studies has
developed multiple models explaining how information search works
(\protect\hyperlink{ref-wilson1999models}{T. D. Wilson, 1999}). A few of them are presented in Figure
\protect\hyperlink{fig_info_search_models}{1.2}. Across many of these models, we
observe that most major Information Retrieval (IR) systems have three
fundamental ways of letting users interact with the system, and the
underlying information: \emph{(1)} an interface for entering search
\textbf{queries}; \emph{(2)} an interface for viewing and evaluating a \textbf{list} of
retrieved information-objects, or search results; \emph{(3)} an interface for
viewing and evaluating \textbf{individual information-objects}. For instance,
(\protect\hyperlink{ref-marchionini1995information}{Marchionini, 1995})'s ISP model hints at these three
interfaces in the fourth, sixth and seventh stages, namely ``formulate
query'', ``examine results'', and ``extract info''. (\protect\hyperlink{ref-spink1997study}{Spink, 1997})'s model
of the IR interaction process consists of sequential steps or cycles,
and each cycle comprises one or more interactive feedback occurrences of
user input (query), IR system output (list), and user interpretation and
judgement (of individual information-objects). Consequently, findings
from the large body of empirical research in interactive IR (especially
those with web based search systems) can be grouped around these thee
stages of interactions with search systems:

\begin{enumerate}
\def\labelenumi{\arabic{enumi}.}
\item
  \emph{Stage 1:} search query formulation / reformulation
\item
  \emph{Stage 2:} search results evaluation (or source selection)
\item
  \emph{Stage 3:} content page evaluation (or, interacting with sources)
\end{enumerate}

The discussions in the following subsections are based around these
three stages of interactions. The empirical studies discussed below
generally follow some common principles of user studies in Interactive
IR (IIR) (\protect\hyperlink{ref-borlund2013interactive}{Borlund, 2013}; \protect\hyperlink{ref-kelly2009methods}{Kelly, 2009}): participants are
presented with a search task or search topic, and then they are asked to
search the internet (or a simulation of the open web) for information.
During the search, the various interactions (queries, clicks, webpages
opened etc.) are recorded, and these are analysed and correlated with
other sources of data to answer research questions.

\hypertarget{sec:bg_search_query}{%
\subsection{Stage 1: Query (Re)formulation}\label{sec:bg_search_query}}

\emph{How do users behave when submitting search queries (to an IR system)?}

\textbf{Query formulation} is the process of composing a search query that
describes the information need of a searcher. \textbf{Query reformulation}
refers to the act of either modifying a previous query, or creating a
new query. Query reformulation typically occurs due to a searcher's
improved understanding of how to better translate their information need
into a search query. The relationship between two successively issued
queries have been classified in a number of ways. These classifications
are called \emph{Query Reformulation Types}, or QRTs. Amongst many other,
Boldi et al. (\protect\hyperlink{ref-boldi2009dango}{2009}) used cognitive aspects of the searchers issuing the
query to propose a taxonomy of QRTs, while C. Liu et al. (\protect\hyperlink{ref-liu2010analysis}{2010}) proposed a
similar taxonomy focusing more on the linguistic properties of the two
successive queries. These are compared and contrasted in
Table~\protect\hyperlink{tab_res_Q_QRT_txnmy}{\[tab_res_Q\_QRT_txnmy\]}.

\begin{figure}
\hypertarget{fig_int_Q}{%
\centering
\includegraphics{figs/int_Q.pdf}
\caption{Investigating user-interactions with queries: \textbf{(a)} Visualizing the
distribution of retrieved search results prior to running a query, for
helping searchers understand their queries' effectiveness (\protect\hyperlink{ref-121}{Qvarfordt et al., 2013}). The
visualization is a stacked column chart with ten columns. Each column
represents ten search results: first column represents results ranked
1-10, second column represents results 11-20, etc. Individual columns
have three divisions, indicating the counts of results that: are already
seen by the searcher (dark blue, top), will be re-retrieved, but have
not been seen by the searcher (medium blue, middle), and will be
newly-retrieved (bright teal blue, bottom). The system evaluates the
searcher's query continuously as it is being typed, and updates the
visualization in real-time. \textbf{(b)} Interfaces for examining
interactions with query auto-completion (QAC), by \textbf{\emph{(i)}} Smith et al. (\protect\hyperlink{ref-129}{2016}), and
\textbf{\emph{(ii)}} Hofmann et al. (\protect\hyperlink{ref-125}{2014}) (overlaid with heatmaps of eye fixations for all
participants). This figure is best viewed in colour.}\label{fig_int_Q}
}
\end{figure}

Task-type, task-topic, task-goal, and domain-expertise were found to
influence query reformulation patterns of searchers (\protect\hyperlink{ref-127}{Eickhoff et al., 2015}; \protect\hyperlink{ref-126}{Jiang et al., 2014}; \protect\hyperlink{ref-133}{Mao et al., 2018}).
At first glance, a significant portion of the query reformulation terms
(\(\sim86\%\)) seemed to be coming from the task-description itself
(\protect\hyperlink{ref-126}{Jiang et al., 2014}; \protect\hyperlink{ref-133}{Mao et al., 2018}). This was characterized by significantly more fixations on
the task-description, rather than other SERP elements. Jiang et al. (\protect\hyperlink{ref-126}{2014}) and Mao et al. (\protect\hyperlink{ref-133}{2018})
investigated this phenomenon further. Jiang et al. (\protect\hyperlink{ref-126}{2014}) controlled for the task-type
and task-goal, using the faceted-framework by Li \& Belkin (\protect\hyperlink{ref-li2008faceted}{2008}). Mao et al. (\protect\hyperlink{ref-133}{2018})
controlled for the task-topic and the domain-expertise of the searchers.

If search tasks had \emph{factual} goals, searchers relied heavily on the
task-description for reformulating their queries (\protect\hyperlink{ref-126}{Jiang et al., 2014}). For
\emph{interpretive} tasks (intellectual tasks with specific goals), users
spent more time reading search result surrogates, before reformulating
their queries. This was observed by increased eye-fixations (indicative
of visual attention) and dwell time on search result snippets
(surrogates). For exploratory tasks, searchers fixated the longest on
query-autocompletion (QAC) suggestions, indicating that they were
possibly looking for help and suggestion based on their specific query,
as the search-task had non-specific (amorphous) goals.

Searchers also relied on the task-description for reformulating queries,
when the search-task was outside their domain of expertise (\protect\hyperlink{ref-133}{Mao et al., 2018}). For
in-domain tasks, they used query terms from their own knowledge, that
were not fixated on in visited SERPs and content pages. Eickhoff et al. (\protect\hyperlink{ref-127}{2015}) reported
that a significant share of new query terms came from visited SERPs and
content pages, and query reformulation (specialization) often did not
literally re-use previously encountered terms, but highly related
ones\footnote{\url{https://developer.chrome.com/docs/extensions/reference/history/\#transition-types}} instead. These observations can possibly be explained by Mao et al. (\protect\hyperlink{ref-133}{2018})'s
findings: when exploring a new domain, the searcher may accumulate
vocabulary and learn how to query during the search; when performing
in-domain search-tasks, the searcher may have enough prior knowledge to
come up with effective query terms. It was also seen that searchers from
medicine domain used more unread query terms for their in-domain
search-tasks, compared to politics and environment domains (\protect\hyperlink{ref-133}{Mao et al., 2018}). This
suggested that domain knowledge and expertise is more important for
formulating good search queries in highly technical disciplines (e.g.,
medicine), compared to less technical domains (e.g., politics).

\textbf{Query Auto Completion (QAC)} is a technological feature that suggests
possible queries to web search users from the moment they start typing a
query. It is nearly ubiquitous in modern search systems, and is thought
to reduce physical and cognitive effort when formulating a query. QAC
suggestions are usually displayed as a list
(Figure~\protect\hyperlink{fig_int_Q}{1.3}(b)
and (c)), and users interact in a variety of ways with the list. Hofmann et al. (\protect\hyperlink{ref-125}{2014})
observed a strong position bias among searchers who examined the QAC
list: the top suggestions received the highest visual attention, even
when the ordering of the suggestions were randomized. Average fixation
time decreased consistently on suggested items from top to bottom. Even
when the ranking of suggestions were randomized, time taken to formulate
queries did not significantly differ.

Search topics were found to have a large effect on QAC usage
(\protect\hyperlink{ref-126}{Jiang et al., 2014}; \protect\hyperlink{ref-129}{Smith et al., 2016}). Search was easiest for the topics with the highest QAC
usage. Total eye-gaze duration was longest when visual attention was
shared between the QAC suggestions and the actual search query input
box. Some additional time was probably due to decision making on whether
to use a QAC suggestion. Typing was faster when a QAC was not used.
However, the IR system's retrieval performance (measured using \href{mailto:NDCG@3}{\nolinkurl{NDCG@3}}),
was greater when QAC was used. So Smith et al. (\protect\hyperlink{ref-129}{2016}) speculated that the value of
using QAC suggestions was realized later in the search session by users,
when they saw a reduction in the number of additional queries needed, or
an increase in the value of the information found.

Several user behavioural profiles were identified by exploring
associations between visual attention from eye-tracking, search
interactions from mouse and keyboard activity, and the use of QAC
suggestions (\protect\hyperlink{ref-125}{Hofmann et al., 2014}; \protect\hyperlink{ref-129}{Smith et al., 2016}). These profiles are described in
Table~\protect\hyperlink{tab_res_Q_QAC_profiles}{\[tab_res_Q\_QAC_profiles\]}. An interesting, yet common-sense
observation was that participants' touch-typing ability greatly
influenced their interactions with QAC suggestions.

The native language of searchers was found to influence their overall
querying and searching behaviour. Ling et al. (\protect\hyperlink{ref-132}{2018}) explored this space using four
variations of a multi-lingual search interface. They observed that
participants strongly preferred to issue queries in their first or
native language. A second or non-native language was the next preferred
choice. Mixing of first and second-languages occurred very rarely. In
80\% of the total 300 tasks (25 users \(\times\) 4 interfaces \(\times\) 3
task-types), participants used a single language for querying. In the
rest 20\% of the tasks, participants switched languages for querying,
with a transition from first language to second language being the most
common.

\hypertarget{sec:bg_search_list}{%
\subsection{Stage 2: Search Results Evaluation}\label{sec:bg_search_list}}

\emph{How do users behave when examining a list of information-objects
(returned by an IR system)?}

After a user submits a query to an IR system, the next action they
generally perform is examining and evaluating the list of search results
returned by the IR system. In this section, we discuss empirical studies
which investigated information-searching behaviour around a list of
information-objects, or a representation of information-objects (also
called \emph{surrogates}). We identified some common themes in the research
questions investigated. The discussion below is grouped along these
themes, as relationships between search behaviour and: \emph{(i)} ranking of
search results; \emph{(ii)} information shown in search results; \emph{(iii)}
individual user characteristics; and \emph{(iv)} relevance judgement and
feedback.

\begin{figure}
\hypertarget{fig_int_L_serp}{%
\centering
\includegraphics{figs/int_L_serp.pdf}
\caption{Example interfaces for studying user-interactions with a
search-engine results page (SERP): \textbf{(a)} a simplified SERP without
query input facility, to judge relevance of search results (on a 4-level
scale) for pre-determined search queries (in this case `why do airplanes
have differently shaped wings?'), from Scharinger et al. (\protect\hyperlink{ref-63}{2016}); \textbf{(b)} eye-tracking heatmap
on an organic SERP from Buscher et al. (\protect\hyperlink{ref-115}{2010}; \protect\hyperlink{ref-117}{Dumais et al., 2010}), showing the F-shaped pattern of
visual attention; \textbf{(c)} a multilingual SERP from Ling et al. (\protect\hyperlink{ref-132}{2018}). This figure is
best viewed in colour.}\label{fig_int_L_serp}
}
\end{figure}

~\\
\textbf{Relationships between \uline{ranking of search results}, and
search behaviour:}

Most search engines display results in a rank ordered list, with the
highest \emph{algorithmically} relevant results placed at the top, and others
results ordered below. Granka et al. (\protect\hyperlink{ref-101}{2004}; \protect\hyperlink{ref-108}{Lorigo et al., 2008}) studied eye-movement behaviour of
searchers examining SERPs, and reported observations from three user
studies. They saw that in 96\% of the queries, participants looked at
only the first result page, containing the top 10 results. No
participant looked beyond the third result page for a given query.
Participants looked primarily at the first few results, with nearly
equal attention (dwell time) given to the first and the second results.
However, despite equal attention, the first result was clicked 42\% of
the time, while the second was clicked only 8\% of the time. If none of
the top three results appeared to be relevant, then users chose not to
explore further results, but issued a reformulated query instead. When
the ranking of the search results were reversed (i.e.~placing less
relevant results in the higher ranked positions), participants spent
considerably more time scrutinizing and comparing results (more
fixations and regressions) before making a decision to click or
reformulate.

Some effects of gender were found to influence SERP examination (\protect\hyperlink{ref-108}{Lorigo et al., 2008}).
Females clicked on the second result twice as often, and made more
regressions or repeat viewings of already visited abstracts, compared to
males. Males were more likely to click on lower ranked results, from
entries 7 through 10, and also look beyond the first 10 results
significantly more often than women. Males were also more linear in
their scanning patterns, with less regressions. Pupil dilation did not
differ significantly between gender groups.

Effects of task-type and task-goals also influenced SERP examination
behaviour. Guan \& Cutrell (\protect\hyperlink{ref-105}{2007}) used Broder (\protect\hyperlink{ref-broder2002taxonomy}{2002})'s taxonomy of navigational vs.
informational searches. The authors reported that when users could not
find the target results for navigational searches, they either selected
the first result, or switched to a new query. However, for informational
searches, users rarely issued a new query and were more likely to try
out the top-ranked results, even when those results had lower relevance
to the task. This illustrated possible strong confidence of searchers in
the search engine's relevance ranking, even though searchers clearly saw
target results at lower positions. Thus, people were more likely to
deprecate their own sense of objective relevance and obeyed the ranking
determined by the search engine. Jiang et al. (\protect\hyperlink{ref-126}{2014}) used Li \& Belkin (\protect\hyperlink{ref-li2008faceted}{2008})'s framework of
search-tasks, and saw that in tasks having specific goals, searchers
fixated more on lower ranked results after some time. On the other hand,
for tasks having amorphous goals, there was a wider breadth in viewing
the SERP, and less effort spent in viewing the content pages. Fixations
tended to decrease as search session progressed, indicating decreased
interest and increasing mental effort, which could demonstrate
\emph{satisficing} behaviour (\protect\hyperlink{ref-simon1956rational}{Simon, 1956}). A comprehensive overview
of various behavioural traits associated with task-types and task-goals
can be found in (\protect\hyperlink{ref-126}{Jiang et al., 2014} Table 8).

~\\
\textbf{Relationships between \uline{information shown in search
results}, and search behaviour:}

The amount and quality of different kinds of information shown on SERPs
also affected user's information searching behaviour. Cutrell \& Guan (\protect\hyperlink{ref-104}{2007}) saw that as
the length of the surrogate information (result snippets) was increased,
user's search performance improved for informational tasks, but degraded
for navigational tasks (\protect\hyperlink{ref-broder2002taxonomy}{Broder, 2002}). Analyzing eye-tracking
data, they posited that the difference in performance was due to users
paying more attention to the snippet, and less attention to the URL
located at the bottom of the search result. This led to performance
deterioration in navigational searches. Buscher et al. (\protect\hyperlink{ref-115}{2010}) studied the effects of the
quality of advertisements placed in the SERPs (Figure
\protect\hyperlink{fig_int_L_serp}{1.4}(b)). Similar to findings discussed above, a
strong position bias of visual attention was found towards the top few
organic result entries --- the well known F-shaped pattern of visual
attention --- which was stronger for informational than for navigational
tasks. However, a strong bias \emph{against} sponsored links was observed in
general. Even for informational tasks, where participants generally had
a harder time finding a solution, the ads did not receive any additional
attention from the participants. Lorigo et al. (\protect\hyperlink{ref-108}{2008}) compared the visual attention
patterns of searchers using two different search engines: Google, and
Yahoo!. Behavioural trends followed similar patterns for both search
engines, even though Google was rated as the primary search engine of
all but one of the participants. They found slight variations in some
eye-tracking measures (reading time of surrogates, time to click
results, and query reformulation time), and some self-reported measures
(perceived ease of use, perceived satisfaction, and success rate).
However, none of these differences were statistically significant.

The novel query-preview interface by Qvarfordt et al. (\protect\hyperlink{ref-121}{2013}) was discussed in
Section~\protect\hyperlink{sec:bg_search_query}{1.2.1} and in
Figure~\protect\hyperlink{fig_int_Q}{1.3}(a).
The authors also reported several observations about user behaviour on
SERPs. They saw that the presence of the preview visualization enabled
participants to look deeper into the results lists. Participants tried
to use the preview as a navigation tool, although it was not designed as
such. The tool increased the rates at which participants examined
documents at middle ranks in query results, and thus helped discover
more useful documents in those middle ranks than without the preview
widget. The preview tool also helped to increase the diversity of
documents found in a search session, which could in turn lead to better
performance in terms of recall and precision. Thus, the tool helped
searchers overcome the strong position bias towards top-ranked results,
as observed by other studies discussed previously.

\begin{figure}
\hypertarget{fig_res_L_serp_user_chars}{%
\centering
\includegraphics{figs/res_L_serp_user_chars.pdf}
\caption{Effects of differences in user characteristics on interactions with
SERPs: \textbf{(a)} exhaustive or \emph{depth-first} user (User 1), vs.~economic
or \emph{breadth-first} user (User 2), examining mostly irrelevant results in
Task A, and mostly relevant results in Task B (both users followed the
second link in Task B); vertical axis denotes vertical location on SERP,
and horizontal axis denotes temporal ordering of result examination;
from Aula et al. (\protect\hyperlink{ref-102}{2005}); (similar patterns were identified by Bilal \& Gwizdka (\protect\hyperlink{ref-139}{2016}), in the SERP
examination behaviour of children) \textbf{(b)} children vs.~adults examining
SERPs from a German search engine for children (left), and Google
(right); differently from adults, children exhaustively explored all
search results, paid more attention to thumbnails and embedded media,
and read less text-only snippets; from Gossen et al. (\protect\hyperlink{ref-124}{2014}). Similar observations as
with children were reported for searchers with dyslexia
(\protect\hyperlink{ref-palani2020eye}{Palani et al., 2020}). This figure is best viewed in colour.}\label{fig_res_L_serp_user_chars}
}
\end{figure}

~\\
\textbf{Relationships between \uline{individual user characteristics},
and search behaviour:}

Individual traits of searchers also influence their pattern of
interactions with a SERP, and these patterns can be revealed by
analyzing eye-tracking data. For instance, searchers have been
classified as \emph{economic} vs.~\emph{exhaustive}, based on their style of
evaluating SERPs (\protect\hyperlink{ref-102}{Aula et al., 2005}). \emph{Economic} searchers were found to scan less
than half (three) of the displayed results above the fold, before making
their first action (query re-formulation, or following a link).
\emph{Exhaustive} searchers evaluated more than half of the visible results
above the fold, or even scrolled the results page to view all of the
results, before performing the first action. Thus, economic searchers
demonstrated depth-first search strategy, while exhaustive users
favoured the breadth-first approach
(Figure~\protect\hyperlink{fig_res_L_serp_user_chars}{1.5}(a)). Dumais et al. (\protect\hyperlink{ref-117}{2010}) demonstrated the use of
unsupervised clustering to re-identify the \emph{economic}-\emph{exhaustive} user
groups, based on differences in total fixation impact\footnote{\url{https://youtu.be/RkBUZ4At8Qg}}, scanpaths,
task outcomes, and questionnaire data. The \emph{economic} cluster was
further broken down by users who looked primarily at results
(\emph{economic-results} cluster), and users who viewed both results and ads
(\emph{economic-ads} cluster). All three groups spent the highest amount of
time on the first three results, with the \emph{exhaustive} group being
substantially slower than the other two groups. The \emph{exhaustive} and
\emph{economic-results} groups spent the second-highest amount of time on
results four through six, while the \emph{economic-eds} group spent this time
on the main advertisements. This group spent more than twice as much
time on the main ads as the \emph{economic-results} group, and even more time
on main ads than the \emph{exhaustive group}. This observation is incongruent
to Buscher et al. (\protect\hyperlink{ref-115}{2010})`s findings, as they observed a generally strong bias \emph{against}
viewing sponsored links. Abualsaud \& Smucker (\protect\hyperlink{ref-135}{2019}) conducted further analysis using these
user types, and, in general, reconfirmed the previous findings. They
found that the results above the fold, especially, \textbf{\emph{the first three
search results are special}}, more so for economic users. On submitting
a 'weak' query, if economic users did not find a correct result within
the first three results, they abandoned examination, and reformulated
their query.

Age of searchers also influence SERP evaluation behaviour. Gossen et al. (\protect\hyperlink{ref-124}{2014})
demonstrated differences in SERP evaluation for children and adults
(Figure~\protect\hyperlink{fig_res_L_serp_user_chars}{1.5}(b)). When answers were not found
within the top search results, the adults reformulated the query
starting a new search, while young users exhaustively explored all the
ten results, and used the navigation buttons between results pages to
continue further examination. Children also paid more attention to
thumbnails and embedded media, and focused less on textual snippets.
Children saw the query suggestions at the bottom of the Google SERP
(because they navigated to the bottom), while the adults did not. Bilal \& Gwizdka (\protect\hyperlink{ref-139}{2016}; \protect\hyperlink{ref-140}{Gwizdka \& Bilal, 2017}) investigated this phenomenon further, and observed that even
within children, age plays a role in SERP evaluation behaviour. Younger
children (grade six, age 11) clicked more often on results in
lower-ranked positions than older children (grade eight, age 13). Older
children's clicking behaviour was based more often on reading result
snippets, and not just on the ranked position of a result in a SERP.
Whereas, younger children made less deliberate choices in choosing which
result to click, and were more exhaustive in the exploration of results.
Thus, using Aula et al. (\protect\hyperlink{ref-102}{2005})'s classification and Dumais et al. (\protect\hyperlink{ref-117}{2010})'s observations, it can be
posited that (younger) children start out as \emph{exhaustive} searchers.
With increase in age and maturity, older children and adults evolve into
\emph{economic} searchers. Interestingly, very similar behaviour patterns as
with children (scrolling further down on SERPs, exhaustive exploration,
etc.) were also observed recently for searchers with dyslexia
(\protect\hyperlink{ref-palani2020eye}{Palani et al., 2020}).

Searcher's native language also influenced SERP interaction behaviour
(\protect\hyperlink{ref-132}{Ling et al., 2018}) (Figure~\protect\hyperlink{fig_int_L_serp}{1.4}(c)). We discussed in
Section~\protect\hyperlink{sec:bg_search_query}{1.2.1} that users strongly preferred issuing
queries in a single language, especially their native language. However,
while examining SERPs, they marked search results in both their first
language and second language to be relevant, to an equal degree. This
confirms the usefulness of search result pages that integrate results
from multiple languages. However, a clear separation in the language of
the search results was strongly preferred, and an `interleaved'
presentation (e.g.~odd numbered results in one language and even
numbered results in another language) was least preferred.

\begin{figure}
\hypertarget{fig_int_L_serp_new_vertical}{%
\centering
\includegraphics{figs/int_L_serp_new_vertical.pdf}
\caption{Google search engine result page (SERP) for the queries: \textbf{(a)}
``coronavirus'' \textbf{(b)} ``toyota'' \textbf{(c)} ``evaporation'', and \textbf{(d)} ``life
of pie''. All screenshots are from `above-the-fold', viewed on a
\(2560 \times 1440\) monitor. These examples highlight that modern SERPs
have come a long way from a list of ``ten blue links''. SERPs are becoming
consumable information-objects in their own right, and thus require
different kinds of cognitive processing and interactions, than from the
early days of the internet. Inspired and adapted from
(\protect\hyperlink{ref-yue2018optimizing}{Wang et al., 2018}). Accessed on May 5, 2020. This figure is best
viewed in colour.}\label{fig_int_L_serp_new_vertical}
}
\end{figure}

~\\
\textbf{Relationships between \uline{relevance judgement}, and search
behaviour:}

(\protect\hyperlink{ref-114}{Balatsoukas \& Ruthven, 2010}, \protect\hyperlink{ref-119}{2012}; Balatsoukas \& Ruthven) proposed a list of `relevance criteria' for understanding
how searchers evaluate search results, or perform \emph{relevance judgement}.
These criteria were developed based on literature reviews and their
empirical findings from eye-tracking studies. The final list contains 15
relevance criteria (e.g., \emph{topicality}, \emph{quality}, \emph{recency}, \emph{scope},
\emph{availability}, etc.) and can be found in (\protect\hyperlink{ref-119}{Balatsoukas \& Ruthven, 2012} Appendix B).

Search engines are increasingly adding different modalities of
information on the SERP, besides the ``ten blue links''. These include
images, videos, encyclopaedic information, and maps
(Figure~\protect\hyperlink{fig_int_L_serp_new_vertical}{1.6}). Z. Liu et al. (\protect\hyperlink{ref-128}{2015}) studied the influence of
these different forms of SERP information -- called `verticals' -- on
searcher's relevance judgements. A general observation was that if
verticals were present in a SERP, they created strong attraction biases.
The attraction effect was influenced by the type of verticals, while the
vertical quality (relevant or not) did not have a major impact. For
instance, `images' and `software download' verticals had higher visual
attention, while news verticals had equal attention as the ``ten blue
links'' search results.

\hypertarget{sec:bg_search_content_page}{%
\subsection{Stage 3: Content Page Evaluation}\label{sec:bg_search_content_page}}

\emph{How do users behave when examining a single information-object (e.g., a
a non-search-engine webpage, aka content page) obtained from an IR
system?}

In online information searching, searchers repeatedly interact with
individual webpages, a.k.a. `content pages' in IR terminology. These
webpages can be visited by following links from a search engine,
following links between different webpages, or directly typing the URL
in the browser.

The first group of papers we discuss investigated users' \textbf{visual
attention} and \textbf{reading behaviour} on webpages. Pan et al. (\protect\hyperlink{ref-pan2004determinants}{2004})
studied whether eye-tracking scanpaths on webpages varied based on
task-type, webpage type (business, news, search, or shopping), viewing
order of webpages, and gender of users. The found significant
differences for all factors, except for task-type, which seemed to have
no effect on scanpaths. They used weak task-types: remembering what was
on a webpage vs.~no specific task. In a later work on using
informational vs.~navigational search-tasks, they again saw limited
effect of task-type on visual attention (\protect\hyperlink{ref-lorigo2006influence}{Lorigo et al., 2006}). Findings
from Josephson \& Holmes (\protect\hyperlink{ref-josephson2002visual}{2002})'s study suggested that users possibly follow
habitually preferred scanpaths on a webpage, which can be influenced by
factors like webpage characteristics and memory. However, they used only
three webpages, making the findings difficult to generalize.
Goldberg et al. (\protect\hyperlink{ref-goldberg2002eye}{2002}) studied eye movements on Web portals during
search-tasks, and saw that header bars were typically not viewed before
focusing the main part of the page. So they suggested placing navigation
bars on the left side of a page. Beymer et al. (\protect\hyperlink{ref-beymer2007eye}{2007}) focused on a very
specific feature on webpages: images that are placed next to text
content and how they influence eye movements during a reading task. They
found significant influence on fixation location and duration. Those
influences were dependent on how the image contents related to the text
contents (i.e., whether they showed ads or text-related images). Buscher et al. (\protect\hyperlink{ref-110}{2009})
presented findings from a large scale study where users performed
information-foraging and page-recognition tasks. They observed that in
the first few moments, users quickly scanned the top left of the page,
presumably looking for clues about the content, provenance, type of
information, etc. for that page. The elements that were normally
displayed in the upper left third of webpages (e.g., logos, headlines,
titles or perhaps an important picture related to the content) seemed to
be important for recognizing and categorizing a page. After these
initial moments, influence of task-type set in. For page-recognition
tasks, the attention remained in the top-left corner of the webpage.
However, for information-foraging tasks, fixations moved to the
center-left region of the webpage, where the user was possibly trying to
find task-specific information. The right third of webpages attracted
almost no visual attention during the first one-second of each page
view. Afterwards as well, most users seemed to entirely ignore this
region, or only occasionally look at it. This suggested that users had
low expectations of information-content or general relevance on the
right side of most webpages. As many webpages display advertisements on
the right side, this was a plausible observation, and are in line with
the observed ``F-shaped-patterns'' \footnote{\url{https://www.nngroup.com/articles/f-shaped-pattern-reading-web-content}} on webpages.

Buscher et al. (\protect\hyperlink{ref-110}{2009}) also proposed an eye-tracking measure called \emph{fixation impact}.
This measure first appends a circular Gaussian distribution around each
fixation on a webpage element, to create a fuzzy area of interest. This
is called the \emph{distance impact} value. If a webpage element completely
covers the fixation circle (Gaussian distribution), it gets a \emph{distance
impact} value of 1. If the element partially covers the fixation circle,
its \emph{distance impact} value is smaller. Multiplying the \emph{distance
impact} value with the fixation duration gives the fixation impact for
the given webpage element. Thus, an element that completely covers the
fixation circle gets the full fixation duration as \emph{fixation impact}
value. Elements which are partially inside the circle get a value
proportional to the Gaussian distribution. The authors posited that the
rationale behind creating the fixation impact measure was motivated by
observations from human vision research, which indicates that fixation
duration correlates with the amount of visual information processed; the
longer a fixation, the more information is processed around the fixation
centre. Using the fixation impact measure, Buscher et al. (\protect\hyperlink{ref-110}{2009}) proposed a model for
predicting the amount visual attention that individual webpage elements
may receive (i.e.~visual salience).

Another group of studies investigated how users judged \textbf{relevance of
webpages} w.r.t. an assigned search-task or information need.
(\protect\hyperlink{ref-74}{Gwizdka, 2018}; \protect\hyperlink{ref-48}{Gwizdka \& Zhang, 2015a}, \protect\hyperlink{ref-47}{2015b}) observed that when relevant pages were revisited, the
webpages were read more carefully. Pupil dilations were significantly
larger on visits and revisits to relevant pages, and just before
relevance judgements were made. Certain conditions of visits and
revisits also showed significant differences in EEG alpha frequency band
power, and EEG-derived attention levels. Relevance of individual webpage
elements were also assessed as \emph{click-intention}: whether users would
click on an element they were looking at. Slanzi et al. (\protect\hyperlink{ref-69}{2017}) used pupillometry and EEG
signals to predict whether a mouse click was present for each eye
fixation. EEG features included simple statistical features of signals
(mean, SD, power, etc.), as well as sophisticated mathematical features
(Hjorth features, Fractal Dimensions, Entropy, etc.). A battery of
classifier models were tested. However, the results were not promising.
Logistic Regression had the highest accuracy (71\%), but very low F1
score (0.33), while neural network based classifiers the had highest F1
score (0.4). The authors suspected that the low sampling rate of their
instruments (30 Hz eye-tracker and 128 Hz 14-channel EEG) impacted their
classifier performances. González-Ibáñez et al. (\protect\hyperlink{ref-81}{2019}) compared relevance prediction performances
in the presence and absence of eye-tracking data, and argued that when
eye-tracking data collection is not feasible, mouse left-clicks can be
used a good alternative indicator of relevance.

The `\emph{Competition for Attention}' theory states that items in our visual
field compete for our attention (\protect\hyperlink{ref-desimone1995neural}{Desimone \& Duncan, 1995}). Djamasbi et al. (\protect\hyperlink{ref-30}{2013}) studied web
search and browsing from the perspective of this theory. Theoretical
models suggest that in goal-directed searches, information-salience
and/or information-relevance drives search behaviour (i.e.~competition
for attention does not hold true), whereas exploratory search behaviour
is influenced by competition among stimuli that attracts a user's
attention (i.e.~competition for attention holds true). However, in
practice, information search behaviour often becomes a combination of
both types of visual search activities (\protect\hyperlink{ref-groner1984looking}{Groner et al., 1984}). Djamasbi et al. (\protect\hyperlink{ref-30}{2013}) found
that, despite the goal directed nature of their search-task (finding the
best snack place in Boston to take their friends) \emph{competition for
attention} had some effect at the content page level. Some of the users'
attention was diverted to non-focal areas on content pages. However,
there was little effect of \emph{competition for attention} on how the
results were viewed on SERPs. Users exhibited the familiar top-to-bottom
pattern of viewing
(Section~\protect\hyperlink{sec:bg_search_list}{1.2.2}), paying the most attention to the top
two entries.

\hypertarget{sec:bg_search_expertise}{%
\section{Effects of Expertise and Working Memory on Search Behaviour}\label{sec:bg_search_expertise}}

\begin{figure}
\centering
\includegraphics{figs/search_behaviours.pdf}
\caption{image}
\end{figure}

Our focus of discussion in this proposal is information searching and
learning. As we saw in Chapter
\protect\hyperlink{ch:bg_learn}{\[ch:bg_learn\]}, learning and expertise are closely connected:
expertise is an evolving characteristic of users that reflects learning
over time, rather than being a static property
(\protect\hyperlink{ref-rieh2016searching}{Rieh et al., 2016}; \protect\hyperlink{ref-sawyer2005cambridge}{Sawyer, 2005}). (\protect\hyperlink{ref-white2016interactions}{White, 2016a, Chapter 7}) considers three types of expertise, that are relevant in
information seeking settings: \emph{(i)} domain or subject-matter expertise;
\emph{(ii)} search expertise; and \emph{(iii)} task expertise. \textbf{Domain or
subject-matter expertise} describes people's knowledge in a specialised
subject area such as a domain of interest. \textbf{Search expertise} refers
to people's skill level at performing information-seeking activities,
both in a Web search setting and in other settings such as specialised
domains. \textbf{Task expertise} describes people's expertise in performing
particular search tasks, potentially independent of domain. Although
considered distinctly, the boundaries between these expertise types are
quite blurred, and therefore difficult to estimate at the time of
search, and model it in a way that can be consumed by search systems.

Previous work on domain knowledge and expertise have linked\footnote{and continue to link} domain
expertise and search behaviour in terms of metrics, behavioural
patterns, and criteria
(\protect\hyperlink{ref-cole2013inferring}{M. J. Cole et al., 2013}; \protect\hyperlink{ref-133}{Mao et al., 2018}; \protect\hyperlink{ref-o2020role}{O'Brien et al., 2020}; \protect\hyperlink{ref-white2009characterizing}{White et al., 2009}). A
representative summary is presented in Table
\protect\hyperlink{tab_search_behaviours}{\[tab_search_behaviours\]}, and is adapted from literature
reviews by (\protect\hyperlink{ref-rieh2016searching}{Rieh et al., 2016}) and (\protect\hyperlink{ref-vakkari2016searching}{Vakkari, 2016}). Briefly,
(\protect\hyperlink{ref-wildemuth2004effects}{Wildemuth, 2004}) showed that novices converge toward the same
search patterns as experts, as they are exposed to a topic and learn
more about it. (\protect\hyperlink{ref-zhang2011predicting}{X. Zhang et al., 2011}) found that features such as
document retention, query length, and the average rank of results
selected could be predictive of domain expertise. (\protect\hyperlink{ref-cole2013inferring}{M. J. Cole et al., 2013})
showed that eye-gaze patterns could be used to predict an individual's
level of domain expertise using estimates of cognitive effort associated
with reading. (\protect\hyperlink{ref-white2009characterizing}{White et al., 2009}) showed that measures such as
diverse website visitation, more narrow topical focus, less diversity
(or entropy), more `branchiness' of search sessions, less dwell time,
and higher query and session complexity are indicative of expert
knoweldge and/or search behaviour.

As a stark contrast, (\protect\hyperlink{ref-zlatkin2021students}{Zlatkin-Troitschanskaia et al., 2021}) reviewed literature on
higher education \textbf{students' information search behaviour}. Students
can be considered as novices in all three respects:
domain/subject-matter, search skills, and task. The authors report that
across literature, higher education students' information search
behaviour tends to follow some general general patterns: \emph{(i) foraging:}
no explicit (task-specific) research plan and little understanding of
the differences (pros/cons) between various IR systems; \emph{(ii) Google
dependence:} no intention to use any search tool other than Google,
causing students to struggle to understand library information
structures and engage with scholarly literature effectively; \emph{(iii)
rudimentary search heuristic:} reliance on one and the same simple
search strategy, regardless of search context; \emph{(iv) habitual topic
changing:} students change the search topic after rather superficial
skimming, and before evaluating all search results; and \emph{(v) overuse of
natural language:} students type questions into the search box that are
phrased as if posing them to a person. Highly ranked online sources
accessed via a well-known search engine were perceived as trustworthy.

Effects of memory span and working memory capacity have also been found
to influence search effort and search behaviour
(\protect\hyperlink{ref-arguello2019effects}{Arguello \& Choi, 2019}; \protect\hyperlink{ref-CHIIR19}{Bhattacharya \& Gwizdka, 2019a}; \protect\hyperlink{ref-cole2020more}{L. Cole et al., 2020}; \protect\hyperlink{ref-gwizdka2013effects}{Gwizdka, 2013}, \protect\hyperlink{ref-gwizdka2017can}{2017}).
\textbf{Working memory} (WM) is considered a core executive function is
defined as someone's ability to hold information in short-term memory
when it is no longer perceptually present
(\protect\hyperlink{ref-diamond2013executive}{Diamond, 2013}; \protect\hyperlink{ref-miller1956magical}{G. A. Miller, 1956}). (\protect\hyperlink{ref-bailey2011amount}{Bailey \& Kelly, 2011}) showed
that the amount of effort was a good indicator of user success on search
tasks. (\protect\hyperlink{ref-smith2008user}{Smith \& Kantor, 2008}) studied searcher adaptation to poorly performing
systems and found that searchers changed their search behaviors between
difficult and easy topics in a way that could indicate that users are
satisficing. Differences in search effort between different types
systems (higher effort invested in searching library database vs.~web)
were found by (\protect\hyperlink{ref-rieh2012amount}{Rieh et al., 2012}). A couple of studies showed that mental
effort involved in judging document relevance is lower for irrelevant
and higher for relevant documents (\protect\hyperlink{ref-37}{Gwizdka, 2014}; \protect\hyperlink{ref-villa2013relevance}{Villa \& Halvey, 2013}).
(\protect\hyperlink{ref-gwizdka2017can}{Gwizdka, 2017}) found that that higher WM searchers perform more
actions and that most significant differences are in time spent on
reading results pages. Behaviour of high and low WM searchers were also
found to change differently in the course of a search task performance.

\hypertarget{assessing-learning-during-search}{%
\section{Assessing Learning during Search}\label{assessing-learning-during-search}}

In order for IR systems to foster user-learning at scale, while
respecting individual differences of searchers, there is a need for
measures to represent, assess, and evaluate the learning process,
possibly in an automated fashion. Consequently, a variety of assessment
tools have been used in prior studies. These include self reports, close
ended factual questions (multiple choice), open ended questions (short
answers, summaries, essays, free recall, sentence generation), and
visual mapping techniques using concept maps or mind maps. Each approach
has its own associated advantages and limitations. \textbf{Self-report} asks
searchers to rate their self-perceived pre-search and post-search
knowledge levels (\protect\hyperlink{ref-ghosh2018SearchingLearningExploring}{Ghosh et al., 2018}; \protect\hyperlink{ref-o2020role}{O'Brien et al., 2020}).
This approach is the easiest to construct, and can be generalised over
any search topic. However, self-perceptions may not objectively
represent true learning. \textbf{Closed ended questions} test searchers'
knowledge using factual multiple choice questions (MCQs). The answer
options can be a mixture of fact-based responses (\emph{TRUE}, \emph{FALSE}, or \emph{I
DON'T KNOW}),
(\protect\hyperlink{ref-gadiraju2018AnalyzingKnowledgeGain}{Gadiraju et al., 2018}; \protect\hyperlink{ref-xu2020does}{Xu et al., 2020}; \protect\hyperlink{ref-yu2018PredictingUserKnowledgea}{Yu et al., 2018})
or recall-based responses (\emph{I remember / don't remember seeing this
information}) (\protect\hyperlink{ref-kruikemeier2018learning}{Kruikemeier et al., 2018}; \protect\hyperlink{ref-roy2020exploring}{Roy et al., 2020}).
Constructing topic-dependant MCQs may take time and effort, since they
are topic dependant. Recent work on automatic question generation may be
leveraged to overcome this limitation (\protect\hyperlink{ref-syed2020improving}{Syed et al., 2020}). Evaluating
close ended questions is the easiest, and generally automated in various
online learning platforms. Multiple choice questions, however, suffer
from a limitation: they allow respondents to answer correctly by
guesswork. \textbf{Open ended questions} assess learning by letting searchers
write natural language summaries or short answers
(\protect\hyperlink{ref-bhattacharya2018relating}{Bhattacharya \& Gwizdka, 2018}; \protect\hyperlink{ref-o2020role}{O'Brien et al., 2020}; \protect\hyperlink{ref-roy2021note}{Roy et al., 2021}). Depending on
experimental design, prompts for writing such responses can be generic
(least effort) (\protect\hyperlink{ref-bhattacharya2018relating}{Bhattacharya \& Gwizdka, 2018}, \protect\hyperlink{ref-bhattacharya2019measuring}{2019b}),
or topic-specific (some effort) (\protect\hyperlink{ref-syed2020improving}{Syed et al., 2020}). While this
approach can provide the richest information about the searcher's
knowledge state, evaluating such responses is the most challenging, and
requires extensive human intervention
(\protect\hyperlink{ref-kanniainen2021assessing}{Kanniainen et al., 2021}; \protect\hyperlink{ref-leu2015new}{Leu et al., 2015}; \protect\hyperlink{ref-wilson2013comparison}{M. J. Wilson \& Wilson, 2013}) (as
discussed in Section
\protect\hyperlink{sec:bg_learn_artefact}{\[sec:bg_learn_artefact\]}). \textbf{Visual mapping} techniques such
as mind maps and concept maps have also been used to assess learning
during search
(\protect\hyperlink{ref-egusa2010usingb}{Egusa et al., 2010}, \protect\hyperlink{ref-egusa2014howd}{2014a}, \protect\hyperlink{ref-egusa2014howe}{2014b}, \protect\hyperlink{ref-egusa2017evaluating}{2017}; \protect\hyperlink{ref-halttunen2005assessing}{Halttunen \& Jarvelin, 2005}).
Concept maps have been discussed at length in Section
\protect\hyperlink{sec:bg_concept_maps}{\[sec:bg_concept_maps\]}. Learning has also been measured in
\textbf{other ways}, such as user's familiarity with concepts and
relationships between concepts (\protect\hyperlink{ref-pirolli1996scatter}{Pirolli et al., 1996}), gains in user's
understanding of the topic structure, e.g., via conceptual changes
described in pre-defined taxonomies (\protect\hyperlink{ref-zhang2016process}{P. Zhang \& Soergel, 2016}), and user's
ability to formulate more effective queries
(\protect\hyperlink{ref-chen2020understanding}{Chen et al., 2020}; \protect\hyperlink{ref-pirolli1996scatter}{Pirolli et al., 1996}).

\hypertarget{limitations-of-current-search-systems-in-fostering-learning}{%
\section{Limitations of Current Search Systems in Fostering Learning}\label{limitations-of-current-search-systems-in-fostering-learning}}

\hypertarget{sec:bg_search_longitudinal_studies}{%
\subsection{Longitudinal studies}\label{sec:bg_search_longitudinal_studies}}

Learning is a longitudinal process, occurring gradually over time
(Sections
\protect\hyperlink{sec:bg_learn_sensemaking}{\[sec:bg_learn_sensemaking\]} and
\protect\hyperlink{sec:bg_learn_principles}{\[sec:bg_learn_principles\]}). Therefore, information
researchers have studied participant's search behaviour in prior,
\textbf{albeit few}, longitudinal studies. Examples include studies by
(\protect\hyperlink{ref-kelly2006measuring_a}{Kelly, 2006a}, \protect\hyperlink{ref-kelly2006measuring_b}{2006b}; \protect\hyperlink{ref-kuhlthau2004seeking}{Kuhlthau, 2004}; \protect\hyperlink{ref-vakkari2001changes}{Vakkari, 2001b}; \protect\hyperlink{ref-white2009characterizing}{White et al., 2009}; \protect\hyperlink{ref-wildemuth2004effects}{Wildemuth, 2004}).

(\protect\hyperlink{ref-wildemuth2004effects}{Wildemuth, 2004}) examined the search behaviour of medical
students in microbiology. In this experiment, students were observed at
three points of time (at the beginning of the course, at the end of the
course, and six months after the course), under the assumption that
domain expertise changes during a semester. Some search strategies, most
notably the gradual narrowing of the results through iterative query
modification, were the same throughout the observation period. Other
strategies varied over time as individuals gained domains knowledge.
Novices were less efficient in selecting concepts to include in search
and less accurate in their tactics for modifying searches.
(\protect\hyperlink{ref-pennanen2003students}{Pennanen \& Vakkari, 2003}; \protect\hyperlink{ref-vakkari2001theory}{Vakkari, 2001a}, \protect\hyperlink{ref-vakkari2000cognition}{2000}, \protect\hyperlink{ref-vakkari2001changes}{2001b})
also examined students at multiple points in time, as they were
developing their thesis proposal. One important change in behaviour was
the use of a more varied and more specific vocabulary as students
learned more about their research topic. (\protect\hyperlink{ref-weber2019informationseeking}{Weber et al., 2019})
examined a large sample of German students from all academic fields in a
two wave study and found that the more advanced they are in their
studies, the more students show a more advanced search behaviour (e.g.,
using more English queries and accessing academic databases more
frequently). \textbf{Advanced search behaviour predicted better university
grades.} (\protect\hyperlink{ref-weber2018can}{Weber et al., 2018}) also provide mixed evidence on the potential
long-term effects of such interventions, as some of their participants
reverted to their previous habits two weeks after the study and
therefore exhibited only short-term changes in their information-seeking
behaviour.

Overall, results regarding the promotion of user' search and evaluation
skills are encouraging. But there is a clear need for more longitudinal
studies. The general body of search-as-learning literature examines the
learner in the short-term, typically over the course of a single lab
session (\protect\hyperlink{ref-kelly2009evaluation}{Kelly et al., 2009}; \protect\hyperlink{ref-zlatkin2021students}{Zlatkin-Troitschanskaia et al., 2021}). The trend is
similar in other Human-Computer Interaction (HCI) research venues. A
meta-analysis of 1014 user studies reported in the ACM CHI 2020
conference revealed that more than 85\% of the studies observed
participants for a day or less. To this day, ``longitudinal studies are
the exception rather than the norm'' (\protect\hyperlink{ref-HCIUXres81_online}{Koeman, 2020}). ``An
over-reliance on short studies risks inaccurate findings, potentially
resulting in prematurely embracing or disregarding new concepts''
(\protect\hyperlink{ref-HCIUXres81_online}{Koeman, 2020}).

\hypertarget{sec:bg_search_sensemaking}{%
\subsection{Supporting sensemaking and reflection}\label{sec:bg_search_sensemaking}}

As we saw in Section
\protect\hyperlink{sec:bg_learn_sensemaking}{\[sec:bg_learn_sensemaking\]}, learning \emph{is} sensemaking. Yet,
modern search systems are still quite far from supporting sensemaking
and learning, and rather, at best are good \emph{locators} of information.
(\protect\hyperlink{ref-rieh2016searching}{Rieh et al., 2016}) says that modern search systems should support
sensemaking by offering more interactive functions, such as tagging for
annotation, or tracking individuals' search history, so that a learner
could return to a particular learning point. In addition, a system could
provide new features that allow learners to reflect upon their own
learning process and search outcomes, thus facilitating the development
of critical thinking skills.

It's easy to be impressed by the scientific and engineering feats that
have produced web search engines. They are, unquestionably, one of the
most impactful and disruptive information technologies of our time.
However, it's critical to remember their many limitations: they do not
help us \textbf{know what we want to know}; they do not help us \textbf{choose the
right words} to find it; they do not help us know if what we've found
is \textbf{relevant or true}; and they do not help us \textbf{make sense of it}.
All they do is quickly retrieve what other people on the internet have
shared. While this is a great feat, all of the content on the internet
is far from everything we know, and quite often a poor substitute for
expertise.

-- (\protect\hyperlink{ref-ko2021seeking}{Ko, 2021}) (emphasis our own)

\hypertarget{sec:bg_search_summary}{%
\section{Summary}\label{sec:bg_search_summary}}

In this second chapter of the background literature review, we discussed
\emph{(i)} how searchers interact with three stages / interfaces of modern
information retrieval system: query formulation, search results
evaluation, and content page evaluation; \emph{(ii)} how expertise and
working memory influence overall search behaviour; \emph{(iii)} how learning
or knowledge gain during search has been assessed in recent search as
learning literature; and \emph{(iv)} what are the limitations of current
search systems to foster learning, including gaps in literature about
long term search behaviour and learning outcomes, as well as lack of
support for sensemaking.

We saw that while we have a plethora of studies investigating search
behaviour searchers in the short term, we have merely a handful of
studies observing the same participant for more than a day. To the best
of the author's knowledge, most of these studies were conducted over a
decade ago. Thus, while we have excellent knowledge of short term nature
of influence of searching on learning, we do not know what are the
longer term effects. Furthermore, we we have gaps in our knowledge of
\emph{(i)} how practices like articulation and externalization, and user
attributes like metacognition, motivation, and self regulation moderate
the searching as learning process; \emph{(ii)} how these moderator variables
change over time; and \emph{(iii)} what these phenomena collectively entail
for the design of future learning-centric IR systems. In the next
chapter, we take these gaps in knowledge and use them to inform our
research questions and hypotheses.

\hypertarget{research-questions-and-hypotheses}{%
\chapter{Research Questions and Hypotheses}\label{research-questions-and-hypotheses}}

\hypertarget{sec:rq}{%
\section{Research Questions}\label{sec:rq}}

Combining empirical findings and gaps in the literature from the
disciplines of Education (Chapter
\protect\hyperlink{ch:bg_learn}{\[ch:bg_learn\]}) and Information (Chapter
\protect\hyperlink{ch:bg_search}{\[ch:bg_search\]}), we saw that:

\begin{itemize}
\item
  searching for information online is an integral part of new learning
  (Section
  \protect\hyperlink{sec:bg_learn_info_eval}{\[sec:bg_learn_info_eval\]})
\item
  learning happens when students connect new pieces of information to
  their existing knowledge structures via assimilation, restructuring,
  or tuning (Section
  \protect\hyperlink{sec:bg_learn_sensemaking}{\[sec:bg_learn_sensemaking\]}), and this process is
  influenced by the learner's individual traits (Section
  \protect\hyperlink{sec:bg_learn_promoting_learning}{\[sec:bg_learn_promoting_learning\]})
\item
  modern knowledge-work requires less of long term memory, and more of
  creation of knowledge-artefacts, which should be treated as better
  assessors and outcomes of learning (Section
  \protect\hyperlink{sec:bg_learn_artefact}{\[sec:bg_learn_artefact\]})
\item
  domain expertise and search behaviour are strongly linked (Section
  \protect\hyperlink{sec:bg_search_expertise}{\[sec:bg_search_expertise\]})
\item
  learning is a process that takes place longitudinally over time
  (Sections
  \protect\hyperlink{sec:bg_learn_sensemaking}{\[sec:bg_learn_sensemaking\]} and
  \protect\hyperlink{sec:bg_learn_principles}{\[sec:bg_learn_principles\]}), yet only a handful of studies
  (mostly over a decade ago) have investigated the intertwined process
  of searchers' learning and their information searching behaviour
  over time (Section
  \protect\hyperlink{sec:bg_search_longitudinal_studies}{\[sec:bg_search_longitudinal_studies\]})
\item
  this creates acute gaps in our knowledge about long term information
  searching and learning behaviour, which is crucial for building
  learning-centric search systems of the future, which can support
  sensemaking and knowledge-gain
\end{itemize}

Guided by the above insights, we ask the following research questions in
this dissertation proposal, and aim to answer them via a longitudinal
study of students' information search behaviour and learning outcomes
over the course of a university semester (Section
\protect\hyperlink{sec:method_exp_design}{\[sec:method_exp_design\]}). For the purposes of this
dissertation, we consider learning as change in a student's knowledge
about certain topics over the duration of a university semester.

The research questions are first stated in this section, to put them all
together in one place for easy reference. Then the overarching
hypotheses are discussed in Section
\protect\hyperlink{sec:rq_hypotheses}{1.2}.

\begin{quote}
\textbf{RQ1:} What kind of longitudinal information search behaviours are
correlated to the degree of change in students' knowledge levels and
learning outcomes?
\end{quote}

\begin{quote}
\textbf{RQ2:} What are the similarities and differences in information
search behaviours for tasks where the learning goals are new (new
search tasks), versus those where the learning goals are repeated
(repeated search tasks)?
\end{quote}

\begin{quote}
\textbf{RQ3:} How does externalisation and articulation affect students'
learning outcomes and experiences during search?
\end{quote}

\begin{quote}
\textbf{RQ4:} How do (changing) individual differences of students moderate
their information search behaviours and learning outcomes?
\end{quote}

\hypertarget{sec:rq_hypotheses}{%
\section{Overarching Hypotheses}\label{sec:rq_hypotheses}}

In this Section, we discuss the research framework and hypotheses behind
the research questions. The study is primarily planned to be
exploratory, therefore the hypotheses are exploratory in nature as well.

\hypertarget{sec:framework_rq1_rq2}{%
\subsection{Learning as Students' Transition from Novice to Expert (RQ1, RQ2)}\label{sec:framework_rq1_rq2}}

Learning and expertise are closely connected: expertise is an evolving
characteristic of learners that reflects learning over time, rather than
being a static property (\protect\hyperlink{ref-rieh2016searching}{Rieh et al., 2016}). Domain expertise and
search behaviour has been studied, albeit mostly during single lab
sessions, and sometimes longitudinally (Section
\protect\hyperlink{sec:bg_search_expertise}{\[sec:bg_search_expertise\]}). There is a clear gap in
understanding how higher education students search for information in
the long term, how their information use behaviour develop over time,
and how it affects their learning (\protect\hyperlink{ref-zlatkin2021students}{Zlatkin-Troitschanskaia et al., 2021}). RQ1 and RQ2
aims to address some of these gaps.

\textbf{Hypothesis for RQ1:} Search behaviours described in Table
\protect\hyperlink{tab_search_behaviours}{\[tab_search_behaviours\]} will occur both within individual
search sessions, and across progressive search sessions recorded over a
semester, as domain expertise of students increases
(\protect\hyperlink{ref-eickhoff2014lessons}{Eickhoff et al., 2014}).

\textbf{Hypotheses for RQ2:} This research question stems from the idea of
lifelong or continuous learning: how do search behaviours evolve over
time when gaining knowledge about perpetual life skills (e.g., financial
literacy). We hypothesize that

\begin{itemize}
\item
  relevance judgement of previously viewed information on this topic
  will change over time, as searcher gains more knowledge and
  expertise
\item
  the decision or choice to put effort into searching again, or
  satisfice with previously found information, will have links to
  motivation and self-regulation
\end{itemize}

\hypertarget{sec:framework_rq3_rq4}{%
\subsection{Promoting Better Learning (RQ3, RQ4)}\label{sec:framework_rq3_rq4}}

Better learning takes place when students articulate and their unformed
and still developing understanding, and continue to articulate it
throughout the learning process (Section
\protect\hyperlink{sec:bg_learn_articulation}{\[sec:bg_learn_articulation\]}). Also, students' motivation,
self-regulation and metacognition capabilities determine, direct, and
sustain the approaches they take to learn (Section
\protect\hyperlink{sec:bg_learn_promoting_learning}{\[sec:bg_learn_promoting_learning\]}). Effective searching for
learning is affected by students' search tactics and information
evaluation capabilities (Section
\protect\hyperlink{sec:bg_learn_info_eval}{\[sec:bg_learn_info_eval\]}) as well as cognitive capabilities
such as memory span (Section
\protect\hyperlink{sec:bg_search_expertise}{\[sec:bg_search_expertise\]}).

\textbf{Hypothesis for RQ3:} articulation during the search as learning
process (via concurrent think aloud) will lead to better learning (and
possibly better searching) outcomes, than working silently.

\textbf{Hypotheses for RQ4:} with respect to the individual differences and
contexts in which students search to learn, we speculate the following
hypotheses:

\begin{itemize}
\item
  students showing sustained or increasing metacognition,
  self-regulation, and motivation over the duration of the semester
  will put more ``effort'' into their searches, and demonstrate better
  learning and search outcomes
\item
  students with higher memory span will demonstrate more `branchiness'
  and parallel browsing in their search behaviour
\item
  students with better information evaluation capabilities will
  demonstrate better learning and search outcomes
\end{itemize}

\hypertarget{sec:rq_diss_contributions}{%
\section{Anticipated Contributions}\label{sec:rq_diss_contributions}}

We anticipate by answering the proposed research hypotheses and
question, the results can greatly contribute to the existing knowledge
of Interactive Information Retrieval and Educational Sciences in
general, and Search as Learning in particular. Referring back to some of
the research agenda advocated by the multiple workshops and journal
special issues on Search as Learning (Section
\protect\hyperlink{sec:intro_outline}{\[sec:intro_outline\]}), our research questions aim to
investigate \emph{(i)} the contexts in which students search to learn; \emph{(ii)}
the factors that influence their learning outcomes; and \emph{(iii)} whether
students are more critical consumers of information.

Many researchers have expressed their concern with the lack of
longitudinal studies in IIR and related domains
(\protect\hyperlink{ref-kelly2009evaluation}{Kelly et al., 2009}; \protect\hyperlink{ref-HCIUXres81_online}{Koeman, 2020}; \protect\hyperlink{ref-zlatkin2021students}{Zlatkin-Troitschanskaia et al., 2021}). If
significant relationships were to be found between students' information
search behaviours and learning outcomes, the results of this
dissertation can provide great insights and contributions towards \emph{(i)}
understanding how search behaviours can predict learning outcomes;
\emph{(ii)} creating reliable measures, methods, and instruments for
capturing changes in people's knowledge level, learning experiences, and
learning outcomes (\protect\hyperlink{ref-url_rieh_homepage}{Rieh, 2020}); and \emph{(iii)} developing search
systems that better support learning and sensemaking.

\hypertarget{method-proposed-longitudinal-study}{%
\chapter{Method: Proposed Longitudinal Study}\label{method-proposed-longitudinal-study}}

We propose to conduct a remote longitudinal study to investigate the
research questions and hypotheses discussed in Section
\protect\hyperlink{sec:rq}{\[sec:rq\]}. The
study will primarily be exploratory in nature. The following sections
discuss the apparatus, procedure, measurement of variables, plans for
data analysis, anticipated limitations, and the expected timeline for
the data collection and dissertation defence.

\hypertarget{sec:method_exp_design}{%
\section{Study Design}\label{sec:method_exp_design}}

\begin{figure}
\hypertarget{fig_final_project_description}{%
\centering
\includegraphics{figs/final_project_description.pdf}
\caption{Final project description, setting up the search tasks throughout the
semester.}\label{fig_final_project_description}
}
\end{figure}

The remote longitudinal study is planned to span over the course of a
typical university semester (approximately 16 weeks). Participants will
be recruited from a course (study site) at the University of Texas at
Austin (UT Austin). The ideal study site will be a final-project based
course, where the final-project report (artefact, Section
\protect\hyperlink{sec:bg_learn_artefact}{\[sec:bg_learn_artefact\]}) will be built over time throughout
the semester, with periodic check-ins. Completing the final project will
require searching and navigating information online, finding relevant
sources for a set of assigned or self-chosen topics, and weaving a
narration around the information found in the selected sources.

The study design was informed by running a pilot study during Summer
2021 semester, in partnership with two courses at UT Austin School of
Information: \emph{Information in Cyberspace}, and \emph{Academic Success in the
Digital University}. More details of the pilot study are presented in
Appendix \protect\hyperlink{ch:pilot_study}{\[ch:pilot_study\]}. The final project overview from the
\emph{Information in Cyberspace} course is presented in Figure
\protect\hyperlink{fig_final_project_description}{1.1}. The words or phrases
enclosed in \[\[square brackets\]{]} will be appropriately modified when
the study site is finalised. We aim to recruit upwards of 30
participants. Since the study is exploratory in nature, and is intended
to inform future confirmatory studies, we are unable to conduct a power
analysis.

\hypertarget{apparatus}{%
\section{Apparatus}\label{apparatus}}

\hypertarget{yasbil-browsing-logger}{%
\subsection{YASBIL Browsing Logger}\label{yasbil-browsing-logger}}

The YASBIL browsing logger (\protect\hyperlink{ref-bhattacharya2021yasbil}{Bhattacharya \& Gwizdka, 2021}) will be utilised
for this study. YASBIL (Yet Another Search Behaviour and Interaction
Logger) is a two-component logging solution for ethically recording a
user's browsing activity for Interactive IR user studies. It was
developed by the author in early Spring 2021, and was employed in the
pilot study for data collection and testing. YASBIL comprises a Firefox
browser extension and a WordPress plugin. The browser extension logs the
browsing activity in the participants' machines. The WordPress plugin
collects the logged data into the researcher's data server. YASBIL
captures participant's behavioural data, such as webpage visits, time
spent on pages, identification of popular search engines and their
SERPs, identifying rank of result clicked on SERPs, tracking mouse
clicks and scrolls, and the order and sequences of this events. The
logging works on any webpage, without the need to own or have knowledge
about the HTML structure of the webpage. To protect the privacy of
participants, the logger software can be switched on or off, and
participants will be instructed (and encouraged) to switch on the logger
only when they are performing search activities related to the
experiment. YASBIL also offers ethical data transparency and security
towards participants, by enabling them to view and obtain copies of the
logged data, as well as securely upload the data to the researcher's
server over an HTTPS connection. Although developed using the
cross-browser WebExtension API (\protect\hyperlink{ref-url_cross_browser_extn}{Mozilla Developer Network, 2021}), YASBIL
currently works in the Firefox Web Browser. So participants will be
instructed to install Firefox and YASBIL on their machines if they
choose to participate in the study.

\hypertarget{online-concept-mapping-software}{%
\subsection{Online Concept Mapping Software}\label{online-concept-mapping-software}}

As discussed in Section
\protect\hyperlink{sec:bg_concept_maps}{\[sec:bg_concept_maps\]}, concept maps are an appropriate way to
understand and assess sense-making and change in knowledge structures.
In this study, participants will produce concept maps whenever they are
working on an information search task. An appropriate browser based
concept-mapping tool will be used for this purpose. The ``Sero!''
platform\footnote{\url{https://developer.chrome.com/docs/extensions/reference/history/\#transition-types}} is a promising choice for this purpose. Sero! ``uses concept
mapping for knowledge and learning assessment'' and ``facilitates
big-picture comprehension and gives educators deeper insight into
students' thinking''.

\hypertarget{sec:method_search_task_template}{%
\section{Search Task Template}\label{sec:method_search_task_template}}

\begin{figure}
\hypertarget{fig_search_task_template}{%
\centering
\includegraphics{figs/search_task_template.pdf}
\caption{Template for each search task, adapted from the \emph{Information in
Cyberspace} course at the UT Austin School of Information. Phrases
enclosed in \[\[square brackets\]{]} will be modified for individual
tasks.}\label{fig_search_task_template}
}
\end{figure}

Participants of the longitudinal study will be performing several search
tasks throughout the duration of the semester (Section
\protect\hyperlink{sec:method_procedure}{1.4}). The search tasks mentioned will
generally be on the topics of the material taught in the course. The
generic template for each search task is presented in Figure
\protect\hyperlink{fig_search_task_template}{1.2}, and is adapted from the
\emph{Information in Cyberspace} course. The words or phrases enclosed in
\[\[square brackets\]{]} will be appropriately modified when the study
site is finalized.

\hypertarget{sec:method_procedure}{%
\section{Procedure}\label{sec:method_procedure}}

\begin{figure}
\hypertarget{fig_study_procedure}{%
\centering
\includegraphics{figs/study_procedure.pdf}
\caption{Overview of the study procedure.}\label{fig_study_procedure}
}
\end{figure}

The longitudinal study will consist of six data collection stages, as
outlined in Figure~\protect\hyperlink{fig_study_procedure}{1.3}. Each of these stages are described
below.

\hypertarget{sec:method_sur1}{%
\subsection{SUR1: Entry Survey}\label{sec:method_sur1}}

Participants will be recruited for the study via the entry survey
\[SUR1\]. The description of the study and the link to the survey will
be posted in the Learning Management System used for the course
(Canvas). The entry survey will serve dual purpose: recruit
participants, as well as capture their individual-differences, or
moderating variables. This is done so as to avoid having to capture all
of this information in the initial session (SES1), which would have made
the SES1 session overly long. Details of the data captured in SUR1 are
described below, with references to sections in the Appendix, where the
full-text of the questionnaire can be found.

\begin{enumerate}
\def\labelenumi{\arabic{enumi}.}
\item
  \textbf{Consent Form:} The first page of the entry survey will be the
  online consent form for participating in the study. Participants
  will be able to proceed once they provide consent.
\item
  \textbf{Demographics:} (Appendix
  \protect\hyperlink{app:demographics}{\[app:demographics\]}) demographic information
\item
  \textbf{Search and IT proficiency:} (Appendix
  \protect\hyperlink{app:search_it_proficiency}{\[app:search_it_proficiency\]}) Captures previous search
  experience, and proficiency in navigating the web. Some items are
  adapted from the \emph{Digital Health Literacy Instrument (DHLI)} by
  (\protect\hyperlink{ref-van2017development}{Van Der Vaart \& Drossaert, 2017}), and the \emph{Search Self-Efficacy scale (SSE)} by
  (\protect\hyperlink{ref-brennan2016factor}{Brennan et al., 2016}).
\item
  \textbf{Course Load and other engagements:} (Appendix
  \protect\hyperlink{app:course_load}{\[app:course_load\]}) To determine how busy the participant
  will be in the semester, and how much time they plan to allocate for
  the course with which the study is integrated. This will help to
  establish the learner's context.
\item
  \textbf{Note-taking Strategies:} (Appendix
  \protect\hyperlink{app:note_taking_strategies}{\[app:note_taking_strategies\]}) Captures styles and
  strategies used by participants to take notes. Adapted from
  \emph{Listening and Note Taking Survey} by
  (\protect\hyperlink{ref-note_taking_survey_penn_state}{Penn State Learning, 2021}), and \emph{Note Taking Strategies
  Inventory} by (\protect\hyperlink{ref-note_taking_strategies_umass}{UMass Amherst Student Success, 2021}).
\item
  \textbf{Motivation:} (Appendix
  \protect\hyperlink{app:imi}{\[app:imi\]})
  Adapted from the \emph{Intrinsic Motivation Inventory (IMI)} by
  (\protect\hyperlink{ref-ryan1982control}{Ryan, 1982}), which is a multidimensional measurement device
  intended to assess participants' subjective experience related to a
  target activity (the assignments for the course they are taking).
  The instrument assesses participants' interest/enjoyment, perceived
  competence, effort/importance, pressure/tension, perceived choice,
  and value/usefulness, while performing a given activity, thus
  yielding six subscale scores. Three items in the value/usefulness
  subscale will be completed with contextual information when the
  study site is finalized. The pressure/tension and the perceived
  choice components will not be included in the entry survey, and will
  be present in the mid-term \[SUR2\] and exit \[SUS3\] surveys.
\item
  \textbf{Self-regulation:} (Appendix
  \protect\hyperlink{app:srq}{\[app:srq\]})
  Adapted from the \emph{Self-Regulation Questionnaire (SRQ)} by
  (\protect\hyperlink{ref-brown1999self}{J. M. Brown et al., 1999}), which assess seven self-regulatory processes
  through self-report: receiving relevant information, evaluating the
  information and comparing it to norms, triggering change, searching
  for options, formulating a plan, implementing the plan, and
  assessing the plan's effectiveness (Section
  \protect\hyperlink{sec:bg_learn_self_regulation}{\[sec:bg_learn_self_regulation\]}).
\item
  \textbf{Metacognition:} (Appendix
  \protect\hyperlink{app:mai}{\[app:mai\]})
  Adapted from the \emph{Metacognivite Awareness Inventory (MAI)},
  originally proposed by (\protect\hyperlink{ref-schraw1994assessing}{Schraw \& Dennison, 1994}) as a 52-item true /
  false questionnaire, and later revised by (\protect\hyperlink{ref-terlecki2018call}{Terlecki \& McMahon, 2018}) to use
  five-point Likert scales. The instrument measures two components of
  cognition through self-report: knowledge about cognition, and
  regulation of cognition (Section
  \protect\hyperlink{sec:bg_learn_metacognition}{\[sec:bg_learn_metacognition\]}).
\end{enumerate}

After completing the entry survey, participants will be asked to prepare
for the initial synchronous session, \[SES1\], by \emph{(i)} installing
Firefox web browser and the YASBIL extension on their machines, \emph{(ii)}
get a quick introduction to concept maps (by watching a short video),
and \emph{(iii)} familiarising themselves with the \emph{Sero!} learning platform
(for creating and assessing concept maps). This is a one-time step. If a
participant cannot find the time for this step, they will be informed
that an extra 5-10 minutes will be taken in the beginning of SES1 to
complete this step.

The entry survey and the software installation is expected to take about
10-15 minutes to complete. Participants will be compensated with USD 5
for their time for completing this step. The survey will be floated in
the first week of the semester, and will be closed after sufficient a
number participants have been recruited.

\hypertarget{ses1-initial-session}{%
\subsection{SES1: Initial Session}\label{ses1-initial-session}}

\begin{figure}
\hypertarget{fig_search_task_repeated}{%
\centering
\includegraphics{figs/search_task_repeated.pdf}
\caption{Prompts for the search task repeated in Initial Session \[SES1\] and
Final Session \[SES3\].}\label{fig_search_task_repeated}
}
\end{figure}

For the Initial Session, (aka SES1), participants will be invited to a
remote study session over a video-conferencing platform (Zoom). The
purpose of the the pre-test will be to establish baseline search
behaviour and domain knowledge of the participants. The study will
consist of a training task, two actual search tasks, one website
reliability assessment, and a memory span test. Each of these components
are described below.

\begin{enumerate}
\def\labelenumi{\arabic{enumi}.}
\item
  \textbf{Training Task:} Participants will perform a training task to
  familiarize themselves with how to operate the YASBIL browser
  extension to log their browsing activity, and how to create concept
  maps while they are searching. The training task is expected to take
  around 5-10 minutes.
\item
  \textbf{Two Search Tasks:} Participants will perform two search tasks, of
  which one will be repeated in the final session \[SES3\] at the end
  of the semester, and the other will not be repeated. This will help
  to answer research question RQ2 (Section
  \protect\hyperlink{sec:rq}{\[sec:rq\]}). The
  order of the two search tasks will be randomized. The generic format
  the search tasks are described in Figure
  \protect\hyperlink{fig_search_task_template}{1.2}.

  The repeated search task will be on the topic of financial literacy,
  a topic that may be considered to be universally important to
  college students, and part of lifelong learning. The SES1 and SES3
  versions of the repeated task are presented in Figure
  \protect\hyperlink{fig_search_task_repeated}{1.4}. The non-repeated search task
  will be on topics that will be taught in the course serving as the
  site for the study. Examples of search tasks used in the Pilot Study
  are in Appendix
  \protect\hyperlink{ch:pilot_study}{\[ch:pilot_study\]}. Each search task is expected to take
  around 20 minutes.

  To answer RQ3 (effect of externalisation and articulation in
  learning), each participant will perform one of the search tasks
  while thinking aloud (concurrent think-aloud, or \emph{CTA condition}),
  and perform the other one in silence (\emph{silent condition}). The
  choice of the task for each of the conditions will be randomized and
  balanced.

  Each search task will begin with a pre-task questionnaire (Appendix
  \protect\hyperlink{app:pre_task_qsn}{\[app:pre_task_qsn\]}), which asks participants to self-rate
  their pre-search topic knowledge and interest. Then participants
  will turn on the YASBIL browsing logger and start searching. The
  deliverable for each search task will be a written summary
  (artefact), and a concept map. After participants are satisfied with
  the quality of the deliverable, they will turn off YASBIL browsing
  logger, and proceed to the post-task questionnaire. The post-task
  questionnaire (Appendix
  \protect\hyperlink{app:post_task_qsn}{\[app:post_task_qsn\]}) asks participants to self-rate their
  post-search topic knowledge, search experience, interest and
  motivation, and overall perceptions. The pre-task and post-task
  questionnaires are adapted from
  (\protect\hyperlink{ref-collins2016assessing}{Collins-Thompson et al., 2016}; \protect\hyperlink{ref-crescenzi2020adaptation}{Crescenzi, 2020}). After the two
  search tasks are completed, participants will answer questions about
  whether they preferred the think-aloud condition or the silent
  condition, and why (Appendix
  \protect\hyperlink{app:cta_v_silent}{\[app:cta_v\_silent\]}).
\item
  \textbf{Webpage Comparison Assessment:} To assess participants'
  (mis)information evaluation capabilities, they will perform a
  Website Comparison assessment created by the
  (\protect\hyperlink{ref-sheg2021webpage_comparison}{Stanford History Education Group, 2021}). This task asks students to compare
  two websites and select the one that they would use to begin
  research on a topic. One of the pages is a Wikipedia article. The
  other has ``.edu'' in its URL, but the page reveals that the content
  is a student-written blog post created as part of a university
  course. Many students have been taught that Wikipedia pages are
  completely unreliable and should be avoided. Many have also been
  taught that sites with a .edu domain are trustworthy. This
  assessment gauges their ability to think in more nuanced ways about
  these kinds of sites. The website reliability assessment is expected
  to take 10 minutes.
\item
  \textbf{Memory Span Test:} The session will conclude with the assessment
  of the participant's working memory capacity (WMC) using a memory
  span task (\protect\hyperlink{ref-francis2004coglab}{Francis et al., 2004}). Memory span assessment is kept in
  the synchronous session because it is a timed task, and needs to
  conducted in a controlled (experimenter observed) condition.
\end{enumerate}

Participants will be asked to share their screen for the whole duration
of the session. Their screens and audio will be recorded for the entire
duration. They may choose to turn off their video. The total time for
SES1 is expected to not exceed 1.5 hours (90 minutes). Participants will
be compensated with USD 25 for this session.

\hypertarget{ses2-longitudinal-tracking}{%
\subsection{SES2: Longitudinal Tracking}\label{ses2-longitudinal-tracking}}

The longitudinal tracking \[SES2\] will be conducted asynchronously over
the duration of the semester, to understand the change (or lack thereof)
of participants' search behaviour and knowledge gain over time. Whenever
participants will work on different parts of their final project
research paper (which will be termed as SES2a, SES2b, \ldots etc.), as
described in Figure
~\protect\hyperlink{fig_final_project_description}{1.1}, they will use Firefox web
browser, and will log their browsing activity using the YASBIL browsing
logger. To protect their privacy, participants will be instructed to
turn YASBIL on only when they are searching for information related to
the course. In addition to the submitted working-draft of their research
paper, participants will submit a cumulative concept map with each
document submission. The cumulative concept map will help to track
participants' evolution of knowledge about the final project topic(s)
over the course of the semester. Participants will receive reminder
emails before the deadline of each assignment, to remind them to use
Firefox, turn YASBIL on, and incrementally update the concept map.
Participants will receive USD 5 per each assignment for which they log
data, up to a maximum of USD 20 for four assignments.

\hypertarget{sec:method_sur2}{%
\subsection{SUR2: Mid-Term Survey}\label{sec:method_sur2}}

The mid-term survey \[SUR2\] will take place around the mid point of the
semester (Week 8-9). The purpose is to track whether any of the
participants' individual difference measures (e.g.~motivation,
metacognition, course load etc.) changed during the first half of the
semester. This survey will essentially be a replica of the Entry Survey
\[SUR1\] (Section \protect\hyperlink{sec:method_sur1}{1.4.1}), with two modifications. First, the
consent form and the demographics sections will be absent. Second,
Intrinsic Motivation Inventory (IMI) will include the `pressure/tension'
and the `perceived choice' subscales, as these scales are more
meaningful after an activity has taken place (\protect\hyperlink{ref-ryan1982control}{Ryan, 1982}). The IMI
will also be reworded to reflect the mid-point of the semester (Appendix
\protect\hyperlink{app:midterm_survey}{\[app:midterm_survey\]}). Participants will be compensated with
USD 5 for their time for completing this step.

\hypertarget{ses3-final-session}{%
\subsection{SES3: Final Session}\label{ses3-final-session}}

The Final Session \[SES3\] will be similar in structure to the Initial
Session (SES1), and will take place at the end of the semester, after
all the course related tasks are completed by the student. The purpose
of the session is to record the `evolved' search behaviour, and final
knowledge state. Participants will perform two search tasks, one website
reliability assessment task, and take the memory span test once again.
In the end, there will be a short semi-structured interview.

Of the two search tasks, the topic of one will be repeated from SES1
(financial literacy, Figure
\protect\hyperlink{fig_search_task_repeated}{1.4}), while the topic of the other
will come from the course material. In both search tasks, participants
will be given the option of \textbf{\emph{not searching}} if they feel confident
enough to answer the search task questions from their prior knowledge.
The deliverables for each search-task will be a written summary
(artefact) and a concept map (Figure
\protect\hyperlink{fig_search_task_template}{1.2}).

Following the two search task, participants will perform another website
comparison/reliability task created by the
(\protect\hyperlink{ref-sheg2021webpage_comparison}{Stanford History Education Group, 2021}), which will assess their evolved
information evaluation skills. Then they will retake the memory span
test (\protect\hyperlink{ref-francis2004coglab}{Francis et al., 2004}).

A semi-structured interview will be conducted in the end, where
participants will reflect on their overall searching and learning
experience. Certain `interesting' handpicked sessions from their
submitted logs may be identified and questions about them can also be
asked to participants. A list of the interview questions asked in the
Pilot Study (Appendix
\protect\hyperlink{ch:pilot_study}{\[ch:pilot_study\]}) are presented in Appendix
\protect\hyperlink{app:post_task_interview}{\[app:post_task_interview\]}, which can be reused.

Similar to SES1, participants will be asked to share their screen for
the whole duration of the session, except for the interview, whence they
can stop screen-sharing. Their screens and audio will be recorded for
the same. They may choose to turn off their video. The total time for
SES3 is expected to not exceed 1.5 hours (90 minutes). Participants will
be compensated with USD 25 for this session, and will be asked to
complete the Exit Survey \[SUR3\] as soon as convenient.

\hypertarget{sur3-exit-survey}{%
\subsection{SUR3: Exit Survey}\label{sur3-exit-survey}}

The exit survey \[SUR3\] will take place after the Final Session
\[SES3\]. The purpose is to record the final state of the participants'
individual difference measures (e.g.~motivation, metacognition, course
load etc.), and whether they changed during the second half of the
semester. This survey will essentially be a replica of the mid-term
survey \[SUR2\] (Section \protect\hyperlink{sec:method_sur2}{1.4.4}), with the Intrinsic Motivation Inventory
(IMI) reworded to reflect the end-point of the semester (Appendix
\protect\hyperlink{app:final_survey}{\[app:final_survey\]}). Participants will be compensated with
USD 5 for their time for completing this step. If participants do not
take the survey within three days (say) after appearing for SES3, they
will be sent reminder emails.

Participants will be compensated with a bonus payment of USD 15, if they
complete all the parts of the study without missing any component.

\hypertarget{measurement-and-variables}{%
\section{Measurement and Variables}\label{measurement-and-variables}}

\hypertarget{independent-explanatory-search-interaction-and-process-measures}{%
\subsection{Independent / Explanatory: Search Interaction and Process Measures}\label{independent-explanatory-search-interaction-and-process-measures}}

The independent variables will be the search process behavioural
measures. Information searching behaviour will be operationalized using
a battery of search process measures, based on the three-stages of user
interaction discussed in Section
\protect\hyperlink{sec:bg_search_3_stage}{\[sec:bg_search_3\_stage\]}. These measures include query
reformulation types and measures
(Table~\protect\hyperlink{tab_res_Q_QRT_txnmy}{\[tab_res_Q\_QRT_txnmy\]}), SERP examination measures, content
page examination measures, and overall search session measures (Table
\protect\hyperlink{tab_search_behaviours}{\[tab_search_behaviours\]}). A non-exhaustive list of such
search process measures which have been used in prior literature is
presented in Appendices
\protect\hyperlink{sec:app_vars_qry}{\[sec:app_vars_qry\]} through
\protect\hyperlink{sec:app_vars_overall_search}{\[sec:app_vars_overall_search\]}.

\hypertarget{dependant-outcome-learning-measures}{%
\subsection{Dependant / Outcome: Learning Measures}\label{dependant-outcome-learning-measures}}

Learning (knowledge gain) will constitute the dependant variables, or
factor variables, when dichotomized via median split. Learning outcomes
are planned to be assessed by: \emph{(i)} analysis of concept maps
(\protect\hyperlink{ref-halttunen2005assessing}{Halttunen \& Jarvelin, 2005}) (Appendix
\protect\hyperlink{sec:app_vars_concept_maps}{\[sec:app_vars_concept_maps\]}); \emph{(ii)} analysis of written
summaries / knowledge artefacts (\protect\hyperlink{ref-wilson2013comparison}{M. J. Wilson \& Wilson, 2013}); and \emph{(iii)}
instructor awarded scores and grades received by students in the course,
which will be obtained via FERPA release. Time, resources and
feasibility permitting, other possible ways of assessing learning can be
by using \emph{(iv)} Online Research and Comprehension Assessment (ORCA)
(\protect\hyperlink{ref-leu2015new}{Leu et al., 2015} Table 3), (\protect\hyperlink{ref-kanniainen2021assessing}{Kanniainen et al., 2021} Appendix A); and \emph{(v)}
information-use from websites in written artefacts
(\protect\hyperlink{ref-vakkari2020usefulness}{Vakkari, 2020}; \protect\hyperlink{ref-vakkari2019modeling}{Vakkari et al., 2019}).

\hypertarget{moderator-individual-differences}{%
\subsection{Moderator: Individual Differences}\label{moderator-individual-differences}}

The variables of individual differences that are hypothesized to
moderate learning are \emph{(i)} motivation, scored using Intrinsic
Motivation Inventory (IMI) (\protect\hyperlink{ref-ryan1982control}{Ryan, 1982}); \emph{(ii)} self-regulation,
scored using Self-Regulation Questionnaire (SRQ) (\protect\hyperlink{ref-brown1999self}{J. M. Brown et al., 1999});
\emph{(iii)} metacognition, scored using revised Metacognivite Awareness
Inventory (MAI) (\protect\hyperlink{ref-schraw1994assessing}{Schraw \& Dennison, 1994}; \protect\hyperlink{ref-terlecki2018call}{Terlecki \& McMahon, 2018}) \emph{(iv)} memory
span, scored using memory span test (\protect\hyperlink{ref-francis2004coglab}{Francis et al., 2004}); \emph{(v)} search
proficiency, scored using Digital Health Literacy Instrument (DHLI)
(\protect\hyperlink{ref-van2017development}{Van Der Vaart \& Drossaert, 2017}) and Search Self-Efficacy Scale (SSE)
(\protect\hyperlink{ref-brennan2016factor}{Brennan et al., 2016}); \emph{(vi)} information evaluation capabilities
(mastery / emerging / beginner), scored according to rubrics provided by
(\protect\hyperlink{ref-sheg2021webpage_comparison}{Stanford History Education Group, 2021}) assessments (an example grading rubric is
present in Appendix
\protect\hyperlink{sec:app_pilot_ses3}{\[sec:app_pilot_ses3\]} Task 3).

\hypertarget{data-analysis-plans}{%
\section{Data Analysis Plans}\label{data-analysis-plans}}

Exploratory data analysis (such as time series plotting) and descriptive
statistics will be used to identify if changes in search process /
interaction measures can be visually observed over the course of time.
Inferential statistics (difference between groups) will be employed to
test if there are significant differences in the learning measures
between student groups who learn more versus learn less. Pattern Mining
and clustering approaches may also be used to identify clusters or
patterns in the search process (time-series) data, and see if these
clusters correlate with high and low learning. An example of measuring
such changes in variables can be found in (\protect\hyperlink{ref-133}{Mao et al., 2018, sec. 3.2}). Advanced
search interactions such as parallel browsing behaviour (multi-tabbed
and multi-windowed browsing) may also be analysed
(\protect\hyperlink{ref-huang2010parallel}{J. Huang \& White, 2010}; \protect\hyperlink{ref-labaj2012modeling}{Labaj \& Bielikova, 2012}).

\hypertarget{anticipated-limitations}{%
\section{Anticipated Limitations}\label{anticipated-limitations}}

There are foreseeable limitations to this proposed longitudinal study.
First, there may not be enough participants who sign up for the study.
The remedy for this is choosing a course with a large number of
students, and using appealing messaging in the recruitment material
(e.g., an attractive video message was used to recruit participants in
the Pilot Study\footnote{\url{https://youtu.be/RkBUZ4At8Qg}}). Second, participants may drop off due to various
reasons during the study. This can be tackled by regularly communicating
with the participants, keeping them engaged with affectionate, caring
and encouraging messaging, and letting them know that their
participation is valued highly by the researchers. Third, participants
may not show any changes in their search behaviours, of the changes may
be random. As one anonymous reviewer put it ``the smart kids will show up
smart and with good search skills and they will leave smart with good
search skills and they will not change their inherent behavior over the
time. Similarly, the dumb kids will show up dumb, do whatever keeps them
dumb and end up dumb.'' Alternatively, there may be are not changes over
time, but rather strategies that are consistently followed by students
who will learn more, and students who will learn less. There is no easy
fix to this, and if this happens, it will be treated as a finding from
the study. A possible rescue from this situation would then be to use
the (semi) qualitative data -- the semi structured interview at the end,
the concurrent think alouds, and others -- to derive interesting
findings.

\hypertarget{proposed-dissertation-timeline}{%
\section{Proposed Dissertation Timeline}\label{proposed-dissertation-timeline}}

\begin{figure}
\centering
\includegraphics{figs/timeline_data_collection.pdf}
\caption{image}
\end{figure}

\begin{enumerate}
\def\labelenumi{\arabic{enumi}.}
\item
  \textbf{November 2021:} prepare and submit IRB proposal
\item
  \textbf{December 2021:} defend proposal, and if changes to the study
  protocol are suggested by the committee, submit them as an IRB
  amendment
\item
  \textbf{January - May 2022:} conduct longitudinal study; a week-by-week
  schedule is in Table
  \protect\hyperlink{tab_timeline_data_collection}{\[tab_timeline_data_collection\]}
\item
  \textbf{Summer and Fall 2022:} two backup semesters for data collection,
  if something goes wrong in Spring 2022, especially due to COVID-19
  pandemic situations.
\item
  \textbf{August 2022 - February 2023:} analyze data and write dissertation
\item
  \textbf{April 2023:} Dissertation defense
\item
  \textbf{May 2023:} revise and complete dissertation
\end{enumerate}

\hypertarget{ch:pilot_study}{%
\chapter{Prior Work: Pilot Study}\label{ch:pilot_study}}

A pilot study was conducted in the Summer of 2021 at the School of
Information, University of Texas at Austin (Texas iSchool). This was
mainly a feasibility study to determine the technical logistics and
participant retention rates. It included SES1, SES2, and SES3 from the
proposed study procedure (Figure
\protect\hyperlink{fig_study_procedure}{\[fig_study_procedure\]}). There was no recording of individual
differences via recruitment, mid-term, and final surveys. It also did
not include submission of concept maps. Eight students from two courses
at the Texas iSchool -- \emph{Academic Success in the Digital University}
(ACS), and \emph{Information in Cyberspace} (CYB) -- participated in the
pilot study. The study ran from start of June 2021 to mid-August 2021.
There was no participant drop-off. Synchronous sessions (SES1 and SES3)
were conducted over the Zoom video conferencing platform. Log data was
captured using the YASBIL browser extension (\protect\hyperlink{ref-bhattacharya2021yasbil}{Bhattacharya \& Gwizdka, 2021}).
All setup and technical logistics worked out properly, without any major
technological issues. Details of the search task descriptions are
presented below. Participants were compensated with USD 15 for SES1, 15
points of extra course credit for longitudinal tracking in SES2, and USD
15 for SES3.

\hypertarget{ses1-initial-session-1}{%
\section{SES1: Initial Session}\label{ses1-initial-session-1}}

Participants performed a training task to familiarise themselves with
the YASBIL browser extension. The they performed two search tasks as
described below. Each search task was followed by measurement of mental
workload using NASA-TLX.

~\\
\textbf{Task 1: Financial Literacy} (Repeated in SES3)

Money management and financial literacy are essential life skills, and
what better time to learn about them than in college? Write a note to
your future self, about essential money-related advice and skills that
college students should know and practice.

\textbf{What to do:}

\begin{itemize}
\item
  Find at least 3 unique, good quality online resources that are
  relevant to this topic
\item
  Look for resources that help establish connections and develop a
  narrative
\end{itemize}

\textbf{What to deliver:}

\begin{itemize}
\item
  Write a summary of the lessons, advice, and/or tips you found across
  the different resources. This is a note to your future self, so the
  narrative can be in a format that is most useful and interesting to
  YOU
\item
  Paste the links of ALL the resources that you finally selected to
  develop your narrative, in the second text box, one link per line
\end{itemize}

~\\
\textbf{Task 2: Social Media during COVID-19} (Topic was part of course
content in ACS and CYB)

``What was the role of Social Media during the COVID-19 pandemic? How did
it affect people's lives during quarantine and social distancing?''
Suppose a family member (say your aunt) or a friend asked you these
question over a phone call, and you want to talk to them on this topic
for a couple of minutes.

\textbf{What to do:}

\begin{itemize}
\item
  Find at least 3 unique, good quality online resources that are
  relevant to this topic
\item
  Look for resources that help establish connections and develop a
  narrative
\end{itemize}

\textbf{What to deliver:}

\begin{itemize}
\item
  Write a short summary of the content that you found across the
  different resources. The length and writing style can be such that
  you can read it out to your family member/friend over a phone call,
  without them losing interest.
\item
  In the summary, briefly mention your thoughts about each resource -
  do you agree or disagree with the content in the resource? Anything
  else?
\item
  Paste the links of ALL the resources that you finally selected to
  develop your narrative, in the second text box, one link per line
\end{itemize}

\hypertarget{ses2-longitudinal-tracking-1}{%
\section{SES2: Longitudinal Tracking}\label{ses2-longitudinal-tracking-1}}

The longitudinal tracking session SES2 involved student participants
submitting log data for two final-project assignments for the ACS
course, and four final project assignments for the CYB course.
Participants received reminder emails to log and sync their data a few
days before each assignment was due. Seven (out of 8) participants
logged their data and synced it with our data server in a timely
fashion, without major technical issues. One participant CYB course
forgot to log their data for the first two assignments, despite the
email reminder. However, upon following up with them, they remembered to
log their data for the third and fourth sessions.

\hypertarget{sec:app_pilot_ses3}{%
\section{SES3: Final Session}\label{sec:app_pilot_ses3}}

All eight participants from SES1 completed SES3 (no drop off).
Participants performed two search tasks, and one website reliability
evaluation task as described below. All three tasks were followed by
measurement of mental workload using NASA-TLX. At the end of the tasks,
they underwent a short semi-structured interview to reflect on their
overall study experience. The interview questions can be found in
Appendix
\protect\hyperlink{app:post_task_interview}{\[app:post_task_interview\]}.

~\\
\textbf{Task 1: Financial Literacy} (Repeated from SES1)

At the start of the semester, you wrote a note to your future self,
about essential money-related advice and skills that college students
should know and practice.

Here is what you wrote:\\
\texttt{{[}dynamic\ content\ showing\ participants’\ previous\ responses{]}}

Here are the resources you took help from:\\
\texttt{{[}dynamic\ content\ showing\ participants’\ previous\ responses{]}}

Now you have a chance to \textbf{update or revise the note with more
information}. You can either choose to write afresh, or copy-paste the
note from above into the first textbox below and add to it /edit it.
Feel free to search the web if you need to, after turning YASBIL on. You
can choose \textbf{NOT to search}, as well.

If you do choose to search, please paste the links of ALL the resources
that you finally selected for updating your note, one link per line, in
the second textbox. The links can be the same ones you visited earlier,
or different.

Did you need to search the web for updating the note? Why?

~\\
\textbf{Task 2: HTML CSS} (Topic was part of course content in ACS and CYB)

In your course, you studied about websites, HTML, and CSS. Therefore,
for answering the questions below, \textbf{you may choose NOT to search the
web}, if you feel you can answer the questions reasonably well. If you
do need to search the web, feel free to do so, after turning on YASBIL.

As you understand these concepts, please explain (with examples if
necessary)

\begin{enumerate}
\def\labelenumi{\arabic{enumi}.}
\item
  what is the purpose of HTML?
\item
  what is the purpose of CSS?
\item
  how do HTML and CSS come together when someone visits a website?
\end{enumerate}

List as many HTML tags as you can, one per line, in the following
format:\\
\texttt{{[}HTML\ tag{]}\ -\ {[}few\ words\ explaining\ the\ function\ of\ the\ tag{]}}

List as many CSS properties as you can, one per line, in the following
format:\\
\texttt{{[}CSS\ property{]}\ -\ {[}few\ words\ explaining\ the\ function\ of\ the\ property{]}}

Did you need to search the web for this task? Why?

~\\
\textbf{Task 3: Website Reliability Evaluation} (From Stanford History
Education Group\footnote{\url{https://developer.chrome.com/docs/extensions/reference/history/\#transition-types}})

You are researching children's health and come across this website:

\href{https://acpeds.org}{\texttt{https://acpeds.org}}.

Please decide if this website is a trustworthy source of information on
children's health. You may use any information on this website, or you
can open a new tab and do an Internet search if you want. Take about 5
minutes to complete this task. Turn YASBIL on before proceeding.

Is this website a trustworthy source to learn about children's health?
Explain your answer, citing evidence from the webpages you used. Be sure
to provide the URLs to the webpages you cite in the next textbox.

Please paste the URLs to the webpages you used to explain your answer
above, one per line.

\textbf{Grading Rubric for Task 3 (as provided by SHEG):}

This task presents students with the website of the American College of
Pediatricians (ACPeds.org) and asks them whether it is a trustworthy
source to learn about children's health. Despite the site's professional
title and appearance, the American College of Pediatricians is not the
nation's major professional organization of pediatricians. That
designation belongs to the similarly named American Academy of
Pediatrics. The American College of Pediatricians is a conservative
advocacy organization established in 2002 in response to the American
Academy of Pediatrics' support of adoption by same-gender couples. The
American College of Pediatricians website features a mission statement
that reads, in part, ``We recognize the basic father-mother family unit,
within the context of marriage, to be the optimal setting for childhood
development.'' News releases on the site include headlines like,
``Same-Sex Marriage -- Detrimental to Children'' and ``Know Your ABCs: The
Abortion Breast Cancer Link.''

~\\
This exercise is an open web search in which students are free to stay
on the American College of Pediatricians site or leave it to search for
information about the group. Successful students will look beyond the
surface features of the site and detect its agenda from its new releases
or other focus issues. A faster route, however, is to leave the site
almost immediately to search for reliable information about the true
agenda of this organization.

~\\
\textbf{Mastery:} Student rejects the website as a trustworthy source and
provides a clear rationale. Student provides reliable supporting
evidence and cites the source of information.

\textbf{Emerging:} Student rejects the website as a trustworthy source and
provides supporting evidence. However, the response falls short of
Mastery because: \emph{(i)} Student provides relevant evidence and says where
the evidence is from, but the explanation is incomplete; \emph{(ii)} Student
provides a complete explanation that is supported by relevant evidence
but does not say where the evidence is from.

\textbf{Beginning:} Student rejects the source but provides an incoherent,
irrelevant, or unreasonable explanation; or the student simply accepts
the source as trustworthy.

After Task 3, participants participated in a semi-structured interview
to reflect on their overall searching and learning experience, as well
as participating in the study. The questions asked in the interview are
presented in Appendix
\protect\hyperlink{app:post_task_interview}{\[app:post_task_interview\]}.

\hypertarget{appendix:signup_survey}{%
\chapter{SUR1: Entry Survey}\label{appendix:signup_survey}}

\hypertarget{app:demographics}{%
\section{Demographics}\label{app:demographics}}

\begin{enumerate}
\def\labelenumi{\arabic{enumi}.}
\item
  Please select the degree level/name of the program you are in.
\item
  Please state which year of the program you are in.
\item
  Please state your major(s)
\item
  Do you have native-level familiarity with English language? Yes / No
  / Other:
\item
  Please state your age (in years)
\item
  Please state your gender
\item
  With which ethnicities do you identify? Please check all that apply:

  \begin{itemize}
  \item
    African
  \item
    African American / Black
  \item
    Asian - East
  \item
    Asian - South East
  \item
    Asian - South
  \item
    Asian - Middle East
  \item
    Caucasian / White
  \item
    Hispanic / Latinx
  \item
    Native American
  \item
    Pacific Islander
  \item
    Mixed
  \item
    Other:
  \end{itemize}
\item
  Are you an international student? Yes / No; If Yes, where are you
  originally from?
\item
  Please enter an email address that you check regularly. We will send
  communications and compensation information to this email address.
\item
  Your name as you would like us to address you.
\end{enumerate}

\hypertarget{app:search_it_proficiency}{%
\section{Search and IT Proficiency}\label{app:search_it_proficiency}}

\begin{enumerate}
\def\labelenumi{\arabic{enumi}.}
\item
  Which device(s) and browser(s) do you normally use to surf the
  internet?

  \begin{longtable}[]{@{}llllllll@{}}
  \toprule()
  & Chrome & Safari & Firefox & Edge & Opera & Other & None \\
  \midrule()
  \endhead
  Desktop & & & & & & & \\
  Laptop & & & & & & & \\
  Tablet & & & & & & & \\
  Smartphone & & & & & & & \\
  \bottomrule()
  \end{longtable}
\item
  How comfortable are you with using Mozilla Firefox to search
  information on the internet?

  \begin{itemize}
  \item
    I do not know how to use Mozilla Firefox.
  \item
    I have never used Mozilla Firefox.
  \item
    I feel very uncomfortable to use Mozilla Firefox.
  \item
    I feel uncomfortable to use Mozilla Firefox.
  \item
    I feel neither comfortable nor uncomfortable to use Mozilla
    Firefox.
  \item
    I feel comfortable to use Mozilla Firefox.
  \item
    I feel very comfortable to use Mozilla Firefox.
  \item
    Other:
  \end{itemize}
\item
  Which search engines do you normally use?

  \begin{itemize}
  \item
    Google
  \item
    Bing
  \item
    Baidu
  \item
    Yahoo!
  \item
    Yandex
  \item
    DuckDuckGo
  \item
    Other:
  \end{itemize}
\end{enumerate}

~\\
The following items are adapted from the Digital Health Literacy
Instrument (DHLI) by (\protect\hyperlink{ref-van2017development}{Van Der Vaart \& Drossaert, 2017}).

\emph{On a scale of 1 to 5 \ldots{}}

\emph{(1) Very difficult/Very seldom \textbar{} Difficult/Seldom \textbar{} Neutral \textbar{}
Easy/Often \textbar{} Very easy/Very often (5)}

\emph{How easy or difficult is it for you to\ldots{}}

\begin{enumerate}
\def\labelenumi{\arabic{enumi}.}
\setcounter{enumi}{3}
\item
  Use the keyboard of a computer (e.g., to type words)?
\item
  Use the mouse (e.g., to put the cursor in the right field or to
  click)?
\item
  Use the buttons or links and hyperlinks on websites?
\end{enumerate}

\emph{When you search the Internet for information, how easy or difficult is
it for you to \ldots{}}

\begin{enumerate}
\def\labelenumi{\arabic{enumi}.}
\setcounter{enumi}{6}
\item
  Make a choice from all the information you find?
\item
  Use the proper words or search query to find the information you are
  looking for
\item
  Find the exact information you are looking for?
\item
  Decide whether the information is reliable or not?
\item
  Decide whether the information is written with commercial interests
  (e.g., by people trying to sell a product)?
\item
  Check different websites to see whether they provide the same
  information?
\item
  Decide if the information you found is applicable to your situation?
\item
  Apply the information you found in your daily life?
\item
  Use the information you found to make decisions about your life
\end{enumerate}

\emph{When you search the Internet for information, how often does it happen
that\ldots{}}

\begin{enumerate}
\def\labelenumi{\arabic{enumi}.}
\setcounter{enumi}{14}
\item
  You lose track of where you are on a website or the Internet?
\item
  You do not know how to return to a previous page?
\item
  You click on something and get to see something different than you
  expected?
\end{enumerate}

~\\
The following items are adapted from the Search Self-Efficacy Scale
(SSE) by (\protect\hyperlink{ref-brennan2016factor}{Brennan et al., 2016}).

\emph{On a scale of 1 to 5, how confident are you that you can \ldots{}}

\emph{(1) Not at all confident \textbar{} Neither confident nor unconfident \textbar{}
Totally confident (5)}

\begin{enumerate}
\def\labelenumi{\arabic{enumi}.}
\setcounter{enumi}{17}
\item
  Identify the major requirements of the search from the initial
  statement of the topic.
\item
  Correctly develop search queries to reflect my requirements.
\item
  Use special syntax in advanced searching (e.g., AND, OR, NOT).
\item
  Evaluate the resulting list to monitor the success of my approach.
\item
  Develop a search query which will retrieve a large number of
  appropriate articles.
\item
  Find an adequate number of articles.
\item
  Find articles similar in quality to those obtained by a professional
  searcher.
\item
  Devise a query which will result in a very small percentage of
  irrelevant items on my list.
\item
  Efficiently structure my time to complete the task.
\item
  Develop a focused search query that will retrieve a small number of
  appropriate articles.
\item
  Distinguish between relevant and irrelevant articles.
\item
  Complete the search competently and effectively.
\item
  Complete the individual steps of the search with little difficulty.
\item
  Structure my time effectively so that I will finish the search in
  the allocated time.
\end{enumerate}

\hypertarget{app:course_load}{%
\section{Course Load and Other Engagements}\label{app:course_load}}

\begin{enumerate}
\def\labelenumi{\arabic{enumi}.}
\item
  How many total weekly hours of coursework are you registered for
  this semester?
\item
  How many weekly hours do you anticipate putting in for studying this
  course?
\item
  What are your other time commitments, as hours per week? (enter 0 if
  not applicable)

  \begin{itemize}
  \item
    jobs
  \item
    extra-curriculars
  \item
    other
  \end{itemize}
\item
  Do you hold a position of responsibility (officer / committee
  member) in any (student) organisation? Yes / No
\end{enumerate}

\hypertarget{app:note_taking_strategies}{%
\section{Note-taking Strategies}\label{app:note_taking_strategies}}

Adapted from \emph{Listening and Note Taking Survey} by
(\protect\hyperlink{ref-note_taking_survey_penn_state}{Penn State Learning, 2021}), and \emph{Note Taking Strategies Inventory}
by (\protect\hyperlink{ref-note_taking_strategies_umass}{UMass Amherst Student Success, 2021}).

\emph{For each question, choose the response that best describes your actions
(not the one that describes what you think you should be doing). There
are no right or wrong answers. In general (not specifically for this
course)}

\begin{enumerate}
\def\labelenumi{\arabic{enumi}.}
\item
  I take notes using (check all that apply)

  \begin{itemize}
  \item
    Paper and Pen / Pencil
  \item
    Laptop / Desktop
  \item
    Tablet with Keyboard
  \item
    Tablet with Stylus / Digital Pen
  \end{itemize}
\item
  When taking notes on the laptop, I minimize distractions by:
\item
  Are you familiar with ``concept maps''?

  \begin{itemize}
  \item
    Never heard of it
  \item
    Heard of it, never used it
  \item
    Used it a few times in the past
  \item
    Used it quite often in the past; now do not use it
  \item
    Use it regularly
  \end{itemize}
\end{enumerate}

\emph{On a scale of 1 to 5 \ldots{}}

\emph{(1) Never \textbar{} Rarely \textbar{} Sometimes \textbar{} Often \textbar{} Always (5)}

\begin{enumerate}
\def\labelenumi{\arabic{enumi}.}
\setcounter{enumi}{3}
\item
  I read my assignments before I go to lecture.
\item
  I find lectures interesting and/or challenging.
\item
  My lecture notes are well organized.
\item
  I recognize main ideas in lectures.
\item
  I recognize supporting details of main ideas.
\item
  I recognize patterns in lectures, e.g., cause-effect,
  concept-example.
\item
  My lecture notes are complete.
\item
  I recognize relationships between lecture and readings.
\item
  I integrate my lecture notes with my reading notes.
\item
  I summarize my notes, both lecture and reading, in my own words.
\item
  I review my notes immediately after class.
\item
  I conduct weekly reviews of my notes.
\item
  I edit my notes within 24 hours after class.
\item
  I take notes
\item
  I put dates on my notes
\item
  I makes notes in the margins of the text when I read (on paper /
  digital medium, e.g.~iPad and Apple Pencil)
\item
  I pause periodically while reviewing notes to summarize or
  paraphrase the information.
\item
  I use diagrams in my notes
\item
  I use different colours when writing notes
\item
  I create outlines, concept maps or organizational charts of how
  ideas fit together.
\item
  I write down questions I want to ask the instructor
\item
  I reorganize and fill in notes I took in class
\item
  I put things in my own words
\item
  I rewrite my notes
\item
  I use abbreviations in my notes
\item
  I write out my own descriptions of the main concepts
\item
  I keep track of things I do not understand and note when they
  finally become clear and what made that happen
\item
  I understand my notes
\item
  I refer back to my notes
\end{enumerate}

\begin{enumerate}
\def\labelenumi{\arabic{enumi}.}
\setcounter{enumi}{32}
\item
  How do you organise your notes?
\item
  How often do you refer back to your notes?
\item
  Have you ever wished that you had written better notes? Yes / No.
  Why?
\item
  How long do you store your notes for?

  \begin{itemize}
  \item
    Till the end of the semester
  \item
    End of academic year
  \item
    End of college
  \item
    Lifelong
  \end{itemize}
\item
  How do you search for a bit of information in your notes?
\end{enumerate}

\hypertarget{app:imi}{%
\section{Motivation}\label{app:imi}}

Adapted from Intrinsic Motivation Inventory (IMI) (\protect\hyperlink{ref-ryan1982control}{Ryan, 1982}).
Items will be randomly ordered.

\emph{For each of the following statements, please indicate how true it is
for you, using the following scale:}

\emph{(1) not at all true \textbar{} somewhat true \textbar{} very true (5)}

\textbf{Interest/Enjoyment}\\
I will enjoy taking this course very much.\\
This course will be fun to do.\\
I think this will be a boring course. (R)\\
This course will not hold my attention at all. (R)\\
I would describe this course as very interesting.\\
I think this course will be quite enjoyable.

~\\
\textbf{Perceived Competence}\\
I think I will be pretty good at this course.\\
I think I will be doing pretty well at this course, compared to other
students.\\
After working at this course for awhile, I will feel pretty competent.\\
I think I will be satisfied with my performance in this course.\\
I think I am pretty skilled at this course.\\
This is a course that I think would not be able to do very well. (R)

~\\
\textbf{Effort/Importance}\\
I plan to put a lot of effort into this course.\\
I don't think I will try very hard to do well at this course. (R)\\
I will try very hard on this course.\\
It is important to me to do well in this course.\\
I do not plan to put much energy into this course. (R)

~\\
\textbf{Value/Usefulness} \footnote{\url{https://developer.chrome.com/docs/extensions/reference/history/\#transition-types}}\strut \\
I believe this course could be of some value to me.\\
I think that doing this course is useful for learning about\\
I think this course is important to take because it can\\
I think taking this course could help me to\\
I believe taking this course could be beneficial to me.\\
I think this is an important course.

~\\
\textbf{Scoring directions:} Score each response from 1 (not at all true) to
5 (very true). Then reverse score the items marked with (R). To do that,
subtract the item response from 6, and use the resulting number as the
item score. Then, calculate subscale scores by averaging across all of
the items on that subscale. The subscale scores are then used in the
analyses of relevant research questions.

\hypertarget{app:srq}{%
\section{Self-regulation}\label{app:srq}}

Self-Regulation Questionnaire (SRQ) by (\protect\hyperlink{ref-brown1999self}{J. M. Brown et al., 1999}).

\emph{Please answer the following questions by selecting the option that best
describes how you are. There are no right or wrong answers. Work quickly
and don't think too long about your answers.}

\emph{(1) Strongly Disagree \textbar{} Disagree \textbar{} Neutral \textbar{} Agree \textbar{} Strongly Agree
(5)}

2

\begin{enumerate}
\def\labelenumi{\arabic{enumi}.}
\item
  I usually keep track of my progress toward my goals.
\item
  My behavior is not that different from other people's. (R)
\item
  Others tell me that I keep on with things too long. (R)
\item
  I doubt I could change even if I wanted to. (R)
\item
  I have trouble making up my mind about things. (R)
\item
  I get easily distracted from my plans. (R)
\item
  I reward myself for progress toward my goals.
\item
  I don't notice the effects of my actions until it's too late. (R)
\item
  My behavior is similar to that of my friends. Evaluating
\item
  It's hard for me to see anything helpful about changing my ways.
  Triggering R
\item
  I am able to accomplish goals I set for myself. Searching
\item
  I put off making decisions. Planning R
\item
  I have so many plans that it's hard for me to focus on any one of
  them. (R)
\item
  I change the way I do things when I see a problem with how things
  are going.
\item
  It's hard for me to notice when I've ``had enough'' (alcohol, food,
  sweets, internet, social media) (R)
\item
  I think a lot about what other people think of me.
\item
  I am willing to consider other ways of doing things.
\item
  If I wanted to change, I am confident that I could do it.
\item
  When it comes to deciding about a change, I feel overwhelmed by the
  choices. (R)
\item
  I have trouble following through with things once I've made up my
  mind to do something. (R)
\item
  I don't seem to learn from my mistakes. (R)
\item
  I'm usually careful not to overdo it when working, eating, drinking,
  or being on social media.
\item
  I tend to compare myself with other people.
\item
  I enjoy a routine, and like things to stay the same. (R)
\item
  I have sought out advice or information about changing. Searching
\item
  I can come up with lots of ways to change, but it's hard for me to
  decide which one to use. (R)
\item
  I can stick to a plan that's working well.
\item
  I usually only have to make a mistake one time in order to learn
  from it.
\item
  I don't learn well from punishment. (R)
\item
  I have personal standards, and try to live up to them.
\item
  I am set in my ways. (R)
\item
  As soon as I see a problem or challenge, I start looking for
  possible solutions.
\item
  I have a hard time setting goals for myself. (R)
\item
  I have a lot of willpower.
\item
  When I'm trying to change something, I pay a lot of attention to how
  I'm doing.
\item
  I usually judge what I'm doing by the consequences of my actions.
\item
  I don't care if I'm different from most people. (R)
\item
  As soon as I see things aren't going right I want to do something
  about it.
\item
  There is usually more than one way to accomplish something.
\item
  I have trouble making plans to help me reach my goals. (R)
\item
  I am able to resist temptation.
\item
  I set goals for myself and keep track of my progress.
\item
  Most of the time I don't pay attention to what I'm doing. (R)
\item
  I try to be like people around me.
\item
  I tend to keep doing the same thing, even when it doesn't work. (R)
\item
  I can usually find several different possibilities when I want to
  change something.
\item
  Once I have a goal, I can usually plan how to reach it.
\item
  I have rules that I stick by no matter what.
\item
  If I make a resolution to change something, I pay a lot of attention
  to how I'm doing.
\item
  Often I don't notice what I'm doing until someone calls it to my
  attention. (R)
\item
  I think a lot about how I'm doing.
\item
  Usually I see the need to change before others do.
\item
  I'm good at finding different ways to get what I want.
\item
  I usually think before I act.
\item
  Little problems or distractions throw me off course. (R)
\item
  I feel bad when I don't meet my goals.
\item
  I learn from my mistakes.
\item
  I know how I want to be.
\item
  It bothers me when things aren't the way I want them.
\item
  I call in others for help when I need it.
\item
  Before making a decision, I consider what is likely to happen if I
  do one thing or another.
\item
  I give up quickly. (R)
\item
  I usually decide to change and hope for the best. (R)
\end{enumerate}

\textbf{Scoring Directions:} Score each response from 1 (strongly disagree)
to 5 (strongly agree), and calculate the following subscale scores by
summing the items on that subscale. Items marked (R) are reverse-coded
(i.e.~1 = strongly agree and 5 = strongly disagree).

\begin{enumerate}
\def\labelenumi{\arabic{enumi}.}
\item
  \emph{Receiving relevant information:} 1, 8, 15, 22, 29, 36, 43, 50, 57
\item
  \emph{Evaluating the information and comparing it to norms:} 2, 9, 16,
  23, 30, 37, 44, 51, 58
\item
  \emph{Triggering change:} 3, 10, 17, 24, 31, 38, 45, 52, 59
\item
  \emph{Searching for options:} 4, 11, 18, 25, 32, 39, 46, 53, 60
\item
  \emph{Formulating a plan:} 5, 12, 19, 26, 33, 40, 47, 54, 61
\item
  \emph{Implementing the plan:} 6, 13, 20, 27, 34, 41, 48, 55, 62
\item
  \emph{Assessing the plan's effectiveness:} 7, 14, 21, 28, 35, 42, 49, 56,
  63
\end{enumerate}

\hypertarget{app:mai}{%
\section{Metacognition}\label{app:mai}}

Metacognivite Awareness Inventory (MAI) proposed by
(\protect\hyperlink{ref-schraw1994assessing}{Schraw \& Dennison, 1994}) and revised by (\protect\hyperlink{ref-terlecki2018call}{Terlecki \& McMahon, 2018}).

\emph{Think of yourself as a \textbf{learner}. Read each statement carefully, and
rate it as it generally applies to you when you are in the role of a
learner (student, attending classes, university etc.) Please indicate
how true each reason is for you using the following scale:}

\textbar{} x0.18 \textbar{} x0.18 \textbar{} x0.18 \textbar{} x0.18 \textbar{} x0.18 \textbar{} \emph{I \textbf{NEVER} do this} \&
\emph{I do this \textbf{infrequently}} \& \emph{I do this \textbf{inconsistently}} \& \emph{I do
this \textbf{frequently}} \& \emph{I \textbf{ALWAYS} do this}\\

2

\begin{enumerate}
\def\labelenumi{\arabic{enumi}.}
\item
  I ask myself periodically if I am meeting my goals.
\item
  I consider several alternatives to a problem before I answer.
\item
  I try to use strategies that have worked in the past.
\item
  I pace myself while learning in order to have enough time.
\item
  I understand my intellectual strengths and weaknesses.
\item
  I think about what I really need to learn before I begin a task.
\item
  I know how well I did once I finish a test.
\item
  I set specific goals before I begin a task.
\item
  I slow down when I encounter important information.
\item
  I know what kind of information is most important to learn.
\item
  I ask myself if I have considered all options when solving a
  problem.
\item
  I am good at organizing information.
\item
  I consciously focus my attention on important information.
\item
  I have a specific purpose for each strategy I use.
\item
  I learn best when I know something about the topic.
\item
  I know what the teacher expects me to learn.
\item
  I am good at remembering information.
\item
  I use different learning strategies depending on the situation.
\item
  I ask myself if there was an easier way to do things after I finish
  a task.
\item
  I have control over how well I learn.
\item
  I periodically review to help me understand important relationships.
\item
  I ask myself questions about the material before I begin.
\item
  I think of several ways to solve a problem and choose the best one.
\item
  I summarize what I've learned after I finish.
\item
  I ask others for help when I don't understand something.
\item
  I can motivate myself to learn when I need to.
\item
  I am aware of what strategies I use when I study.
\item
  I find myself analyzing the usefulness of strategies while I study.
\item
  I use my intellectual strengths to compensate for my weaknesses.
\item
  I focus on the meaning and significance of new information.
\item
  I create my own examples to make information more meaningful.
\item
  I am a good judge of how well I understand something.
\item
  I find myself using helpful learning strategies automatically.
\item
  I find myself pausing regularly to check my comprehension.
\item
  I know when each strategy I use will be most effective.
\item
  I ask myself how well I accomplish my goals once I'm finished.
\item
  I draw pictures or diagrams to help me understand while learning.
\item
  I ask myself if I have considered all options after I solve a
  problem.
\item
  I try to translate new information into my own words.
\item
  I change strategies when I fail to understand.
\item
  I use the organizational structure of the text to help me learn.
\item
  I read instructions carefully before I begin a task.
\item
  I ask myself if what I'm reading is related to what I already know.
\item
  I reevaluate my assumptions when I get confused.
\item
  I organize my time to best accomplish my goals.
\item
  I learn more when I am interested in the topic.
\item
  I try to break studying down into smaller steps.
\item
  I focus on overall meaning rather than specifics.
\item
  I ask myself questions about how well I am doing while I am learning
  something new.
\item
  I ask myself if I learned as much as I could have once I finish a
  task.
\item
  I stop and go back over new information that is not clear.
\item
  I stop and reread when I get confused.
\end{enumerate}

\textbf{Scoring Directions:} Score each response from 1 (never) to 5
(always), and calculate the following subscale scores by summing the
items on that subscale.

\emph{Knowledge about Cognition:}

\begin{enumerate}
\def\labelenumi{\arabic{enumi}.}
\item
  \emph{Declarative Knowledge:} 5, 10, 12, 16, 17, 20, 32, 46 (score out of
  \(8\times5 = 40\))
\item
  \emph{Procedural Knowledge:} 3, 14, 27, 33 (score out of \(4\times5 = 20\))
\item
  \emph{Conditional Knowledge:} 15, 18, 26, 29, 35 (score out of
  \(5\times5 = 25\))
\end{enumerate}

\emph{Regulation of Cognition:}

\begin{enumerate}
\def\labelenumi{\arabic{enumi}.}
\item
  \emph{Planning:} 4, 6, 8, 22, 23, 42, 45 (score out of \(7\times5 = 35\))
\item
  \emph{Information Management Strategies:} 9, 13, 30, 31, 37, 39, 41, 43,
  47, 48 (score out of \(10\times5 = 50\))
\item
  \emph{Comprehension Monitoring:} 1, 2, 11, 21, 28, 34, 49 (score out of
  \(7\times5 = 35\))
\item
  \emph{Debugging Strategies:} 25, 40, 44, 51, 52 (score out of
  \(5\times5 = 25\))
\item
  \emph{Evaluation:} 7, 19, 24, 36, 38, 50 (score out of \(6\times5 = 30\))
\end{enumerate}

\hypertarget{appendix:pre_post_tasks}{%
\chapter{Questionnaires for Initial (SES1) and Final (SES3) Sessions}\label{appendix:pre_post_tasks}}

Pre-Test session (SES1) is conducted at the beginning of the semester,
and the Post-Test session (SES3) is conducted at the end of the
semester.

\hypertarget{app:pre_task_qsn}{%
\section{Pre-Task Questionnaire (for SES1 and SES3)}\label{app:pre_task_qsn}}

The following items are adapted from (\protect\hyperlink{ref-collins2016assessing}{Collins-Thompson et al., 2016}).

\begin{enumerate}
\def\labelenumi{\arabic{enumi}.}
\item
  How much do you know about this topic?\\
  \emph{(1) nothing \textbar{} I know a lot (5)}
\item
  How interested are you to learn more about this topic?\\
  \emph{(1) not at all \textbar{} very much (5)}
\item
  How difficult do you think it will be to search for information
  about this topic?\\
  \emph{(1) very easy \textbar{} very difficult (5)}
\end{enumerate}

The following items are adapted from (\protect\hyperlink{ref-crescenzi2020adaptation}{Crescenzi, 2020}).

\emph{Indicate your agreement with the following statements.}

\emph{(1) Strongly Disagree \textbar{} Neutral \textbar{} Strongly Agree (5)}

\begin{enumerate}
\def\labelenumi{\arabic{enumi}.}
\setcounter{enumi}{3}
\item
  I am interested to learn more about the topic of this task.
\item
  I know a lot about this topic.
\item
  I can write a good summary now without needing to look for
  information.
\item
  It will be difficult to determine when I have enough information to
  write my summary.
\item
  I think this will be a difficult task.
\item
  I am confident I know (or can find) adequate information to write a
  good summary.
\end{enumerate}

\hypertarget{app:post_task_qsn}{%
\section{Post-Task Questionnaire (for SES1 and SES3)}\label{app:post_task_qsn}}

The following items are adapted from (\protect\hyperlink{ref-collins2016assessing}{Collins-Thompson et al., 2016}).

\emph{Indicate your agreement with the following statements.}

\emph{(1) Not at all \textbar{} Unlikely \textbar{} Somewhat \textbar{} Likely \textbar{} Very Likely (5)}

\emph{Search for information exploration:}

\begin{enumerate}
\def\labelenumi{\arabic{enumi}.}
\item
  I was cognitively engaged in search task.
\item
  I made an effort at performing the search task.
\item
  The time for search was spent productively on meaningful tasks.
\item
  I was able to explore relationships among multiple concepts.
\item
  I was able to expand the scope of my knowledge about the topic.
\item
  I feel that I was able to put together pieces of information into
  one big concept.
\end{enumerate}

\emph{Learner interest and motivation:}

\begin{enumerate}
\def\labelenumi{\arabic{enumi}.}
\setcounter{enumi}{6}
\item
  I feel that I have full understanding of the topic of this task
\item
  I became more interested in this topic.
\item
  I would like to find more information about this topic
\item
  I would like to share what I learned with my people I know.
\item
  I feel that I learned useful information as a result of this search.
\item
  I was able to develop new ideas or perspectives.
\end{enumerate}

\emph{Perceived learning and search success:}

\emph{On a scale of 0 - 100}

\begin{enumerate}
\def\labelenumi{\arabic{enumi}.}
\setcounter{enumi}{12}
\item
  How would you grade your learning outcome?
\item
  How would you grade your search outcome?
\end{enumerate}

~\\
The following items are adapted from (\protect\hyperlink{ref-crescenzi2020adaptation}{Crescenzi, 2020}).

\emph{Indicate your agreement with the following statements.}

\emph{(1) Strongly Disagree \textbar{} Neutral \textbar{} Strongly Agree (5)}

\begin{enumerate}
\def\labelenumi{\arabic{enumi}.}
\setcounter{enumi}{14}
\item
  Overall, it was difficult to search for information to make the
  summary.
\item
  It was difficult to determine search terms to use to find relevant
  information.
\item
  It was difficult to decide whether to continue inspecting the search
  results or to search again.
\item
  It was difficult to choose which search results to view.
\item
  It was difficult to determine when to stop looking for information.
\item
  I would have preferred to think longer about my summary.
\item
  If I had more time, I would have considered more information.
\item
  I felt anxious while completing this task.
\item
  I did not have enough time.
\item
  It was difficult to decide which sources to select.
\item
  I felt hurried or rushed during this task.
\item
  I had adequate information to make a good summary.
\item
  I felt I had enough information.
\item
  My understanding of the topic was no longer changing.
\item
  I collected enough information to make a summary.
\item
  I was no longer learning about the topic.
\item
  I felt I had adequate information to make a summary.
\item
  I was focused on getting information about one thing.
\item
  I felt continuing the search was a waste of time, as the same
  information was showing up.
\item
  I had a list of certain things I was interested in.
\item
  I stopped searching because I was not finding new information.
\item
  I stopped searching when I had an option that satisfied the things
  that were important to me.
\item
  I only considered looking for the piece of information most
  important to me.
\item
  I kept finding the same information in every search.
\item
  My view of the topic was no longer changing.
\item
  I was most concerned about finding information on one specific
  aspect.
\end{enumerate}

\hypertarget{app:cta_v_silent}{%
\section{Preference for CTA vs Silent Condition}\label{app:cta_v_silent}}

\begin{enumerate}
\def\labelenumi{\arabic{enumi}.}
\item
  You were asked to talk-aloud for one task, and work in silence for
  the other. Which one was better?

  \begin{itemize}
  \item
    Talk aloud much was a lot better
  \item
    Talk aloud was slightly better
  \item
    I did not feel any difference
  \item
    Working in silence was slightly better
  \item
    Working in silence was a lot better
  \end{itemize}
\item
  Why?
\end{enumerate}

\hypertarget{app:post_task_interview}{%
\section{Semi-structured Interview Questions (at the end of SES3)}\label{app:post_task_interview}}

\begin{enumerate}
\def\labelenumi{\arabic{enumi}.}
\item
  What role did searching the web for information play in the course
  you just completed?
\item
  How about other courses?
\item
  How do you take notes, and manage your notes? PROBE: Do you have
  some strategies to organise information?
\item
  Over the course of the semester, do you think your information
  search behaviour and strategies changed? Or remained the same?
\item
  I understand that Firefox may not be your usual browser. How was
  your experience of using Firefox and YASBIL to complete the
  assignments?
\item
  Did you feel restricted while using Firefox?
\item
  Did you ever need to switch back to your old browser while
  completing the assignments?
\item
  How did you feel about using YASBIL to record your browsing
  activity?
\item
  Did you feel ever ``under surveillance'' while using YASBIL?
\item
  If you were asked to redesign the way students interact with Google
  or any other Search Engine, what would you do? (PROBE: think about
  the tasks you had to do for your course, or in your daily life, or
  perhaps this study; redesign to better support student education)
\item
  Imagine a scenario where search engines can measure how much
  information students already know about a certain topic. Do you
  think it will be useful or desirable?
\item
  (if yes to previous question): What are some ways by which a search
  engine could possibly assess a student's existing knowledge?
\end{enumerate}

\hypertarget{app:midterm_survey}{%
\chapter{SUR2: Midterm Survey}\label{app:midterm_survey}}

\textbf{Overall Instructions:}

\emph{You may have seen these questions before. We ask you to carefully
consider the questions again, reflect back on your activities and
experiences since the beginning of the semester, and answer the
questions accordingly. We all grow and evolve with time. These questions
will help us to understand how you have evolved over the past few
weeks.}

~\\
\textbf{Reworded Version of Intrinsic Motivation Inventory:}\\
(slightly different from SUR1 to indicate midpoint of the semester)

\emph{For each of the following statements, please indicate how true it is
for you, using the following scale:}

\emph{(1) not at all true \textbar{} somewhat true \textbar{} very true (5)}

\textbf{Interest/Enjoyment}\\
I am enjoying doing the final project activities very much\\
The final project activities are fun to do.\\
I think this is a boring course. (R)\\
The course and the final project activities are not holding my attention
at all. (R)\\
I would describe the final project activities as very interesting.\\
I think the final project activities were quite enjoyable.\\
While I was doing the final project activities, I was thinking about how
much I was enjoying them.

~\\
\textbf{Perceived Competence}\\
I think I am pretty good at the final project activities.\\
I think I am doing pretty well in the final project activities, compared
to other students.\\
After working at the final project activities for awhile, I am feeling
pretty competent.\\
I am satisfied with my performance in this course.\\
I am pretty skilled at the final project activities.\\
This is a course that I am not able to do very well. (R)

~\\
\textbf{Effort/Importance}\\
I am putting a lot of effort into this course.\\
I am not trying very hard to do well at the final project activities.
(R)\\
I am trying very hard on the final project activities.\\
It is important to me to do well in this course.\\
I am not putting much energy into the final project activities. (R)

~\\
\textbf{Pressure/Tension}\\
I was not feeling nervous at all while doing the final project
activities. (R)\\
I was feeling very tensed while doing the final project activities.\\
I was very relaxed while doing the final project activities. (R)\\
I was anxious while working on the final project parts.\\
I felt pressured while doing the final project activities.

~\\
\textbf{Perceived Choice}\\
I believe I have some choice about doing the final project activities.\\
I feel like it is not my own choice to do the final project parts. (R)\\
I didn't really have a choice about doing the final project tasks. (R)\\
I feel like I have to do the final project tasks. (R)\\
I did the final project activities because I had no choice. (R)\\
I did the final project activities because I wanted to.\\
I did the final project activities because I had to. (R)

~\\
\textbf{Value/Usefulness} \footnote{\url{https://developer.chrome.com/docs/extensions/reference/history/\#transition-types}}\strut \\
I believe the course and the final project activities could be of some
value to me.\\
I think that doing the final project activities is useful for\\
I think this is important to do because it can\\
I would be willing to do research on the final project topic again
because it has some value to me.\\
I think doing the final project activities will help me to\\
I believe doing the final project activities will be beneficial to me.\\
I think this is an important course.

\hypertarget{app:final_survey}{%
\chapter{SUR3: Exit Survey}\label{app:final_survey}}

\textbf{Overall Instructions:}

\emph{You may have seen these questions before. We ask you to carefully
consider the questions again, reflect back on your activities and
experiences since the beginning of the semester, and answer the
questions accordingly. We all grow and evolve with time. These questions
will help us to understand how you have evolved over the past few
months.}

~\\
\textbf{Reworded Version of Intrinsic Motivation Inventory:}\\
(Slightly different from SUR2 to indicate end of the semester)

\emph{For each of the following statements, please indicate how true it is
for you, using the following scale:}

\emph{(1) not at all true \textbar{} somewhat true \textbar{} very true (5)}

\textbf{Interest/Enjoyment}\\
I enjoyed taking the course and doing the final project activities very
much.\\
The course and final project activities were fun to do.\\
I think this was a boring course. (R)\\
The course and the final project activities were not holding my
attention at all. (R)\\
I would describe the final project activities as very interesting.\\
I think the final project activities were quite enjoyable.\\
While I was doing the final project activities, I was thinking about how
much I enjoyed them.

~\\
\textbf{Perceived Competence}\\
I think I am pretty good at this course and the final project
activities.\\
I think I did pretty well in the course, and in the final project
activities, compared to other students.\\
After working at the final project activities for awhile, I felt pretty
competent.\\
I am satisfied with my performance in this course.\\
I am pretty skilled at the final project activities.\\
This is a course that I was not able to do very well. (R)

~\\
\textbf{Effort/Importance}\\
I put a lot of effort into this course.\\
I did not try very hard to do well in this course, and in the final
project activities. (R)\\
I tried very hard in the course, and in the final project activities.\\
It was important to me to do well in this course.\\
I did not put much energy into the course and the final project
activities. (R)

~\\
\textbf{Pressure/Tension}\\
I did not feel nervous at all while doing the final project activities.
(R)\\
I felt very tensed while doing the final project activities.\\
I felt very relaxed while doing the final project activities. (R)\\
I was anxious while working on the final project parts.\\
I felt pressured while doing the final project activities.

~\\
\textbf{Perceived Choice}\\
I believe I have some choice about doing the final project.\\
I felt like it is not my own choice to do the final project parts. (R)\\
I didn't really have a choice about doing the final project tasks. (R)\\
I felt like I had to do the final project tasks. (R)\\
I did the final project activities because I had no choice. (R)\\
I did the final project activities because I wanted to.\\
I did the final project activities because I had to. (R)

~\\
\textbf{Value/Usefulness} \footnote{\url{https://developer.chrome.com/docs/extensions/reference/history/\#transition-types}}\strut \\
I believe the course and the final project activities could be of some
value to me.\\
I think that doing the final project activities is useful for\\
I think this is important to do because it can\\
I would be willing to do research on the final project topics/subtopics
again because it has some value to me.\\
I think doing the final project activities will help me to\\
I believe doing the final project activities will be beneficial to me.\\
I think this was an important course and final project for me.

\hypertarget{app:variables}{%
\chapter{Variables and Measures}\label{app:variables}}

This Section contains a non-exhaustive list of possible variables and
operationalizations of information search behaviour collated from recent
literature.

\hypertarget{sec:app_vars_qry}{%
\section{Query Reformulation Variables}\label{sec:app_vars_qry}}

\begin{itemize}
\item
  \textbf{\#Queries:} the number of issued queries in a session
  (\protect\hyperlink{ref-133}{Mao et al., 2018}; \protect\hyperlink{ref-vakkari2016searching}{Vakkari, 2016})
\item
  \textbf{Query Reformulation Type (QRT)}, and corresponding counts:

  \begin{itemize}
  \item
    \textbf{automated:} generalization, specialization, word
    substitution, repeat, new (\protect\hyperlink{ref-liu2010analysis}{C. Liu et al., 2010})
  \item
    \textbf{partially manual:} generalization, specialization, parallel
    move, mission change, error correction (\protect\hyperlink{ref-boldi2009dango}{Boldi et al., 2009})
  \end{itemize}
\item
  \textbf{\#Terms per q.:} the number of terms in a query (\protect\hyperlink{ref-133}{Mao et al., 2018})
\item
  \textbf{\#Unique terms:} the number of unique terms in a session (\protect\hyperlink{ref-133}{Mao et al., 2018})
\item
  \textbf{\#Unique terms per q.:} (\#Unique terms / \#queries) , the number of
  unique terms per query (query vocabulary richness, QVR) (\protect\hyperlink{ref-133}{Mao et al., 2018})
\item
  \textbf{\#Synonyms:} number of synonyms in terms (\protect\hyperlink{ref-vakkari2016searching}{Vakkari, 2016})
\item
  \textbf{\%Terms from desc.:} the ratio of terms from the task description
  / assignment description (\protect\hyperlink{ref-133}{Mao et al., 2018})
\item
  \textbf{\%Terms from SERP:} the proportion of novel query terms found in
  the text of SERPs (\protect\hyperlink{ref-133}{Mao et al., 2018})
\item
  \textbf{\%Terms from content page:} the proportion of novel query terms
  found in the text of content pages (\protect\hyperlink{ref-133}{Mao et al., 2018})
\item
  \textbf{\%Others:} the proportion of novel query terms from other sources
  (i.e., not webpage text) (\protect\hyperlink{ref-133}{Mao et al., 2018})
\item
  \textbf{Query type time:} time taken to issue the query (may not be a
  signal; will be dependent on typing speed)
\item
  (\protect\hyperlink{ref-downey2007models}{Downey et al., 2007} Table 3) has more linguistic features of query
  terms: web frequency, geo mean, max bigram, etc. of query terms
\end{itemize}

\hypertarget{sec:app_vars_serp_content_pg}{%
\section{SERP / Content Page Examination variables}\label{sec:app_vars_serp_content_pg}}

\begin{itemize}
\item
  \textbf{Transition Type:} how the browser navigated to a particular URL
  on a particular visit (Appendix
  \protect\hyperlink{sec:app_vars_transition_typ}{1.6})
\item
  \textbf{Dwell time:} time spent on the page (seconds)
\item
  \textbf{Avg. page display time:} average length of time for which a
  webpage is viewed during a session (\protect\hyperlink{ref-white2009characterizing}{White et al., 2009})
\item
  \textbf{\#Clicks:} number of clicks on page
\item
  \textbf{Scroll depth:} how far down the page was scrolled (pixels,
  viewport proportion, \% of total webpage length)
\item
  \textbf{Click depth:} how far down in the page was a click made
\item
  \textbf{Scrolled Distance:} total distance scrolled
\item
  \textbf{\%Scroll Time:} \% of time spent in scrolling
\item
  \textbf{Scroll speed:} indicative of reading vs scanning
\item
  \textbf{Scroll pattern:} entropy / scroll chaosness
\item
  \textbf{\#Scroll direction change:} the number of times scrolling
  direction was changed --- indicative of hesitancy
\item
  \textbf{Non-scroll time:} \% of time NOT spent in scrolling, which is
  indicative of reading
\item
  \textbf{\#Visit-in-session:} whether first visit or revisit, in this
  session
\item
  \textbf{\#Visit-overall:} whether first visit or revisit, in the whole
  data
\item
  \textbf{Number of unclicked hovers:} Median number of times for which the
  query was issued and the URL is hovered on but not clicked, per the
  earlier definition. ``We selected the number of unclicked hovers as a
  feature because we found that it was correlated with clickthrough in
  our previous analysis.'' (\protect\hyperlink{ref-huang2011no}{J. Huang et al., 2011})
\item
  \textbf{Cursor trail length:} Total distance (in pixels) travelled by the
  cursor on the SERP. (\protect\hyperlink{ref-huang2011no}{J. Huang et al., 2011})
\item
  \textbf{Movement time:} Total time (in seconds) for which the cursor was
  being moved on the SERP. (\protect\hyperlink{ref-huang2011no}{J. Huang et al., 2011})
\item
  \textbf{Cursor speed:} The average cursor speed (in pixels per second) as
  a function of trail length and movement time (\protect\hyperlink{ref-huang2011no}{J. Huang et al., 2011})
\item
  All distances / depths can be measured using raw pixel values,
  viewport proportions and \% of total webpage lengths
\end{itemize}

\hypertarget{sec:app_vars_serp}{%
\section{SERP-only variables}\label{sec:app_vars_serp}}

\begin{itemize}
\item
  \textbf{Rank:} rank of result clicked
\item
  \textbf{Avg. SERP dwell time per query:} Total SERP Dwell Time / \#queries
  entered (\protect\hyperlink{ref-collins2016assessing}{Collins-Thompson et al., 2016})
\item
  \textbf{Focus:} ``degree to which a SERP is covered by a single topic.
  Topics may be derived based on existing class hierarchies such as
  the Open Directory Project; Experts tend to explore more narrow
  topical spaces than non-experts''
  (\protect\hyperlink{ref-eickhoff2014lessons}{Eickhoff et al., 2014}; \protect\hyperlink{ref-rieh2016searching}{Rieh et al., 2016})
\item
  \textbf{Entropy:} ``captures diversity across multiple topics on search
  results pages. Experts typically have higher focus, less diversity,
  and thus, lower entropy across topics'' (\protect\hyperlink{ref-rieh2016searching}{Rieh et al., 2016})
\end{itemize}

\hypertarget{sec:app_vars_content_pg}{%
\section{Content page-only variables}\label{sec:app_vars_content_pg}}

\begin{itemize}
\tightlist
\item
  \textbf{Avg. Content dwell time per query:} Total content page dwell time
  / \#queries (\protect\hyperlink{ref-collins2016assessing}{Collins-Thompson et al., 2016})
\end{itemize}

\hypertarget{sec:app_vars_overall_search}{%
\section{Overall Search Behaviour}\label{sec:app_vars_overall_search}}

\begin{itemize}
\item
  \textbf{Session length (time):} total time spent in the session, from
  logging software turn on to turn-off (or last page-visit, if
  turn-off time is null)
\item
  \textbf{Session length (pages):} no. of pages visited in session,
  including search engine home pages and result pages
  (\protect\hyperlink{ref-white2009characterizing}{White et al., 2009})
\item
  \textbf{Session length (queries):} no. of queries issued in session
  (\protect\hyperlink{ref-white2009characterizing}{White et al., 2009})
\item
  \textbf{\#Search engines:} number of unique search engines used (will
  probably be one?)
\item
  \textbf{\#Tabs:} number of tabs opened during the session
\item
  \textbf{\#Windows:} number of windows opened during the session
\item
  \textbf{Branchiness:} no. of revisits to previous pages in the session
  that were then followed by a forward motion to a previously
  unvisited page in the session(\protect\hyperlink{ref-white2009characterizing}{White et al., 2009})
\item
  \textbf{\#Unqiue domains:} number of unique websites visited during the
  session; diversity of websites / breadth of
  coverage(\protect\hyperlink{ref-white2009characterizing}{White et al., 2009})
\item
  \textbf{Ratio of querying to browsing:} proportion of the session that is
  devoted to querying versus browsing pages retrieved by the search
  engine or linked to from search results. A high number (much greater
  than one) means that the session was query-intensive. In contrast, a
  low number (much less than one) means that the session was
  browse-intensive(\protect\hyperlink{ref-white2009characterizing}{White et al., 2009})
\item
  \textbf{\#Bookmarks-add:} no. of bookmarks added
\item
  \textbf{\#Bookmarks-delete:} no. of bookmarks deleted
\item
  \textbf{Search success:} ``If the final event in a search session was a
  URL click, we scored the session as a success, and if the final
  action was a query, we score the session as a failure.''
  (\protect\hyperlink{ref-white2009characterizing}{White et al., 2009})
\item
  \textbf{Last Page Visited:} Where was the last click in a session?
  (\protect\hyperlink{ref-eickhoff2014lessons}{Eickhoff et al., 2014}) treated sessions ending at ehow.com as
  procedural search sessions (users visiting tutorial articles), and
  sessions ending at Wikipedia as declarative search sessions.
\item
  \textbf{Search visualization:} as described by (\protect\hyperlink{ref-bateman2012search}{Bateman et al., 2012})
\end{itemize}

\hypertarget{sec:app_vars_transition_typ}{%
\section{Webpage Transition Types}\label{sec:app_vars_transition_typ}}

In browser history API parlance, a \emph{transition type} is used to describe
how the browser navigated to a particular URL on a particular visit.
E.g., if a user visits a page by clicking a link on another page, the
transition type is ``link''. The following definitions are taken from
Google Chrome Developer Website\footnote{\url{https://developer.chrome.com/docs/extensions/reference/history/\#transition-types}}.

\begin{itemize}
\item
  \textbf{link:} The user got to this page by clicking a link on another
  page.
\item
  \textbf{typed:} The user got this page by typing the URL in the address
  bar. Also used for other explicit navigation actions. See also
  generated, which is used for cases where the user selected a choice
  that didn't look at all like a URL.
\item
  \textbf{auto\_bookmark:} The user got to this page through a suggestion in
  the UI---for example, through a menu item.
\item
  \textbf{auto\_subframe:} Subframe navigation. This is any content that is
  automatically loaded in a non-top-level frame. For example, if a
  page consists of several frames containing ads, those ad URLs have
  this transition type. The user may not even realize the content in
  these pages is a separate frame, and so may not care about the URL
  (see also manual\_subframe).
\item
  \textbf{manual\_subframe:} For subframe navigations that are explicitly
  requested by the user and generate new navigation entries in the
  back/forward list. An explicitly requested frame is probably more
  important than an automatically loaded frame because the user
  probably cares about the fact that the requested frame was loaded.
\item
  \textbf{generated:} The user got to this page by typing in the address
  bar and selecting an entry that did not look like a URL. For
  example, a match might have the URL of a Google search result page,
  but it might appear to the user as "Search Google for ...". These
  are not quite the same as typed navigations because the user didn't
  type or see the destination URL. See also keyword.
\item
  \textbf{auto\_toplevel:} The page was specified in the command line or is
  the start page.
\item
  \textbf{form\_submit:} The user filled out values in a form and submitted
  it. Note that in some situations---such as when a form uses script
  to submit contents---submitting a form does not result in this
  transition type.
\item
  \textbf{reload:} The user reloaded the page, either by clicking the
  reload button or by pressing Enter in the address bar. Session
  restore and Reopen closed tab use this transition type, too.
\item
  \textbf{keyword:} The URL was generated from a replaceable keyword other
  than the default search provider. See also keyword\_generated.
\item
  \textbf{keyword\_generated:} Corresponds to a visit generated for a
  keyword. See also keyword.
\end{itemize}

\hypertarget{sec:app_vars_concept_maps}{%
\section{Concept Map Analysis Measures}\label{sec:app_vars_concept_maps}}

The following concept map analysis measures are adapted from
(\protect\hyperlink{ref-halttunen2005assessing}{Halttunen \& Jarvelin, 2005}).

\begin{itemize}
\item
  The number of concepts in the beginning of instruction (beginning),
\item
  The number of new concepts presented in second essay (new),
\item
  The number of concepts in the end of instruction (end),
\item
  The difference between the number of concepts in the beginning and
  end of instruction (difference),
\item
  The number of concepts which remained same along the time (stable),
\item
  The number of concepts that were ignored or changed fundamentally
  along the instruction (changed/ignored),
\item
  The number of top-level concepts in the end (top level),
\item
  The number of new top-level concepts presented in the second essay
  (new top level),
\item
  The number of links between concept hierarchies in the end (links),
\item
  Maximum depth of hierarchy levels in the end,
\item
  The number of concepts per different levels of hierarchy in the end,
\item
  The number of concepts per top-level concepts i.e.~hierarchies in
  the end, and
\item
  The level where new concepts were introduced.
\end{itemize}

\startappendices

\hypertarget{the-first-appendix}{%
\chapter{The First Appendix}\label{the-first-appendix}}

This first appendix includes an R chunk that was hidden in the document (using \texttt{echo\ =\ FALSE}) to help with readibility:

\textbf{In 02-rmd-basics-code.Rmd}

\textbf{And here's another one from the same chapter, i.e.~Chapter \ref{code}:}

\hypertarget{the-second-appendix-for-fun}{%
\chapter{The Second Appendix, for Fun}\label{the-second-appendix-for-fun}}

\hypertarget{references}{%
\chapter*{References}\label{references}}
\addcontentsline{toc}{chapter}{References}

\markboth{References}{}

\hypertarget{refs}{}
\begin{CSLReferences}{1}{0}
\leavevmode\vadjust pre{\hypertarget{ref-135}{}}%
Abualsaud, M., \& Smucker, M. D. (2019). Patterns of search result examination: {Query} to first action. \emph{Proceedings of the 28th {ACM} International Conference on Information and Knowledge Management}, 1833--1842. \url{https://doi.org/10.1145/3357384.3358041}

\leavevmode\vadjust pre{\hypertarget{ref-agosti2014evaluation}{}}%
Agosti, M., Fuhr, N., Toms, E., \& Vakkari, P. (2014). Evaluation methodologies in information retrieval dagstuhl seminar 13441. \emph{ACM SIGIR Forum}, \emph{48}, 36--41.

\leavevmode\vadjust pre{\hypertarget{ref-allan2012frontiers}{}}%
Allan, J., Croft, B., Moffat, A., \& Sanderson, M. (2012). Frontiers, challenges, and opportunities for information retrieval: Report from SWIRL 2012 the second strategic workshop on information retrieval in lorne. \emph{ACM SIGIR Forum}, \emph{46}, 2--32.

\leavevmode\vadjust pre{\hypertarget{ref-ambrose2010howa}{}}%
Ambrose, S. A., Bridges, M. W., DiPietro, M., Lovett, M. C., \& Norman, M. K. (2010). \emph{How {Learning Works}: Seven {Research}-{Based Principles} for {Smart Teaching}}. {John Wiley \& Sons}.

\leavevmode\vadjust pre{\hypertarget{ref-amina2017active}{}}%
Amina, T. (2017). Active knowledge making: Epistemic dimensions of e-learning. In \emph{E-learning ecologies} (pp. 65--87). Routledge.

\leavevmode\vadjust pre{\hypertarget{ref-arguello2019effects}{}}%
Arguello, J., \& Choi, B. (2019). The effects of working memory, perceptual speed, and inhibition in aggregated search. \emph{ACM Transactions on Information Systems}, \emph{37}(3). \url{https://doi.org/10.1145/3322128}

\leavevmode\vadjust pre{\hypertarget{ref-102}{}}%
Aula, A., Majaranta, P., \& Räihä, K.-J. (2005). Eye-tracking reveals the personal styles for search result evaluation. In M. F. Costabile \& F. Paternò (Eds.), \emph{Human-computer interaction - {INTERACT} 2005} (pp. 1058--1061). {Springer Berlin Heidelberg}.

\leavevmode\vadjust pre{\hypertarget{ref-ausubel2012acquisition}{}}%
Ausubel, D. P. (2012). \emph{The acquisition and retention of knowledge: A cognitive view}. Springer Science \& Business Media.

\leavevmode\vadjust pre{\hypertarget{ref-ausubel1968educational}{}}%
Ausubel, D. P., Novak, J. D., Hanesian, H., et al. (1968). \emph{Educational psychology: A cognitive view} (Vol. 6). Holt, Rinehart; Winston New York.

\leavevmode\vadjust pre{\hypertarget{ref-bailey2011amount}{}}%
Bailey, E., \& Kelly, D. (2011). Is amount of effort a better predictor of search success than use of specific search tactics? \emph{Proceedings of the American Society for Information Science and Technology}, \emph{48}(1), 1--10.

\leavevmode\vadjust pre{\hypertarget{ref-114}{}}%
Balatsoukas, P., \& Ruthven, I. (2010). The use of relevance criteria during predictive judgment: {An} eye tracking approach. \emph{Proceedings of the American Society for Information Science and Technology}, \emph{47}(1), 1--10. \url{https://doi.org/10.1002/meet.14504701145}

\leavevmode\vadjust pre{\hypertarget{ref-119}{}}%
Balatsoukas, P., \& Ruthven, I. (2012). An eye-tracking approach to the analysis of relevance judgments on the {Web}: {The} case of {Google} search engine. \emph{Journal of the American Society for Information Science and Technology}, \emph{63}(9), 1728--1746. \url{https://doi.org/10.1002/asi.22707}

\leavevmode\vadjust pre{\hypertarget{ref-bateman2012search}{}}%
Bateman, S., Teevan, J., \& White, R. (2012). The search dashboard: How reflection and comparison impact search behavior. \emph{Proceedings of the 2012 {ACM} Annual Conference on {Human Factors} in {Computing Systems} - {CHI} '12}, 1785. \url{https://doi.org/10.1145/2207676.2208311}

\leavevmode\vadjust pre{\hypertarget{ref-belkin1982ask}{}}%
Belkin, N. J., Oddy, R. N., \& Brooks, H. M. (1982). ASK for information retrieval: Part i. Background and theory. \emph{Journal of Documentation}.

\leavevmode\vadjust pre{\hypertarget{ref-beymer2007eye}{}}%
Beymer, D., Orton, P. Z., \& Russell, D. M. (2007). An eye tracking study of how pictures influence online reading. \emph{IFIP Conference on Human-Computer Interaction}, 456--460.

\leavevmode\vadjust pre{\hypertarget{ref-bhattacharya2021longitudinal}{}}%
Bhattacharya, N. (2021). A longitudinal study to understand learning during search. \emph{Proceedings of the 2021 Conference on Human Information Interaction and Retrieval}, 363--366.

\leavevmode\vadjust pre{\hypertarget{ref-bhattacharya2018relating}{}}%
Bhattacharya, N., \& Gwizdka, J. (2018). Relating eye-tracking measures with changes in knowledge on search tasks. \emph{Symposium on Eye Tracking Research \& Applications (ETRA)}.

\leavevmode\vadjust pre{\hypertarget{ref-bhattacharya2019measuring}{}}%
Bhattacharya, N., \& Gwizdka, J. (2019b). Measuring learning during search: Differences in interactions, eye-gaze, and semantic similarity to expert knowledge. \emph{Proceedings of the 2019 Conference on Human Information Interaction and Retrieval}, 63--71.

\leavevmode\vadjust pre{\hypertarget{ref-CHIIR19}{}}%
Bhattacharya, N., \& Gwizdka, J. (2019a). Measuring learning during search: Differences in interactions, eye-gaze, and semantic similarity to expert knowledge. \emph{CHIIR'19}.

\leavevmode\vadjust pre{\hypertarget{ref-bhattacharya2021yasbil}{}}%
Bhattacharya, N., \& Gwizdka, J. (2021). YASBIL: Yet another search behaviour (and) interaction logger. \emph{Proceedings of the 44th International ACM SIGIR Conference on Research and Development in Information Retrieval}, 2585--2589.

\leavevmode\vadjust pre{\hypertarget{ref-139}{}}%
Bilal, D., \& Gwizdka, J. (2016). Children's {Eye}-fixations on {Google Search Results}. \emph{Proceedings of the 79th {ASIS}\&{T Annual Meeting}}, \emph{79}, 89:1--89:6. \url{https://doi.org/10.1002/pra2.2016.14505301089}

\leavevmode\vadjust pre{\hypertarget{ref-blanken2017metacognition}{}}%
Blanken-Webb, J. (2017). Metacognition: Cognitive dimensions of e-learning. In \emph{E-learning ecologies} (pp. 163--182). Routledge.

\leavevmode\vadjust pre{\hypertarget{ref-boldi2009dango}{}}%
Boldi, P., Bonchi, F., Castillo, C., \& Vigna, S. (2009). From" dango" to" japanese cakes": Query reformulation models and patterns. \emph{2009 IEEE/WIC/ACM International Joint Conference on Web Intelligence and Intelligent Agent Technology}, \emph{1}, 183--190.

\leavevmode\vadjust pre{\hypertarget{ref-borlund2013interactive}{}}%
Borlund, P. (2013). Interactive {Information Retrieval}: {An Introduction}. \emph{Journal of Information Science Theory and Practice}, \emph{1}(3), 12--32. \url{https://doi.org/10.1633/JISTAP.2013.1.3.2}

\leavevmode\vadjust pre{\hypertarget{ref-breakstone2018we}{}}%
Breakstone, J., McGrew, S., Smith, M., Ortega, T., \& Wineburg, S. (2018). Why we need a new approach to teaching digital literacy. \emph{Phi Delta Kappan}, \emph{99}(6), 27--32.

\leavevmode\vadjust pre{\hypertarget{ref-breakstone2021students}{}}%
Breakstone, J., Smith, M., Wineburg, S., Rapaport, A., Carle, J., Garland, M., \& Saavedra, A. (2021). Students' {Civic Online Reasoning}: A {National Portrait}. \emph{Educational Researcher}. \url{https://doi.org/10.3102/0013189X211017495}

\leavevmode\vadjust pre{\hypertarget{ref-brennan2016factor}{}}%
Brennan, K., Kelly, D., \& Zhang, Y. (2016). Factor analysis of a search self-efficacy scale. \emph{Proceedings of the 2016 ACM on Conference on Human Information Interaction and Retrieval}, 241--244.

\leavevmode\vadjust pre{\hypertarget{ref-broder2002taxonomy}{}}%
Broder, A. (2002). A taxonomy of web search. \emph{SIGIR Forum}, \emph{36}(2), 3--10. \url{https://doi.org/10.1145/792550.792552}

\leavevmode\vadjust pre{\hypertarget{ref-brookes1980foundations}{}}%
Brookes, B. C. (1980). The foundations of information science. Part i. Philosophical aspects. \emph{Journal of Information Science}, \emph{2}(3-4), 125--133.

\leavevmode\vadjust pre{\hypertarget{ref-brown1998self}{}}%
Brown, J. (1998). \emph{Self-regulation and the addictive behaviours}. New York: Plenum Press.

\leavevmode\vadjust pre{\hypertarget{ref-brown1999self}{}}%
Brown, J. M., Miller, W. R., \& Lawendowski, L. A. (1999). The self-regulation questionnaire. In V. L. \& J. T. L. (Eds.), \emph{Innovations in clinical practice: A sourcebook} (Vol. 17, pp. 281--292). Professional Resource Press/Professional Resource Exchange.

\leavevmode\vadjust pre{\hypertarget{ref-110}{}}%
Buscher, G., Cutrell, E., \& Morris, M. R. (2009). What {Do You See When You}'re {Surfing}? {Using Eye Tracking} to {Predict Salient Regions} of {Web Pages}. \emph{Proceedings of the SIGCHI Conference on Human Factors in Computing Systems}, 10.

\leavevmode\vadjust pre{\hypertarget{ref-115}{}}%
Buscher, G., Dumais, S. T., \& Cutrell, E. (2010). The good, the bad, and the random: {An} eye-tracking study of ad quality in web search. \emph{Proceedings of the 33rd International {ACM SIGIR} Conference on Research and Development in Information Retrieval}, 42--49. \url{https://doi.org/10.1145/1835449.1835459}

\leavevmode\vadjust pre{\hypertarget{ref-chen2020understanding}{}}%
Chen, Y., Zhao, Y., \& Wang, Z. (2020). Understanding online health information consumers' search as a learning process. \emph{Library Hi Tech}.

\leavevmode\vadjust pre{\hypertarget{ref-cherry2020what}{}}%
Cherry, K. (2020). What {Is Motivation}? In \emph{Verywell Mind}. \url{https://www.verywellmind.com/what-is-motivation-2795378}

\leavevmode\vadjust pre{\hypertarget{ref-cole2020more}{}}%
Cole, L., MacFarlane, A., \& Makri, S. (2020). More than words: The impact of memory on how undergraduates with dyslexia interact with information. \emph{Proceedings of the 2020 Conference on Human Information Interaction and Retrieval}, 353--357. \url{https://doi.org/10.1145/3343413.3378005}

\leavevmode\vadjust pre{\hypertarget{ref-cole2013inferring}{}}%
Cole, M. J., Gwizdka, J., Liu, C., Belkin, N. J., \& Zhang, X. (2013). Inferring user knowledge level from eye movement patterns. \emph{Information Processing \& Management}, \emph{49}(5), 1075--1091.

\leavevmode\vadjust pre{\hypertarget{ref-collins2021reimagining}{}}%
Collins, C. (2021). Reimagining {Digital Literacy Education} to {Save Ourselves}. \emph{Learning for Justice}, \emph{Fall 2021}. \url{https://www.learningforjustice.org/magazine/fall-2021/reimagining-digital-literacy-education-to-save-ourselves}

\leavevmode\vadjust pre{\hypertarget{ref-collins2017search}{}}%
Collins-Thompson, K., Hansen, P., \& Hauff, C. (2017). Search as learning (dagstuhl seminar 17092). \emph{Dagstuhl Reports}, \emph{7}.

\leavevmode\vadjust pre{\hypertarget{ref-collins2016assessing}{}}%
Collins-Thompson, K., Rieh, S. Y., Haynes, C. C., \& Syed, R. (2016). Assessing learning outcomes in web search: A comparison of tasks and query strategies. \emph{Proceedings of the 2016 ACM on Conference on Human Information Interaction and Retrieval}, 163--172.

\leavevmode\vadjust pre{\hypertarget{ref-cope2017elearningc}{}}%
Cope, B., \& Kalantzis, M. (2017). \emph{E-{Learning Ecologies}: Principles for {New Learning} and {Assessment}}. {Taylor \& Francis}.

\leavevmode\vadjust pre{\hypertarget{ref-cope2013new}{}}%
Cope, B., \& Kalantzis, M. (2013). Towards a {New Learning}: The {\emph{Scholar}} {Social Knowledge Workspace}, in {Theory} and {Practice}. \emph{E-Learning and Digital Media}, \emph{10}(4), 332--356. \url{https://doi.org/10.2304/elea.2013.10.4.332}

\leavevmode\vadjust pre{\hypertarget{ref-crescenzi2020adaptation}{}}%
Crescenzi, A. M. C. (2020). \emph{Adaptation in {Information Search} and {Decision}-{Making} under {Time Pressure}} {[}PhD thesis, The University of North Carolina at Chapel Hill University Libraries{]}. \url{https://doi.org/10.17615/YT6K-AC37}

\leavevmode\vadjust pre{\hypertarget{ref-104}{}}%
Cutrell, E., \& Guan, Z. (2007). What are you looking for? {An} eye-tracking study of information usage in web search. \emph{Proceedings of the {SIGCHI} Conference on Human Factors in Computing Systems}, 407--416. \url{https://doi.org/10.1145/1240624.1240690}

\leavevmode\vadjust pre{\hypertarget{ref-deci2013intrinsic}{}}%
Deci, E. L., \& Ryan, R. M. (2013). \emph{Intrinsic motivation and self-determination in human behavior}. Springer Science \& Business Media.

\leavevmode\vadjust pre{\hypertarget{ref-dervin2010sensemaking}{}}%
Dervin, B., \& Naumer, C. M. (2010). Sense-making. In M. J. Bates \& M. M. N. (Eds.), \emph{Encyclopedia of library and information sciences (3rd ed.)} (pp. 4696-\/-4707). Taylor; Francis.

\leavevmode\vadjust pre{\hypertarget{ref-desimone1995neural}{}}%
Desimone, R., \& Duncan, J. (1995). Neural mechanisms of selective visual attention. \emph{Annual Review of Neuroscience}, \emph{18}(1), 193--222.

\leavevmode\vadjust pre{\hypertarget{ref-diamond2013executive}{}}%
Diamond, A. (2013). Executive functions. \emph{Annual Review of Psychology}, \emph{64}, 135--168.

\leavevmode\vadjust pre{\hypertarget{ref-dicerbo2014impacts}{}}%
DiCerbo, K. E., \& Behrens, J. T. (2014). Impacts of the digital ocean on education. \emph{London: Pearson}, \emph{1}.

\leavevmode\vadjust pre{\hypertarget{ref-30}{}}%
Djamasbi, S., Hall-Phillips, A., \& Yang, R. (Rachel). (2013). Search {Results Pages} and {Competition} for {Attention Theory}: {An Exploratory Eye}-{Tracking Study}. In S. Yamamoto (Ed.), \emph{Human {Interface} and the {Management} of {Information}. {Information} and {Interaction Design}} (pp. 576--583). {Springer Berlin Heidelberg}. \url{http://link.springer.com.ezproxy.lib.utexas.edu/chapter/10.1007/978-3-642-39209-2_64}

\leavevmode\vadjust pre{\hypertarget{ref-downey2007models}{}}%
Downey, D., Dumais, S. T., \& Horvitz, E. (2007). Models of searching and browsing: Languages, studies, and application. \emph{IJCAI}, \emph{7}, 2740--2747.

\leavevmode\vadjust pre{\hypertarget{ref-117}{}}%
Dumais, S. T., Buscher, G., \& Cutrell, E. (2010). Individual differences in gaze patterns for web search. \emph{Proceedings of the Third Symposium on Information Interaction in Context}, 185--194. \url{https://doi.org/10.1145/1840784.1840812}

\leavevmode\vadjust pre{\hypertarget{ref-egusa2010usingb}{}}%
Egusa, Y., Saito, H., Takaku, M., Terai, H., Miwa, M., \& Kando, N. (2010). Using a {Concept Map} to {Evaluate Exploratory Search}. \emph{Proceedings of the {Third Symposium} on {Information Interaction} in {Context}}, 175--184. \url{https://doi.org/10.1145/1840784.1840810}

\leavevmode\vadjust pre{\hypertarget{ref-egusa2014howd}{}}%
Egusa, Y., Takaku, M., \& Saito, H. (2014a). How {Concept Maps Change} if a {User Does Search} or {Not}? \emph{Proceedings of the 5th {Information Interaction} in {Context Symposium}}, 68--75. \url{https://doi.org/10.1145/2637002.2637012}

\leavevmode\vadjust pre{\hypertarget{ref-egusa2014howe}{}}%
Egusa, Y., Takaku, M., \& Saito, H. (2014b). How to evaluate searching as learning. \emph{Searching as {Learning Workshop} ({IIiX} 2014 Workshop)}. \url{http://www.diigubc.ca/IIIXSAL/program.html}

\leavevmode\vadjust pre{\hypertarget{ref-egusa2017evaluating}{}}%
Egusa, Y., Takaku, M., \& Saito, H. (2017). Evaluating {Complex Interactive Searches Using Concept Maps}. \emph{{SCST}@ {CHIIR}}, 15--17.

\leavevmode\vadjust pre{\hypertarget{ref-127}{}}%
Eickhoff, C., Dungs, S., \& Tran, V. (2015). An eye-tracking study of query reformulation. \emph{Proceedings of the 38th International {ACM SIGIR} Conference on Research and Development in Information Retrieval}, 13--22. \url{https://doi.org/10.1145/2766462.2767703}

\leavevmode\vadjust pre{\hypertarget{ref-eickhoff2017introduction}{}}%
Eickhoff, C., Gwizdka, J., Hauff, C., \& He, J. (2017). Introduction to the special issue on search as learning. \emph{Information Retrieval Journal}, \emph{20}(5), 399--402.

\leavevmode\vadjust pre{\hypertarget{ref-eickhoff2014lessons}{}}%
Eickhoff, C., Teevan, J., White, R., \& Dumais, S. (2014). Lessons from the journey: A query log analysis of within-session learning. \emph{Proceedings of the 7th ACM International Conference on Web Search and Data Mining}, 223--232.

\leavevmode\vadjust pre{\hypertarget{ref-francis2004coglab}{}}%
Francis, G., MacKewn, A., \& Goldthwaite, D. (2004). \emph{{CogLab} on a {CD}}. {Wadsworth Publishing Company}.

\leavevmode\vadjust pre{\hypertarget{ref-freund2013searching}{}}%
Freund, L., Gwizdka, J., Hansen, P., Kando, N., \& Rieh, S. Y. (2013). From searching to learning. \emph{Evaluation Methodologies in Information Retrieval. Dagstuhl Reports}, \emph{13441}, 102--105.

\leavevmode\vadjust pre{\hypertarget{ref-freund2014searching}{}}%
Freund, L., He, J., Gwizdka, J., Kando, N., Hansen, P., \& Rieh, S. Y. (2014). Searching as learning (SAL) workshop 2014. \emph{Proceedings of the 5th Information Interaction in Context Symposium}, 7--7.

\leavevmode\vadjust pre{\hypertarget{ref-gadiraju2018AnalyzingKnowledgeGain}{}}%
Gadiraju, U., Yu, R., Dietze, S., \& Holtz, P. (2018). Analyzing knowledge gain of users in informational search sessions on the web. \emph{Conference on Human Information Interaction \& Retrieval (CHIIR)}.

\leavevmode\vadjust pre{\hypertarget{ref-ghosh2018SearchingLearningExploring}{}}%
Ghosh, S., Rath, M., \& Shah, C. (2018). Searching as learning: Exploring search behavior and learning outcomes in learning-related tasks. \emph{Conference on Human Information Interaction \& Retrieval (CHIIR)}.

\leavevmode\vadjust pre{\hypertarget{ref-goldberg2002eye}{}}%
Goldberg, J. H., Stimson, M. J., Lewenstein, M., Scott, N., \& Wichansky, A. M. (2002). Eye tracking in web search tasks: Design implications. \emph{Proceedings of the 2002 Symposium on Eye Tracking Research \& Applications}, 51--58.

\leavevmode\vadjust pre{\hypertarget{ref-81}{}}%
González-Ibáñez, R., Esparza-Villamán, A., Vargas-Godoy, J. C., \& Shah, C. (2019). A comparison of unimodal and multimodal models for implicit detection of relevance in interactive {IR}. \emph{Journal of the Association for Information Science and Technology}, \emph{0}(0). \url{https://doi.org/10.1002/asi.24202}

\leavevmode\vadjust pre{\hypertarget{ref-124}{}}%
Gossen, T., Höbel, J., \& Nürnberger, A. (2014). A comparative study about children's and adults' perception of targeted web search engines. \emph{Proceedings of the {SIGCHI} Conference on Human Factors in Computing Systems}, 1821--1824. \url{https://doi.org/10.1145/2556288.2557031}

\leavevmode\vadjust pre{\hypertarget{ref-grabowski1996generative}{}}%
Grabowski, B. L. (1996). Generative learning: Past, present, and future. \emph{Handbook of Research for Educational Communications and Technology}, 897--918.

\leavevmode\vadjust pre{\hypertarget{ref-101}{}}%
Granka, L. A., Joachims, T., \& Gay, G. (2004). Eye-tracking analysis of user behavior in {WWW} search. \emph{Proceedings of the 27th Annual International {ACM SIGIR} Conference on Research and Development in Information Retrieval}, 478--479. \url{https://doi.org/10.1145/1008992.1009079}

\leavevmode\vadjust pre{\hypertarget{ref-groner1984looking}{}}%
Groner, R., Walder, F., \& Groner, M. (1984). Looking at faces: Local and global aspects of scanpaths. In \emph{Advances in psychology} (Vol. 22, pp. 523--533). Elsevier.

\leavevmode\vadjust pre{\hypertarget{ref-105}{}}%
Guan, Z., \& Cutrell, E. (2007). An eye tracking study of the effect of target rank on web search. \emph{Proceedings of the {SIGCHI} Conference on Human Factors in Computing Systems}, 417--420. \url{https://doi.org/10.1145/1240624.1240691}

\leavevmode\vadjust pre{\hypertarget{ref-guyan2013improving}{}}%
Guyan, M. (2013). Improving {Learner Motivation} for {eLearning}. In \emph{Learning Snippets}. \url{https://learningsnippets.wordpress.com/2013/10/30/improving-learner-motivation-for-elearning/}

\leavevmode\vadjust pre{\hypertarget{ref-gwizdka2013effects}{}}%
Gwizdka, J. (2013). Effects of working memory capacity on users' search effort. \emph{Proceedings of the {International Conference} on {Multimedia}, {Interaction}, {Design} and {Innovation}}, 11:1--11:8. \url{https://doi.org/10.1145/2500342.2500358}

\leavevmode\vadjust pre{\hypertarget{ref-37}{}}%
Gwizdka, J. (2014). Characterizing {Relevance} with {Eye}-tracking {Measures}. \emph{Proceedings of the 5th {Information Interaction} in {Context Symposium}}, 58--67. \url{https://doi.org/10.1145/2637002.2637011}

\leavevmode\vadjust pre{\hypertarget{ref-gwizdka2017can}{}}%
Gwizdka, J. (2017). I {Can} and {So I Search More}: Effects {Of Memory Span On Search Behavior}. \emph{Proceedings of the 2017 {Conference} on {Conference Human Information Interaction} and {Retrieval}}, 341--344. \url{https://doi.org/10.1145/3020165.3022148}

\leavevmode\vadjust pre{\hypertarget{ref-74}{}}%
Gwizdka, J. (2018). Inferring {Web Page Relevance Using Pupillometry} and {Single Channel EEG}. In F. D. Davis, R. Riedl, J. vom Brocke, P.-M. Léger, \& A. B. Randolph (Eds.), \emph{Information {Systems} and {Neuroscience}} (pp. 175--183). {Springer International Publishing}. \url{https://doi.org/10.1007/978-3-319-67431-5_20}

\leavevmode\vadjust pre{\hypertarget{ref-140}{}}%
Gwizdka, J., \& Bilal, D. (2017). Analysis of {Children}'s {Queries} and {Click Behavior} on {Ranked Results} and {Their Thought Processes} in {Google Search}. \emph{Proceedings of the 2017 {Conference} on {Conference Human Information Interaction} and {Retrieval}}, 377--380. \url{https://doi.org/10.1145/3020165.3022157}

\leavevmode\vadjust pre{\hypertarget{ref-gwizdka2016search}{}}%
Gwizdka, J., Hansen, P., Hauff, C., He, J., \& Kando, N. (2016). Search as learning (SAL) workshop 2016. \emph{Proceedings of the 39th International ACM SIGIR Conference on Research and Development in Information Retrieval}, 1249--1250.

\leavevmode\vadjust pre{\hypertarget{ref-48}{}}%
Gwizdka, J., \& Zhang, Y. (2015a). Towards {Inferring Web Page Relevance} \textendash{} {An Eye}-{Tracking Study}. \emph{Proceedings of {iConference}'2015}, 5. \url{https://www.ideals.illinois.edu/handle/2142/73709}

\leavevmode\vadjust pre{\hypertarget{ref-47}{}}%
Gwizdka, J., \& Zhang, Y. (2015b). Differences in {Eye}-{Tracking Measures Between Visits} and {Revisits} to {Relevant} and {Irrelevant Web Pages}. \emph{Proceedings of the 38th {International ACM SIGIR Conference} on {Research} and {Development} in {Information Retrieval}}, 811--814. \url{https://doi.org/10.1145/2766462.2767795}

\leavevmode\vadjust pre{\hypertarget{ref-halttunen2005assessing}{}}%
Halttunen, K., \& Jarvelin, K. (2005). Assessing learning outcomes in two information retrieval learning environments. \emph{Information Processing \& Management}, \emph{41}(4), 949--972. \url{https://doi.org/10.1016/j.ipm.2004.02.004}

\leavevmode\vadjust pre{\hypertarget{ref-hansen2016editorial}{}}%
Hansen, P., \& Rieh, S. Y. (2016). Editorial: Recent advances on searching as learning: An introduction to the special issue. \emph{Journal of Information Science}, \emph{42}(1), 3--6. \url{https://doi.org/10.1177/0165551515614473}

\leavevmode\vadjust pre{\hypertarget{ref-125}{}}%
Hofmann, K., Mitra, B., Radlinski, F., \& Shokouhi, M. (2014). An eye-tracking study of user interactions with query auto completion. \emph{Proceedings of the 23rd {ACM} International Conference on Conference on Information and Knowledge Management}, 549--558. \url{https://doi.org/10.1145/2661829.2661922}

\leavevmode\vadjust pre{\hypertarget{ref-huang2010parallel}{}}%
Huang, J., \& White, R. (2010). Parallel browsing behavior on the web. \emph{Proceedings of the 21st {ACM} Conference on {Hypertext} and Hypermedia - {HT} '10}, 13. \url{https://doi.org/10.1145/1810617.1810622}

\leavevmode\vadjust pre{\hypertarget{ref-huang2011no}{}}%
Huang, J., White, R., \& Dumais, S. (2011). No clicks, no problem: Using cursor movements to understand and improve search. \emph{Proceedings of the SIGCHI Conference on Human Factors in Computing Systems}, 1225--1234.

\leavevmode\vadjust pre{\hypertarget{ref-huang2013relevance}{}}%
Huang, X., \& Soergel, D. (2013). Relevance: {An} improved framework for explicating the notion. \emph{Journal of the American Society for Information Science and Technology}, \emph{64}(1), 18--35. \url{https://doi.org/10.1002/asi.22811}

\leavevmode\vadjust pre{\hypertarget{ref-126}{}}%
Jiang, J., He, D., \& Allan, J. (2014). Searching, browsing, and clicking in a search session: {Changes} in user behavior by task and over time. \emph{Proceedings of the 37th International {ACM SIGIR} Conference on Research \& Development in Information Retrieval}, 607--616. \url{https://doi.org/10.1145/2600428.2609633}

\leavevmode\vadjust pre{\hypertarget{ref-josephson2002visual}{}}%
Josephson, S., \& Holmes, M. E. (2002). Visual attention to repeated internet images: Testing the scanpath theory on the world wide web. \emph{Proceedings of the 2002 Symposium on Eye Tracking Research \& Applications}, 43--49.

\leavevmode\vadjust pre{\hypertarget{ref-jossberger2010challenge}{}}%
Jossberger, H., Brand-Gruwel, S., Boshuizen, H., \& Van de Wiel, M. (2010). The challenge of self-directed and self-regulated learning in vocational education: A theoretical analysis and synthesis of requirements. \emph{Journal of Vocational Education and Training}, \emph{62}(4), 415--440.

\leavevmode\vadjust pre{\hypertarget{ref-kahne2012digital}{}}%
Kahne, J., Lee, N.-J., \& Feezell, J. T. (2012). Digital media literacy education and online civic and political participation. \emph{International Journal of Communication}, \emph{6}, 24.

\leavevmode\vadjust pre{\hypertarget{ref-kalantzis2012newa}{}}%
Kalantzis, M., \& Cope, B. (2012). \emph{New {Learning}: Elements of a {Science} of {Education}}. {Cambridge University Press}.

\leavevmode\vadjust pre{\hypertarget{ref-kanfer1970self_b}{}}%
Kanfer, F. H. (1970a). \emph{Self-monitoring: Methodological limitations and clinical applications.}

\leavevmode\vadjust pre{\hypertarget{ref-kanfer1970self_a}{}}%
Kanfer, F. H. (1970b). Self-regulation: Research, issues, and speculations. \emph{Behavior Modification in Clinical Psychology}, \emph{74}, 178--220.

\leavevmode\vadjust pre{\hypertarget{ref-kanniainen2021assessing}{}}%
Kanniainen, L., Kiili, C., Tolvanen, A., Aro, M., Anmarkrud, Ø., \& Leppänen, P. H. T. (2021). Assessing reading and online research comprehension: Do difficulties in attention and executive function matter? \emph{Learning and Individual Differences}, \emph{87}, 101985. \url{https://doi.org/10.1016/j.lindif.2021.101985}

\leavevmode\vadjust pre{\hypertarget{ref-karapanos2021advances}{}}%
Karapanos, E., Gerken, J., Kjeldskov, J., \& Skov, M. B. (Eds.). (2021). \emph{Advances in {Longitudinal HCI Research}}. {Springer International Publishing}. \url{https://doi.org/10.1007/978-3-030-67322-2}

\leavevmode\vadjust pre{\hypertarget{ref-kelly2006measuring_a}{}}%
Kelly, D. (2006a). Measuring online information seeking context, {Part} 1: Background and method. \emph{Journal of the American Society for Information Science and Technology}, \emph{57}(13), 1729--1739. \url{https://doi.org/10.1002/asi.20483}

\leavevmode\vadjust pre{\hypertarget{ref-kelly2006measuring_b}{}}%
Kelly, D. (2006b). Measuring online information seeking context, {Part} 2: Findings and discussion. \emph{Journal of the American Society for Information Science and Technology}, \emph{57}(14), 1862--1874. \url{https://doi.org/10.1002/asi.20484}

\leavevmode\vadjust pre{\hypertarget{ref-kelly2009methods}{}}%
Kelly, D. (2009). Methods for evaluating interactive information retrieval systems with users. \emph{Foundations and Trends in Information Retrieval}, \emph{3}(1---2), 1--224.

\leavevmode\vadjust pre{\hypertarget{ref-kelly2009evaluation}{}}%
Kelly, D., Dumais, S., \& Pedersen, J. O. (2009). Evaluation challenges and directions for information-seeking support systems. \emph{IEEE Computer}, \emph{42}(3).

\leavevmode\vadjust pre{\hypertarget{ref-knowles1975self}{}}%
Knowles, M. S. (1975). \emph{Self-directed learning: A guide for learners and teachers.} New York: Association press.

\leavevmode\vadjust pre{\hypertarget{ref-ko2021seeking}{}}%
Ko, A. J. (2021). Seeking information. In \emph{Foundations of {Information}}. \url{https://faculty.washington.edu/ajko/books/foundations-of-information/\#/seeking}

\leavevmode\vadjust pre{\hypertarget{ref-HCIUXres81_online}{}}%
Koeman, L. (2020). \emph{HCI/UX research: What methods do we use? -- lisa koeman -- blog}. \url{https://lisakoeman.nl/blog/hci-ux-research-what-methods-do-we-use/}.

\leavevmode\vadjust pre{\hypertarget{ref-kruikemeier2018learning}{}}%
Kruikemeier, S., Lecheler, S., \& Boyer, M. M. (2018). Learning from news on different media platforms: An eye-tracking experiment. \emph{Political Communication}, \emph{35}(1), 75--96.

\leavevmode\vadjust pre{\hypertarget{ref-kuhlthau2004seeking}{}}%
Kuhlthau, C. C. (2004). \emph{Seeking meaning: A process approach to library and information services} (Vol. 2). Libraries Unlimited Westport, CT.

\leavevmode\vadjust pre{\hypertarget{ref-labaj2012modeling}{}}%
Labaj, M., \& Bielikova, M. (2012). Modeling parallel web browsing behavior for web-based educational systems. \emph{2012 {IEEE} 10th {International Conference} on {Emerging eLearning Technologies} and {Applications} ({ICETA})}, 229--234. \url{https://doi.org/10.1109/ICETA.2012.6418330}

\leavevmode\vadjust pre{\hypertarget{ref-lei2015effect}{}}%
Lei, P.-L., Sun, C.-T., Lin, S. S., \& Huang, T.-K. (2015). Effect of metacognitive strategies and verbal-imagery cognitive style on biology-based video search and learning performance. \emph{Computers \& Education}, \emph{87}, 326--339.

\leavevmode\vadjust pre{\hypertarget{ref-leu2015new}{}}%
Leu, D. J., Forzani, E., Rhoads, C., Maykel, C., Kennedy, C., \& Timbrell, N. (2015). The {New Literacies} of {Online Research} and {Comprehension}: Rethinking the {Reading Achievement Gap}. \emph{Reading Research Quarterly}, \emph{50}(1), 37--59. \url{https://doi.org/10.1002/rrq.85}

\leavevmode\vadjust pre{\hypertarget{ref-li2008faceted}{}}%
Li, Y., \& Belkin, N. J. (2008). A faceted approach to conceptualizing tasks in information seeking. \emph{Information Processing \& Management}, \emph{44}(6), 1822--1837.

\leavevmode\vadjust pre{\hypertarget{ref-132}{}}%
Ling, C., Steichen, B., \& Choulos, A. G. (2018). A comparative user study of interactive multilingual search interfaces. \emph{Proceedings of the 2018 Conference on Human Information Interaction \& Retrieval}, 211--220. \url{https://doi.org/10.1145/3176349.3176383}

\leavevmode\vadjust pre{\hypertarget{ref-liu2010analysis}{}}%
Liu, C., Gwizdka, J., Liu, J., Xu, T., \& Belkin, N. J. (2010). Analysis and evaluation of query reformulations in different task types. \emph{Proceedings of the American Society for Information Science and Technology}, \emph{47}(1), 1--9.

\leavevmode\vadjust pre{\hypertarget{ref-128}{}}%
Liu, Z., Liu, Y., Zhou, K., Zhang, M., \& Ma, S. (2015). Influence of vertical result in web search examination. \emph{Proceedings of the 38th International {ACM SIGIR} Conference on Research and Development in Information Retrieval}, 193--202. \url{https://doi.org/10.1145/2766462.2767714}

\leavevmode\vadjust pre{\hypertarget{ref-108}{}}%
Lorigo, L., Haridasan, M., Brynjarsdóttir, H., Xia, L., Joachims, T., Gay, G., Granka, L., Pellacini, F., \& Pan, B. (2008). Eye tracking and online search: {Lessons} learned and challenges ahead. \emph{Journal of the American Society for Information Science and Technology}, \emph{59}(7), 1041--1052. \url{https://doi.org/10.1002/asi.20794}

\leavevmode\vadjust pre{\hypertarget{ref-lorigo2006influence}{}}%
Lorigo, L., Pan, B., Hembrooke, H., Joachims, T., Granka, L., \& Gay, G. (2006). The influence of task and gender on search and evaluation behavior using google. \emph{Information Processing \& Management}, \emph{42}(4), 1123--1131.

\leavevmode\vadjust pre{\hypertarget{ref-loyens2008selfdirected}{}}%
Loyens, S. M. M., Magda, J., \& Rikers, R. M. J. P. (2008). Self-{Directed Learning} in {Problem}-{Based Learning} and its {Relationships} with {Self}-{Regulated Learning}. \emph{Educational Psychology Review}, \emph{20}(4), 411--427. \url{https://doi.org/10.1007/s10648-008-9082-7}

\leavevmode\vadjust pre{\hypertarget{ref-mannion2020metacognition}{}}%
Mannion, J. (2020). Metacognition, self-regulation and self-regulated learning: What's the difference? In \emph{impact.chartered.college}. \url{https://impact.chartered.college/article/metacognition-self-regulation-regulated-learning-difference/}

\leavevmode\vadjust pre{\hypertarget{ref-133}{}}%
Mao, J., Liu, Y., Kando, N., Zhang, M., \& Ma, S. (2018). How does domain expertise affect users' search interaction and outcome in exploratory search? \emph{ACM Transactions on Information Systems}, \emph{36}.

\leavevmode\vadjust pre{\hypertarget{ref-marchionini1995information}{}}%
Marchionini, G. (1995). \emph{Information {Seeking} in {Electronic Environments}}. {Cambridge University Press}.

\leavevmode\vadjust pre{\hypertarget{ref-marchionini2006toward}{}}%
Marchionini, G. (2006). Toward human-computer information retrieval. \emph{Bulletin of the American Society for Information Science and Technology}, \emph{32}(5), 20--22.

\leavevmode\vadjust pre{\hypertarget{ref-marton1976qualitative_b}{}}%
Marton, F., \& Säaljö, R. (1976). On qualitative differences in learning---ii outcome as a function of the learner's conception of the task. \emph{British Journal of Educational Psychology}, \emph{46}(2), 115--127.

\leavevmode\vadjust pre{\hypertarget{ref-marton1976qualitative_a}{}}%
Marton, F., \& Säljö, R. (1976). On qualitative differences in learning: I---outcome and process. \emph{British Journal of Educational Psychology}, \emph{46}(1), 4--11.

\leavevmode\vadjust pre{\hypertarget{ref-mcgrew2020learning}{}}%
McGrew, S. (2020). Learning to evaluate: An intervention in civic online reasoning. \emph{Computers \& Education}, \emph{145}, 103711.

\leavevmode\vadjust pre{\hypertarget{ref-mcgrew2021skipping}{}}%
McGrew, S. (2021). Skipping the source and checking the contents: An in-depth look at students' approaches to web evaluation. \emph{Computers in the Schools}, \emph{38}(2), 75--97.

\leavevmode\vadjust pre{\hypertarget{ref-mcgrew2018can}{}}%
McGrew, S., Breakstone, J., Ortega, T., Smith, M., \& Wineburg, S. (2018). Can students evaluate online sources? Learning from assessments of civic online reasoning. \emph{Theory \& Research in Social Education}, \emph{46}(2), 165--193.

\leavevmode\vadjust pre{\hypertarget{ref-mcgrew2021click}{}}%
McGrew, S., \& Glass, A. C. (2021). Click {Restraint}: Teaching {Students} to {Analyze Search Results}. \emph{Proceedings of the 14th {International Conference} on {Computer}-{Supported Collaborative Learning}-{CSCL} 2021}.

\leavevmode\vadjust pre{\hypertarget{ref-mcgrew2017challenge}{}}%
McGrew, S., Ortega, T., Breakstone, J., \& Wineburg, S. (2017). The challenge that's bigger than fake news: Civic reasoning in a social media environment. \emph{American Educator}, \emph{41}(3), 4.

\leavevmode\vadjust pre{\hypertarget{ref-mihailidis2013media}{}}%
Mihailidis, P., \& Thevenin, B. (2013). Media literacy as a core competency for engaged citizenship in participatory democracy. \emph{American Behavioral Scientist}, \emph{57}(11), 1611--1622.

\leavevmode\vadjust pre{\hypertarget{ref-miller1956magical}{}}%
Miller, G. A. (1956). The magical number seven, plus or minus two: Some limits on our capacity for processing information. \emph{Psychological Review}, \emph{63}(2), 81.

\leavevmode\vadjust pre{\hypertarget{ref-miller1991self}{}}%
Miller, W. R., \& Brown, J. M. (1991). Self-regulation as a conceptual basis for the prevention and treatment of addictive behaviours. \emph{Self-Control and the Addictive Behaviours}, 3--79.

\leavevmode\vadjust pre{\hypertarget{ref-moon2011applieda}{}}%
Moon, B., Hoffman, R. R., Novak, J., \& Canas, A. (Eds.). (2011). \emph{Applied {Concept Mapping}: Capturing, {Analyzing}, and {Organizing Knowledge}} (Zeroth). {CRC Press}. \url{https://doi.org/10.1201/b10716}

\leavevmode\vadjust pre{\hypertarget{ref-url_cross_browser_extn}{}}%
Mozilla Developer Network. (2021). \emph{Building a cross-browser extension - mozilla \textbar{} MDN}. \url{https://developer.mozilla.org/en-US/docs/Mozilla/Add-ons/WebExtensions/Build_a_cross_browser_extension}.

\leavevmode\vadjust pre{\hypertarget{ref-council2000how}{}}%
National Research Council. (2000). \emph{How people learn: {Brain}, mind, experience, and school: {Expanded} edition}. {The National Academies Press}. \url{https://doi.org/10.17226/9853}

\leavevmode\vadjust pre{\hypertarget{ref-newlondon1996pedagogy}{}}%
New London Group. (1996). A pedagogy of multiliteracies: Designing social futures. \emph{Harvard Educational Review}, \emph{66}(1), 60--92.

\leavevmode\vadjust pre{\hypertarget{ref-ngss_sensemaking}{}}%
Next Generation Science Standards. (2021). \emph{Task annotation project in science \textbar{} sense-making}. \url{https://www.nextgenscience.org/sites/default/files/TAPS\%20Sense-making.pdf}.

\leavevmode\vadjust pre{\hypertarget{ref-novak2002meaningful}{}}%
Novak, J. D. (2002). Meaningful learning: The essential factor for conceptual change in limited or inappropriate propositional hierarchies leading to empowerment of learners. \emph{Science Education}, \emph{86}(4), 548--571.

\leavevmode\vadjust pre{\hypertarget{ref-novak2010learninga}{}}%
Novak, J. D. (2010). \emph{Learning, creating, and using knowledge: Concept maps as facilitative tools in schools and corporations} (2nd ed). {Routledge}.

\leavevmode\vadjust pre{\hypertarget{ref-novak1984learning}{}}%
Novak, J. D., \& Gowin, D. B. (1984). \emph{Learning how to learn}. Cambridge University Press. \url{https://doi.org/10.1017/CBO9781139173469}

\leavevmode\vadjust pre{\hypertarget{ref-o2020role}{}}%
O'Brien, H. L., Kampen, A., Cole, A. W., \& Brennan, K. (2020). The role of domain knowledge in search as learning. \emph{Conference on Human Information Interaction and Retrieval (CHIIR)}.

\leavevmode\vadjust pre{\hypertarget{ref-palani2020eye}{}}%
Palani, S., Fourney, A., Williams, S., Larson, K., Spiridonova, I., \& Morris, M. R. (2020). An eye tracking study of web search by people with and without dyslexia. \emph{Proceedings of the 43rd International ACM SIGIR Conference on Research and Development in Information Retrieval}, 729--738. \url{https://doi.org/10.1145/3397271.3401103}

\leavevmode\vadjust pre{\hypertarget{ref-pan2004determinants}{}}%
Pan, B., Hembrooke, H. A., Gay, G. K., Granka, L. A., Feusner, M. K., \& Newman, J. K. (2004). The determinants of web page viewing behavior: An eye-tracking study. \emph{Proceedings of the 2004 Symposium on Eye Tracking Research \& Applications}, 147--154.

\leavevmode\vadjust pre{\hypertarget{ref-pea2014learning}{}}%
Pea, R., \& Jacks, D. (2014). \emph{The learning analytics workgroup: A report on building the field of learning analytics for personalized learning at scale}. \url{https://ed.stanford.edu/sites/default/files/law_report_complete_09-02-2014.pdf}; Stanford, CA: Stanford University.

\leavevmode\vadjust pre{\hypertarget{ref-note_taking_survey_penn_state}{}}%
Penn State Learning. (2021). \emph{Listening and note taking survey \textbar{} penn state learning}. \url{https://pennstatelearning.psu.edu/listening-and-note-taking-survey}; Pennsylvania State University.

\leavevmode\vadjust pre{\hypertarget{ref-pennanen2003students}{}}%
Pennanen, M., \& Vakkari, P. (2003). Students' conceptual structure, search process, and outcome while preparing a research proposal: A longitudinal case study. \emph{Journal of the American Society for Information Science and Technology}, \emph{54}(8), 759--770.

\leavevmode\vadjust pre{\hypertarget{ref-piaget1936origins}{}}%
Piaget, J. (1936). \emph{Origins of intelligence in children.}

\leavevmode\vadjust pre{\hypertarget{ref-pirolli1996scatter}{}}%
Pirolli, P., Schank, P., Hearst, M., \& Diehl, C. (1996). Scatter/gather browsing communicates the topic structure of a very large text collection. \emph{Conference on Human Factors in Computing Systems (CHI'96)}.

\leavevmode\vadjust pre{\hypertarget{ref-121}{}}%
Qvarfordt, P., Golovchinsky, G., Dunnigan, T., \& Agapie, E. (2013). Looking ahead: {Query} preview in exploratory search. \emph{Proceedings of the 36th International {ACM SIGIR} Conference on Research and Development in Information Retrieval}, 243--252. \url{https://doi.org/10.1145/2484028.2484084}

\leavevmode\vadjust pre{\hypertarget{ref-url_rieh_homepage}{}}%
Rieh, S. Y. (2020). \emph{Research area 1: Searching as learning}. \url{https://rieh.ischool.utexas.edu/research}.

\leavevmode\vadjust pre{\hypertarget{ref-rieh2016searching}{}}%
Rieh, S. Y., Collins-Thompson, K., Hansen, P., \& Lee, H.-J. (2016). Towards searching as a learning process: A review of current perspectives and future directions. \emph{Journal of Information Science}, \emph{42}(1), 19--34. \url{https://doi.org/10.1177/0165551515615841}

\leavevmode\vadjust pre{\hypertarget{ref-rieh2012amount}{}}%
Rieh, S. Y., Kim, Y.-M., \& Markey, K. (2012). Amount of invested mental effort (AIME) in online searching. \emph{Information Processing \& Management}, \emph{48}(6), 1136--1150.

\leavevmode\vadjust pre{\hypertarget{ref-roy2020exploring}{}}%
Roy, N., Moraes, F., \& Hauff, C. (2020). Exploring users' learning gains within search sessions. \emph{Conference on Human Information Interaction and Retrieval (CHIIR)}.

\leavevmode\vadjust pre{\hypertarget{ref-roy2021note}{}}%
Roy, N., Torre, M. V., Gadiraju, U., Maxwell, D., \& Hauff, C. (2021). Note the highlight: Incorporating active reading tools in a search as learning environment. \emph{Proceedings of the 2021 Conference on Human Information Interaction and Retrieval}, 229--238.

\leavevmode\vadjust pre{\hypertarget{ref-rumelhart1981accretion}{}}%
Rumelhart, D. E., \& Norman, D. A. (1981). Accretion, tuning and restructuring: Three modes of learning. In J. W. Cotton \& K. R. (Eds.), \emph{Semantic factors in cognition} (pp. 37--90).

\leavevmode\vadjust pre{\hypertarget{ref-rumelhart1977representation}{}}%
Rumelhart, D. E., \& Ortony, A. (1977). The representation of knowledge in memory. In R. C. Anderson, S. R. J., \& M. W. E. (Eds.), \emph{Schooling and the acquisition of knowledge} (pp. 99--135). Hillsdale, NJ: Erlbaum.

\leavevmode\vadjust pre{\hypertarget{ref-ryan1982control}{}}%
Ryan, R. M. (1982). Control and information in the intrapersonal sphere: An extension of cognitive evaluation theory. \emph{Journal of Personality and Social Psychology}, \emph{43}(3), 450.

\leavevmode\vadjust pre{\hypertarget{ref-ryan2000intrinsic}{}}%
Ryan, R. M., \& Deci, E. L. (2000a). Intrinsic and extrinsic motivations: Classic definitions and new directions. \emph{Contemporary Educational Psychology}, \emph{25}(1), 54--67.

\leavevmode\vadjust pre{\hypertarget{ref-ryan2000self}{}}%
Ryan, R. M., \& Deci, E. L. (2000b). Self-determination theory and the facilitation of intrinsic motivation, social development, and well-being. \emph{American Psychologist}, \emph{55}(1), 68.

\leavevmode\vadjust pre{\hypertarget{ref-ryan2017self}{}}%
Ryan, R. M., \& Deci, E. L. (2017). \emph{Self-determination theory: Basic psychological needs in motivation, development, and wellness}. Guilford Publications.

\leavevmode\vadjust pre{\hypertarget{ref-saks2014distinguishing}{}}%
Saks, K., \& Leijen, Ä. (2014). Distinguishing {Self}-directed and {Self}-regulated {Learning} and {Measuring} them in the {E}-learning {Context}. \emph{Procedia - Social and Behavioral Sciences}, \emph{112}, 190--198. \url{https://doi.org/10.1016/j.sbspro.2014.01.1155}

\leavevmode\vadjust pre{\hypertarget{ref-saracevic1975relevance}{}}%
Saracevic, T. (1975). Relevance: A review of and a framework for the thinking on the notion in information science. \emph{Journal of the American Society for Information Science}, \emph{26}(6), 321--343.

\leavevmode\vadjust pre{\hypertarget{ref-saracevic2007relevance}{}}%
Saracevic, T. (2007a). Relevance: A review of the literature and a framework for thinking on the notion in information science. {Part II}: Nature and manifestations of relevance. \emph{Journal of the American Society for Information Science and Technology}, \emph{58}(13), 1915--1933. \url{https://doi.org/10.1002/asi.20682}

\leavevmode\vadjust pre{\hypertarget{ref-saracevic2007relevancea}{}}%
Saracevic, T. (2007b). Relevance: A review of the literature and a framework for thinking on the notion in information science. {Part III}: Behavior and effects of relevance. \emph{Journal of the American Society for Information Science and Technology}, \emph{58}(13), 2126--2144.

\leavevmode\vadjust pre{\hypertarget{ref-saracevic2016notion}{}}%
Saracevic, T. (2016). The {Notion} of {Relevance} in {Information Science}: {Everybody} knows what relevance is. {But}, what is it really? \emph{Synthesis Lectures on Information Concepts, Retrieval, and Services}.

\leavevmode\vadjust pre{\hypertarget{ref-sawyer2005cambridge}{}}%
Sawyer, R. K. (2005). \emph{The {Cambridge} handbook of the learning sciences}. {Cambridge University Press}.

\leavevmode\vadjust pre{\hypertarget{ref-63}{}}%
Scharinger, C., Kammerer, Y., \& Gerjets, P. (2016). Fixation-{Related EEG Frequency Band Power Analysis}: {A Promising Neuro}-{Cognitive Methodology} to {Evaluate} the {Matching}-{Quality} of {Web Search Results}? \emph{{HCI International} 2016 \textendash{} {Posters}' {Extended Abstracts}}, 245--250. \url{https://doi.org/10.1007/978-3-319-40548-3_41}

\leavevmode\vadjust pre{\hypertarget{ref-schraw1994assessing}{}}%
Schraw, G., \& Dennison, R. S. (1994). Assessing {Metacognitive Awareness}. \emph{Contemporary Educational Psychology}, \emph{19}(4), 460--475. \url{https://doi.org/10.1006/ceps.1994.1033}

\leavevmode\vadjust pre{\hypertarget{ref-simon1956rational}{}}%
Simon, H. A. (1956). Rational choice and the structure of the environment. \emph{Psychological Review}, \emph{63}(2), 129.

\leavevmode\vadjust pre{\hypertarget{ref-69}{}}%
Slanzi, G., Balazs, J. A., \& Velásquez, J. D. (2017). Combining eye tracking, pupil dilation and {EEG} analysis for predicting web users click intention. \emph{Information Fusion}, \emph{35}, 51--57. \url{https://doi.org/10.1016/j.inffus.2016.09.003}

\leavevmode\vadjust pre{\hypertarget{ref-129}{}}%
Smith, C. L., Gwizdka, J., \& Feild, H. (2016). Exploring the use of query auto completion: {Search} behavior and query entry profiles. \emph{Proceedings of the 2016 {ACM} on Conference on Human Information Interaction and Retrieval}, 101--110. \url{https://doi.org/10.1145/2854946.2854975}

\leavevmode\vadjust pre{\hypertarget{ref-smith2008user}{}}%
Smith, C. L., \& Kantor, P. B. (2008). User adaptation: Good results from poor systems. \emph{Proceedings of the 31st Annual International ACM SIGIR Conference on Research and Development in Information Retrieval}, 147--154.

\leavevmode\vadjust pre{\hypertarget{ref-spink1997study}{}}%
Spink, A. (1997). Study of interactive feedback during mediated information retrieval. \emph{Journal of the American Society for Information Science}.

\leavevmode\vadjust pre{\hypertarget{ref-sheg2021webpage_comparison}{}}%
Stanford History Education Group. (2021). \emph{Webpage comparison \textbar{} {Civic Online Reasoning}}. \url{https://cor.stanford.edu/curriculum/assessments/webpage-comparison}.

\leavevmode\vadjust pre{\hypertarget{ref-syed2020improving}{}}%
Syed, R., Collins-Thompson, K., Bennett, P. N., Teng, M., Williams, S., Tay, D. W. W., \& Iqbal, S. (2020). Improving learning outcomes with gaze tracking and automatic question generation. \emph{The Web Conference (WWW)}.

\leavevmode\vadjust pre{\hypertarget{ref-tahamtan2019effect}{}}%
Tahamtan, I. (2019). The effect of motivation on web search behaviors of health consumers. \emph{Proceedings of the 2019 Conference on Human Information Interaction and Retrieval}, 401--404.

\leavevmode\vadjust pre{\hypertarget{ref-terlecki2020revising}{}}%
Terlecki, M. (2020). Revising the {Metacognitive Awareness Inventory} ({MAI}) to be {More User}-{Friendly}. In \emph{Improve with Metacognition}. \url{https://www.improvewithmetacognition.com/revising-the-metacognitive-awareness-inventory/}

\leavevmode\vadjust pre{\hypertarget{ref-terlecki2018call}{}}%
Terlecki, M., \& McMahon, A. (2018). A {Call} for {Metacognitive Intervention}: Improvements {Due} to {Curricular Programming} in {Leadership}. \emph{Journal of Leadership Education}, \emph{17}(4), 130--145. \url{https://doi.org/10.12806/V17/I4/R8}

\leavevmode\vadjust pre{\hypertarget{ref-note_taking_strategies_umass}{}}%
UMass Amherst Student Success. (2021). \emph{Note taking strategies inventory \textbar{} success@UMass}. \url{https://www.umass.edu/studentsuccess/sites/default/files/inline-files/note-taking-strategies_0.pdf}.

\leavevmode\vadjust pre{\hypertarget{ref-vakkari2001theory}{}}%
Vakkari, P. (2001a). A theory of the task-based information retrieval process: A summary and generalisation of a longitudinal study. \emph{Journal of Documentation}, \emph{57}(1), 44--60. \url{https://doi.org/10.1108/EUM0000000007075}

\leavevmode\vadjust pre{\hypertarget{ref-vakkari2016searching}{}}%
Vakkari, P. (2016). Searching as learning: A systematization based on literature. \emph{Journal of Information Science}, \emph{42}(1), 7--18. \url{https://doi.org/10.1177/0165551515615833}

\leavevmode\vadjust pre{\hypertarget{ref-vakkari2020usefulness}{}}%
Vakkari, P. (2020). The {Usefulness} of {Search Results}: A {Systematization} of {Types} and {Predictors}. \emph{Proceedings of the 2020 {Conference} on {Human Information Interaction} and {Retrieval}}, 243--252. \url{https://doi.org/10.1145/3343413.3377955}

\leavevmode\vadjust pre{\hypertarget{ref-vakkari2000cognition}{}}%
Vakkari, P. (2000). Cognition and changes of search terms and tactics during task performance: A longitudinal case study. In \emph{Content-based multimedia information access-volume 1} (pp. 894--907).

\leavevmode\vadjust pre{\hypertarget{ref-vakkari2001changes}{}}%
Vakkari, P. (2001b). Changes in search tactics and relevance judgements when preparing a research proposal a summary of the findings of a longitudinal study. \emph{Information Retrieval}, \emph{4}(3), 295--310.

\leavevmode\vadjust pre{\hypertarget{ref-vakkari2019modeling}{}}%
Vakkari, P., Völske, M., Potthast, M., Hagen, M., \& Stein, B. (2019). Modeling the usefulness of search results as measured by information use. \emph{Information Processing \& Management}, \emph{56}(3), 879--894. \url{https://doi.org/10.1016/j.ipm.2019.02.001}

\leavevmode\vadjust pre{\hypertarget{ref-van2017development}{}}%
Van Der Vaart, R., \& Drossaert, C. (2017). Development of the digital health literacy instrument: Measuring a broad spectrum of health 1.0 and health 2.0 skills. \emph{Journal of Medical Internet Research}, \emph{19}(1), e27.

\leavevmode\vadjust pre{\hypertarget{ref-villa2013relevance}{}}%
Villa, R., \& Halvey, M. (2013). Is relevance hard work? Evaluating the effort of making relevant assessments. \emph{Proceedings of the 36th International ACM SIGIR Conference on Research and Development in Information Retrieval}, 765--768.

\leavevmode\vadjust pre{\hypertarget{ref-yue2018optimizing}{}}%
Wang, Y., Yin, D., Jie, L., Wang, P., Yamada, M., Chang, Y., \& Mei, Q. (2018). Optimizing whole-page presentation for web search. \emph{ACM Trans. Web}, \emph{12}(3). \url{https://doi.org/10.1145/3204461}

\leavevmode\vadjust pre{\hypertarget{ref-weber2019informationseeking}{}}%
Weber, H., Becker, D., \& Hillmert, S. (2019). Information-seeking behaviour and academic success in higher education: Which search strategies matter for grade differences among university students and how does this relevance differ by field of study? \emph{Higher Education}, \emph{77}(4), 657--678. \url{https://doi.org/10.1007/s10734-018-0296-4}

\leavevmode\vadjust pre{\hypertarget{ref-weber2018can}{}}%
Weber, H., Hillmert, S., \& Rott, K. J. (2018). Can digital information literacy among undergraduates be improved? Evidence from an experimental study. \emph{Teaching in Higher Education}, \emph{23}(8), 909--926. \url{https://doi.org/10.1080/13562517.2018.1449740}

\leavevmode\vadjust pre{\hypertarget{ref-white2016interactions}{}}%
White, R. (2016a). \emph{Interactions with search systems}. Cambridge University Press.

\leavevmode\vadjust pre{\hypertarget{ref-white_2016_iwss_learning}{}}%
White, R. (2016b). Learning and use. In \emph{Interactions with search systems} (pp. 231--248). {Cambridge University Press}. \url{https://doi.org/10.1017/CBO9781139525305.010}

\leavevmode\vadjust pre{\hypertarget{ref-white2009characterizing}{}}%
White, R., Dumais, S., \& Teevan, J. (2009). Characterizing the influence of domain expertise on web search behavior. \emph{Proceedings of the {Second ACM International Conference} on {Web Search} and {Data Mining} - {WSDM} '09}, 132. \url{https://doi.org/10.1145/1498759.1498819}

\leavevmode\vadjust pre{\hypertarget{ref-wildemuth2004effects}{}}%
Wildemuth, B. M. (2004). The effects of domain knowledge on search tactic formulation. \emph{Journal of the American Society for Information Science and Technology}, \emph{55}(3), 246--258. \url{https://doi.org/10.1002/asi.10367}

\leavevmode\vadjust pre{\hypertarget{ref-wilson2013comparison}{}}%
Wilson, M. J., \& Wilson, M. L. (2013). A comparison of techniques for measuring sensemaking and learning within participant-generated summaries. \emph{Journal of the American Society for Information Science and Technology}, \emph{64}(2), 291--306.

\leavevmode\vadjust pre{\hypertarget{ref-wilson1999models}{}}%
Wilson, T. D. (1999). Models in information behaviour research. \emph{Journal of Documentation}, \emph{55}(3), 249--270.

\leavevmode\vadjust pre{\hypertarget{ref-wineburg2016students}{}}%
Wineburg, S., \& McGrew, S. (2016). Why students can't google their way to the truth. \emph{Education Week}, \emph{36}(11), 22--28.

\leavevmode\vadjust pre{\hypertarget{ref-wineburg2017lateral}{}}%
Wineburg, S., \& McGrew, S. (2017). \emph{Lateral reading: Reading less and learning more when evaluating digital information}.

\leavevmode\vadjust pre{\hypertarget{ref-wittrock1989generative}{}}%
Wittrock, M. C. (1989). Generative processes of comprehension. \emph{Educational Psychologist}, \emph{24}(4), 345--376.

\leavevmode\vadjust pre{\hypertarget{ref-xu2020does}{}}%
Xu, L., Zhou, X., \& Gadiraju, U. (2020). How does team composition affect knowledge gain of users in collaborative web search? \emph{Conference on Hypertext and Social Media (HT)}.

\leavevmode\vadjust pre{\hypertarget{ref-yu2018PredictingUserKnowledgea}{}}%
Yu, R., Gadiraju, U., Holtz, P., Rokicki, M., Kemkes, P., \& Dietze, S. (2018). Predicting {User Knowledge Gain} in {Informational Search Sessions}. \emph{The 41st {International ACM SIGIR Conference} on {Research} \& {Development} in {Information Retrieval}}, 75--84. \url{https://doi.org/10.1145/3209978.3210064}

\leavevmode\vadjust pre{\hypertarget{ref-zhang2014towards}{}}%
Zhang, P., \& Soergel, D. (2014). Towards a comprehensive model of the cognitive process and mechanisms of individual sensemaking. \emph{Journal of the Association for Information Science and Technology}, \emph{65}(9), 1733--1756. \url{https://doi.org/10.1002/asi.23125}

\leavevmode\vadjust pre{\hypertarget{ref-zhang2016process}{}}%
Zhang, P., \& Soergel, D. (2016). Process patterns and conceptual changes in knowledge representations during information seeking and sensemaking: A qualitative user study. \emph{Journal of Information Science}, \emph{42}(1), 59--78.

\leavevmode\vadjust pre{\hypertarget{ref-zhang2011predicting}{}}%
Zhang, X., Cole, M., \& Belkin, N. (2011). Predicting {Users}' {Domain Knowledge} from {Search Behaviors}. \emph{Proceedings of the 34th {International ACM SIGIR Conference} on {Research} and {Development} in {Information Retrieval}}, 1225--1226. \url{https://doi.org/10.1145/2009916.2010131}

\leavevmode\vadjust pre{\hypertarget{ref-zimmerman1989social}{}}%
Zimmerman, B. J. (1989). A social cognitive view of self-regulated academic learning. \emph{Journal of Educational Psychology}, \emph{81}(3), 329.

\leavevmode\vadjust pre{\hypertarget{ref-zlatkin2021students}{}}%
Zlatkin-Troitschanskaia, O., Hartig, J., Goldhammer, F., \& Krstev, J. (2021). Students' online information use and learning progress in higher education \textendash{} {A} critical literature review. \emph{Studies in Higher Education}, 1--26. \url{https://doi.org/10.1080/03075079.2021.1953336}

\end{CSLReferences}

%%%%% REFERENCES


\end{document}
